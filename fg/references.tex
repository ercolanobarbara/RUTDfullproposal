\renewcommand{\hdrtitle}{References}

\clearpage

\section{References}
\begin{literature}
\item 
Ahrens, T.~J. and Rubin, A.~M. (1993) Impact-induced tensional failure in rock.
\textit{J. Geophysical Res.} \textbf{98}, 1185

\item 
Amelin, Y., Krot, A., Hutcheon, I.~D. and Ulyanov, A.~A. (2002)
Lead Isotopic Ages of Chondrules and Calcium-Aluminum-Rich Inclusions. 
\sci \textbf{297}, 1678

\item
Apai, D., Pascucci, I., Bouwman, J., Natta, A., Henning, Th. and
Dullemond, C.P. (2005) The onset of planet formation in Brown Dwarf disks.
\sci, \textbf{310}, 834

\item 
Balbus, S.~A. and Hawley, J.~F. (1998) Instability, turbulence and enhanced
transport in accretion disks, \rmf \textbf{70}, 1

\item 
Barge, P. and Sommeria, J. (1995) Did planet formation begin inside
persistent gaseous vortices?, \aap \textbf{295}, L1

\item 
Beckwith, S.~V.~W. and Sargent, A.~I. (1991) Particle emissivity in
circumstellar disks. \apj \textbf{381}, 250

\item 
Bell, K.~R., Cassen, P.~M., Klahr, H.~H. and Henning, Th. (1997) 
The Structure and Appearance of Protostellar Accretion Disks: 
Limits on Disk Flaring. \apj \textbf{486}, 372

\item 
Benz, W. (2000) Low Velocity Collisions and the Growth of Planetesimals.
\ssr \textbf{92}, 279

\item 
Benz, W. and Asphaug, E. (1994) Impact simulations with fracture. I - Method
and tests. \ica \textbf{107}, 98

\item 
Bizzarro, M., Baker, J.~A. and Haack, H. (2004) Mg isotope evidence for
contemporaneous formation of chondrules and refractory inclusions. \nat
\textbf{431}, 275

\item 
Blum, J., Wurm, G., Poppe, T. and Heim, L.-O. (1999) Aspects of Laboratory Dust
Aggregation with Relevance to the Formation of Planetesimals. In:
\textit{Laboratory Astrophysics and Space Research}, Astrophysics and Space
Science Library, Vol. \textbf{236} (Eds. P. Ehrenfreund, K. Krafft, H. Kochan,
V. Pirronello) Kluwer Academic Publishers, Dordrecht, 399

\item 
Blum, J. and Wurm, G. (2000) Experiments on Sticking, Restructuring and
Fragmentation of Preplanetary Dust Aggregates. \ica \textbf{143}, 138-146

\item 
Blum, J., Wurm, G., Kempf, S., Poppe, T., Klahr, H., et al. (2000) Growth
and Form of Planetary Seedlings: Results from a Microgravity Aggregation
Experiment.
\prl \textbf{85}, 2426-2429

\item 
Blum, J. (2004) Grain Growth and Coagulation. In: \textit{Astrophysics of
Dust}, ASP Conference Series, Vol. \textbf{309} (Eds. A. Witt, G. Clayton and
B. Draine), 369-391

\item 
Blum, J. and Schr\"apler, R. (2004) Structure and Mechanical Properties of
High-Porosity Macroscopic Agglomerates Formed by Random Ballistic Deposition.
\prl \textbf{93}, 115503

\item 
Bockel\'ee-Morvan, D., Gauthier, D., Hersant, F., Hur\'e J.-M. and Robert,
F.  (2002) Turbulent radial mixing in the solar nebula as the source of
crystalline silicates in comets. \aap \textbf{384}, 1107

\item 
Bouwman, J., de Koter, A., van den Ancker, M.~E. and Waters,
L.~B.~F.~M. (2000) The composition of the circumstellar dust around the
Herbig Ae stars AB Aur and HD 163296. \aap \textbf{360}, 213

\item 
Bouwman, J., Meeus, G., de Koter, A., Hony, S., Dominik, C. and Waters,
L.~B.~F.~M. (2001) Processing of silicate dust grains in Herbig Ae/Be
systems.
\aap \textbf{375}, 950

\item 
Brazzle, R.~H. (1999) Verification and interpretation of the I-Xe chronometer.
\gca \textbf{63}, 739

\item 
Bridges, F. G., Supulver K. D., Lin, D. N. C. {\it et al.}
(1996) Energy loss and sticking mechanisms in particle aggregation in
planetesimal formation. \ica \textbf{123}, 422

\item
Brownlee D. E., Burnett D. {\em et~al.} (1996).  Stardust: Comet and
interstellar dust sample return mission, in \textit{Physics, Chemistry,
and Dynamics of Interplanetary Dust}, ed. B.A.S.~Gustafson and
M.S.~Hanner (ASP Conf. Series 104).

\item 
Brucato, J.~R., Colangeli, L., Mennella, V., Palumbo, P. and Bussoletti, E.
(1999) Mid-infrared spectral evolution of thermally annealed amorphous
pyroxene. \aap \textbf{348}, 1012

\item 
Brucato, J.~R., Mennella, V., Colangeli, L., Rotundi, A. and Palumbo,
P. (2000) Production and processing of silicates in laboratory and in
space. \pss \textbf{50}, 829

\item 
Carpenter, J.~M., Wolf, S., Schreyer, K., Launhardt, R. and Henning,
Th. (2005) Evolution of Cold Circumstellar Dust around Solar-type Stars. \aj
\textbf{129}, 1049

\item 
Chokshi, A.,  Tielens A. G. G. M. and Hollenbach, D. (1993)
Dust coagulation \apj \textbf{407}, 806 

\item 
Clayton, R.~N. (1993) Oxygen isotopes in meteorites. \textit{Ann. Rev. Earth
\& Planetary Sci.} \textbf{21}, 115

\item 
Cuzzi, J.~N., Dobrovolskis, A.~R. and Champney, J.~M. (1993) 
Particle-gas dynamics in the midplane of a protoplanetary nebula
\ica \textbf{106}, 102

\item 
Cuzzi, J., Hogan, R. C., Paque, J. M., Dobrovolskis, A. R. (2001)
Size-selective concentration of chondrules and other small particles in
protoplanetary nebula turbulence \apj \textbf{546}, 496

\item 
Dahneke (1975)   J.~Colloid Interf.~Sci. \textbf{1}, 58 

\item D'Alessio, P., Calvet, N. and Woolum, D.S. (2005) Thermal structure of
  protoplanetary disks. In: \textit{Chondrites and the protoplanetary disk\/},
  ASP Conf. Ser. Vol. 341 (Eds. A.N. Krot, E.R.D. Scott, B. Reipurth), 353

\item 
Dominik, C. and Tielens, A.~G.~G.~M. (1997) The Physics of Dust Coagulation
and the Structure of Dust Aggregates in Space. \ica \textbf{480}, 647

\item 
Dullemond,~C.~P. (2002) The 2-D structure of dusty disks around Herbig Ae/Be
stars. I. Models with grey opacities. \aap \textbf{395}, 853

\item 
Dullemond, C.~P. and Natta, A. (2003) The effect of scattering on the
structure and SED of protoplanetary disks. \aap \textbf{408}, 161

\item 
Dullemond, C.~P. and Dominik, C. (2004) Dust settling in disks: from flaring
to self-shadowing. \aap \textbf{421}, 1075

\item 
Dullemond, C.~P. and Dominik, C. (2004) Flaring vs. self-shadowed disks: The
SEDs of Herbig Ae/Be stars. \aap \textbf{417}, 159

\item 
Dullemond C.~P. and Dominik, C. (2005) Dust coagulation in protoplanetary
disk: a too rapid depletion of small grains. \aap \textbf{434}, 971

\item {Dullemond}, C.~P. and {Apai}, D. and {Walch}, S. (2006)
  Crystalline Silicates as a Probe of Disk Formation History, 
  \apjl \textbf{640}, 67

\item Dullemond, C.P., Hollenbach, D., Kamp, I. \& D'Alessio, P. (2006)
  Models of the structure and evolution of protoplanetary disks, 
  in ``Protostars and Planets V'', eds. Reipurth, Jewitt \& Keil,\\ {\tt
  http://ifa.hawaii.edu/UHNAI/ppv.htm}

\item
Eisner, J.~A., Lane, B.~F., Akeson, R.~L., Hillenbrand, L.~A. and 
Sargent, A.~I. (2003) Near-Infrared Interferometric Measurements of 
Herbig Ae/Be Stars \apj \textbf{588}, 360

\item
Fabian, D., J\"ager, C., Henning, Th., Dorschner, J. and Mutschke, H. (2000)
Steps toward interstellar silicate mineralogy. V. Thermal Evolution of
Amorphous Magnesium Silicates and Silica. \aap \textbf{364}, 282

\item
Finocchi, F., Gail, H.-P. and Duschl, W.~J. (1997a) Chemical reactions in
protoplanetary disks II. Carbon dust oxidation. \aap \textbf{325}, 1264
 
\item
Finocchi, F. and Gail, H.-P. (1997b) Chemical reactions in protoplanetary
accretion disks III. The role of ionisation processes. \aap \textbf{327},
825

\item
Fischer, O., Henning, T. and Yorke, H.~W. (1996) Simulation of polarization
maps. II. The circumstellar environment of pre-main sequence objects.
\aap \textbf{308}, 863

\item
Forrest, W.~J., Sargent, B., Furlan, E., D'Alessio, P., Calvet, N.,
Hartmann, L., Uchida, K.~I., Green, J.~D., Watson, D.~M., Chen, C.~H. and 11
coauthors (2004) Mid-infrared Spectroscopy of Disks around Classical T Tauri
Stars.
\apjs \textbf{154}, 443

\item
Gail, H.-P. (2001) Radial mixing in protoplanetary accretion disks. I.
Stationary disc models with annealing and carbon combustion. \aap
\textbf{378}, 192-213

\item
Gail, H.-P. (2002) Radial mixing in protoplanetary accretion
disks. III. Carbon dust oxidation and abundance of hydrocarbons in
comets. \aap \textbf{390}, 253-265

\item
Gail, H.-P. (2003) Formation and Evolution of Minerals in Accretion Disks
and Stellar Outflows. In: \textit{Astromineralogy} (Ed. Th. Henning) Lecture
Notes in Physics Vol. \textbf{609}. Springer, Heidelberg. p. 55-120

%\item
%Gail, H.-P. (2003b) Radial mixing in protoplanetary accretion disks. IV.
%Metamorphosis of the silicate dust complex. \aap \textbf{413}, 571-591

\item
Gail, H.-P. (2004) Radial mixing in protoplanetary accretion disks. IV.
Metamorphosis of the silicate dust complex. \aap \textbf{413}, 571-591

\item
Gail, H.-P. and Tscharnuter, W.~M. (2005) Evolution of protoplanetary disks
including detailed chemistry and mineralogy. In: \textit{Reactive Flow,
Diffusion and Transport} (Ed. R. Rannacher et. al.) Springer,
Berlin-Heidelberg.  (accepted)

\item
Gilmour, J.~D., Pravdivtseva, O.~V., Busfield, A. and Hohenberg C.~M. (2004)
I-Xe and the chronology of the early solar system.  \textit{Workshop on
chondrites and the protoplanetary disk} abstr. no. 9054.

\item 
Gilmour J.D., Pravdivtseva O.V., Busfield A., and Hohenberg C.M. (2006).
The I-Xe chronometer and the early solar system.  
\textit{Met. Planet. Sci.} \textbf{41}, 19

\item
Gil, C., Malbet, F., Sch\"oller, M., Chesneau, O., Leinert, Ch. (2005)
Observations of 51 Ophiuchi with MIDI at the VLTI astro-ph/0508052

\item
Goldreich, P. and Ward, W.~R. (1973) The Formation of Planetesimals. \apj
\textbf{183}, 1051

\item
G\"unther, R., Sch\"afer, C. and Kley, W. (2004)  Evolution of irradiated
circumbinary disks \aap \textbf{423}, 559

\item
G\"unther, R. and Kley, W. (2002) Circumbinary disk evolution \aap
\textbf{387}, 550

\item
Haisch Jr., K.~E.,  Lada, E.~A. and Lada, C.~J. (2001) Disk Frequencies and
Lifetimes in Young Clusters. \apj \textbf{553}, L153

\item
Hallenbeck, S.~L.; Nuth, J.~A. and Daukantas, P.~L. (1998) Mid-Infrared
Spectral Evolution of Amorphous Magnesium Silicate Smokes Annealed in
Vacuum: Comparison to Cometary Spectra. \ica \textbf{131}, 198

\item
Harker, D.~E. and Desch, S.~J. (2002) Annealing of Silicate Dust by Nebular 
Shocks at 10 AU. \apj \textbf{565}, L109

\item Harker, D.E., Woodward, C.E., and Wooden, D.H. (2005) The dust grains
  from 9P/Tempel 1 before and after the encounter with deep
  impact. \textit{Science\/}, \textbf{310}, 278

\item
Hashimoto, A. (1990) Evaporation kinetics of forsterite and implications for
the early solar nebula. \nat \textbf{347}, 53

\item
Hashimoto, A. (1991) Evaporation of melilite. \textit{Meteoritics\/}
\textbf{26}, 344

\item
Hatzes, A. P., Bridges, F., Lin, D. N. C., Sachtjen, S. (1991)
Coagulation of particles in Saturn's rings - Measurements of the cohesive
force of water frost \ica \textbf{89}, 113

\item
Heim, L.-O., Blum, J., Preuss, M. and Butt, H.-J. (1999) Adhesion and
Friction Forces Between Spherical Micrometer-Sized Particles. \prl
\textbf{83}, 3328

\item
Henning, Th., Il'In, V. B., Krivova, N.A., Michel, B., Voshchinnikov, N.V.
(1999) WWW database of optical constants for astronomy \aaps \textbf{136},
405.

\item
Henning, Th. (2003) Astromineralogy Lecture Notes
in Physics Vol. \textbf{609}. Springer, Heidelberg.

\item
Hines, D. C., Backman, D. E., Bouwman, J. et al. (2006) 
Discovery of an unusual debris system associated with HD 12039 \apj,
\textbf{638}, 1070

\item
Honda, M., Kataza, H., Okamoto, Y.K., Miyata, T. et al. (2003) 
Detection of Crystalline Silicates around the T Tauri Star Hen 3-600A
\apj \textbf{585}, L59

\item
Hueso, R. and Guillot, T. (2003) Evolution of the Protosolar Nebula and
Formation of the Giant Planets. \ssr \textbf{106}, 105

\item
Ilgner, M., Henning, Th., Markwick, A.~J. and Millar, T.~J. (2004) Transport
processes and chemical evolution in steady accretion disk flows. \aap
\textbf{415}, 643

\item
Jessberger, E.K., A. Christoforidis, J. Kissel (1988): 
Aspects of the major element composition of Halley's dust.
Nature \textbf{332}, 691

Jessberger E.K. (1999) 
On the elemental, isotopic and mineralogical ingredients of
ROCKY cometary particulates.
Space Science Reviews \textbf{90}, 91

\item
Johansen, A., Andersen, A.~C., Brandenburg, A. (2004) Simulations of
dust-trapping vortices in protoplanetary discs. \aap \textbf{417}, 361

\item
Johansen, A. and Klahr, H. (2005) Dust diffusion in protoplanetary discs by
magnetorotational turbulence. \apj, \textbf{634}, 1353

\item
Johansen, A., Klahr, H. and  Henning, Th. (2006) Gravoturbulent formation of
planetesimals. \apj  \textbf{636}, 1121

\item 
Johansen, A., Henning, Th. and  Klahr, H. (2006) 
Dust sedimentation and self-sustained Kelvin-Helmholtz turbulence
in protoplanetary disk mid-planes. 
I. Radially symmetric simulations. \apj, in press

\item Johansen, A., Klahr, H. and Mee, A.J. (2006) 
Diffusion properties of magnetorotational turbulence in accretion disks:
Effects of an imposed magnetic field. \mn, in press, 
ArXiv Astrophysics e-prints arXiv:astro-ph/0603765

\item
Keller, Ch. and Gail, H.-P. (2004) Radial mixing in protoplanetary accretion
disks. VI. Mixing by large-scale radial flows. \aap \textbf{415}, 1177

\item
Kempf, S., Pfalzner, S., Henning, Th. (1999) N-Particle-Simulations of dust
growth. I. Growth driven by Brownian Motion \ica \textbf{141}, 388

\item
Kessler-Silacci, J.~E., Hillenbrand, L.~A., Blake, G.~A. and Meyer, M.~R.
(2005) 8-13 $\mu$m spectroscopy of young stellar objects: evolution of
the silicate feature \apj \textbf{622}, 404

\item 
Kessler-Silacci, J.~E., Augereau, J.-C., Dullemond, C.P., et al. (2006) c2d
Spitzer IRS spectra of disks around T Tauri stars. I. Silicate emission and
grain growth, \apj \textbf{639}, 275

\item
Kim, J. S., Hines, D. C. et al. (2005) Cold outer disks associated with
sun-like stars. \apj, \textbf{632}, 659

\item
Kita, N.~T., Nagahara, H., Togashi, S. and Morishita, Y. (2000)
A short duration of chondrule formation in the solar nebula: evidence from 
$^{26}$Al in Semarkona ferromagnesian chondrules. \gca \textbf{64}, 3913

%\item
%Kita, N.~T., Ikeda, Y., Togashi, S., Liu, Y., Morishita, Y. and Weisberg,
%M.~K.  (2004) Origin of ureilites inferred from a SIMS oxygen isotopic and
%trace element study of clasts in the Dar al Gani 319 polymict ureilite. \gca
%\textbf{68}, 4213

\item
Kita N.~T., Huss G.~R. et al. (2004). Constraints on the origin of
chondrules and CAIs from short-lived and long-lived
radionuclides.   \textit{Workshop on chondrites and the protoplanetary
disk, abstr.\ no \#9064}.

\item
Klahr, H.~H. and Henning, Th. (1997) Particle-Trapping Eddies in
Protoplanetary Accretion Disks. \ica \textbf{128}, 213.

\item
Klahr, H., Henning, Th. and Kley, W. (1999) On the Azimuthal Structure of
Thermal Convection in Circumstellar Disks. \apj \textbf{514}, 325

\item
Klahr, H. and Lin, D.N.C. (2001) Dust Distribution in Gas Disks. A Model for
the Ring Around HR 4796A. \apj \textbf{554}, 1095

\item
Klahr, H.~H. \& Bodenheimer, P. (2003)  Turbulence in accretion disks:
vorticity generation and angular momentum transport via the global
baroclinic instability  \apj \textbf{582}, 869

\item
Klahr, H. and Kley, W. (2006) 3D-Radiation Hydro Calculations of Disk-Planet
Interaction. \aap \textbf{445}, 747

\item
Klahr H. and Lin, D.~N.~C. (2005) Dust Distribution in Gas Disks II: Self 
Induced Ring Formation through a Clumping Instability. \apj \textbf{632},
1113

\item Klahr, H. \& Bodenheimer, P.\ (2006)\ 
Formation of Giant Planets by Concurrent Accretion of Solids and Gas inside
an Anticyclonic Vortex.\ \apj \textbf{639}, 432-440

\item 
Klein, R., Apai, D., Pascucci, I., Henning, Th., Waters,
L.B.F.M. (2003) First Detection of Millimeter Dust Emission from Brown
Dwarf Disks, Astrophys. J. \textbf{593}, L57-L60.

%\item
%Kleine, T., Mezger, K., Palme, H., Scherer, E. und M\"unker, C. (2005) The W
%isotope composition of eucrite metals: constraints on the timing and cause
%of the thermal metamorphism of basaltic eucrites. \epsl \textbf{231}, 41

\item 
Kleine, T., Mezger, K., Palme, H., Scherer, E. and M\"unker, C.
(2005).  Early core formation in asteroids and late accretion of
chondrite parent bodies: Evidence from 182Hf-182W in CAIs, metal-rich
chondrites and iron meteorites.  \textit{Geochim. Cosmochim. Acta},
\textbf{69}, 5805


\item
Kley, W. \& Lin, D.N.C. (1992)  Two-dimensional viscous accretion disk models:
 I. On meridional circulations in radiative regions  \aap \textbf{397}, 600

\item
Kley, W., D'Angelo, G., Henning, Th. (2001)  Three-dimensional simulations
 of a planet embedded in a protoplanetary disk  \apj \textbf{547}, 457

\item
Kornet, K., Stepinski, T.~F. and R\'o\.zyczka, M. (2001) Diversity of
planetary systems from evolution of solids in protoplanetary disks. \aap
\textbf{378}, 180

\item
Kouchi, A., Kudo, T., Nakano, H., Arakawa, M., Watanabe, et al.
(2002) Rapid growth of asteroids owing to very sticky interstellar 
organic gases \apj \textbf{566}, L121

\item
Krause, M. and Blum J., (2004) Growth and Form of Planetary Seedlings: Results
from a Sounding Rocket Microgravity Aggregation Experiment.
\prl \textbf{93}, 021103

\item 
Krot A., Scott E. and Reipurth B. (2005)
         \textit{Proceedings of Chondrites and the Protoplanetary Disk}
         (ASP Conference Series 341, Astronomical Society of the 
         Pacific, San Francisco).

\item
Kunihiro T., Rubin A.~E., McKeegan K.~D. and Wasson J.~T. (2004)
Initial $^{26}$Al/\,$^{27}$Al in carbonaceous-chondrite chondrules: Too little
$^{26}$Al to melt asteroids. \gca \textbf{68}, 2947

\item
Lauretta D., Leshin L.A. and McSween H. Y. Jr. (2006)
         \textit{Meteorites and the Early Solar System II}
          (University of Arizona Press, Tucson)

\item
Leinert, Ch., van Boekel, R., Waters, L.~B.~F.~M., Chesneau, O., Malbet, F.,
K\"ohler, R., Jaffe, W., Ratzka, Th., Dutrey, A., Preibisch, Th. and 28
coauthors (2004) Mid-infrared sizes of circumstellar disks around Herbig
Ae/Be stars measured with MIDI on the VLTI. \aap \textbf{423}, 537

\item
Lodders, K. (2003) Solar System Abundances and Condensation Temperatures of
the Elements. \apj \textbf{591}, 1220

\item
Lugmair, G.~W. and Shukolyukov, A. (1998) Early solar system timescales
according to $^{53}$Mn-$^{53}$Cr systematics. \gca \textbf{62}, 2863

\item
Lynden-Bell, D. and Pringle, J.~E. (1974) The evolution of viscous discs and
the origin of the nebular variables. \mn \textbf{168}, 603

\item
Markiewicz, W.~J., Mizuno, H. and Voelk, H.~J.\ (1991) Turbulence induced
relative velocity between two grains \aap \textbf{242}, 286

\item
Markwick, A.~J., Ilgner, M., Millar, T.~J. and Henning, Th. (2002) Molecular
distributions in the inner regions of protostellar disks. \aap \textbf{385},
632

\item
Meakin, P., Donn, B., Aerodynamic properties of fractal grains:
implications for the primordial solar nebula \apj \textbf{329}, L39

\item
Meeus, G., Waters, L.~B.~F.~M., Bouwman, J., van den Ancker, M.~E.,
Waelkens, C.  and Malfait, K. (2001) ISO spectroscopy of circumstellar dust
in 14 Herbig Ae/Be systems: Towards an understanding of dust
processing. \aap \textbf{365}, 476

\item
Miura, H. and Nakamoto, T. (2005) A shock-wave heating model for chondrule
formation II. Minimum size of chondrule precursors  \ica \textbf{175}, 289

\item
Mizuno, H., Markiewicz, W.~J. and Voelk, H.~J. (1988) Grain growth in
turbulent protoplanetary accretion disks. \aap \textbf{195}, 183

\item
Miyake, K., Nakagawa, Y. (1993) Effects of particle size distribution on
opacity curves of protoplanetary disks around T Tauri stars. \ica
\textbf{106}, 20

\item Mizuno, H., Markiewicz, W.J. \& V\"olk, H.J. (1988) 
  Grain growth in turbulent protoplanetary accretion disks,
  \aap \textbf{195}, 183

\item
Mizuno, H. (1989) Grain growth in the turbulent accretion disk solar
nebula \ica \textbf{80}, 189

\item
Monnier, J.~D., Millan-Gabet, R., Billmeier, R. et al. (2005)
The Near-Infrared Size-Luminosity Relations for Herbig Ae/Be Disks
\apj \textbf{624}, 832

\item
Morfill G.~E., Tscharnuter. W.~M. and V\"olk H.-J. (1985) Dynamical and
chemical evolution of the protoplanetary nebula. In: \textit{Protostars \&
Planets II} (Eds. D.~C. Black, M.~S. Matthews). University of Arizona Press,
Tucson. p. 493

\item
Morfill, G.~E. (1988) Protoplanetary accretion disks with coagulation and
evaporation \ica \textbf{75}, 371

\item
Mostefaoui, S. Kita N.~T. et al.  (2002) The relative formation ages of
ferromagnesian chondrules inferred from their initial aluminum-26/aluminum-27
ratios. \mps \textbf{37}, 421  

\item
Mysen, B.~O. and Kushiro, I. (1988) Condensation, evaporation melting and
crystallization in the primitive solar nebula: Experimental data in the system
MgO-SiO$_2$-H$_2$ to $1.0\times10^{-9}$ bar and 1870$^\circ$C with variable
oxygen fugacity. \textit{American Mineralogist} \textbf{73}, 1

\item
Mysen, B.~O. and Kushiro, I. (1989) Oxygen fugacity and evaporation phase
relations in the solar nebula. \textit{Annual Report Geophysical Laboratory,
Carnegie Institution} 1988-1989, 33                                  

\item
Nagahara, H., Kushiro, I. and Mysen, B.~O. (1994) Evaporation of olivine -- Low
pressure phase relations of the olivine system and its implication for the
origin of chondritic components in the solar nebula. \gca,  \textbf{58}, 1951 

\item
Nakagawa, Y., Nakazawa, K. and Hayashi, C. (1981) Growth and sedimentation of
dust grains in the primordial solar nebula. \ica \textbf{45}, 517

\item
Nakagawa, Y., Sekiya, M. and Hayashi, C. (1986) Settling and growth of dust
particles in a laminar phase of a low-mass solar nebula \ica \textbf{67}, 375

\item
Nakamoto, T. and Nakagawa, Y. (1994) Formation, early evolution, and
gravitational stability of protoplanetary disks. \apj \textbf{421}, 640

\item
Nakamoto, T. and Nakagawa, Y. (1995) Growth of protoplanetary disks around
young stellar objects. \apj \textbf{445}, 330

\item
Nuth, J.A., Faris, J., Wasilewski, P. and Berg, O. (1994)  Magnetically
enhanced coagulation of very small iron grains \ica \textbf{107}, 155

\item 
Nuth, J.A. and Johnson, N.M. (2006) Crystalline silicates in comets:
how did they form? \ica \textbf{180}, 243

\item
Ossenkopf, V. (1993)  Dus coagulation in dense molecular clouds: the
formation of fluffy aggregates  \aap \textbf{280}, 617

\item
Palme, H. (2001) Chemical and isotopic heterogeneity in protosolar 
matter. \textit{Phil. Trans. R. Soc. Lond.} \textbf{A 359}, 2061--2075

\item
Pascucci, I., Apai, D., Henning, Th. and Dullemond, C.P. (2003) 
The first detailed look at a Brown Dwarf disk \apj \textbf{590}, L111

\item
Pascucci, I., Wolf, S., Steinacker, J., Dullemond, C. P., Henning, Th.,
Niccolini, G., Woitke, P. Lopez, B. (2004) The 2-D continuum radiative
transfer problem. Benchmark results for disk configurations \aap
\textbf{417}, 793

\item
Poppe, T. (2003) Sintering of highly porous silica-particle samples:
analogues of early Solar-System aggregates. \ica \textbf{164}, 139

\item
Poppe, T., Blum, J. and Henning, Th. (1997) Generating a jet of
deagglomerated small particles in vacuum. \textit{Review of Scientific
Instruments\/} \textbf{68}, 2529

\item
Poppe, T., Blum, J. and Henning, Th. (2000a) Analogous Experiments on the
Stickiness of Micron-Sized Preplanetary Dust. \apj \textbf{533}, 454-471

\item
Poppe, T., Blum, J. and Henning, Th. (2000b) Experiments on Collisional Grain
Charging of Micron-sized Preplanetary Dust. \apj \textbf{533}, 472-480

\item
Poppe, T. and Schr\"apler, R. (2005) Further Experiments on Collisional
Tribocharging of Cosmic Grains. \aap \textbf{438}, 1

\item
Praburam, G. and Goree, J. (1995) Cosmic dust synthesis by accretion and
coagulation  \apj \textbf{441}, 830

\item
Przygodda, F., van Boekel, R., \`Abrah\`am, P., Melnikov, S.~Y.; Waters,
L.~B.~F.~M., Leinert, Ch. (2003) Evidence for grain growth in T Tauri disks.
\aap \textbf{412}, L43

\item
Rodmann, J., Henning, Th., Chandler, C.~J., Mundy, L.~G.,
Wilner, D.~J. (2006) Large dust particles in disks around T Tauri stars 
\aap, \textbf{446}, 211 

\item
Ruden, S.~P. and Lin, D.~N.~C. (1986) The global evolution of the primordial
solar nebula. \apj \textbf{308}, 883

\item
Russell, S.~S., Srinivasan, G., Huss, G.~R., Wasserburg, G.~J. and MacPherson,
G.~J. (1996) Evidence for widespread $^{26}$Al in the solar nebula and
constraints for nebula time scales. \sci \textbf{273}, 757

\item
Sano, T., Miyama, S.~M.; Umebayashi, T. and Nakano, T. (2000)
Magnetorotational Instability in Protoplanetary Disks. II. Ionization State
and Unstable Regions.
\apj \textbf{543}, 486

\item
Sch\"afer, C., Speith, R., Hipp, M. and Kley, W. (2004) Simulations of 
planet-disk interactions using Smoothed Particle Hydrodynamics 
\aap \textbf{418}, 325

\item
Sch\"afer, Chr. (2005) Application of Smooth Particle Hydrodynamics to
selected Aspects of Planet Formation. PhD thesis, Universit\"at T\"ubingen

\item
Schmitt, W., Henning, Th. und Mucha, R. (1997) Dust evolution in
protoplanetary accretion disks. \aap \textbf{325}, 569

\item
Schr\"apler, R. and Henning, Th. (2004) Dust Diffusion, Sedimentation, and
Gravitational Instabilities in Protoplanetary Disks. \apj \textbf{614}, 960

\item
Sears, D. (2004) The origin of chondrules and chondrites. Cambridge:
Cambridge University Press

\item
Sekiya, M. and Takeda, H. (2003) Were planetesimals formed by dust accretion
in the solar nebula?  \textit{Earth, Planets and Space\/} \textbf{55}, 263

\item
Semenov, D., Henning, Th., Helling, Ch., Ilgner, M. and Sedlmayr, E. (2003)
Rosseland and Planck mean opacities for protoplanetary discs. \aap
\textbf{410}, 611

\item
Semenov, D., Wiebe, D. and Henning, Th. (2004) Reduction of chemical
networks.  II. Analysis of the fractional ionisation in protoplanetary
discs. \aap \textbf{417}, 93

\item
Semenov, D., Pavlyuchenkov, Ya, Schreyer, K., Henning, Th., Dullemond,
C. P., Bacmann, A. (2005) Millimeter observations and modeling of the AB
Aurigae system \apj \textbf{621}, 853

\item
Silverstone, M. D., Meyer, M. R., et al.~(2006) Primordial warm dust
evolution from 3-30 Myr around sun-like stars \apj \textbf{639}, 1138

\item
Sirono, S.-I. (2004) Conditions for collisional growth of a grain aggregate.
\ica \textbf{167}, 431

\item
Stephan T., Arndt P., Jessberger E. K., Kl\"ock W., Nakamura K., Maetz M.,
Rost D., Thomas-Keprta K. L., Warren J. L., Weber I., and Wies C. (2001)
Comprehensive consortium study of interplanetary dust particles from
collector U2071. Lunar Planet. Sci. \textbf{32}, \#1267 (abstr.).

\item 
Stephan T., Jessberger E. K., Kl\"ock W., Rulle H., and Zehnpfenning
J. (1994a) TOF-SIMS analysis of in-terplanetary dust. Earth
Planet. Sci. Lett. \textbf{128}, 453--467.

\item 
Stephan T., Weber I., and Hoppe P. (2005) TOF-SIMS, NanoSIMS, and TEM
analysis of interplanetary dust particle sections. Lunar
Planet. Sci. \textbf{36}, \#1645 (abstr.).

\item
Supulver, K., Bridges, F., Tiscareno, S., Lievore, J., 
Lin, D. N. C. (1997) The sticking properties of water frost produced under
various ambient conditions \ica \textbf{129}, 539

\item
Suttner, G. and Yorke, H.~Y. (2001) Early dust evolution in protostellar
accretion disks \apj \textbf{551}, 461

\item
Tachibana, S., Tsuchiyama, A. and Nagahara, H. (2002) Experimental study of
incongruent evaporation kinetics of enstatite in vacuum and in hydrogen gas. 
\gca \textbf{66}, 713

\item
Tanaka, H., Himeno, Y. and Ida, S. (2005) Dust Growth and Settling in
Protoplanetary Disks and Disk Spectral Energy Distributions. I. Laminar Disks.
\apj \textbf{625}, 414

\item
Testi, L., Natta, A., Shepherd, D. S. and Wilner, D. J. (2003)
Large grains in the disk of CQ Tau. \aap \textbf{403}, 323

\item
Thompson, S.~P., Fonti, S., Verrienti, C., Blanco, A., Orofino, V. and Tang,
C.~C. (2002) Laboratory study of annealed amorphous MgSiO$_3$ silicate using
IR spectroscopy and synchrotron X-ray diffraction. \aap \textbf{395}, 705

\item
Trieloff, M., Kunz, J., Clague, D.~A., Harrison, D. and All\'egre, C.~J. (2000)
The nature of pristine noble gases in mantle plumes. \sci \textbf{288}, 1036

\item
Trieloff, M., Kunz, J. and All\'egre, C.~J. (2002) Noble gas systematics of
the R\'eunion mantle plume source and the origin of primordial noble gases
in Earth´s mantle. \epsl \textbf{200}, 297

\item
Trieloff, M., Jessberger, E.~K., Herrwerth, I., Hopp, J., Fi\'eni, C.,
Gh\'elis, M., Bourot-Denise, M. and Pellas, P. (2003b) $^{244}$Pu and
$^{40}$Ar-$^{39}$Ar thermochronometries reveal structure and thermal history
of the H-chondrite parent asteroid. \nat \textbf{422}, 502

\item
Trieloff, M. and Palme, H. (2006) The origin of solids in the early solar
system. In: Planet Formation: Theory, Observations, and Experiments
(Eds. H.  Klahr \& W. Brandner), Cambridge University Press (in press)

\item
Tscharnuter, W. (1987) A collapse model of the turbulent presolar nebula
\aap \textbf{188}, 55

\item
Tsuchiyama, A., Tachibana, S. and Takahashi, T. (1999) Evaporation of
forsterite in the primordial solar nebula; rates and accompanied isotopic
fractionation.
\gca \textbf{63}, 2451

\item
van Boekel, R., Min, M., Leinert, Ch., Waters, L.~B.~F.~M., Richichi, A.,
Chesneau, O., Dominik, C., Jaffe, W., Dutrey, A., Graser, U. and 13
coauthors (2004) The building blocks of planets within the `terrestrial'
region of protoplanetary disks. \nat \textbf{432}, 479

\item
van Boekel, R., Min, M., Waters, L.~B.~F.~M., de Koter, A., Dominik, C., van
den Ancker, M.~E. and Bouwman, J. (2005) A 10 $\mu$m spectroscopic survey of
Herbig Ae star disks: Grain growth and crystallization. \aap \textbf{437},
189

\item
V\"olk, H. J., Morfill, G. E., Roeser, S., Jones, F. C.
(1980) Collisions between grains in a turbulent gas \aap \textbf{85}, 316

\item
Wang, J., Davis, A.~M., Clayton, R.~N. and Hashimoto, A., (1999),
Evaporation of single crystal forsterite: evaporation kinetics, magnesium
isotope fractionation, and implications of mass-dependent isotopic
fractionation of a diffusion-controlled reservoir. \gca \textbf{63}, 953

\item
Wilner, D.~J., D'Alessio, P., Calvet, N., Claussen, M.~J. and Hartmann, L.
(2005) Toward Planetesimals in the Disk around TW Hydrae: 3.5 Centimeter Dust
Emission. \apj \textbf{626}, L109

\item
Wehrstedt, M. and Gail, H.-P. (2002) Radial mixing in protoplanetary accretion
disks. II. Time dependent disk models with annealing and carbon combustion.
\aap \textbf{385}, 181-204 

\item
Wehrstedt, M. and Gail, H.-P. (2003) Radial mixing in protoplanetary accretion
disks. V. Models with different element mixtures. \aap \textbf{410}, 917

\item
Weidenschilling, S.~J. (1980) Dust to planetesimals - Settling and coagulation
in the solar nebula. \ica  \textbf{44}, 172

\item
Weidenschilling, S.~J. (1984) Evolution of grains in a turbulent solar
nebula \ica \textbf{60},553

\item
Weidenschilling, S.~J. (1990) Early stages of accumulation in the solar
nebula
\adsr \textbf{10}, 101

\item
Weidenschilling, S.~J. (1997) The Origin of Comets in the Solar Nebula: A
Unified Model \ica \textbf{127}, 290

\item
Weidenschilling, S.~J. and Cuzzi, J.~N. (1993) Formation of Planetesimals in
the Solar Nebula. In: \textit{Protostars and Planets III} (Eds. E.~H. Levy,
J.~I.  Lunine) University of Arizona Press, Tucson, 1031

\item
Wolf, S., Henning, Th. and Stecklum, B. (1999) Multidimensional
self-consistent radiative transfer simulations based on the Monte-Carlo
method. \aap \textbf{349}, 839

\item
Wolf, S. (2003) MC3D-3D continuum radiative transfer, Version 2. \cpc
\textbf{150}, 99

\item
Wolf, S., Padgett, D.~L. and Stapelfeldt, K.~R. (2003) The Circumstellar
Disk of the Butterfly Star in Taurus. \apj \textbf{588}, 373

\item
Wooden, D.~H., Harker, D.~E., Woodward, C.~E., Butner, H.~M., Koike, C.,
Witteborn, F.~C. and McMurtry, C.W. (1999) Silicate Mineralogy of the Dust in
the Inner Coma of Comet C/1995 01 (Hale-Bopp) Pre- and Postperihelion. 
\apj \textbf{517}, 1034

\item
Wooden, D.~H., Butner, H.~M., Harker, D.~E., Woodward, C.~E. (2000) Mg-Rich
Silicate Crystals in Comet Hale-Bopp: ISM Relics or Solar Nebula Condensates?
\ica \textbf{43}, 126

\item
Wooden, D.~H., Woodward, C.~E. and Harker, D.~E. (2004) Discovery of
Crystalline Silicates in Comet C/2001 Q4 (NEAT). \apj \textbf{612}, L77

\item
W\"unsch, R., Klahr, H.~H., and R\'o\.zyczka, M. (2005) 2-D models of layered
protoplanetary disks: I. The ring instability. \mn, (in press)

\item
Wurm, G. and Blum, J. (1998) Experiments on Preplanetary Dust Aggregation.
\ica \textbf{132}, 125

\item
Wurm, G., Blum, J. and Colwell, J.~E. (2001) Aerodynamical sticking of dust
aggregates. \phre \textbf{64}, 046301

\item
Wurm, G. and Blum, J. (2000) An Experimental Study on the Structure of
Cosmic Dust Aggregates and Their Alignment by Motion Relative to Gas. \apj
\textbf{529}, L57

\item
Wurm, G., Paraskov, G. and Krauss, O. (2004) On the Importance of Gas Flow
through Porous Bodies for the Formation of Planetesimals. \apj \textbf{606},
983

\item
Wurm, G., Paraskov, G. and Krauss, O. (2005a) Ejection of dust by elastic
waves in collisions between millimeter- and centimeter-sized dust aggregates
at 16.5 to 37.5 m/s impact velocities. \phre \textbf{71}, 21304

\item
Wurm, G., Paraskov, G. and Krauss, O. (2005b) Growth of Planetesimals by
Impacts at ~25m/s. \ica \textbf{178}, 253

\item Whipple, F.L. (1972) 
  On certain aerodynamic processes for asteroids and comets,
  in ``From Plasma to Planet'', Ed.\ Evlius,
  Wiley Interscience Division, p.\ 211

\item
Yorke, H.~W., Bodenheimer, P. and Laughlin, G. (1993) The formation of
protostellar disks. I - 1 M(solar). \apj \textbf{411}, 274

\item
Youdin, A.~N., Shu, F.~H. (2002) Planetesimal Formation by Gravitational
Instability. \apj \textbf{580}, 494

\item
Zinner, E. and G\"opel, Ch. (2002) Aluminum-26 in H4 chondrites:
Implications for its production and its usefulness as a fine-scale
chronometer for early solar system events. \mps \textbf{37}, 1001

\end{literature}
