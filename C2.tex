
%%%%%%%%%%%%%%%%%%%%%%%%%%%%%%%%%%%%%%%%%%%%%%%%%%%%%%%%
% LaTex Template for proposals within the              %
% DFG Research Unit Program                            %         
%Planet Formation Witnesses and Probes: Transition Discs
% August 2016                                            %                           
%                                                      %
%%%%%%%%%%%%%%%%%%%%%%%%%%%%%%%%%%%%%%%%%%%%%%%%%%%%%%%%
%
% 
%
% This template may be used to prepare proposals in latex.
%
%
% The project description, including publication list, should be no more than 20 pages
% in length. It should be self-explanatory and not require reviewers to read the 
% literature that is quoted or enclosed.

\documentclass[10pt,fleqn,twoside]{article}

%%%% USE ARIAL FONT %%%%%%%%%%%%%%%%%%%%%%%%%%%%%%%%%%%%%%%%%%%%%%%%%%%%%%
\usepackage{helvet}
\renewcommand\familydefault{phv}

%%%% INCLUDE NECESSARY PACKAGES %%%%%%%%%%%%%%%%%%%%%%%%%%%%%%%%%%%%%%%%%%
%\usepackage{babel}
\usepackage[UKenglish]{babel}
\usepackage{amsmath}
\usepackage{amssymb}
\usepackage{fancyhdr}
\usepackage{natbib}
\usepackage[usenames,dvipsnames]{xcolor}
\usepackage{ae,aecompl}
\usepackage{graphicx}
\usepackage{palatino}
\usepackage[T1]{fontenc}
\usepackage{rotating}
\usepackage{epsf}
\usepackage{setspace}
\usepackage{sfmath}

%%%% PAGE LAYOUT %%%%%%%%%%%%%%%%%%%%%%%%%%%%%%%%%%%%%%%%%%%%%%%%%%%%%%%%%
\setlength{\textheight}{22cm}
\setlength{\topmargin}{-1.2cm}
\setlength{\textwidth}{15.6cm}
\setlength{\oddsidemargin}{0.0cm}
\setlength{\evensidemargin}{0.0cm}
\setlength{\mathindent}{1.5cm}
\setlength{\parindent}{0.0cm}
\setlength{\parskip}{0.08cm}

%%%% PAGE HEADER %%%%%%%%%%%%%%%%%%%%%%%%%%%%%%%%%%%%%%%%%%%%%%%%%%%%%%%%%%
\pagestyle{fancy}
\fancyhead[RE,RO]{}
\fancyfoot[RO]{ \thepage}
\fancyfoot[LE]{ \thepage}
\fancyfoot[CE,CO]{}

%%% FONTS FOR THE TITLE PAGE %%%%%%%%%%%%%%%%%%%%%%%%%%%%%%%%%%%%%%%%%%%%%%
\newfont{\tpfonta}{cmssbx10 scaled 1600}
\newfont{\tpfontb}{cmssbx10 scaled 3200}

%%%% EURO SIGN %%%%%%%%%%%%%%%%%%%%%%%%%%%%%%%%%%%%%%%%%%%%
\newcommand\euro{{\sffamily C%
\makebox[0pt][l]{\kern-.70em\mbox{--}}%
\makebox[0pt][l]{\kern-.68em\raisebox{.25ex}{--}}}}
\newcommand\keuro{k{\sffamily C%
\makebox[0pt][l]{\kern-.70em\mbox{--}}%
\makebox[0pt][l]{\kern-.68em\raisebox{.25ex}{--}}}}

%%%% COLOR DEFINITIONS %%%%%%%%%%%%%%%%%%%%%%%%%%%%%%%%%%%%
\definecolor{blue} {rgb} {0.25,0.25,0.75}

%%%% ADDITONAL EMPHASIS %%%%%%%%%%%%%%%%%%%%%%%%%%%%%%%%%%%
\newcommand{\cem}{\color{blue}}
\newcommand{\eem}{\sl\color{blue}}

%%%% SET THE COLOR OF THE (SUB-) SECTION TITLES %%%%%%%%%%% 
\newcommand{\Tcol}{\color{blue}}

%%%% SET THE COLOR OF THE TITLE BOX BACKGROUND %%%%%%%%%%%%
\definecolor{Background}{rgb} {0.62,0.75,0.5}


%%%%%%%%%%%%%%%%%%%%%%%%% COMMENT BOXES %%%%%%%%%%%%%%%%%%%%%%%%%%%%%
\usepackage[colorinlistoftodos,textsize=scriptsize,textwidth=2cm]{todonotes}
\newcommand{\til}[1]{\todo[color=LimeGreen,inline]{Til: #1}}
\newcommand{\barbara}[1]{\todo[inline]{Barbara: #1}}

%%%%%%%%%%%%%%%%%%%%%%% REMOVE SPACE IN LISTS %%%%%%%%%%%%%%%%%%%%%%%

\usepackage{enumitem}
\setlist{nosep}

%%%% REFERENCE SECTION NAME %%%%%%%%%%%%%%%%%%%%%%%%%%%%%%%
\renewcommand\refname{\Tcol 9. Bibliography}

%%%% COLOR THE SECTION NUMBERS %%%%%%%%%%%%%%%%%%%%%%%%%%%%%%%
\makeatletter
\renewcommand\@seccntformat[1]{\color{blue} {\csname the#1\endcsname}\hspace{0.5em}}
\makeatother
\renewcommand\thesection{\arabic{section}.}
\renewcommand\thesubsection{\arabic{section}.\arabic{subsection}}

%%%% CHANGE THE APPEARANCE OF THE \PARAGRAPH COMMAND  %%%%%%%%%%%%%%%%%%%%%%%%%%%%%%%
\makeatletter
\renewcommand\paragraph{\@startsection{paragraph}{4}{\z@}%
            {-2.5ex\@plus -1ex \@minus -.25ex}%
            {1.25ex \@plus .25ex}%
            {\normalfont\normalsize\bfseries}}
\makeatother
\setcounter{secnumdepth}{4}     % how many sectioning levels to assign numbers to
\setcounter{tocdepth}{4}        % how many sectioning levels to show in ToC


\fancyhead[LE,LO]{\slshape
%%%%  Please edit
%
Ercolano \&
Birnstiel: RU Transition Discs project C2}
%
%
%%%%%


\begin{document}


\newpage

%%%% PROJECT DESCRIPTION STARTS HERE %%%%%%%%%%%%%%%%%%%%%%%%%%%%%%%%%%%

\setcounter{page}{1}

\centerline{\huge\bf\Tcol
%
%
%
%
%%%%  Please edit
%
 Project C2:}

\centerline{\huge\bf\Tcol Gone with the wind:}
\centerline{\huge\bf\Tcol Dust entrainment in photoevaporative winds}

%
%%%%
%
%
%
%
\vskip1.0cm

%%%%  Please edit

\noindent{\bf Authors:}\\
\begin{tabular}{ll}
{\textsf{PI:}}                   & B.~Ercolano (LMU) \& T.~Birnstiel (LMU)\\
{\textsf{Co-I:}}                & K.~Dullemond (Heidelberg)\\
{\textsf{Collaborations:}}      &  James Owen (Princeton, USA), P.~Caselli (MPE), G. Picogna (LMU)\\

\end{tabular}

%%%%  Please edit

\vspace{1em}
\noindent{\bf Requested positions: 1PhD student} \\

\vspace{1em}
\noindent{\bf Abstract:}\\

The search for the smoking gun of disc dispersal via photoevaporative
winds, which destroy discs via the formation of Type 1 TDs,  has until
now failed to identify suitable tracers. Quantitative spectroscopy of
YSOs to search for blue-shifted emission lines produced in the wind
relies on an accurate characterisation of the thermochemical
properties of the winds. A central ingredients for the chemical
calculations is the dust content of the wind as micron sized grains
provide the dominant opacity channel in the far-ultraviolet,
furthermore small particles are important players in the temperature
balance of the gas via the photoelectric process.  

We will use realistic radiation-hydrodynamic models of
photoevaporative winds coupled to dust evolution models for the
underlying grain distribution in the disc, to calculate the dust
entrainment in winds to feed to chemical models. The observability of
the continuum emission due to the dust grains in winds from edge-on
discs, a potential new diagnostic, will be estimated both for Herbig
Ae stars and for their fainter T-Tauri counterparts.  

\section{\Tcol State of the art and preliminary work}
\renewcommand{\leftmark}{\sc State of the Art and preliminary work}

The dispersal of protoplanetary discs plays a crucial role in the
planet formation process, and leads to the formation of Type
1 TDs. While photoevaporation from the central star has been proposed
as the dominant disc-dispersal mechanism around low-mass stars
(e.g. Clarke et 2001), to date only tentative evidence exists of a
wind detection, via blue-shifted forbidden line emission of mostly
NeII and OI (e.g. Hartigan, Edwards \& Ghandour 1995; Alexander 2008;
Pascucci \& Sterzik 2009; Schisano, Ercolano \& Guedel 2010; Ercolano
\& Owen 2010, 2016). These lines can only probe the wind on very local
scales and they cannot be inverted to obtain mass loss rates, which
are crucial to pin down the dominating mechanism which drives the disc
photoevaporative wind
(i.e. EUV, FUV or X-ray - or a combination). Different driving
mechanism induce more or less vigorous mass loss at different disc
radii, which  can have dramatic effect on planet formation, both at
the times of planetesimal assembly and for the later dynamical
evolution of planet(esimal)s (e.g. Ercolano \& Rosotti 2015).  

Owen, Ercolano \& Clarke (2011b) demonstrated that in the case of
Herbig Ae/Be stars an EUV-driven wind, the wind selectively entrains
grains of different sizes at different radii resulting in a dust
population that varies spatially and increases with height above the
disc at radii larger than about 10~AU. At near infrared wavelengths
this variable grain population produces a 'wingnut' morphology which
may have already been observed in the case of PDS 144N (Perrin et al.
2006). The work of Owen et al. (2011b) could not however reproduce the
colour gradient of the observations, which show redder emission at
larger heights above the disc. Possibly, the problem was due to the
fact that the synthetic images were dominated by emission from the
smallest grains entrained in the flow. Grain growth, neglected in the
Owen et al. (2011b) calculations in the disc is a natural solution to
the colour problem, which needs to be taken into account in future
simulations. \til{There are two ways in which coagulation could affect
this: either by changing the initial size distribution on which the
flow ``feeds'', or by chagning the dust size/dynamics during the
flow. Which one do you mean and should we discuss this here (or
later)?}

While it is currently not clear if the PDS 144N observation can be
explained by dust entrainment in a photoevaporative wind, the work of
Owen, Ercolano \& Clarke (2011b) has clearly demonstrated that a
significant amount of small grains (which dominate the opacity in the
FUV) do populate disc winds, hence playing an important part in the
chemistry there and at the base of the flow. 

\subsection{Dust-entrainment in winds: modelling strategies}

\subsubsection{Analytical approach}

{\color{red}Describe approach used in Owen, Ercolano \& Clarke 2011b

--why can we decouple photoevaporation calculations from dust
entrainment calculation and why is it necessary to do so for our aims

-- discuss shortcomings of current models: The Owen, Ercolano \& Clarke (2011b) calculation are limited to the EUV-case
only and do not include dust-evolution in the underlying disc, for
these reasons their applicability to models aiming at quantitative
spectroscopy of disc winds is rather limited. 
}

\subsubsection{Numerical approach}

{\color{red}Dust could be modeled self-consistently in the
  radiation hydrodynamical simulations - 

-- mention if this could be done in B1 -- Giovanni??? 2 fluid approach
or particle approach? -- should we do that for selected cases? 

-- Recently Hutchison et al. (2016a, b) have employed SPH
.... blahblahblah 

-- describe briefly technique and most important results 

-- describe why we do not wish to use sph for our models 

}

\subsection{Grain sizes and abundances at the base of the wind}

{\color{green}There are basically two approaches - to zeroth order one
  can use 1d dust evolution models to set the abundance and size
  distribution at the base of the flow. For that it is enough to use
  Birnstiel, Klahr \& Ercolano 2012. 

-- describe what this paper does 

A more sofisticated approach is to use 2d models which resolve the
vertical abundance and size distribution of the grains. 

-- describe such a model -- what is the state of the art on this? -- 

-- is the development of a 2d model the  object of the C1 project? 

-- how likely it is that one could also find some kind of
parameterisation that could be easily coupled to the grian entrainment calculations?
}

In this project
we aim to determine the dust content of photoevaporative winds for the
EUV and X-ray case for a range of stellar, disc and wind parameters,
using realistic descriptions for grain growth in the underlying
disc. 


\subsection{\Tcol Project-related publications}

% Please list your own publications related to the proposed project, 
% adhering to the rules of the DFG guidelines 1.91. In brief, please note: 
% - Up to 10 publications
% - The work must be published or accepted.
% - Publications on astro-ph (arXive, SPIRES or articles with a DOI) count as published. 
% - Any work that is only in the status ``accepted'' MUST be attached to the proposal
%    together with the acceptance letter.
% - All publications in this section CAN be attached to the proposal. Please limit these
%    attachments to a minimum and please note that the reviewers may not read the attachments -
%    the proposal has to speak for itself.
% - The number of allowed publications refers to the sum of the publications listed
%    in ``1.1.1 Articles published or officially accepted by publication outlets...'' and 
%    in ``1.1.2 Other publications''. Publications which only exist on public repositories 
%    belong into the category ``Other Publications''.
[Text]

\subsubsection{\Tcol 
Articles published or officially accepted by publication outlets with scientific quality assurance;
book publications}

[Text]

\subsubsection{\Tcol Other publications}

[Text]

\subsubsection{\Tcol Patents}

\paragraph{\Tcol Pending}

[Text]

\paragraph{\Tcol Issued}

[Text]

\section{\Tcol Objectives and work programme}
\renewcommand{\leftmark}{\sc Objectives and work programme}


\subsection{\Tcol Anticipated total duration of the project}

36 months

\subsection{\Tcol Objectives}
In this project
we aim to determine the dust content of photoevaporative winds for the
EUV and X-ray case for a range of stellar, disc and wind parameters,
using realistic descriptions for grain growth in the underlying
disc. 

This will allow us to : 
\begin{enumerate}
\item Build a dust model for photoevaporative winds to be used in
chemical calculations.
\item Estimate the observability and observation characteristics of
the dust phase in photoevaporative winds.
\end{enumerate}

\subsection{\Tcol Work programme including proposed research methods}
% for each applicant

In this project we will use the analytical approach described in
Section 1.1.1 coupled to models for the evolution of dust grains in
the disc in order to efficiently produce dust entrainment models for the
complete set of X-ray photoevaporation wind solutions calculated in
project B1. 

The work will be carried out by a PhD student co-supervised by
Prof. Ercolano (LMU) and Prof. Birnstiel (LMU), who will provide
guidance, respectively, on the dust entrainment calculations and the
coupling with the dust evolution models for the underlying disc. 

\subsection{Research Tools}

For this project we will need the following tools:
\begin{enumerate}
\item Photoevaporative wind solutions for EUV-driven
models for T-Tauri and Herbig stars. These are
available in the literature (e.g. Font et al. ???).
\item Photoevaporative wind solutions for X-ray photoevaporated
models. These will be provided from project B1.
\item Parameterised dust growth models in one dimension. These
are available in the literature (Birnstiel, Klahr \& Ercolano, 2012) 
\item Two-dimensional dust evolution results from project C1.
\item A 3D radiative transfer code to post-postprocess the wind
models with the calculated grain populations. We will make use  of the
RadMC code developed and maintained by Prof. Dullemond (e.g. REF?????).
\end{enumerate}

\subsection{Research Plan} 

The project will proceed in stages of increasing complexity. We will start
by setting up a framework that can be benchmarked against available
calculations and progressively adding new elements, as they become
available from other sub-projects. The plan has been designed to fit a
PhD student, who will have the opportunity to develop a new theoretical
model as well as acquainting her/himself with standard numerical
techniques (e.g. radiation-hydrodynamics, dust evolution models,
radiative transfer). 

The most important science product from this project is the set of
grain models developed for the  X-ray driven wind. These are needed by
project B2 for the chemical calculations, as the dust
grains dominating the opacities in the FUV are not equally distributed
in the wind (see e.g. Owen, Ercolano \& Clarke, 2011b, Hutchison et
al. 2016ab) and they thus affect the chemistry in the wind
differently in different parts. 

\subsubsection{Months 1-12}
The student will start by producing wind solution for the EUV case
from the work of Font et al. (2004), which may be applicable to Herbig
stars. She/he will then proceed to calculate the dust distribution in
the wind, under simplifying assumptions for the underlying dust
distributions as in Owen et al (2011b). In brief, streamlines from the
base of the flow to the edge of the grid will be computed and along
each of them, the force balance between the drag force, gravity and
the centrifugal force will be calculated. A positive net force on a
grain along the streamline will indicate that the grain is
entrained. 

Once dust abundance and size distributions have been obtained for the
wind, radiative transfer calculations will be performed to produce
synthetic continuum observations at several disc inclinations. 
These first models will be benchmarked against the solutions
of Owen et al. (2011b).

After the benchmarking tests, the framework will be applied, for the
first time to the X-ray phtoevaporation case. For testing purposes the student may at first make use of the
existing wind solutions of Owen et al. (2010, 2011, 2012), until new
wind solutions become available form sub-project B1. This step will
probably however be skipped as the new wind models for the X-ray case
are already being calculated by Dr Picogna, who is employed to do the
preparatory work from project B1. It is therefore likely that an
initial set of high resolution new X-ray wind solution will already be
available to the student right from the start of the project.
 

\subsubsection{Months 13-24}
The student will then be in a position to
significantly improve on this work by considering more realistic grain
abundances and size distributions for the underlying disc. She/He will
first couple the grain entrainment calculations to of simple
prescriptions of dust evolution (e.g.  Birnstiel,
Klahr \& Ercolano 2012) obtained from the one-dimensional models of
REF?????. 

{\color{green} At a later stage the models will use the
results from the two-dimensional calculations of dust evolution carried out in project
C1. 

-- will we also get prescriptions from these??? 

-- if not how can these be coupled??? 

}


\subsubsection{Months 25-36}

In the last year the whole machinery will be in place. The student
will now be able apply it to a wide parameter
space of disc winds, performing radiative transfer calculations of the obtained
structures to compare with available observations or to make
observability predictions which may guide future observing
proposals. We will join forces with expert collaborators on scattered light
observations (e.g. Prof. Henning) to plan new proposals, however we note
that failure to obtain new observations does not preclude the main
aims of this projects to be achieved. 

If time allows, the student will collaborate with Dr Picogna (B1) to
produce full hydrodynamical simulations of disc winds, where the dust
component in the disc and wind is treated as particles (e.g. Picogna
\& Kley 2016). These calculations, which are computationally very expensive
will be useful as a comparison to the simpler methods previously
developed by the student in the project.  

\vspace{0.5em}


\subsection{\Tcol Data handling}

The model data-grids will be made available on the Research Unit
dedicated server for use within the team. Furthermore we will provide
a set of diagnostic models to guide observers in the wider community
on the public partition of the server.
\barbara{Til can we also provide perhaps other useful data from the
dust models?}
\til{Not sure what things observers would like to use. I think we
could provide anything from the full size distribution at every 2D
point to mean grain sizes (averaged in different ways) to calculated
images.}

\subsection{\Tcol Other information}
% Please use this section for any additional information you feel is
% relevant which has not been provided elsewhere.

Not Relevant

\subsection{\Tcol Information on scientific and financial involvement of international cooperation partners}

Not Relevant

\section{\Tcol Bibliography}

[Text]

\section{\Tcol Requested modules/funds}
\renewcommand{\leftmark}{\sc  Requested modules/funds}
% Explain each item for each applicant (stating last name, first name).

\subsection{\Tcol Basic Module}

\subsubsection{\Tcol Funding for Staff}
% Please note that funds for your own temporary position (“Eigene Stelleâ€)
% are not to be included here; this belongs to the separate “Module Temporary Positionâ€.

We require funding for one PhD student to be supervised at the LMU
jointly by Prof. Birnstiel and Prof. Ercolano.

\subsubsection{\Tcol Direct Project Costs}

\paragraph{\Tcol Equipment up to EUR 10,000, Software and Consumables}

Will be provided by the host institution. 

\paragraph{\Tcol Travel Expenses}

Total: 9900 Euro Justification : Each year one national trip (meeting of Astronomical Society, national
meetings) and one international trip (conference, visit
collaborators). 
During the course of the PhD 2 one week long visits to our main
international collaborator, Dr J. Owen (currently at Princeton
University, will move to Imperial College London in 2017). 

Cost estimate: 
\begin{itemize}
\item National trip: 5 overnight stays, train/airfare,
conference fee; 1000 Euro (3000 over 3 years).
\item International trip: 6 overnight stays, airfare, conference fee;
  1500 Euro (4500 over 3 years).
\item Visit to/from J. Owen: airfare, 6 overnight stay 1200 Euro (2400
  for 2 visits)
\end{itemize}

\paragraph{\Tcol Visiting Researchers (excluding Mercator Fellows)}

Not Relevant

\paragraph{\Tcol Other Costs}

None

\paragraph{\Tcol Project-related publication expenses}

We request 750 Euro py (total 2250 Euro) for publication expenses.

\subsubsection{\Tcol Instrumentation}

None

\paragraph{\Tcol Equipment exceeding EUR 10,000} 

None

\paragraph{\Tcol Major Instrumentation exceeding EUR 100,000} 

None 

\subsection{\Tcol Module Temporary Position}

Not Relevant 

\subsection{\Tcol Module Replacement Funding}

Not Relevant 

\subsection{\Tcol Module Mercator Fellows}

Not Relevant 

\subsection{\Tcol Module Public Relations Funding}

Not Relevant 

\section{\Tcol Project requirements}
\renewcommand{\leftmark}{\sc Project requirements}

\subsection{\Tcol Employment status information}
% For each applicant, state the last name, first name, and employment
% status (including duration of contract and funding body, if on a
% fixed-term contract).

Barbara Ercolano, Professor at the Ludwig-Maximilians-Universit\"at
M\"unchen  (permanent)

Tilmann Birnstiel, Professor at the Ludwig-Maximilians-Universit\"at M\"unchen starting 02/2017  (tenure-track)

\subsection{\Tcol First-time proposal data}
% Only if applicable: Last name, first name of first-time applicant.

Not Relevant

\subsection{\Tcol Composition of the project group}
% List only those individuals who will work on the project but will not
% be paid out of the project funds. State each person’s name, academic
% title, employment status, and type of funding.

[Text]

\subsection{\Tcol Cooperation with other researchers}

\subsubsection{\Tcol Planned cooperation on this project}

\paragraph{\Tcol Collaborating researchers for this project within the
  Research Unit}
%Each proposal must be accompanied by a description of how the project
%is integral to the Research Unit, %both in terms of subject matter
%and organisation. This includes a description of the cooperation with
%%others participating within the Research Unit. 

{\color{red} Expand below - include the researchers that are linked to
  the various projects}

The project will use the wind models calculated in project B1 and then
feed back the dust model to the same project (B1) and to the reduced
chemical network tests of project B2. Dust evolution calculations from C1 will also be used. Observational constraints will
be obtained in collaboration with experts working on project A1 and
stellar properties to guide the models 
from project A2. 


\paragraph{\Tcol Collaborating researchers for this project outside of
  the Research Unit}

Dr. James Owen, currently at Princeton University, from 2017 at
Imperial College London. 

{\color{red} Expand above - mention James' contributions to the state
  of the art and his likely contribution to the project - mention visits}

\subsubsection{\Tcol Researchers with whom you have collaborated scientifically within the past three years}
% This information is important for DFG to exclude possible conflicts of interest.
% Please mention not only the names of the cooperation partners but also their institution and city.
% Scientists already mentioned in the previous two subsubsections do not have to be mentioned
% again.

F. Niederhofer (STSci, USA); M. Hilker (ESO, Garching); N. Bastian (U. Liverpool,
UK); M. Guarcello (U. Palermo, Italy); M. Tazzari (U. Cambridge, UK);
A. Natta (Florence, Italy); R. Alexander (U. Leicester); D. Hubber
(LMU); J. Dale (U. Hertfordshire, UK); C. Koepferl (LMU); I. Bonnell
(U. St. Andrews, UK); A. McLeod (ESO, Garching); D. Boneberg
(U. Cambridge, UK); R. Parker (U. Liverpool, UK); R. Wesson (UCL,
London, UK); M. Barlow (UCL, London, UK); A. Glassgold (u. Berkeley,
USA); C. Manara (ESA, Noordwjik, Netherlands); A. Danekhar (CfA,
Harward, USA); Q. Parker (Sidney, Australia); S. Casassus
(U. de Chile, Santiago, Chile); I. Pascucci (U. Arizona, USA);
A. Bevan (UCL, London, UK).

\subsection{\Tcol Scientific equipment}
% List larger instruments that will be available to you for the
% project. These may include large computer facilities if computing
% capacity will be needed. 

The group of Prof. Ercolano has two own computer clusters comprising 

\begin{itemize}
\item 2 CPU Intel Xeon X5650 (Westmere, beginning
2010, 2.66 GHz) 6 cores each 12 cores total (24 virtual) 74 GB ram.

\item 4 CPU Intel Xeon E7-4850 (Ivy Bridge, beginning 2014, 2.30 GHz)
12 cores each 48 cores total (96 virtual) 660 GB ram.

\end{itemize}

Further computational power is provided through the C2PAP facility of the Excellence Cluster to which
the group has guaranteed time. This comprises 126 nodes, each node with 2 CPU Intel Xeon E5-2680 (Sandy
Bridge, beginning 2012, 2.7 GHz) 8 cores each 16 cores total (32
virtual) 64 GB ram. Note that while the future of the Excellence
Cluster Universe is uncertain, the C2PAP facilities will be in any
case supported by the LMU. 

The Leibniz Rechnung Zentrum (LRZ) is also available to us, where still
larger facilities are available with somewhat more constrained and longer queues.

\subsection{\Tcol Project-relevant interests in commercial enterprises}
% Information on connections between the project and the production
% branch of the enterprise.

Not Relevant

\subsection{\Tcol Additional information}
% If applicable, please list proposals requesting major
% instrumentation and/or those previously submitted to a third party
% here.

Not Relevant

\end{document}
