%----------------------------------------------------------------------
%                        PROJECT DEFINITION
%----------------------------------------------------------------------
\renewcommand{\projnr}{A2}
\renewcommand{\projtitleshort}{Mineralogical and chemical composition}
\renewcommand{\projauth}{Tscharnuter, Gail}
%
\setcounter{section}{0}
\noindent{\normalfont\sffamily\Large\bfseries Project \projnr: \projtitleshort}
%
\section{Full title:}
\hspace{1\baselineskip}\\
\centerline{\large ``Evolution of the mineralogical and}\\
\centerline{\large chemical composition of pre-planetary disks''}
%
\section{General information}\mbox{}

\subsection{Principle investigators:}
\hspace{-\baselineskip}\\\noindent
%
{\bfseries\itshape Tscharnuter}, Werner~M., Prof.~Dr.\\
C4, tenure\\
03.06.1945, \"osterreichisch\\
DFG Code number of latest application: SFB 439 - 05\\
Zentrum f\"ur Astronomie der Universit\"at Heidelberg\\
Institut f\"ur Theoretische Astrophysik\\
Albert-\"Uberle Str. 2\\
69120 Heidelberg\\
Tel: 06221-544815\\
Fax: 06221-544221\\
Email: wmt@ita.uni-heidelberg.de\\
Private address:\\
Langgewann 38/3\\
69121 Heidelberg\\
Tel.: 06221-475954\\
%
\vspace{1em}\\\noindent
{\bfseries\itshape Gail}, Hans-Peter, Prof.~Dr.\\
Wiss. Ang.\\
23.07.41, deutsch\\
DFG Code number of latest application: SFB 439 - 05\\
Zentrum f\"ur Astronomie der Universit\"at Heidelberg\\
Institut f\"ur Theoretische Astrophysik\\
Albert-\"Uberle Str. 2\\
69120 Heidelberg\\
Tel: 06221-548982\\
Fax: 06221-544221\\
Email: gail@ita.uni-heidelberg.de\\
Private address:\\
Werderstr. 25\\
69120 Heidelberg\\
Tel.: 06221-401936\\
%
\vspace{1em}\\\noindent
{\bfseries\itshape Henning}, Thomas, Prof.~Dr.\\
Direktor MPIA\\
09.04.56, deutsch\\
DFG Code number of latest application: He 1953/21-2\\
Max-Planck-Institut f\"ur Astronomie\\
K\"onigstuhl 17\\
69117 Heidelberg\\
Tel: 06221-528200\\
Fax: 06221-528339\\
Email: henning@mpia.de\\
Private address:\\
Philosophenweg 3\\
69120 Heidelberg\\
Tel.: 06221-589129\\

\subsection{Co-investigators within this Forschergruppe:}
\begin{coilist}
%\item Th.~Henning (MPIA Heidelberg)
\item D.~Lattard (Mineralogisches Institut, Uni Heidelberg)
\item M.~Trieloff (Mineralogisches Institut, Uni Heidelberg)
\end{coilist}

%XXXXXXXXXXXXXXXXXXXXXXXXXXXXXXXXXXXXXXXXXXXXXXXXXXXXXXXXXXXXXXXXXXXXXXXXXXXXXXX
%XXXXXXXXXXXXXXXXXXXXXXXXXXXXXXXXXXXXXXXXXXXXXXXXXXXXXXXXXXXXXXXXXXXXXXXXXXXXXXX

\section{Summary (Zusammenfassung)}
\subsubsection{Summary:}
The project aims at developing comprehensive models based on
numerical simulations for the early mineralogical and chemical
evolution of the material in pre-planetary disks. This includes,
on the one hand, modeling of the disk structure and evolution, and
of processes of matter transport within the disk, and, on the
other hand, the modeling of the chemical and physical alteration
processes to which the pristine interstellar matter is subjected
until its incorporation into large bodies of a planetary system.
The required input data for the mineralogical processes, e.\,g.,
annealing, evaporation, and growth processes, will be retrieved by
mineralogical experiments in \projlattard. The results will allow
to predict the composition of the precursor material from which
planetesimals and later planets are formed and to compare this
with the observational material from bodies in our Solar System,
e.\,g., cometary and meteoritic material, and with observations of
protoplanetary accretion disks.

\subsubsection{Zusammenfassung:}
In dem Projekt sollen theoretische Modelle und numerische Simulationen
f\"ur die fr\"uhe mineralogische und chemische Entwicklung des Materials in den
pr\"a\-planetaren Akkretionsscheiben entwickelt werden. Es sollen sowohl der
Aufbau und die Ent\-wicklung der Akkretionsscheiben einschlie\ss lich des
Stofftransports in der Scheibe, als auch die chemischen und physikalischen
Ver\"anderungen des interstellaren Ausgangsmaterials, z.B. durch Tempern,
Verdampfung und Wachstum, bis zum Einbau des Materials in die K\"orper eines
Plane\-tensystems m\"oglichst vollst\"andig modelliert werden. Die ben\"otigten
Daten f\"ur die Eigenschaften der Minerale sollen durch Laborexperimente im
Projekt \projlattard\ bestimmt werden. Die Resultate werden es erlauben, die
Zusammensetzung des Materials, aus dem Planetesimale und sp\"ater Planeten
gebildet werden, vorherzusagen, und dies mit Beobachtungsdaten f\"ur K\"orper
aus unserem Sonnensystem, z.B.\ Kometen und meteoritisches Material, und mit
Beobachtungsdaten f\"ur protostellare Akkretionsscheiben zu vergleichen.

%XXXXXXXXXXXXXXXXXXXXXXXXXXXXXXXXXXXXXXXXXXXXXXXXXXXXXXXXXXXXXXXXXXXXXXXXXXXXXXX
%XXXXXXXXXXXXXXXXXXXXXXXXXXXXXXXXXXXXXXXXXXXXXXXXXXXXXXXXXXXXXXXXXXXXXXXXXXXXXXX
\section{State of the art (Stand der Forschung)}

Since planetesimals form very rapidly, their composition is a snapshot of
the composition of the disk material at the time and location of their
formation. Hence, variable chemical compositions are expected for small bodies at this stage, and indeed the compositional variety of chondrite parent bodies (Palme 2001) demonstrates the validity of this argument. Afterwards, the material locked in planetesimals is essentially isolated from
further evolution processes in the accretion disk and mixing of material locked
in planetesimals and protoplanets between different zones is of limited
efficiency. It is therefore essential to study both the chemistry of the
gas phase and that of the solid phases during the early disk evolution up to
the instant of planetesimal formation in order to understand the composition
of the present planets and smaller bodies and of composition gradients in
our Solar System and to get an insight into the variety of compositions
which can be expected for extrasolar planetary systems to occur.

\subsubsection{Structure and evolution of protoplanetary disks:} Model
calculations for the structure and evolution of protoplanetary
disks in the one zone and the (1+1)-D approximation have become
standard by now. They are based on the $\alpha$- or
$\beta$-approximation for disk viscosity or some receipt to the
magneto-rotational instability. Dust opacities usually are
calculated by simple analytic approximations or from tabulations.
The application of such models to observations of accretion disks
around young stellar objects have also arrived at a high degree of
sophistication (cf.\ the review of D'Alessio et al.~2005) and
provided a wealth of information on disk properties and their
mineral content.  A number of theoretical studies of chemistry and
mineralogy in protoplanetary disks are based on such models (cf.
the review by Bergin et al.~2005). Hydrodynamic 2-D and 3-D models
of protoplanetary disks which allow to study the importance of
flows and mixing on the chemistry in the disk are just becoming
available (e.\,g., Klahr et al.~1999; Boss 2004; Tscharnuter \&
Gail 2006; Turner et al.~2006). Also sophisticated models based on
MHD simulations start to become available (cf.\ Bouvier et al.
2006).

\subsubsection{Dust opacity:}
Recent millimeter observations of protoplanetary disks around
young stars have revealed that in some of these disks a
substantial fraction of dust grains has grown up to cm-sized
particles within only a few Myr of the evolution (see, e.\,g.,
Dent et al.~2006; Rodmann et al.~2006; Natta et al.~2006). Dust
particles in accretion disks are porous aggregates of a large
number of small ($\leq0.1\mu$m) sub-units with different
compositions. In cold outer regions of the disk, the pores are filled at least
in part with ice. The extinction properties of such composite
particles are difficult to determine and, in the past, opacities
have usually be calculated from Mie theory for single grains.
Recently there has been considerable progress in this field and
receipts for calculating opacities for dust aggregates have been
worked out (e.\,g., Henning \& Stognienko 1996; Voshchinnikov \&
Mathis 1999; Voshchinnikov et al.~2005, 2006; Semenov et al.~2003;
Min et al.~2006; Kimura et al.~2006). It is found that there are
strong differences between opacities of aggregates and of single
particles in certain cases. Since the disk temperature structure
is determined to a large extent by the dust opacity, better
implementations of dust opacity in disk models are required than
those presently used in model calculations.

\subsubsection{Mixing processes:}
Observations of crystalline material in the outer parts of
accretion disks and in comets raised the question of the origin of
this high temperature material in cold disk regions and mixing has
been considered one likely process for this (cf.\ the discussions
in Wooden et al.~2005, 2006; Alexander et al.~2006; Bergin et al.
2006). The importance of mixing processes in accretion disks was
first discussed by Morfill \& V\"olk (1984). Theoretical models for
mixing of dust based on simple disk models were developed by Gail
(2001, 2004) and Bockel{\'e}e-Morvan (2002), which showed that
mixing may account for the observations of crystalline material in
comets. Particle transport during the formation phase of the disk
was discussed by Dullemond et al.~(2006). Mixing processes based
on hydrodynamic disk models have been studied, e.\,g., by Boss
(2004), Tscharnuter \& Gail (2006), Turner et al.~(2006). Mixing
processes of the gas-phase species and their implications were
studied, e.\,g., by Ilgner et al.~(2004), Ilgner \& Nelson (2006),
Ciesla \& Cuzzi (2006), Semenov et al.~(2006).

\subsubsection{Annealing:}
Annealing of amorphous dust is considered as one of the main
sources of crystalline dust seen in cometary nuclei (e.\,g.,
Wooden et al.~2000, 2005, 2006; Gail 2001, 2004) and
interplanetary dust particles (e.\,g., Alexander \& Keller 2006).
This crystalline dust may either be mixed from hot inner disk
regions into the cold comet formation zone (e.\,g., Gail 2001,
2004 Bockel{\'e}e-Morvan 2002) or may be produced in situ by
shocks (e.\,g., Harker \& Desh 2002). The annealing process has
been studied in the laboratory for magnesium rich silicates
(e.\,g., Hallenbeck et al.~1998, Rietmeijer et al.~2002, Fabian et
al.~2000) and the data have been used in model calculations for
dust annealing in protoplanetary disks (e.\,g., Gail 2001, 2004,
Wehrstedt \& Gail 2002, 2003, 2006, Bockel{\'e}e-Morvan et al.
2002). Since the amorphous ISM silicates are partially iron rich,
data for iron bearing silicates are required, but presently not
available (except for the very preliminary results of Brucato et
al.~2002). The same holds for Ca-Al compounds.
%\\[1ex]

The predictions of preliminary models for annealing and mixing
(e.\,g., Gail 2004) seem to be roughly in accord with the observed
distribution of crystalline silicate material in disks around
young stellar objects (e.\,g., van Boekel et al.~2004) or with the
results of the Deep Impact experiment (e.\,g., Harker et al.~2005)
and the Stardust mission%
\footnote{cf.\ \texttt{http://www.nasa.gov/stardust}}.

\subsubsection{Dust evaporation and growth:}
In the hot inner part of accretion disks the mineral dust is
driven by evaporation/condensation into the chemical equilibrium
mineral mixture (Grossman \& Larimer 1974, Petaev \& Wood 1998, 2004). 
These processes are discussed e.\,g., by Gail
(2004) and Wooden et al.~(2005). Laboratory experiments on
evaporation and condensation have been conducted for some
substances of interest during the last decade, e.\,g., for
forsterite (e.\,g., Nagahara \& Ozawa 1996) and pyroxene (e.\,g.,
Tachibana et al.~2002). For many mineral substances, in particular
for the highly refractory Ca-Al-compounds, and Fe-bearing olivine and pyroxene, the data necessary for a
quantitative modeling of the composition of protoplanetary disks
are not yet available.

\subsubsection{Chemistry of gas phase and ices:}
There are two reasons why the chemistry of molecular species in
pre-planetary accretion disks is one of the most important
processes to be studied. On the one hand, observations of
molecular lines from the gas phase belong to the most important
tools for studying the properties of accretion disks (e.\,g.,
Dutrey et al.~2006). On the other hand, molecules and ices form
the volatile component of the disk material. The ices form a major
fraction of the material of the icy bodies in the outer regions of
the accretion disk and a knowledge of the ice composition
therefore is crucial for determining the composition of such
bodies. In big planetesimals and protoplanets subject to
%\marginpar[\hspace*{4em}\tiny
%\parbox{6em}{D. Lattard\\ (03.05.06):\\ Dieser Absatz\\
%ist zu lang\hfill\rule[-4mm]{1mm}{20mm}}]%
%{\hspace*{4em}\tiny\parbox{6em}%
%{D. Lattard\\ (03.05.06):\\ Dieser Absatz\\ ist zu lang}}%
radioactive heating, the initial supply with volatiles is
responsible for the presence of fluid water which triggers a rich
chemistry which in turn is responsible for the existence (or
non-existence) of atmospheres and oceans on the terrestrial
planets in our Solar System and on possible earth like planets in
extrasolar planetary systems. In our Solar System the comets
preserve in their ice component a record of the composition of the
gaseous component of the Solar Nebula at the time and place of
their formation which allows to reconstruct many of the processes
operating in the disk during the early stages of the planetary
formation process. Considerable effort therefore has been gone
already in observations and modeling.%
%\marginpar[\hspace*{4em}\tiny\parbox{6em}{of what?\\comets?}]%
%{\hspace*{4em}\tiny\parbox{6em}{of what?\\comets?}}

The present status of the subject is described, e.\,g., in the
review articles by Langer et al.~(2000) and Bergin et al.~(2006).
Importantly, with Spitzer it now becomes possible to
resolve the inner regions of protoplanetary disks and
study their gas-phase chemistry by observations of
vibration-rotation bands (Lahuis et al.~2006). This is also now
done from the ground (Rettig et al.~2005). Such investigations
will help to constrain future disk models much better than this is
possible from IR dust feature observations alone.
%\\[1ex]

%General discussions on the relevant processes and early attempts to model the
%accretion disk chemistry are described e.\,g., in Prinn \& Fegley (1989), Fegley \%&
%Prinn (1989), Prinn (1993). These considerations are largely based on
%equilibrium chemistry and did not consider the coupling with the disk structure
%and evolution and did not include the typical interstellar ion-molecule
%chemistry operating in the outer parts of the disk. Attempts to model the
%chemistry in pre-planetary disks coupled with models for the disk structure and
%evolution started about 15 years ago. Finocchi et al. (1997a,b) studied the
%chemistry by solving the reaction kinetics in a gas parcel accreting onto the
%star, including mineral dust evaporation and condensation and the chemistry
%of carbon dust combustion.
%\\[1ex]

\subsubsection{Chemical gradients in the Solar System:}
It is observed that volatile rock-forming elements (in particular,
the alkaline elements) are depleted in the inner Solar System.
Potassium, for instance, is definitely depleted on Venus, Earth,
Moon, Mars and many differentiated and also chondritic meteorite parent bodies
(Taylor 1988). Present studies indicate an early loss of volatile
elements in the inner Solar System (Palme 2001; see also Humayun and Clayton 1995).
Furthermore, a systematically increasing depletion of volatile elements
in the sequence of carbonaceous chondrites CI (about solar
abundances, Asplund et al.~2005), CM, CV is well documented. The
terrestrial abundances are very similar to CV chondrites (All\`egre et al.
2001; Palme 2001). On the other hand, in the sequence CI-CM-CV-Earth the
abundance ratio of Al/Si systematically increases. The
origin of the abundance variations of the volatile and the main
rock-forming elements in the inner solar system are presently not well understood.
%XXXXXXXXXXXXXXXXXXXXXXXXXXXXXXXXXXXXXXXXXXXXXXXXXXXXXXXXXXXXXXXXXXXXXXXXXXXXXXX
%XXXXXXXXXXXXXXXXXXXXXXXXXXXXXXXXXXXXXXXXXXXXXXXXXXXXXXXXXXXXXXXXXXXXXXXXXXXXXXX
%\\[1ex]

\section{Preliminary work (Eigene Vorarbeiten)}
Some work has already been done with respect to modeling the growth-
and destruction processes of the dust, modeling of the gas phase
chemistry, the structure and evolution of pre-planetary accretion
disks, and for modeling the radiative transfer in accretion disks.
%\\[1ex]

\subsubsection{Modeling of disks:}
From our previous work there exist a number of computer programs for the
structure and evolution of accretion disks in the one zone approximation, the
(1+1)-D approximation, and a 2-D hydrodynamic code. These codes are already
coupled with detailed routines for computing the chemical composition of the
disk material (gas phase and solids) and for computing mixing processes and
radiative transfer in the disk:
\begin{compactitemize}
\item For stationary disks in one-zone approximation with $\alpha$-viscosity
(Gail 1997, 2001, 2004; Willacy et al.~1998).
\item For the time evolution of one-zone models with $\alpha$- and
$\beta$-viscosity (Wehrstedt \& Gail 2002, 2003).
\item For stationary disks in (1+1)-D approximation with $\alpha$-
and $\beta$-viscosity, coupled with a complete radiative transfer
model analogue to plane parallel stellar atmospheres (Gail 2001;
Bell et al.~1997).
\item For time-dependent evolution of disks in the (1+1)-D
approximation for $\alpha$- and $\beta$-viscosity (Wehrstedt \&
Gail 2006 )
\item For time-dependent 2-D radiative hydrodynamics with
$\beta$-viscosity (Gail \& Tscharnuter 2006, Tscharnuter \& Gail
2006).
\item For time dependent 3-D radiative hydrodynamics with $\alpha$-viscosity
(Klahr et al.~1999)
\end{compactitemize}
\noindent The programs 1,2,3 can be used as tools for testing methods 
and studying simple processes, while 4 to 6 will serve as basis for future
developments.


\subsubsection{Radiative transfer:} For stationary (1+1)-D models a code
package has been developed which solves the radiative transfer
problem self-consistently with the disk structure. Preliminary
results are published in Gail (2001). 2-D time dependent radiative
transfer in the Eddington approximation has been self-consistently
coupled with 2-D hydrodynamics (Tscharnuter \& Gail 2006).
Furthermore, we have a lot of experience with multi-dimensional
radiative transfer codes (e.\,g., Wolf et al.~1999, Manske \&
Henning 1999, Steinacker et al.~2003)
%\\[1ex]

\subsubsection{Growth and destruction of dust:} Extensive experience from
other projects is available for modeling dust-related processes.
Dust vaporization and condensation has been modeled for instance
for stellar winds. Theoretical models have been developed for the
condensation of the silicates olivine
(Mg$_{2x}$Fe$_{2(1-x)}$SiO$_4$), pyroxene (Mg$_x$Fe$_{1-x
}$SiO$_3$), and also quartz (SiO$_2$), which consider variable Mg-
and Fe contents and consider diffusion of the cations Mg$^{2+}$
and Fe$^{2+}$ in the silicates (Gail \& Sedlmayr 1999). They were
extended to other dust species (Ferrarotti \& Gail 2001, 2002,
2003). These models have been applied to carrying out numerical
simulations of dust formation in circumstellar dust shell with
different element mixtures (Ferrarotti \& Gail 2001, 2002, 2003,
2006).
%\\[1ex]

Thermo-chemical data have been collected for numerous compounds and
subroutines have been developed to calculate such heterogeneous
chemical equilibria in solid-gas mixtures. These subroutines were
applied in the previous model computations (Gail 1997, 2001, 2003;
Ferrarotti \& Gail 2001, 2002, 2003; Wehrstedt \& Gail 2002, 2003)
and will also be used in this project.
%\\[1ex]

\subsubsection{Dust processing:} The processing of silicate dust by
annealing in pre-planetary disks was studied theoretically and a
numerical model for calculating the degree of crystallization was
developed for the first time and used in a number of model
calculations (Lenzuni et al.~1995; Duschl et al.~1996, Gail 2001,
2003, 2004; Wehrstedt \& Gail 2002, 2003, 2006). The numerical
experience from these calculations and the codes can also be used
for the present project.
%\\[1ex]

\subsubsection{Mixing processes in accretion disks:} Mixing and transport of
dust and gas phase species between the warm inner and hot outer
parts in accretion disks by turbulent diffusion have been studied
for the first time in Gail (2001, 2002, 2003, 2004), Wehrstedt \&
Gail (2002, 2003, 2006), Ilgner et al.~(2004), Schr\"apler \&
Henning (2004). The necessary codes for treating such processes
simultaneously with model calculations for the disk structure and
evolution have been developed for one-zone models, (1+1)-D models
and for 2-D hydrodynamic models. Numerical experience and codes
from these calculations can be used for the present project.
%\\[1ex]

In a theoretical study the influence of large scale flows in accretion disks
was studied (Keller \& Gail 2005) which showed its importance for mixing
material between widely disparate disk regions. Such flow structures have been
implemented in a (1+1)-D evolution model for accretion disks to study mixing
processes (Wehrstedt \& Gail 2006).
%\\[1ex]

\subsubsection{Hydrodynamics of accretion disks:} A rich experience is
available at the ITA and MPIA for modeling astrophysically
relevant flows, where in particular self-gravity and energy
transport by radiation in moving media are important. As examples,
we may mention collapse flows during star formation and giant
planet formation (Tscharnuter 1987; Wuchterl 1990, 1991a,b;
Wuchterl \& Tscharnuter 2003), flow structures in accretion disks
(Klahr et al.~1999; Klahr \& Kley 2006), or non-linear
hydrodynamic instabilities in cold accretion disks, that probably
explain the small-scale structures in Saturn's B-ring (Schmit and
Tscharnuter 1995, 1999). For these hydrodynamical simulations,
implicit (collapse flows in spherical and axial symmetry) as well
as explicit (cylindrically symmetric instabilities in cold
accretion disk) methods have been used.
%\\[1ex]

For the simulation of axially symmetric pre-planetary disks for which energy
transport by radiation and in particular chemical reactions (including
sublimation and condensation processes) and diffusive matter transport are
of outstanding importance, the (explicit) operator splitting method seems most
appropriate (a description can be found in Norman \& Winkler 1986). The
essential advantage of this method is that source and advection terms in the
hydrodynamic equations can be updated separately within a given time step,
which allows to solve the usually very large system of reaction equations
separately for each cell of the grid. This opens the possibility of massive
parallelization.
%\\[1ex]

The present status of the code development is as follows: An
explicit 2-D hydrodynamics code, coupled with both an implicit
radiative transfer solver based on the diffusion approximation and
a system of advection-diffusion-reaction equations for the
chemistry of the gas phase and dust species, is ready for use.
Viscous processes in the disk are presently implemented in the
approximation of the $\beta$-viscosity (Duschl et al.~2000).
Because of the restrictions by the Courant-condition this code
can, however, only be used for investigation of processes which
operate on shorter timescales than the global disk evolution.
%
%
%\\[1ex]

\subsubsection{Modeling of gas phase chemistry:} We have studied the
chemistry in accretion disks in a number of model calculations
(Finocchi et al.~1997a, 1997b; Markwick et al.~2002; Ilgner et al.
2004; Gail \& Tscharnuter 2006; Tscharnuter \& Gail 2006),
including gas-grain interactions (Willacy et al.~1998) and surface
processes (Semenov et al.~2004), and even turbulent mixing (Ilgner
et al.~2004). A wide range of temperatures, densities, and
high-energy X-ray and UV radiation fluxes encountered in
protoplanetary disks leads to complicated chemistry, with
different chemical processes controlling the evolution of
molecular species in different disk locations. Thus, it is of
particular importance to take all relevant reactions and gas-grain
processes into account in the numerical modeling of the
protoplanetary disk chemical evolution, and, which is even more
crucial, to understand when, where, and why these processes are so
important.

The MPIA group has solid experience in sophisticated modeling of
the outer disk chemistry in a variety of young low-mass objects
surrounding low-mass, Sun-like T~Tau stars as well as more massive
and luminous Herbig Ae and Be stars (e.\,g., Markwick et al.
2002). Semenov et al.~(2004) have studied the evolution of the
fractional ionization in various regions of a T Tau disk subject
to stellar and interstellar UV and X-ray radiation in detail,
using a special reduction technique that allows for any
pre-selected molecule a set of most chemically important reactions
to be isolated (Wiebe et al.~2003). They found that the stellar
X-ray radiation may trigger very peculiar chemistry in the inner
disk zone, leading to the formation of complex carbon and
cyanopolyine chains (e.\,g., HC$_{11}$N). Semenov et al.~(2005)
have consistently modelled for the first time the millimeter
interferometric and single-dish observations of the AB Aur disk
and envelope, using a gas-grain chemical model with surface
chemistry, deuterium fractionation, and self-shielding of CO and
H$_2$.

Presently, our chemical models range from a simple static
gas-grain model with surface chemistry to a complicated
non-stationary model with 2-D turbulent mixing included. All the
models are based on the UMIST\,95 database of reactions
(Millar et al.~1997) and take into account ionization and dissociation
by stellar and interstellar UV and X-ray radiation as well as
cosmic rays and decaying short-lived radionuclides. Gas-grain
interactions include accretion of gas-phase molecules onto and
desorption of frozen species from dust surfaces, and dissociative
recombinations of ions with charged grains. A set of surface
chemical reactions was adopted from Hasegawa et al.~(1992) and
Hasegawa \& Herbst (1993). A unique numerical integration scheme is
used to solve the arising sets of non-linear stiff differential
equations (either with only chemical terms or also with turbulent
transport), which is based on the Backward Differentiation Formula
variable-order solver DVODPK with iterative preconditioned Krylov
methods and sparse Jacobian representation (see, e.\,g., Nejad 2005).
%\\[1ex]

\subsubsection{Opacity of dust:}
%
The dust opacity modeling constitutes a problem of its own for
everybody who aims at studying the disk physics and evolution in
detail. The major problem here is the lack of detailed knowledge
on the composition, mineralogy, size distribution, topology,
degree of porosity, etc. of dust grains in distinct disk regions.
While the numerical recipes to model the optical properties of
porous inhomogeneous aggregate, composite, and layered particles
already are available to us (Semenov et al.~2003; Voshchinnikov et
al.~2006), the dust compositional model of Pollack et al.~(1994)
is outdated and has to be revised in order to achieve further
progress in this field. In Henning \& Stognienko (1996) and Semenov
et al.~(2003), we have compiled opacity tables designed for
hydrodynamical simulations of protoplanetary accretion disks by
unifying the dust-dominated opacity model for cold and warm disk
regions with gas-dominated opacities of Helling et al.~(2000) for
the hot disk interior. A similar strategy for the opacity modeling
and the numerical codes can also be utilized in the proposed
project.
%\\[1ex]

%
\subsubsection{Elemental compositions of planetary systems:} Will be
determined in cooperation with projects A8, B11 of SFB\,439. From
the model calculations of the chemo-dynamical evolution of the
Milky Way the required data on element abundances and dust input
from the ISM into protostellar accretion disks will be taken.


%-------------------------------------------------------------------------------
%
% Here follows the own refereed publications by the PIs in relation to
% the project proposed here.
%
\ownpubltitle{Own publications related to the Forschergruppe:}
%
% BELOW IS ONLY AN EXAMPLE OF TWO ENTRIES. SEE THE ADDITIONAL FILES
% SENT TO YOU WITH ALL THE REFERENCES FROM THE VORANTRAG
%
\begin{ownpubl}

\item
Bell, K.~R., Cassen, P.~M., Klahr, H.~H. and Henning, Th. (1997)
The Structure and Appearance of Protostellar Accretion Disks: Limits on Disk
Flaring. \apj, \textbf{486}, 372--387

\item
Duschl, W.~J., Gail, H.-P. and Tscharnuter, W.~M. (1996) Destruction
processes for dust in protoplanetary accretion disks, \aap, \textbf{312},
624--642

\item
Dutrey, A., Henning, Th., Guilloteau et al.~(2006) CID-chemistry in disks. I. Deep search for N$_2$H$^+$ in protoplanetary disks around LMCa 15, MWC 480, and DM
Tau. \aap, (submitted)

\item
Ferrarotti, A.~S. and Gail, H.-P. (2000) Mineral formation in stellar winds. II.
Effects of Mg/Si abundance variations on dust composition in AGB stars. \aap,
\textbf{371}, 133--151

\item
Ferrarotti, A.~S. and Gail, H.-P. (2002) Mineral formation in stellar winds.
III. Dust formation in S stars. \aap, \textbf{382}, 256--281


\item
Ferrarotti, A.~S. and Gail, H.-P. (2003) Mineral formation in stellar winds. IV.
Formation of magnesiow{\"u}stite. \aap, \textbf{398}, 1029--1039

\item
Ferrarotti, A.~S. and Gail, H.-P. (2006) Composition and quantities of dust
produced by AGB-stars and returned to the interstellar medium. \aap,
\textbf{447}, 553--576

\item
Finocchi, F., Gail, H.-P. and Duschl, W.~J. (1997) Chemical reactions in
protoplanetary disks II. Carbon dust oxidation. \aap, \textbf{325}, 1264--1279

\item
Finocchi, F. and Gail, H.-P. (1997) Chemical reactions in protoplanetary
accretion disks III. The role of ionisation processes. \aap, \textbf{327}, 825--844

\item
Gail, H.-P. (1997) Chemical reactions in protoplanetary disks IV. Multi
component dust mixture. \aap,\textbf{332}, 1099--1122

\item
Gail, H.-P. (2001) Radial mixing in protoplanetary accretion disks. I.
Stationary disk models with annealing and carbon combustion. \aap, \textbf{378},
192--213

\item
Gail, H.-P. (2002)
Radial mixing in protoplanetary accretion disks. III. Carbon dust oxidation and
abundance of hydrocarbons in comets. \aap, \textbf{390}, 253--265

\item
Gail, H.-P. (2003) Formation and Evolution of Minerals in Accretion Disks and
Stellar Outflows. In: \textit{Astromineralogy}, ed. Th. Henning. \textit{Lecture
Notes in Physics}, \textbf{609}. Springer, Heidelberg. p. 55--120

\item
Gail, H.-P. (2004) Radial mixing in protoplanetary accretion disks. IV.
Metamorphosis of the silicate dust complex. \aap, \textbf{413}, 571--591

\item
Gail, H.~P. and Sedlmayr, E. (1999) Mineral formation in stellar winds I.
Condensation sequence of silicate and iron grains in oxygen rich outflows.
\aap, \textbf{347}, 594--616

\item
Gail, H.-P. and Tscharnuter, W.~M. (2006) Evolution of protoplanetary disks
including detailed chemistry and mineralogy. In: \textit{Reactive Flow,
Diffusion and Transport}, ed. R. Rannacher et al.~(Springer, Berlin-Heidelberg)
(in press)

\item Grossman, L. \& Larimer, J.~W. 1974 Early chemical history of the
Solar System. Rev. Geophys Space Phys. \textbf{12}, 71-101


\item
Henning, T. and Stognienko, R. (1996) Dust opacities for protoplanetary accretion
disks: influence of dust aggregates. \aap, \textbf{311}, 291--303

\item
Henning, Th., Dullemond, C.P., Wolf, S. and Dominik, C. (2006) Dust Coagulation
in Protoplanetary Disks. In:  Planet Formation. Theory, Observation and
Experiments, eds. H. Klahr and W. Brandner (Cambridge: Cambridge University
press) (in press)

\item
Ilgner, M., Henning, Th., Markwick, A.~J. and Millar, T.~J. (2004) Transport
processes and chemical evolution in steady accretion disk flows. \aap,
\textbf{415}, 643--659

\item
Keller, Ch. and Gail, H.-P. (2004) Radial mixing in protoplanetary accretion
disks. VI. Mixing by large-scale radial flows. \aap, \textbf{415}, 1177--1185

\item
Klahr, H.~H., Henning, Th. and Kley, W. (1999) On the Azimuthal Structure of
Thermal Convection in Circumstellar Disks. \apj, \textbf{514}, 325--343

\item
Klahr, H. and Kley, W. (2006) 3D-radiation hydro simulations of disk-planet
interactions. I. Numerical algorithm and test cases. \aap, \textbf{445},
747--758

\item
Lenzuni, P., Gail, H.-P. and Henning, Th. (1995) Dust evaporation in
protostellar cores, \apj, \textbf{447}, 848--862

\item
Manske, V. and Henning, Th. (1999) 2D radiative transfer with transiently heated
particles for the circumstellar environment of Herbig Ae/Be stars. \aap,
\textbf{349}, 907--911

\item
Markwick, A.~J., Ilgner, M., Millar, T.~J. and Henning, Th. (2002) Molecular
distributions in the inner regions of protostellar disks. \aap, \textbf{385},
632--646

\item Palme, H. (2001).  Chemical and isotopic heterogeneity in protosolar matter.  \textit{Phil. Trans. R. Soc. Lond.} \textbf{A 359}, 2061.   

\item Petaev, M. I. and J. A. Wood (1998) The Condensation with Partial
Isolation (CWPI) model of nebular condensation in the Solar Nebula.
\textit{Meteorit. Planet. Sci.}
\textbf{33}, 1123-1137.

\item Petaev, M. I. and  Wood, J. A. (2004) Meteoritic constraints on
temperatures, pressures, cooling rates, chemical compositions, and modes of
condensation in the solar nebula. In Chondrites and the Protoplanetary Disk
(Eds. A.~N.  Krot, E.~R.~D. Scott, and B. Reipurth), Vol. 341.


\item
Rodmann, J., Henning, T., Chandler, C.~J., Mundy, L.~G.
and Wilner, D.~J.\ (2006) Large dust particles in disks around T Tauri stars.
\aap, \textbf{446}, 211--221

\item
Schr\"apler, R., Henning, Th. (2004) Dust Diffusion, Sedimentation, and
Gravitational Instabilities in Protoplanetary Disks. \apj, \textbf{614}, 960--978

\item
Schmit, U. and Tscharnuter, W.~M. (1995) A fluid dynamical treatment of the
common action of self-gravitation, collisions, and rotation in Saturn's B Ring.
\ica, \textbf{115}, 304--319

\item
Schmit, U. and Tscharnuter, W.~M. (1999) On the Formation of the Fine-Scale
Structure in Saturn's B Ring. \ica, \textbf{138}, 173--187

\item
Semenov, D., Henning, Th., Helling, C., Ilgner, M. and Sedlmayr, E.
(2003) Rosseland and Planck mean opacities for protoplanetary discs. \aap,
\textbf{410}, 611--621

\item
Semenov, D., Wiebe, D. and Henning, T. (2004) Reduction of chemical networks.
II. Analysis of the fractional ionisation in protoplanetary discs. \aap,
\textbf{417}, 93--106

\item
Semenov, D., Wiebe, D. and Henning, Th. (2006) Gas-phase CO in protoplanetary
disks: A challenge for turbulent mixing. \apjl, (submitted)

\item
Semenov, D., Pavlyuchenkov, Y., Schreyer, K., Henning, T., Dullemond, C., and
Bacmann, A.\ (2005) Millimeter Observations and Modeling of the AB Aurigae
System.  \apj, \textbf{621}, 853--874

\item
Steinacker, J., Henning, Th., Bacmann, A. and Semenov, D. (2003) 3D continuum
radiative transfer in complex dust configurations around stellar objects and
active galactic nuclei. I. Computational methods and capabilities. \aap,
\textbf{401}, 405--418

\item
Tscharnuter, W.~M. (1987) A collapse model of the turbulent presolar nebula.
\aap, \textbf{188}, 55

\item
Tscharnuter, W.~M and Gail, H.-P. (2006) 2-D protoplanetary accretion disks. I.
Hydrodynamics, chemistry, and mixing processes. (in preparation)

\item
van Boekel, R., Min, M.,Leinert, C. and  20 coauthors (2004) The building blocks
of planets within the `terrestrial' region of protoplanetary disks. \nat,
\textbf{432}, 479--482


\item Voshchinnikov, N.~V., Il'in, V.~B. and Henning,Th. (2005) Modeling the
optical properties of composite and porous interstellar grains. \aap,
\textbf{429}, 371--381

\item Voshchinnikov, N.~V., Il'in, V.~B., Henning, Th. and Dubkova, D.~N.
(2006) Dust extinction and absorption: the challenge of porous grains.
\aap, \textbf{445}, 167--177

\item
Wehrstedt, M. and Gail, H.-P. (2002) Radial mixing in protoplanetary accretion
disks. II. Time dependent disk models with annealing and carbon combustion.
\aap, \textbf{385}, 181--204

\item
Wehrstedt, M. and Gail, H.-P. (2003) Radial mixing in protoplanetary accretion
disks. V. Models with different element mixtures. \aap, \textbf{410}, 917--935

\item
Wehrstedt, M. and Gail, H.-P. (2006) Radial mixing in protoplanetary accretion
disks. VII. 2-dimensional transport of tracers. (in preparation)

\item
Wiebe, D., Semenov, D., \& Henning, T.\ (2003) Reduction of chemical networks.
I. The case of molecular cloud. \aap, \textbf{399}, 197--210

\item
Willacy, K., Klahr, H. H., Millar, T. J. and Henning, Th. (1998)
Gas and grain chemistry in a protoplanetary disk. \aap, \textbf{338}, 995--1005

\item
Wolf, S., Henning, Th. and Stecklum, B. (1999) Multidimensional self-consistent
radiative transfer simulations based on the Monte-Carlo method. \aap,
\textbf{349}, 839--850

\item
Wuchterl, G. and Tscharnuter, W.~M. (2003) From clouds to stars.  Protostellar
collapse and the evolution to the pre-main sequence I. Equations and evolution
in the Hertzsprung-Russell diagram. \aap, \textbf{398}, 1081--1090

\end{ownpubl}
%
%XXXXXXXXXXXXXXXXXXXXXXXXXXXXXXXXXXXXXXXXXXXXXXXXXXXXXXXXXXXXXXXXXXXXXXXXXXXXXXX
%XXXXXXXXXXXXXXXXXXXXXXXXXXXXXXXXXXXXXXXXXXXXXXXXXXXXXXXXXXXXXXXXXXXXXXXXXXXXXXX

\section{Goals (Ziele)}
%
The present project is part of a larger one consisting of a
theoretical, astrophysical part which will be conducted at the
Institute for Theoretical Astrophysics (ITA) belonging to the
recently founded Center of Astronomy (ZAH) of the Heidelberg
University, and a complementary experimental, mineralogical part (the project
\projlattard), which will be conducted at the Mineralogical
Institute and at KIP (University of Heidelberg) in cooperation
with the MPIA. One of the main goals is to bring together the
expertise of astrophysics on pre-planetary accretion disks, their
physics and their modeling at one hand, with the expertise of
mineralogy and meteoritics on the formation of the Solar System
and the expertise on phase transformation and solid state
reactions on the other hand. The resulting synergy effects are
expected to result in a significant acceleration of our
understanding of the formation of the Solar System and other
planetary systems.

\subsubsection{First part: Modeling of 2-D disk structure.} (Principle
investigator: W.~M. Tscharnuter)
%\\[1ex]

\noindent The codes existing at the Institute for Theoretical Astrophysics
for modeling protoplanetary accretion disks in the 1-D (one-zone)
and (1+1)-D approximation, respectively, and the 2-D radiation
hydrodynamics code for accretion disks with axial symmetry will be
continued to be developed to incorporate the important chemical
and mineralogical processes parallel to the experimental progress
in project \projlattard, which supplies the necessary input data
for quantitative model calculations. In turn, the model
calculations will show which type of materials and processes are
important for accretion disks and for which ones, for this reason,
high quality data are required to enable their realistic
incorporation into model calculations.
%\\[1ex]

The  1-D and (1+1)-D codes are required for investigating the
disk's evolution and its composition over periods of time of the
order of at least $10^6$\,yrs or more. They will be used to study
in detail the evolution of accretion disks and the chemical and
mineralogical composition of the disk material. The models will
cover the period from the end of the protostellar collapse phase
till the onset of planetesimal formation. For the time being, such
large periods cannot be covered with the available computer power
by explicit 2-D (and the less by 3-D) codes. The existing explicit
2-D code presently can cover periods at most of the order of
$10^4$\,yr. This explicit 2-D code is required to study particular
problems, for instance, to determine detailed flow structures and
to study their implications on transport processes. The 1-D and
(1+1)-D codes will be the working horses for modeling the long
term evolution of pre-planetary disks over periods of the Myr scale
and for providing initial and boundary conditions for studying
particular problems in 2-D models.
%\\[1ex]

The first million years of evolution of pre-planetary disks are of
particular interest since after planetesimals have formed the
solid material is locked in these bodies and is shut off from
further chemical and mixing processes in the disk. The
evolutionary phase up to the formation of the planetesimals
therefore pre-determines the compositions of the planets which in turn are
formed from planetesimals. The subsequent evolution through the planetesimal
phase will be considered in project \projdul, for which the present
model calculations will provide the required information on their
composition. However, in close cooperation with \projdul, early
coagulation of the sub-micron dust particles up to the cm-size region
will already be taken into account both in the (1+1)-D and 2-D models. 
Here, it is of particular interest to model the effect of fractional condensation (Petaev \& Wood 1998, 2004), i.e. that coagulated mineral grains become isolated against solid-gas chemistry, thereby changing the abundance and composition of further condensing minerals.
%\\[1ex]

Model calculations will be performed which cover the domain
of interest of stellar masses and total initial disk angular
momentums, and in particular cover a wide region of
metallicities. On this issue, detailed information will be
available from ongoing projects (A8, B11) in SFB\,439. This will
enable us to study the multitude of planetary systems which can be
formed in the galaxy and their various compositions.
%\\[1ex]

An important goal of the project will be to model the composition
of the primitive matrix material and the bulk composition of chondrites 
and the dust material in cometary nuclei. This will bridge the gap between the
copious information gained from laboratory analysis of the
mineralogical and chemical composition of meteorites on the early
processes in the Solar Nebula, and the ideas developed in
astrophysics on protostellar and pre-planetary accretion disks. It
will further enable to compare predictions from model calculations
for the composition of planetesimals with observations of the ice
and dust composition of cometary nuclei (cooperation with D. Wooden
from NASA-Ames is planned for this comparison). Both types of comparison
will provide important checks on the validity of the model
calculations and the underlying concepts, since only for material
from our own Solar System we have relics from the formation time
of the system which can be analyzed in the laboratory.
%\\[1ex]

The results of the project will also provide information on the
dust composition expected to exist in pre-planetary disks, that can
be used as input for radiative transfer models in project \projwolf\ and
the coagulation experiment \projblumtrie.
%\\[1ex]

In order to develop realistic models for the evolution of the dust
component it is necessary to extend and improve the models for
treating the growth and evaporation of dust and its chemical and
thermal processing in the accretion disk. For many materials
important for the pre-planetary disk only scarce or no information
is available on their properties and on the processes responsible
for their growth, evaporation, and interaction with the gas phase.
This holds in particular for the Ca-Al-compounds and the iron
bearing Mg-silicates. It is planned to close this gap in
cooperation with project \projlattard\ which will study by
laboratory experiments the vaporization, condensation, and
annealing properties,
%and the interaction with the gas phase of the materials
for which presently no or only insufficient data are available for
their incorporation into model calculations.
%\\[1ex]

The very final goal, that is, constructing fully 3-D models of
pre-planetary accretion disks, is certainly beyond the first
funding period of three years. With regard to the apparently
ever-increasing computer power it is intended, however, to start
with prelimininary, explorative 3-D studies. A cooperation with
project C1 is planned. 3-D models could well be one of the main
topics already for the second funding period.

%\vspace{1em}
%-------------------------------------------------------------------------------
\subsubsection{Second part: Modeling of chemistry and dust.}
(Principle investigators: H.-P. Gail and Th. Henning)
%\\[1ex]

\noindent In this part it is planned to study in detail the gas-phase
chemistry, the related chemical reactions on dust grain surfaces,
and the formation of ices of mixed composition. These processes
determine the composition of the volatile materials contained in
planetesimals beyond the snow line in accretion disks and will,
for instance, allow to predict, and to compare with observations,
the composition of the material in cometary nuclei, and the
volatile input in asteroid bodies. But also the long term mineral
evolution in the disk will be studied, for instance to determine
whether chemical gradients develop as they are observed in Solar
System material.
%\\[1ex]

A detailed hydrocarbon chemistry will be implemented to study this
particular chemistry in the warm inner parts of the disk. The
ion-molecule chemistry will be extended to allow for charging of
dust grains.
%\\[1ex]

The calculations will be done by a self-consistent coupling
between disk evolution, chemistry, and transport processes for
time dependent disk models in the (1+1)-D approximation and for
fully 2-D hydrodynamic models. The (1+1)-D models will allow
to to construct spatially extended models covering the full
region of planetary formation (for instance from 0.1 to 50 AU)
and evolution periods of at least $10^6$ years, at the
price of some simplifications for the transport processes which
are unavoidable in this approximation. The 2-D hydrodynamic models
of the first part, on the other hand, will allow to study more
realistically the hydrodynamic phenomena and the transport and
mixing process, but the modeling will be limited to periods of the
order of at most $10^4$ yrs and to moderately extended zones of
the disk because of high computational demands. Experience with
such type of calculations is available from previous work
(Gail \& Tscharnuter~2006, Tscharnuter \& Gail~2006).
%\\[1ex]

Models for the formation and destruction of dirty ice grains
consisting of an icy body with inclusions of dust particles will
be developed in analogy to corresponding models in aerosol
physics. This includes the modeling of the early phases of the
aggregation process, which will be done in close cooperation with
project \projdul. A model for calculating the opacity of such
mixed grains will be developed. The mixture of condensed phases
contained in the dust aggregates will be determined in close
cooperation with the first part of the project.
%\\[1ex]

The focus in this project will be on the early {\em actively
accreting} phase of protoplanetary disks, and on dust grains
smaller than about 1\,cm. This is the phase where the essential
processes operate which determine the composition of the material
of the planetesimals. Because of limited mixing of planetesimals
between different zones in the accretion disk and since nearly all
solid material of an accretion disk is locked in the planetesimals
after their formation, the composition of the planets is already
pre-determined by the composition of the planetesimals and the
radial variation of planetesimal compositions across the accretion
disk.
%\\[1ex]

Models will be calculated not only for Solar System elemental
compositions but also for the spectrum of metallicities
encountered in different regions of the galaxy and different
stages of the galactic chemical evolution.


%XXXXXXXXXXXXXXXXXXXXXXXXXXXXXXXXXXXXXXXXXXXXXXXXXXXXXXXXXXXXXXXXXXXXXXXXXXXXXX
\section{Work schedule (Arbeitsprogramm)}

\subsection{Methods}

%...............................................................................
\subsubsection{Basic modeling tools:}
%
Both parts of this project will start from \emph{existing}
numerical codes for solving the (1+1)-D and/or 2-D radiative
hydrodynamic structure and evolution of protoplanetary accretion
disks, including diffusion and transport processes (Gail 2001,
2004; Wehrstedt \& Gail~2003, 2006; Tscharnuter \& Gail 2006). The
project will focus on the development of detailed and as realistic
as possible solid-state and gas-phase chemistry within these disk
models coupled with transport processes and radiative transfer.
%\\[1ex]

Our model programs will be extended to construct time dependent
evolutionary models of accretion disks in the 1-D, (1+1)-D, and
2-D approximation. The equations for the disk structure are
coupled with the system of advection-diffusion-reaction equations
for the chemistry of the gas phase and the chemical and physical
processes for all dust components.  Since dust grains formed from
the same material, but with different radii, have to be treated as
separate components, we have already in the one-zone approximation
to deal with a system of several hundred nonlinear coupled
parabolic differential equations. Their solution offered no
particular problems in the model calculations performed up to now.
Therefore, there is an excellent prospect that an extension to
even bigger systems for the more realistic modeling of the
chemistry and of transport processes in pre-planetary disks
intended for the present project will be almost straightforward.
%\\[1ex]

Agglomeration processes will be implemented by applying methods
which have proven efficient and stable in aerosol science. It will
also be tested if a simplified treatment of coagulation on the
basis of the method of Deuflhard and Wulkow (1989), the discrete
Garkin approximation, is of sufficient accuracy to treat the
initial agglomeration process up to the point where cm-sized
aggregates are formed. This would be less time consuming than a
direct solution of the set of equations for agglomeration.
%\\[1ex]

In case self-gravity of the disk is to be taken into account
we shall use an efficient Poisson solver which is based on an
appropriate multi-grid algorithm. As far as the numerical solution
of the 2-D hydrodynamical equations is concerned, standard
explicit methods with operator splitting will be applied; the
energy equation together with the moment equations for the
radiation field demands an implicit solver because of the
prohibitively low Courant timesteps forced by optically thin
regions in the disk. The solution is obtained by a straightforward
Newton-Raphson procedure in combination with the GMRES algorithm
for solving the adjoint linearized equations iteratively. The
closure of the moment equations will be achieved by using the
Eddington factors (e.\,g., Stone et al.~1992) which are known
quantities as soon as the 2-D radiative transfer problem is
solved, e.\,g., by the method of short characteristics (e.\,g.,
Dullemond \& Turolla \cit{2000};
van Noort et al.~2002; Kor{\v c}{\'a}kov{\'a} \& {Kub{\'a}t}
2005).
%\\[1ex]

With increasing complexity of the processes to be modeled and the
increasing number of species the computational time requirements
will become serious. For this reason the 1-D model program will be
further developed parallel to the (1+1)-D program for the purpose
of testing models and algorithms before they are implemented in
the (1+1)-D or finally transferred to the fully 2-D hydrodynamic
model. The 1-D model calculations and, at least during the first
phase of the project, the (1+1)-D model calculations can be done
with the local computer resources of the ITA. Later it will
certainly become necessary to perform the (1+1)-D and, \emph{a
fortiori}, 2-D calculations on parallel computers. This can and
will be done, e.\,g., at the parallel clusters which are available
at institutions of the University of Heidelberg (ZAH, IWR) and at
the MPIA.

%*******************************************************************************
\medskip
\subsection{Schedule}
%  Zu 7.2  Schedule

\medskip\noindent%
\textbf{First part: Modeling of 2-D disk structure}

%\bigskip
\subsubsection{First year:}
To put the project on firm grounds, first of all, several, partly
substantial extensions of the existing (1+1)-D and 2-D radiation
hydrodynamical code must be worked
out. They will comprise the following items:\\
%
(i) \emph{Self-gravity} of the disk material has already a
noticeable influence on the vertical pressure structure if the mass of the
disk exceeds only a few percent of the central star's mass. We
plan to implement an appropriate multi-grid algorithm to solve the
Poisson equation.\\
%
(ii) \emph{Radiative transfer} has been dealt with exclusively by using the
Eddington approximation with flux limiter. We plan to implement the
short-characteristics method to solve the 2-D radiation transport
problem. We will use the method described in Dullemond \& Turolla
(\cit{2000}). The code for this will be provided by C.\ P.\ Dullemond (PI of
project C2). The resulting Eddington factors will then be used to close the
moment equations describing the coupling between matter and radiation.\\
%
(iii) \emph{Coagulation} processes which change the size
distribution of the dust particles are the key to form meso- and
macroscopic solid objects out of initially sub-micron dust grains.
To implement a suitable algorithm for solving the coagulation
equations into the 2-D code as efficiently as possible will make
use of the results and the experience to be compiled in project
\projdul.\\
%
(iv) \emph{Sedimentation and general drift motions} of larger dust
grains relative to the gas will be allowed for by using the
concept of the ``asymptotic terminal grain velocity''. The basic
assumption here is that the exchange of momentum between the
various gas and dust components of the disk material is in a
stationary state at every instant of time. Within this approach
the velocities of all components relative to each other are
constant. As an important consequence, we need only deal with the
total momentum of the system. Hence, no extension of the equations
of motion is necessary.
\\
%
The corresponding program units for the 2-D model program will be
developed in part by the PIs of the project, and only in part by the
PhD student. This holds for all parts of the project A2.
%\\[1ex]

%
\subsubsection{Second year:}
We will continue to work on the implementation of chemical and
mineralogical processes, since new relevant results of the
experiments carried out in project A1 will become available
presumably not until the end of the first year. The topics will
be:\\
%
(v) Detailed modeling of condensation and evaporation mechanisms
of Mg-Fe-silicates.\\
%
(vi) Combustion (oxidation) of soot/graphite particles and the
ensuing gas-phase chemistry of hydrocarbons.\\
%
(vii) Growth of enstatite layers on forsterite grains.
\\
(viii) Chemistry and mineralogy of iron and iron-compounds,
e.\,g., troilite (FeS).
%\\[1ex]

\subsubsection{Third year:}
This last period will be mainly devoted to build up more
sophisticated models of the ice mantles covering the refractory
grains at low temperatures. In case new experimental data will be
available, we will be able to model Ca-Al-minerals in more detail.
We also plan to step into the isotope chemistry of
hydrogen/deuterium, carbon and oxygen. However, we will focus our
efforts mainly on the investigation of ice-coated grains:
\\
(ix) ``Pure'' ice layers of H$_2$O, CO, etc.\ are as such
relatively easy to deal with. However, there is the important
complication that, in the course of the condensation of H$_2$O, a
certain amount of still more volatile species, like CO, will be
built into the growing water-ice layer to form so-called
clathrates.

\bigskip
%-------------------------------------------------------------------------------
\medskip\noindent{\bf Second part: Modeling of chemistry and dust}

%...............................................................................
%\bigskip
\subsubsection{First year}
The existing 1-D and (1+1)-D codes will be extended to handle an
as complete as possible gas phase chemistry, in particular with
respect to the chemistry of hydrocarbons, appropriate for the
inner disk regions inside $\approx20$\,AU, and excluding the disks
upper photospheric layer. Freezing of ices on dust grain surfaces
and dust charging will be included. The development of the gas
phase chemistry modeling will be mainly concentrated on the first
year, since a lot of testing and explorative calculations have to
be done for preparing the implementation of a realistic but not
too large chemistry in the more complex 2-D hydro-code.
%\\[1ex]

The chemistry of the upper photospheric layers of accretion disks
will be modelled in detail by accurate computation of the
penetration of high-energy stellar and interstellar radiation (UV,
X-rays) fields into the disk by mean of multi-dimensional line and
continuum radiative transfer. The calculated 2-D distributions of
the impinging radiation fields will be used to compute accurate
photo-dissociation rates of various molecules by spectral averaging
of the corresponding frequency-dependent photo-rates, which likely
will affect the resulting isotopic fractionation. A special
emphasis will be given to the feasible description of the X-ray
chemical reactions in the inner disk zone, using new observational
data on the X-ray activity and variability of the young low-mass
stars. Another important goal will be to take adequate grain size
distribution -- especially PAH-like tiny grains -- into account in
the chemical code.
%\\[1ex]

The development of complex dust extinction models for agglomerate
particles and their integration into the codes will be started.
This work requires a gradual development from simple to complex
models and will probably extend over the whole three year period.
A close cooperation with  N.~V. Voshchinnikov from St. Petersburg
is planned for this part of the project.
%\\[1ex]

Methods for numerical calculation of dust growth and evaporation
processes including growth of surface layers, e.\,g., enstatite
layers on olivine cores, and cation diffusion in mineral grains,
e.\,g., for treating conversion of iron bearing silicates into
magnesium rich silicates, will be tested and implemented in
cooperation with the first part of the project. These developments
serve as test cases for the corresponding developments in the
first part, where limitations of computational time requirements
do not allow extensive tests of different methods to find the most
efficient ones. Such tests will be performed in this second part
of the project.
%\\[1ex]

During the first year the work will concentrate on dust from the
important Si-Mg-Fe system, since part of the required data on
mineral properties for such minerals are already available from
the literature. These data will suffice for our work during the
initial start-up phase.
%\\[1ex]

At the same time the codes will be used to provide first
meaningful results for the mineral processing in the disk. This
will allow to define more definitely which type of silicate dust
mixtures exist in different parts of protoplanetary disks and
which types of experimental data on minerals are needed from
project \projlattard\ for further progress with disk modeling.
%\\[1ex]

%...............................................................................
\subsubsection{Second year}
During the second year first results from the laboratory
experiments in project \projlattard\ are expected to become
available. The development of models for calculating the mineral
composition in the accretion disk will be continued and extended
on the basis of these data. It will be possible to pin down more
exactly which type of experimental data are precisely needed for
further progress in dust modeling.
%\\[1ex]

Also the modeling of the dust species from the Ca-Al-silicate
group will be started during this period, since it is expected
that first results for the data required for modeling these dust
components of protoplanetary disk become available after the first
year.
%\\[1ex]

The results of the modeling of disk structure and evolution in
this part and the first part of the project will allow to make
realistic predictions for the dust mixture in the disk. The
results obtained in the first part of the project from 2-D
hydrodynamic models will also give insight into the global
hydrodynamic mixing and transport processes in the disk, which
determine the spatial and temporal distribution of different dust
components in the disk. This will serve as basis for the further
development of dust extinction models for the accretion disk.
%\\[1ex]

It is expected that, after the first year, the first detailed
results from project \projdul\ also become available which can
serve to determine the degree of agglomeration of dust in
accretion disks. They will be used as a guideline to implement
dust coagulation also in the (1+1)-D models used in this project,
since this is required for the further evolution of the dust
extinction models.
%\\[1ex]

%...............................................................................
\subsubsection{Third year}
In the third year the interactive improvement of models for the
dust evolution between theoretical model calculations in this
project and the laboratory experiments done in \projlattard\  will
be continued and extended. It is expected  that it will be
possible in cooperation with the first part of this project, with
project \projdul, and the other projects to get realistic models
for the dust+ice composition in accretion disks. These models will
be confronted to comparison with observations and laboratory
investigations of dust composition in comets and meteorites.
%\\[1ex]

It is expected that the codes for the calculation of disk
structures and evolution will be sufficiently developed to enable
first comparisons with observations of the early Solar System
formation as derived from analysis of meteorites. The model
calculations will be compared to timings of processes in the early
Solar System as derived for instance in project \projtrie, or to
results for compositional variations of meteoritic parent bodies
in different zone of the early Solar System. This will allow
further constraining of the models for disk evolution.
%\\[1ex]

Models for protoplanetary disks with the full spectrum of dust
processing and the initial stage of coagulation will be calculated
for the range of metallicities and abundances of the rock forming
elements encountered during the chemical evolution of the Milky
Way in order to constrain when and where planetary systems can be
formed in the Milky Way.

%*******************************************************************************
\subsection{Literature}
%
% Here follows a general literature list related to the topic of the
% proposal, just like a literature list for a scientific paper.
%
% AGAIN ONLY EXAMPLES ARE LISTED NOW
%
\begin{literature}

\item
Aikawa, Y., van Zadelhoff, G.~J., van Dishoeck, E.~F., and Herbst, E.\ (2002)
Warm molecular layers in protoplanetary disks. \aap, \textbf{386}, 622--632


\item
Alexander, C.~M.~O., and Keller, L.~P. (2006) Are There Clues to the Dust
'Annealing' Process in Protoplanetary Disks in IDPs? \textit{37th Annual Lunar and Planetary Science Conference}, 2325

\item
Alexander, C.~M.~O., Boss, A.~P., Keller, L.~P., Nuth, J.~A. and Weinberger
(2006) Astronomical and Meteoritic Evidence for the Nature of Interstellar Dust
and its Processing in Protoplanetary Disks. in: \textit{Protostars and
Planets~V\/}, eds. B. Reipurth, D. Jewitt and K. Keil.\\ 
  {\tt http://ifa.hawaii.edu/UHNAI/ppv.htm}
%\\{\tt http://pasp.byu.edu/FTP\%5FSite/E\%2Dbook\%5F2005/341/}

\item
All\`egre, C.~J., Manh\`es, G. and Lewin, E. (2001) Chemical composition of the
Earth and the volatility control on planetary genetics. \epsl, \textbf{185},
49--69

\item
Asplund, M., Grevesse, N. and Sauval, A.~J. (2005) The Solar Chemical
Composition. \textit{ASP Conf. Ser. 336: Cosmic Abundances as Records of Stellar
Evolution and Nucleosynthesis}, eds. T.~G. Barnes and F.~N. Bash, F.~N.
(San Francisco: Astronomical Society of the Pacific), 25--38

\item
Bergin, E.~A., Aikawa, Y., Blake, G.~A. and van Dishoeck, E.~F. (2006) The
Chemical Evolution of Protoplanetary Disks.  in: \textit{Protostars and
Planets~V\/}, eds. B. Reipurth, D. Jewitt and K. Keil.\\ 
  {\tt http://ifa.hawaii.edu/UHNAI/ppv.htm}
%ArXiv Astrophysics e-prints, arXiv:astro-ph/0603358

\item
Bockel{\'e}e-Morvan, D., Gautier, D., Hersant, F., Hur{\'e}, J.-M. and Robert,
F. (2002) Turbulent radial mixing in the solar nebula as the source of
crystalline silicates in comets. \aap, \textbf{384}, 1107--1118

\item
Boss, A.~P. (2004) Evolution of the Solar Nebula. VI. Mixing and Transport of
Isotopic Heterogeneity. \apj, \textbf{616}, 1265-1277

\item
Bouvier, J., Alencar, S.~H.~P., Harries, T.~J., Johns-Krull C.~M. and Romanova, M.~M. (2006) Magnetospheric accretion in classical T Tauri stars.  In:
Protostars and Planets V, eds. B. Reipurth, D. Jewitt and K. Keil.\\ 
  {\tt http://ifa.hawaii.edu/UHNAI/ppv.htm}

\item
Ciesla, F.~J. and Cuzzi, J.~N. (2006) The Evolution of the Water Distribution in
a Viscous Protoplanetary Disk. \textit{ArXiv Astrophysics e-prints},
arXiv:astro-ph/0511372

\item
D'Alessio, P., Calvet, N. and Woolum, D.~S. (2005) Thermal Structure of
Protoplanetary Disks. In:\textit{ASP Conf. Ser. 341: Chondrites and the
Protoplanetary Disk}, eds. A.~N. Krot,  E.~R.~D. Scott and B. Reipurth, p.
353--372
{\tt http://pasp.byu.edu/FTP\%5FSite/E\%2Dbook\%5F2005/341/}

\item
Dent, W.~R.~F., Torrelles, J.~M., Osorio, M., Calvet, N., \& Anglada, G.\ (2006)
The circumstellar disc around the Herbig AeBe star HD169142.
\mnras, \textbf{365}, 1283--1287

\item
Deuflhard, P. and M. Wulkow (1989) Computational Treatment of Polyreaction
Kinetics. \textit{IMPACT Comput. Sci. Eng.}, \textbf{1}, 269

\item
{Dullemond}, C.\ P. and {Turolla}, R. (2000) An efficient algorithm for 
two-dimensional radiative transfer in axisymmetric circumstellar 
envelopes and disks. \aap, \textbf{360}, 1187

\item
Dullemond, C.~P., Apai, D. and {Walch}, S. (2006) Crystalline Silicates as a
Probe of Disk Formation History. \apjl, \textbf{640}, L67-L70

\item
Duschl, W.~J., Strittmatter, P.~A. and Biermann, P.~L. (2000) A note on
hydrodynamic viscosity and selfgravitation in accretion discs. \aap,
\textbf{312}, 1123--1132

\item
Fegley Jr., B. and Prinn, R.~G. (1989) Solar nebula chemistry:
Implications volatiles in the solar system. In: \textit{The Formation and Evolution of Planetary Systems}, eds. H. Weaver and L. Danley (Cambridge University
Press, Cambridge), 171--211

\item
Harker, D.~E. and Desch, S.~J. (2002) Annealing of Silicate Dust by Nebular
Shocks at 10 AU. \apjl, \textbf{565}, L109--L112

\item
Harker, D.~E., Woodward, C.~E., Wooden, D.~H. (2005) The Dust Grains from
9P/Tempel 1 Before and After the Encounter with Deep Impact. \sci, \textbf{310},
278--280

\item
Hasegawa, T.~I., and Herbst, E.\ (1993) Three-Phase Chemical Models of Dense
Interstellar Clouds - Gas Dust Particle Mantles and Dust Particle Surfaces.
\mnras, \textbf{263}, 589--606

\item
Hasegawa, T.~I., Herbst, E., and Leung, C.~M.\ (1992) Models of gas-grain
chemistry in dense interstellar clouds with complex organic molecules.
\apjs, \textbf{82}, 167--195

\item
Helling, C., Winters, J.~M., and Sedlmayr, E.\ (2000) Circumstellar dust shells
around long-period variables. VII. The role of molecular opacities.
\aap, \textbf{358}, 651--664

\item
Humayun, M. and Clayton, R.~N. (1995) Potassium isotope geochemistry: Genetic
implications of volatile element depletion. \gca, \textbf{59}, 2131--2148

\item
Ilgner, M. and Nelson, R.~P. (2006) On the ionisation fraction in protoplanetary
disks. II. The effect of turbulent mixing on gas-phase chemistry. \aap,
\textbf{445}, 223--232

\item
Kimura, H., Kolokolova, L. and Mann, I. (2006) Light scattering by cometary dust
numerically simulated with aggregate particles consisting of identical spheres.
\aap, \textbf{449} 1243--1254

\item
Kor{\v c}{\'a}kov{\'a}, D. and Kub{\'a}t, J. (2005) Radiative transfer in
moving media. II. Solution of the radiative transfer equation in axial symmetry.
\aap, \textbf{440}, 715--725

\item
Lahuis, F., van Dishoeck, E.~F., Boogert, A.~C.~A. and 8 coauthors (2006) Hot
Organic Molecules toward a Young Low-Mass Star: A Look at Inner Disk Chemistry.
\apjl, \textbf{636}, L145-L148

\item
Langer, W.~D., van Dishoeck, E.~F., Bergin, E.~A. and 4 coauthors (2000)
Chemical Evolution of Protostellar Matter. In: 
\textit{Protostars and Planets IV}, eds. V. Mannings, A.~P. Boss, S.~S. Russell (Tucson: University of Arizona
Press) p. 29--58

\item
Millar, T.~J., Farquhar, P.~R.~A., and Willacy, K.\ (1997) The UMIST Database
for Astrochemistry 1995. \aaps, \textbf{121}, 139--185

\item
Morfill, G.~E. and V\"olk, H.~J. (1984) Transport of dust and vapor and chemical
fractionation in the early protosolar cloud.  \apj, \textbf{287}, 371--395

\item
Min, M., Dominik, C., Hovenier, J.~W., de Koter, A. and Waters,
L.~B.~F.~M. (2006) The 10\,$\mu$m amorphous silicate feature of fractal
aggregates and compact particles with complex shapes. \aap, \textbf{445},
1005--1014

\item
Nagahara, H. and Ozawa,  K. (1996) Evaporation of forsterite in H$_2$ gas.
\gca, \textbf{60}, 1445--1459

\item
Natta, A., Testi, L., Calvet, N., Henning, Th., Waters, R. and Wilner, D. (2006)
Dust in Proto-Planetary Disks: Properties and Evolution. In: Protostars and
Planets V, eds. B. Reipurth, D. Jewitt and K. Keil.\\ 
  {\tt http://ifa.hawaii.edu/UHNAI/ppv.htm}

\item
Nejad, L.~A.~M.\ (2005) A Comparison of Stiff ODE Solvers for Astrochemical
Kinetics Problems. \apss, \textbf{299}, 1--29

\item
Norman, M.~L. and Winkler K.-H.~A. (1986) 2-D Eulerian Hydrodynamics with Fluid
Interfaces, Self-Gravity and Rotation. In: \textit{NATO Advanced Study Research Workshop on Astrophysical Radiation Hydrodynamics}, eds.  K.-H.~A. Winkler and
M.~L. Norman (Reidel, Dordrecht), 187--221

\item
Pollack, J.~B., Hollenbach, D., Beckwith, S., Simonelli, D.~P., Roush, T.
and Fong, W. (1994) Composition and radiative properties of grains in molecular
clouds and accretion disks. \apj, \textbf{421}, 615--639

\item
Prinn, R.~G. (1993) Chemistry and evolution of gaseous circumstellar
disks. In \textit{Protostars and Planets III}, Ed. E.~H. Levy, J.~I. Lunine
(University of Arizona Press, Tucson), 1005--1028

\item
Prinn, R.~G. and Fegley Jr., B. (1989) Solar nebula chemistry: Origin of
planetary, satellite, and cometary volatiles. In: \textit{Origin and Evolution
of Planetary and Satellite Atmospheres}, eds. S.~K. Atreya, J.~B. Pollack and
M.~S. Matthews (University of Arizona Press, Tucson),  78--136

\item
Rettig, T.~W., Brittain, S.~D., Gibb, E.~L., Simon, T. and Kulesa, C.(2005)
CO Emission and Absorption toward V1647 Orionis (McNeil's Nebula). \apj,
\textbf{626}, 245--252

\item
Rietmeijer, F.~J.~M., Hallenbeck, S.~L., Nuth, J.~A. and Karner, J.~M. (2002),
Amorphous Magnesiosilicate Smokes Annealed in Vacuum: The Evolution of Magnesium
Silicates in Circumstellar and Cometary Dust. \ica, \textbf{156},  269--286

\item
Stone, J.~M., Mihalas, D. and Norman, M.~L. (1992) ZEUS-2D: A radiation
magnetohydrodynamics code for astrophysical flows in two space dimensions.
III - The radiation hydrodynamic algorithms and tests. \apjs, \textbf{80}, 819--845

\item
Tachibana, S., Tsuchiyama, A. and Nagahara, H. (2002) Experimental study of
incongruent evaporation kinetics of enstatite in vacuum and in hydrogen gas.
\gca, \textbf{66}, 713--728.

\item
Taylor, S.~R. (1988) Planetary compositions. In: \textit{Meteorites and the
Early Solar System}, eds. M.~S. Matthews and J.~F. Kerridge (University of Arizona press, Tucson), p. 512.

\item
Turner, N.~J., Willacy, K., Bryden, G. and Yorke, H.~W. (2006) Turbulent Mixing
in the Outer Solar Nebula. \apj, \textbf{639}, 1218--1226

\item
van Noort, M., Hubeny, I. and Lanz, T. (2002) Multidimensional Non-LTE Radiative
Transfer. I. A Universal Two-dimensional Short-Characteristics Scheme for
Cartesian, Spherical, and Cylindrical Coordinate Systems. \apj, \textbf{568},
1066--1094

\item
Voshchinnikov, N.~V. and Mathis, J.~S. (1999) Calculating cross sections
of composite interstellar grains. \apj, \textbf{526}, 257--264

\item
Willacy, K. and Langer, W.~D.\ (2000) The Importance of Photoprocessing in
Protoplanetary Disks. \apj, \textbf{544}, 903--920

\item
Wooden, D.~H., Butner, H.~M., Harker, D.~E. and Woodward, C.~E. (2000) Mg-Rich
Silicate Crystals in Comet Hale-Bopp: ISM Relics or Solar Nebula Condensates?
\ica, \textbf{143}, 126--137

\item
Wooden, D.~H., Harker, D.~E. and Brearley, A.~J. (2005) Thermal Processing and
Radial Mixing of Dust: Evidence from Comets and Primitive Chondrites.
In: \textit{ASP Conf. Ser. 341: Chondrites and the Protoplanetary Disk},
eds. A.~N. Krot,  E.~R.~D. Scott and B. Reipurth, p. 774--810
\\{\tt  http://pasp.byu.edu/FTP\%5FSite/E\%2Dbook\%5F2005/341/}

\item
Wooden, D.~H., Desch, S., Harker, D., Gail, H.-P. and Keller, L. (2006) Comet
grains and Implications for Heating and Radial Mixing in the Protoplanetary Disk.
In: \textit{Protostars and Planets V}, eds. B. Reipurth, D. Jewitt and K. Keil.\\
 {\tt http://www.ifa.hawaii.edu/UHNAI/ppv.htm}

\item
Wuchterl, G. (1990) Hydrodynamics of Giant Planet Formation I: Overviewing the
$\kappa$-Mechanism. \aap, \textbf{238}, 83--94

\item
Wuchterl, G. (1991) Hydrodynamics of Giant Formation II: Model Equations and
Critical Mass. \ica, \textbf{91}, 39--52

\item
Wuchterl, G., 1991b, Hydrodynamics of Giant Planet Formation III: Jupiter's
Nucleated Instability. \ica, \textbf{91}, 53--64

\end{literature}

%XXXXXXXXXXXXXXXXXXXXXXXXXXXXXXXXXXXXXXXXXXXXXXXXXXXXXXXXXXXXXXXXXXXXXXXXXXXXXXX
\section{External/International collaborations}
\begin{collablist}
\item[D.~Wooden (NASA-Ames)] It is planned to cooperate with
respect to observations of dust from comets. Dr. Wooden is one of
the leading experts in modeling the emission from cometary dust
and analyzing the structure and composition of cometary dust by
remote sensing.

\item[N.~V. Voshchinnikov (Astronomical Institute of St.~Petersburg University)]
It is planned to cooperate with respect to developing models for the opacity of
particle agglomerates. Dr. Voshchinnikov is one of the leading experts in this
field.

\item[E. Herbst (Ohio State University)] A cooperation is planned to develop
models for the formation of complex organic compounds in protoplanetary disks.

\item[H. Mutschke (Uni Jena)] A cooperation is planned for laboratory determination of optical constants and investigations of extinction by complex particles.

\end{collablist}


\section{Link to other projects of the Forschergruppe}
\begin{linkproj}

\item[\projlattard] A strong cooperation is planned between
\projlattard\ and the present project to identify the main
processes responsible for the chemical and mineralogical evolution
of the disk material, to apply the data on mineral properties
determined by laboratory experiments in project \projlattard\ to
evolutionary models for the protoplanetary disk, and to determine
which kind of laboratory input data are most urgently required for
a realistic disk modeling and should be determined by experiments.

\item[\projwolf{}] The results from the models can be used in project
\projwolf{} to predict mineral emission features in the spectra of
protoplanetary disks.
\item[\projtrie{}] Chronometry of planetesimal formation
(project~\projtrie{}) places constraints on the time scales to establish
differences in chemical compositions of the solids.
\item[\projblumtrie{}] The abundances of minerals computed in this
project can be used in project \projblumtrie{} to construct realistic
multi-phase agglomerates for collision experiments.
\item[\projklahr{}] With \projklahr{} we will collaborate on 
the numerical treatment of energy transport through radiation.
\item[\projdul{}] A collaboration with the
coagulation project \projdul{} is envisioned, because coagulation limits the
gas-solid reactions, and because the coagulation modeled in project
\projdul{} depends on properties and abundances of minerals.
\end{linkproj}

\section{Team members (Zusammensetzung der Arbeitsgruppe)}
%
% NOTE: Only list non-DFG-funded team members.
% NOTE: Also list technical assistants, students etc involved in the project
%
\begin{teamlist}
\item[Tscharnuter, W.~M., Prof.~Dr. (ITA, C4)] principal investigator\\
Numerical hydrodynamics, radiative transfer, modeling of accretion disks,
modeling of chemistry.

\item[Gail, H.-P., Prof.~Dr. (ITA, Wiss. Ang.)] principal investigator\\
Modeling of mineralogical processes and of gas phase chemistry, modeling of
accretion disks and radiative transfer

\item[Henning, Prof.~Dr. (MPIA, C4)] principal investigator\\
Modeling of dust extinction, modeling of chemistry in accretion disks,
analysis of observations of accretion disks

\item[Lattard, D.., Prof.~Dr. (Min. Inst., C3)] co-investigator\\
Laboratory investigations of annealing of amorphous materials and
of vaporization/condensation processes of interstellar mineral
analogues.

\item[Trieloff, M., Dr.~habil (Min. Inst, Wiss. Ang.)] co-investigator, \mbox{}\\
Laboratory investigations of vaporization/condensation processes of
interstellar mineral analogues, and of annealing of amorphous materials.

\item[Semenov, D., Dr. (MPIA, Wiss. Ang)] cooperating scientist\\
Cooperation with respect to extinction models for composite dust particles and
with respect to modeling the gas phase chemistry in accretion disks.

\item[J\"ager, C., Dr. (MPIA, Wiss. Ang)] cooperating scientist\\
Investigations of annealing, nanoparticles.

% \item[N.N., PhD-Student 1 (ITA, E13/2)]\mbox{}\\
% This PhD student is one of the two new students for this project. The
% student is financed by DFG.
% 
% \item[N.N., PhD-Student 2 (ITA, E13/2)]\mbox{}\\
% This PhD student is one of the two new students for this project. The
% student is financed by the ITA.

\end{teamlist}
\vspace{1em}



\section{Funding requested}
We request funding for only {\em one} of the two PhD student positions for
this project. The second one will be financed by the host institution. 
%The following table gives an overview of requested
%funding:
Overview over requested funding:
\vspace{1\baselineskip}\\
%
% The table that follows is the overview over the full requested
% funding, including the positions, travel, consumables and ``other
% costs'' (which might include transportation costs of radioactive
% material or the rent of a drop tower or such).
%
\centerline{\begin{tabular}{||l|r|r|r||} \hline \hline & Year 1 & Year 2 &
Year 3 \\ \hline %
Personnel (1 PhD-students: E13/2)   & \hfil24\,000 & 24\,000 & 24\,000 \\
Consumables                        & \hfil 400 & \hfil 400 &\hfil 400 \\
Travel                             & \hfil3\,000 & \hfil3\,000 &\hfil3\,000 \\
Other costs                        & \hfil 0 & \hfil 0 & \hfil 0 \\
\hline
{\bf Total:}                       & \hfil27\,400& \hfil27\,400& \hfil27\,400\\
\hline
\hline
\end{tabular}
}
\vspace{1em}\\
Below these costs are explained in more detail:

\subsection{Personnel (Personalbedarf)}
\begin{teamlist}
\item[PhD-Student 1 (E13/2)]\mbox{}\\
The student will implement into the 2-D hydro-code the processes
for the evolution of the dust components in the accretion disk:
evaporation, condensation, annealing, gas-solid chemical reactions
and -- in cooperation with project \projdul\ -- dust agglomeration
and particle drift. Parallel to this work model calculations for
the evolution of disks and their chemical and mineralogical
composition will be performed. The work will be based in part on
existing 1-D and (1+1)-D codes for the evolution of accretion
disks including chemistry and mineralogical processes. Since a lot
of know-how and software has already been developed in previous
studies the work can be performed within the frame of a PhD
thesis.

\item[PhD-Student 2 (E13/2)]\mbox{}\\
The student will study in detail the gas-phase chemical processes
including the surface reaction on dust particles and the
condensation of ice mixtures in protoplanetary accretion disks. At
the same time, in cooperation with the team members of the
project, models for the extinction by complex dust-ice
agglomerates will be developed. These studies will be combined
with model calculations based on existing 1-D and (1+1)-D codes
and the 2 D hydro-code developed in the first part of the project.
Since a lot of know how and software has already been developed in
previous studies the work can be performed within the frame of a
PhD thesis.

\end{teamlist}

\subsection{Consumables (Verbrauchsmaterial)}
Running costs for computing (supply for inkjet, laser and other printers;
storage media like DVD$\pm$R(W)s, minor software updates and licenses)
on average 400 EUR/year.
%\\[1ex]
\\

%
%Estimated cost per year:\vspace{1\baselineskip}\\
\centerline{\begin{tabular}{||l|r|r|r||}
\hline \hline & Year 1 & Year 2 & Year 3 \\
\hline
%
running costs   & \hfil 400 & 400 & 400 \\
\hline
\hline
\end{tabular}
}

\subsection{Travel expenses in addition to Project Z (Reisekosten)}
%
% Here only travel expenses not related to usual regular Forschergruppe
% meetings and the overall per capita budget for conferences.
%
The PhD students and PIs have to travel for cooperation with other projects
to other nodes. In particular the cooperation with the projects in
T\"ubingen and M\"unster will be important. Estimated expenses per travel
will be about 500 EUR per week (hotel), plus 70 EUR for travel to T\"ubingen
and 150 EUR to M\"unster.  About 2 week-long travels to partner institutes
are planned for each of
the two students per, and totally 2
travels per year for the three PIs.
%\\[1ex]
\\

%
%Estimated cost per year:\vspace{1\baselineskip}\\
\centerline{\begin{tabular}{||l|r|r|r||}
\hline \hline & Year 1 & Year 2 & Year 3 \\
\hline %
Travel between cooperating institutes (students)  &
 \hfil2\,000 & 2\,000 & 2\,000 \\
Travel between cooperating institutes (PIs)  &
 \hfil1\,000 & 1\,000 & 1\,000 \\
\hline
{\bf Total:}& \hfil3\,000& \hfil3\,000& \hfil3\,000\\
\hline
\hline
\end{tabular}
}

\subsection{Other costs (Sonstige Kosten)}
None\\


\section{Preconditions for carrying out the project at home institution}
%
% This is one of the main subsections of a DFG Normalverfahren proposal.
% Several of the subsubsections in this subsection we have placed in their
% own subsections above (like team members, collaborations). What remains
% are the following three subsections. For those not familiar with these,
% we refer to the DFG Merkblatt on Normalverfahren-proposals.
%
\subsection{Scientific equipment available (Apparative Ausstattung)}
%
% Please list those larger instruments available to you for the project (if
% applicable also larger computer equipment in case you need substantial
% amounts of computer time).
%
\begin{itemize}
\item For the model calculations we will use in part the PIA cluster at the
MPIA Heidelberg.  PIA is a Cluster with 128 x 2,6 GHz Dual-Opteron-Processors
by SUN (V20z), with a fast Intercommunication by a 144-port Infiniband-Switch
by Mellanox (MTS14400). Each knot has 4 GB memory and 2 x 73 GB disks.
For storing simulation results totally 10 TeraByte storage on two fibre channel
RAID-systems are available.

\end{itemize}


\subsection{Institution's general contribution (Laufende Mittel f\"ur Sachausgaben)}
%
% Please state the annual fund for consumables which comes from the
% institution's budget or any other third party  (please list separately) to
% pay for the research for which your project is part of.  Use estimates where
% applicable.
%
We estimate that the running costs per year of our equipment
are:\vspace{1\baselineskip}\\
%
\centerline{\begin{tabular}{||l|r|r|r||} \hline \hline & Year 1 & Year 2 &
Year 3 \\ \hline %
running costs   & \hfil4\,000 & 4\,000 & 4\,000 \\
\hline
\hline
\end{tabular}
}




