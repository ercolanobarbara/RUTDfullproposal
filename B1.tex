
%%%%%%%%%%%%%%%%%%%%%%%%%%%%%%%%%%%%%%%%%%%%%%%%%%%%%%%%
% LaTex Template for proposals within the              %
% DFG Research Unit Program                            %         
%Planet Formation Witnesses and Probes: Transition Discs
% August 2016                                            %                           
%                                                      %
%%%%%%%%%%%%%%%%%%%%%%%%%%%%%%%%%%%%%%%%%%%%%%%%%%%%%%%%
%
% 
%
% This template may be used to prepare proposals in latex.
%
%
% The project description, including publication list, should be no more than 20 pages
% in length. It should be self-explanatory and not require reviewers to read the 
% literature that is quoted or enclosed.

\documentclass[10pt,fleqn,twoside]{article}

%%%% USE ARIAL FONT %%%%%%%%%%%%%%%%%%%%%%%%%%%%%%%%%%%%%%%%%%%%%%%%%%%%%%
\usepackage{helvet}
\renewcommand\familydefault{phv}

%%%% INCLUDE NECESSARY PACKAGES %%%%%%%%%%%%%%%%%%%%%%%%%%%%%%%%%%%%%%%%%%
%\usepackage{babel}
\usepackage[UKenglish]{babel}
\usepackage{amsmath}
\usepackage{amssymb}
\usepackage{fancyhdr}
\usepackage{natbib}
\usepackage{xcolor}
\usepackage{ae,aecompl}
\usepackage{graphicx}
\usepackage{palatino}
\usepackage[T1]{fontenc}
\usepackage{rotating}
\usepackage{epsf}
\usepackage{setspace}
%\usepackage{sfmath}

%%%% PAGE LAYOUT %%%%%%%%%%%%%%%%%%%%%%%%%%%%%%%%%%%%%%%%%%%%%%%%%%%%%%%%%
\setlength{\textheight}{22cm}
\setlength{\topmargin}{-1.2cm}
\setlength{\textwidth}{15.6cm}
\setlength{\oddsidemargin}{0.0cm}
\setlength{\evensidemargin}{0.0cm}
\setlength{\mathindent}{1.5cm}
\setlength{\parindent}{0.0cm}
\setlength{\parskip}{0.08cm}

%%%% PAGE HEADER %%%%%%%%%%%%%%%%%%%%%%%%%%%%%%%%%%%%%%%%%%%%%%%%%%%%%%%%%%
\pagestyle{fancy}
\fancyhead[RE,RO]{}
\fancyfoot[RO]{ \thepage}
\fancyfoot[LE]{ \thepage}
\fancyfoot[CE,CO]{}

%%% FONTS FOR THE TITLE PAGE %%%%%%%%%%%%%%%%%%%%%%%%%%%%%%%%%%%%%%%%%%%%%%
\newfont{\tpfonta}{cmssbx10 scaled 1600}
\newfont{\tpfontb}{cmssbx10 scaled 3200}

%%%% EURO SIGN %%%%%%%%%%%%%%%%%%%%%%%%%%%%%%%%%%%%%%%%%%%%
\newcommand\euro{{\sffamily C%
\makebox[0pt][l]{\kern-.70em\mbox{--}}%
\makebox[0pt][l]{\kern-.68em\raisebox{.25ex}{--}}}}
\newcommand\keuro{k{\sffamily C%
\makebox[0pt][l]{\kern-.70em\mbox{--}}%
\makebox[0pt][l]{\kern-.68em\raisebox{.25ex}{--}}}}

%%%% COLOR DEFINITIONS %%%%%%%%%%%%%%%%%%%%%%%%%%%%%%%%%%%%
\definecolor{blue} {rgb} {0.25,0.25,0.75}

%%%% ADDITONAL EMPHASIS %%%%%%%%%%%%%%%%%%%%%%%%%%%%%%%%%%%
\newcommand{\cem}{\color{blue}}
\newcommand{\eem}{\sl\color{blue}}

%%%% SET THE COLOR OF THE (SUB-) SECTION TITLES %%%%%%%%%%% 
\newcommand{\Tcol}{\color{blue}}

%%%% SET THE COLOR OF THE TITLE BOX BACKGROUND %%%%%%%%%%%%
\definecolor{Background}{rgb} {0.62,0.75,0.5}

%%%% REFERENCE SECTION NAME %%%%%%%%%%%%%%%%%%%%%%%%%%%%%%%
\renewcommand\refname{\Tcol 9. Bibliography}

%%%% COLOR THE SECTION NUMBERS %%%%%%%%%%%%%%%%%%%%%%%%%%%%%%%
\makeatletter
\renewcommand\@seccntformat[1]{\color{blue} {\csname the#1\endcsname}\hspace{0.5em}}
\makeatother
\renewcommand\thesection{\arabic{section}.}
\renewcommand\thesubsection{\arabic{section}.\arabic{subsection}}

%%%% CHANGE THE APPEARANCE OF THE \PARAGRAPH COMMAND  %%%%%%%%%%%%%%%%%%%%%%%%%%%%%%%
\makeatletter
\renewcommand\paragraph{\@startsection{paragraph}{4}{\z@}%
            {-2.5ex\@plus -1ex \@minus -.25ex}%
            {1.25ex \@plus .25ex}%
            {\normalfont\normalsize\bfseries}}
\makeatother
\setcounter{secnumdepth}{4}     % how many sectioning levels to assign numbers to
\setcounter{tocdepth}{4}        % how many sectioning levels to show in ToC


\fancyhead[LE,LO]{\slshape
%%%%  Please edit
%
Ercolano: RU Transition Discs Project B1}
%
%
%%%%%


\begin{document}


\newpage

%%%% PROJECT DESCRIPTION STARTS HERE %%%%%%%%%%%%%%%%%%%%%%%%%%%%%%%%%%%

\setcounter{page}{1}

\centerline{\huge\bf\Tcol
%
%
%
%
%%%%  Please edit
%
 Project B1:}

\centerline{\huge\bf\Tcol Disc mass loss from quantitative spectroscopy
of photoevaporative winds}

%
%%%%
%
%
%
%
\vskip1.0cm

%%%%  Please edit
\noindent{\bf Authors:}\\
\begin{tabular}{ll}
{\textsf{PI:}}                  & B.~Ercolano (LMU)\\
{\textsf{Co-I:}}                &P.~Caselli (MPE), K.~Dullemond (Heidelberg), Kley (T\"ubingen)\\
{\textsf{Collaborations:}}      & T.~Preibisch (LMU), L.~Testi (ESO), James Owen (Princeton), \\
& E. van Dishoeck (Leiden, MPE), T. Henning (MPIA)  \\
\end{tabular}

%%%%  Please edit

\vspace{1em}
\noindent{\bf Requested positions: 1 Postdoc} \\

\vspace{1em}
\noindent{\bf Abstract:}\\

\noindent{\bf Abstract:}\\
Type 1 TDs are likely discs in an advanced stage of dispersal. The
dispersal mechanism of discs is of fundamental importance to planet
formation, yet the responsible mechanism is still largely
unconstrained. Photoevaporation from the central star is currently a
promising avenue to investigate. We aim at building the most
up-to-date radiation-hydrodynamical calculations of irradiated discs
coupled to photoionisation, chemistry and radiative transfer
calculations to allow us for the first time to perform quantitative
spectroscopy of disc winds. Comparison with existing observations will
allow us to constrain mass loss rates and emission regions of the wind
which will pin down the underlying driving disc dispersal mechanism.


\section{\Tcol State of the art and preliminary work}
\renewcommand{\leftmark}{\sc State of the Art and preliminary work}

\subsection{Scientific Background}

Understanding disc dispersal is a key piece in the puzzle of planet
formation. Type 1 TDs, which are considered to be objects on the
verge of dispersal provide a tight constrain on the underlaying
dispersal mechanism. One of the favourite models to drive disc
dispersal is photoevaporation by radiation from the central star
(e.g. Clarke et al. 2001, Alexander et al. 2006). The exact nature of
the driving radiation is however still open to debate.  (Extreme and
Far) Ultraviolet (UV) radiation as well as X-ray have been shown to be
able to drive winds from the disc upper layers (Alexander et al. 2006;
Gorti, Hollenbach \& Dullemond 2009; Ercolano et al. 2009; Owen et
al. 2010) able to disperse the discs in the observed
timescales. However both the location and intensity of the wind depend
strongly on the driving radiation, with differences of more than two
orders of magnitude for mass loss rates predicted by different
models. This has profound implications for disc evolution and hence
for the formation of planets and their subsequent evolution
(e.g. Ercolano \& Rosotti 2015). 

While the presence of disc winds has been confirmed via the
observation of a few km/sec blue-shift in the line profiles of a
number of tracers like [NeII]~12.8$\mu$m and [OI] 6300 (e.g. Pascucci
et al 2007), these lines cannot be used to infer the underlying
mass-loss-rate (e.g. Ercolano \& Owen 2010, Ercolano \& Owen 2016).
 For example, the intensity and the profile of the [NeII]~12.8$\mu$m
 can be equally well fitted using an EUV (Alexander ???) or an X-ray
 photoevaporation model (Ercolano \& Owen 2010). The problem with the
 [NeII] line is that the Ne+ formation route can occur both via the
 removal of a valence electron in the fully-ionised winds driven by
 EUV radiation, but also by charge exchange of Ne++ with neutral H
 which is abundant in the quasi-neutral winds driven by X-ray. 
The problem with the [OI] 6300 line and all other ionic collisionally
excited lines considered to date is the strong temperature dependence
imposed by the Boltzmann term in the emissivity. This means that these
lines are mostly just tracing the hot layer of the wind heated by the
EUV radiation and not actually tracing the bulk of the wind where it
matters (Ercolano \& Owen 2016), hence they
cannot be used to infer mass-loss-rates or to constrain the wind
driving mechanism.   

Mid-infrared observations of molecular lines (e.g. CO) provide a new
promising alternative to directly measure disc winds. Indeed recent
observations suggest that these lines may be tracing a disc wind which
is slow and partially molecular (e.g. Pontoppidan et al. 2011; Brown et al. 2013). 

The spectro-astrometric survey of molecular gas in the inner regions of
protoplanetary discs using CRIRES, the high-resolution infrared
imaging spectrometer on the Very Large Telescope (Pontoppidan et
al. 2011), showed that for several sources the astrometric signatures
are dominated by gas with strong non-Keplerian (radial) motions. These
authors concluded that the non-Keplerian spectro-astrometric
signatures are likely indicative of the presence of wide-angle disc
winds. 
More observations of this type are planned after the update of
the CRIRES instrument, which is expected to be completed by
2019. Observations with ALMA in molecular lines like e.g. CO J = 2-1
and J = 3-2 emission are also able to trace the presence of a wind (e.g.Klaassen et al. 2013, 2016).  
Molecular lines are sensitive to the mass loss rates since they
sample a significant area of the wind launching regions. However the
exploitation of molecular tracers is currently severely hampered by
the lack of a suitable hydrodynamic wind model coupled to chemistry
and to dust evolution models (which dominate the opacity in the wind)
to interpret the observations.

 While a number of chemical models exist
of the deeper, denser regions of discs, no model is currently
available for the optically thinner disc winds. The work of Gorti \&
Hollenbach (2009), while carrying out detailed chemical calculations
extending to the disc atmosphere, used a hydrostatic disc model which
was analysed in a 1+1D fashion. Without hydrodynamics no predictions
on line profiles can be made.  

{\color{red} 

here add something about why all models lack elements -

  also perhaps mention something about the work on single streamlines
  for magnetic outflows  -- ewine --
}

\subsection{Preliminary work}
We have performed the only existing radiation hydrodynamic
calculations of X-ray driven photoevaporative winds to date (Owen et
al. 2010, 2011, 2012). We have used these grids to make predictions on
the ionised phase of the wind spectra (Ercolano \& Owen 2010;
Ercolano \& Owen 2016), however the parameter space
available to date is very limited. 

In this project we will
significantly expand on this by constructing a
library of X-ray wind solutions for an extended grid of
X-ray luminosities and stellar masses, covering all observed
values. 
Our previous calculations could only account for
the ionised phase of the wind, hence restricting severely predictions
of interesting line diagnostic. 
We will then lift this limitation by
perform the first simultaneous chemical calculations in
the wind and upper disc atmosphere.  


\subsection{\Tcol Project-related publications}

% Please list your own publications related to the proposed project, 
% adhering to the rules of the DFG guidelines 1.91. In brief, please note: 
% - Up to 10 publications
% - The work must be published or accepted.
% - Publications on astro-ph (arXive, SPIRES or articles with a DOI) count as published. 
% - Any work that is only in the status ``accepted'' MUST be attached to the proposal
%    together with the acceptance letter.
% - All publications in this section CAN be attached to the proposal. Please limit these
%    attachments to a minimum and please note that the reviewers may not read the attachments -
%    the proposal has to speak for itself.
% - The number of allowed publications refers to the sum of the publications listed
%    in ``1.1.1 Articles published or officially accepted by publication outlets...'' and 
%    in ``1.1.2 Other publications''. Publications which only exist on public repositories 
%    belong into the category ``Other Publications''.
[Text]

\subsubsection{\Tcol 
Articles published or officially accepted by publication outlets with scientific quality assurance;
book publications}

[Text]

\subsubsection{\Tcol Other publications}

[Text]

\subsubsection{\Tcol Patents}

\paragraph{\Tcol Pending}

[Text]

\paragraph{\Tcol Issued}

[Text]

\section{\Tcol Objectives and work programme}
\renewcommand{\leftmark}{\sc Objectives and work programme}


\subsection{\Tcol Anticipated total duration of the project}

[Text]

\subsection{\Tcol Objectives}


The main aim of this project is together with project B2 to identify
new wind tracers and use them to constrain mass loss rates and hence
disc dispersal models.

{\color{red} Furthermore an intermediate objective test the X-ray 
photoevaporation theory by performing a higher parameter space grid blahblah}

{\color{red} is this really true? or should this be the aim of B2??
  Perhaps look at how this was described in the ERC program?}

Perform a quantitative spectroscopic comparison between TDs and
primordial disc to provide important constraints on the wind architecture. 
Type 1 TDs, are particularly interesting as the streamline architecture of their winds
and the profiles of the lines that are produced in the wind
differ from those of primordial discs. (e.g. Ercolano \& Owen
2010). Indeed the lines are expected to be broader and brighter for
e.g. inner cavities of a few to 10 AUs. 

Test whether Type 2 TDs (large holes, large accretion rates) may
result from radiative transfer effects on a tilted inner disc. This
idea stems from  the recent suggestion (e.g. Marino et al. 2015, Montesino et
al. 2016) that some Type 2 TDs may have a tilted inner disc. 
A tilted inner disc may strongly influence photoevaporation by
allowing radiation to reach outer disc regions and may produce the
large inner holes of (some) Type 2 TDs. This is certainly a worthwhile
new challenge requiring the development of 3D simulations.  

{\color{red} the marino and montesino idea must also be included in
  the intro and also in the background section of this proposal}

\subsection{\Tcol Work programme including proposed research methods}
% for each applicant

\subsection{Research Tools}

For this project we will need the following tools: 
\begin{enumerate}
\item A 3D hydrodynamical code which we will modify to include the
  effects of X-ray irradiation as we did in Owen et al. (2010). For
  that we will use the Pluto code, for which extensive expertise
  exists in our team. 
\item A 3D photoionisation and chemical code to post-process the wind solutions obtained above. The PI is the author of the MOCASSIN code (Ercolano et al. 2003, 2005, 2008b), which has already been used to calculate the emission line spectra from the ionic phase of X-ray winds (Ercolano \& Owen 2010; Ercolano, Owen \& Testi 2016, in prep). The code has now been coupled and benchmarked to the KROME code to perform arbitrary chemical calculations (Ercolano \& Grassi 2016, in prep)  and needs now only the appropriate reduced chemical network, which we will obtain from project B2. 
\item A 3D radiative transfer code to post-post-process the hydrodynamical grids from step 1, with the appropriate temperatures and abundances obtained from step 2 to produce emission line intensities and profiles to compare with the existing observations and those gathered and reprocessed in project A1. We will make use of the RadMC code developed and maintained by Prof. Dullemond.  
\end{enumerate}

\subsection{Research Plan} 

We have divided the work load into two connected blocks which
also have self-contained immediate objectives. Block 1 is already
being executed by Dr Picogna, employed on a LMUExcellent initiative
grant awarded to B. Ercolano in support of this project (End date
November 2017). 

{\color{red} quickly describe the blocks}

\subsubsection{Block 1: Parameter-space investigations of X-ray
  photoevaporation models.}

The 3D hydrodynamical code PLUTO is being modified to include the
effects of X-ray irradiation (Owen et al. 2010) in order to produce a
library of X-ray wind solutions that will be analysed in Block 2
to produce emission line and continuum spectra of the
wind. The obtained solutions will be first of all benchmarked against
those that are already available for a 0.7 and 0.1 M$_\odot$ central
stars (Owen et al. 2010, 2011, 2012). The parameter space will be then
significantly extended for the mass of the central star and its X-ray
luminosity. 
With the new models we will also test the theoretical relations for X-ray photoevaporation
predicted by means of semi-analytical models and ab-initio arguments
by Owen et al. (2012). While these relations are being widely used in
the literature, they have until now never been tested.\\
As a further step we plan to perform a small set of 3D simulations to explore the effects of asymmetries in the inner disc. We expect to see dramatic effects in the photoevaporation profile and in the wind architecture, which may lead to the formation of large hole TDs. This avenue is never been explored before.\\

\subsubsection{Block 2: Spectral line energy distribution calculations of disc winds.}
The MOCASSIN code (Ercolano et al. 2003, 2005, 2008b), which has
already been used to calculate the emission line spectra from the
ionic phase of X-ray winds (Ercolano \& Owen 2010; Ercolano \& Owen
2016). The code has now been coupled and benchmarked to 
the KROME code (Grassi et al., 2014) to perform arbitrary chemical calculations (Ercolano \&
Grassi 2016, in prep)  and needs now only the appropriate reduced
chemical network, which we will obtain from B2. 

The new KROME-coupled MOCASSIN code will be employed to perform
photoionisation and chemical calculations disc wind solutions starting
off from the available models of Owen et al. (2010, 2011, 2012) and
then moving to the new data obtained in Block 1 (which is 
already being executed). To this aim, under the guidance of Prof Caselli, an initially very
simple network will be included in the MOCASSIN-KROME to perform
initial test calculations, which will then be updated when project B2
begins to provide results. As a final step in the post-processing with
the help of Prof Dullemond 
we will perform 3D radiative
transfer calculations to produce line intensities and profiles to
compare with existing and new observations, which may become available
at the time. 

{\color{red} all of the above needs to be reviewed - see also if some
  of it should go into B2 - and also give approximate timelines}


\subsection{\Tcol Data handling}

A library of models with corresponding profiles for interesting
emission lines will be made available online on the public partition
of the Research Unit server. This will help the community with the
interpretation of upcoming observational datasets. 

\subsection{\Tcol Other information}
% Please use this section for any additional information you feel is
% relevant which has not been provided elsewhere.

Not Relevant

\subsection{\Tcol Information on scientific and financial involvement of international cooperation partners}

Not Relevant

\section{\Tcol Bibliography}

[Text]

\section{\Tcol Requested modules/funds}
\renewcommand{\leftmark}{\sc  Requested modules/funds}
% Explain each item for each applicant (stating last name, first name).

\subsection{\Tcol Basic Module}

\subsubsection{\Tcol Funding for Staff}
% Please note that funds for your own temporary position (“Eigene Stelleâ€)
% are not to be included here; this belongs to the separate “Module Temporary Positionâ€.

We require funding for one Postdoc to work at the LMU in the group of Prof. Ercolano.

\subsubsection{\Tcol Direct Project Costs}


\paragraph{\Tcol Equipment up to EUR 10,000, Software and Consumables}

Will be provided by the host institution

\paragraph{\Tcol Travel Expenses}

Total: 9900 Euro Justification : Each year one national trip (meeting of Astronomical Society, national
meetings) and one international trip (conference, visit
collaborators). 
During the course of the PhD 2 one week long visits to our main
international collaborator, Dr J. Owen (currently at Princeton
University, will move to Imperial College London in 2017). 

Cost estimate: 
\begin{itemize}
\item National trip: 5 overnight stays, train/airfare,
conference fee; 1000 Euro (3000 over 3 years).
\item International trip: 6 overnight stays, airfare, conference fee;
  1500 Euro (4500 over 3 years).
\item Visit to/from J. Oweni: airfare, 6 overnight stay 1200 Euro (2400
  for 2 visits)
\end{itemize}

\paragraph{\Tcol Visiting Researchers (excluding Mercator Fellows)}

Not Relevant 

\paragraph{\Tcol Other Costs}

None

\paragraph{\Tcol Project-related publication expenses}

We request 770 Euro py (total 2250 Euro) for publication expenses.

\subsubsection{\Tcol Instrumentation}

None 

\paragraph{\Tcol Equipment exceeding EUR 10,000} 

None

\paragraph{\Tcol Major Instrumentation exceeding EUR 100,000} 

None 

\subsection{\Tcol Module Temporary Position}

Not Relevant

\subsection{\Tcol Module Replacement Funding}

Not Relevant

\subsection{\Tcol Module Mercator Fellows}

Not Relevant

\subsection{\Tcol Module Public Relations Funding}

Not Relevant

\section{\Tcol Project requirements}
\renewcommand{\leftmark}{\sc Project requirements}

\subsection{\Tcol Employment status information}
% For each applicant, state the last name, first name, and employment
% status (including duration of contract and funding body, if on a
% fixed-term contract).

Barbara Ercolano, Professor at the Ludwig-Maximilians-Universit\"at
M\"unchen  (permanent)

\subsection{\Tcol First-time proposal data}
% Only if applicable: Last name, first name of first-time applicant.

Not Relevant

\subsection{\Tcol Composition of the project group}
% List only those individuals who will work on the project but will not
% be paid out of the project funds. State each person’s name, academic
% title, employment status, and type of funding.

[Text]

\subsection{\Tcol Cooperation with other researchers}

\subsubsection{\Tcol Planned cooperation on this project}

\paragraph{\Tcol Collaborating researchers for this project within the
  Research Unit}
%Each proposal must be accompanied by a description of how the project
%is integral to the Research Unit, %both in terms of subject matter
%and organisation. This includes a description of the cooperation with
%%others participating within the Research Unit. 

This project depends on project B2 for the reduced network and on
project A2 for the observational input. Furthermore Dr
James Owen (Princeton) will be heavily involved in the project.  

{\color{red} expand}


\paragraph{\Tcol Collaborating researchers for this project outside of
  the Research Unit}

Dr. James Owen, currently at Princeton University, from 2017 at
Imperial College London. 


{\color{red} Expand above - mention James' contributions to the state
  of the art and his likely contribution to the project - mention visits}

\subsubsection{\Tcol Researchers with whom you have collaborated scientifically within the past three years}
% This information is important for DFG to exclude possible conflicts of interest.
% Please mention not only the names of the cooperation partners but also their institution and city.
% Scientists already mentioned in the previous two subsubsections do not have to be mentioned
% again.


F. Niederhofer (STSci, USA); M. Hilker (ESO, Garching); N. Bastian (U. Liverpool,
UK); M. Guarcello (U. Palermo, Italy); M. Tazzari (U. Cambridge, UK);
A. Natta (Florence, Italy); R. Alexander (U. Leicester); D. Hubber
(LMU); J. Dale (U. Hertfordshire, UK); C. Koepferl (LMU); I. Bonnell
(U. St. Andrews, UK); A. McLeod (ESO, Garching); D. Boneberg
(U. Cambridge, UK); R. Parker (U. Liverpool, UK); R. Wesson (UCL,
London, UK); M. Barlow (UCL, London, UK); A. Glassgold (u. Berkeley,
USA); C. Manara (ESA, Noordwjik, Netherlands); A. Danekhar (CfA,
Harward, USA); Q. Parker (Sidney, Australia); S. Casassus
(U. de Chile, Santiago, Chile); I. Pascucci (U. Arizona, USA);
A. Bevan (UCL, London, UK).

\subsection{\Tcol Scientific equipment}
% List larger instruments that will be available to you for the
% project. These may include large computer facilities if computing
% capacity will be needed. 

The group of Prof. Ercolano has two own computer clusters comprising 

\begin{itemize}
\item 2 CPU Intel Xeon X5650 (Westmere, beginning
2010, 2.66 GHz) 6 cores each 12 cores total (24 virtual) 74 GB ram.

\item 4 CPU Intel Xeon E7-4850 (Ivy Bridge, beginning 2014, 2.30 GHz)
12 cores each 48 cores total (96 virtual) 660 GB ram.

\end{itemize}

Further computational power is provided through the C2PAP facility of the Excellence Cluster to which
the group has guaranteed time. This comprises 126 nodes, each node with 2 CPU Intel Xeon E5-2680 (Sandy
Bridge, beginning 2012, 2.7 GHz) 8 cores each 16 cores total (32
virtual) 64 GB ram. Note that while the future of the Excellence
Cluster Universe is uncertain, the C2PAP facilities will be in any
case supported by the LMU. 

The Leibniz Rechnung Zentrum (LRZ) is also available to us, where still
larger facilities are available with somewhat more constrained and longer queues.

\subsection{\Tcol Project-relevant interests in commercial enterprises}
% Information on connections between the project and the production
% branch of the enterprise.

Not Relevant


\subsection{\Tcol Additional information}
% If applicable, please list proposals requesting major
% instrumentation and/or those previously submitted to a third party
% here.

Not Relevant


\end{document}
