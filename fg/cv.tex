%--------------------------------------------------------------------
%                      CURRICULUM VITAE
%--------------------------------------------------------------------
%
% Headers and footers for the Curricilum Vitae part
%
\makeatletter
%\newcommand{\cvhdr}{Curricula Vitae}
\newcommand{\cvhdr}{XXXX}
%\lhead[{\bfseries\cvhdr{}}]{}
%\rhead[]{{\bfseries\cvhdr{}}}
%\lhead[{\bfseries Curriculum Vitae \cvhdr{}}]{}
%\rhead[]{{\bfseries Curriculum Vitae \cvhdr{}}}
\lhead[{\bfseries CV -- \cvhdr{}}]{}
\rhead[]{{\bfseries CV -- \cvhdr{}}}
\lfoot[\thepage]{}
\rfoot[]{\thepage}
\cfoot[]{}
\setlength{\headrulewidth}{0pt}
\makeatother
%
%--------------------------------------------------------------------
%
\renewcommand{\cvhdr}{Altherr}
\section*{R.\ Altherr}
\subsection*{Curriculum Vitae}
\begin{cvlayout}
\item[Name:\hfill]
    ALTHERR, Rainer
\item[Address:\hfill]
    Mineralogisches Institut\\
    Universit\"at Heidelberg\\
    Im Neuenheimer Feld 236\\
    69120 Heidelberg\\
    Germany\\
    Phone / Fax: +49-6221-548206/8207\\
    E-mail: raltherr@min.uni-heidelberg.de
\item[Date and Place of Birth:\hfill]
    August 6, 1947, Freiburg, Germany
\item[Nationality:\hfill]
    German
\item[Present position:\hfill]
    University Professor, Director
\item[Background:\hfill]
    Studies in Mathematics and Physics at University of Freiburg (1967-1969)\\
    Studies in Mineralogy and Geology at University of Freiburg (1969-1973; Diploma in Mineralogy)\\
    Dr. rer. nat thesis in Mineralogy (1975), Univ. of Freiburg\\
    Habilitation (1982), Faculty of Sciences, University of Braunschweig
\item[Positions:\hfill]
    University of Freiburg, Germany (1974-1975)\\
    University of Clausthal, Germany (1975-1976)\\
    University of Braunschweig, Germany (1976-1979; 1980-1982)\\
    The University of Chicago, U.S.A. (1979-1980)\\
    University of Karlsruhe (1982-1994): Professor\\
    University of Heidelberg (1994-)
\item[Awards/Fellowships:\hfill]
    G\"odeke Forschungspreis (1974)\\
    Rinne Forschungspreis (1975)\\
    Albert-Maucher-Preis f\"ur Geowissenschaften (DFG) (1981)\\
    NATO fellowship (DAAD) (1979)\\
    Ordinary Member of Heidelberg Academy of Sciences (2003-)\\
    Corresponding Member of Croation Academy of Sciences (2004-)
\end{cvlayout}

\subsection*{Research interests and achievements}
\subsubsection*{Petrology and geochemistry of rocks from the Earth's
    mantle:} 
Mantle xenoliths from various rift systems (French Massif
Central, East Africa, Red Sea/Arabia) and orogenic peridotites from
collisional belts (European Variscides, Alps, Caledonian Orogen in Norway)
have been investigated for their chemical and pressure-temperature
evolution.
\subsubsection*{Geochemical cycles of Li, Be and B:}
The behaviour of these elements and their isotopes during igneous and
metamorphic processes has been investigated.
\subsubsection*{Chemical zoning patterns of mineral grains as archives of
  geological processes:}
The internal chemical structure of a mineral grain is due to (i) changes in
the intensive state variables during its growth and (ii) diffusional
relaxation after growth. Therefore, major and trace element zonations of
mineral grains have been used to deduce the chemical and
pressure-temperature evolution of magmatic and rock systems.
\subsubsection*{Genesis of granitoid rocks and chemical differentiation of
  the continental crust:} 
The formation of granitoid magmas and rocks is the
major process by which the continental crust is internally
differentiated. Numerous granitoid complexes have been investigated and
genetic models have been developed.
\subsubsection*{Genesis of meteorites:} 
Since about 3 years, the applicant has started to work on certain aspects of
chrondrule formation (Li, Be, B and other trace elements).


\subsection*{List of refereed publications in past 5 years}
\begin{ownpubl}

\item
Trieloff, M. and Altherr, R. (2006) He-Ne-Ar isotope systematics in  
Eifel and Pannonian basin mantle xenoliths reveal deep mantle plume- 
lithosphere interaction beneath the European continent. In: Ritter,  
J., Achauer, U. (eds.) Mantle Plumes, Springer, \textbf{in press}

\item
Weinstein, Y., Navon, O., Altherr, R. and Stein, M. (2006) The role  
of lithospheric mantle heterogeneity in the generation of alkali  
basalt suites from NW Harrat Ash Shaam (Israel). Journal of  
Petrology, \textbf{in press}

\item
Marschall, H.~R., Ludwig, T., Altherr, R., Kalt, A. and Tonarini, S.  
(2006) Syros metasomatic tourmaline: evidence for very high d11B  
fluids in subduction zones. Journal of Petrology, \textbf{in press}

\item
Schwarz, W.~H., Trieloff, M. and Altherr, R. (2005) Constraints on  
solar type
noble gases abundances in deep sea sediments and their subducted  
equivalents. \cmp, \textbf{149}, 675-684

\item
Topuz, G., Altherr, R., Schwarz, W.~H., Siebel, W., Satir, M.,
Dokuz, A. (2005) Post-collisional plutonism with adakite-like signatures: the  
Eocene
Saraycik granodiorite (Eastern Pontides, Turkey). \cmp, \textbf{150},  
441-455

\item
Marschall, H.~R., Kasztovszky, Z., Gm\'eling, K. and Altherr, R.  
(2005) Chemical
analysis of high-pressure metamorphic rocks by PGNAA: Comparison with  
results
from XRF and solution ICP-MS. \textit{Journal of Radioanalytical and  
Nuclear
Chemistry\/}, \textbf{265}, 339-348

\item
Trieloff, M., Falter, M., Buikin, A.~I., Korochantseva, E.~V., Jessberger,
E.~K., Altherr, R. (2005) Argon isotope fractionation induced by stepwise
heating.  \gca, \textbf{69}, 1253-1264

\item
Buikin, A.~I., Trieloff, M., Hopp, J., Althaus, T., Korochantseva,  
E.~V., Schwarz, W.~H., Altherr, R.
(2005) Noble gas isotopes suggest deep mantle
plume source of late Cenozoic mafic alkaline volcanism in Europe.
\epsl, \textbf{230}, 143-162

\item
Hopp, J., Trieloff, M. and Altherr, R. (2004)  Neon isotopes in  
mantle rocks from
the Red Sea region reveal large-scale plume�"lithosphere interaction.
\epsl, \textbf{219}, 61-76

\item
Altherr, R., Topuz, G., Marschall, H., Zack, T. and Ludwig, T. (2004)  
Evolution
of a tourmaline-bearing lawsonite eclogite from the Elekdag area  
(Central
Pontides, N Turkey): evidence for infiltration of slab-derived B-rich  
fluids
during exhumation. \cmp, \textbf{148}, 409-425

\item
Altherr, R., Meyer, H.-P., Holl, A., Volker, F., Alibert, C., McCulloch, 
M.~T., Majer, V.
(2004) Geochemical and Sr-Nd-Pb isotopic characteristics of Late
Cenozoic leucite lamproites from the East European Alpine belt  
(Macedonia and
Yugoslavia). \cmp, \textbf{147}, 58-73

\item
Paquin, J., Altherr, R. and Ludwig, T. (2004) Li-Be-B systematics in the
ultrahigh-pressure garnet peridotite from Alpe Arami (Central Swiss  
Alps):
implications for slab-to-mantle transfer. \epsl, \textbf{218}, 507-519

\item
St\"ahle, V., Altherr, R., Koch, M. and Nasdala, L. (2004) Shock-induced
formation of kyanite (Al$_2$SiO$_5$) from sillimanite within a dense  
metamorphic
rock from the Ries crater (Germany). \cmp, \textbf{148}, 150-159

\item
Topuz, G. and Altherr, R. (2004) Pervasive rehydration of high-grade  
gneisses
during exhumation: an example from the Pulur complex, Eastern  
Pontides, Turkey.
\textit{Mineralogy and Petrology\/}, \textbf{81}, 165-185

\item
Topuz, G., Altherr, R., Kalt, A., Satir, M., Werner, O., Schwarz, 
W.~H. (2004)
Hercynian metapelitic gneisses from the Pulur complex, NE Turkey: a  
case for
extensive melting and efficient melt extraction at mid- to lower- 
crustal levels.
\textit{Lithos\/}, \textbf{72}, 183-207

\item
Topuz, G., Altherr, R., Satir, M. and Schwarz, W.~H. (2004) Low-grade  
metamorphic
rocks from the Pulur complex, NE Turkey: implications for the pre- 
Liassic
evolution of the Eastern Pontides. \textit{Int. J. Earth Sci.},  
\textbf{93},
72-91

\item
Olker, B., Altherr, R. and Paquin, J. (2003) Fast exhumation of the
ultrahigh-pressure Alpe Arami garnet peridotite (Central Alps,  
Switzerland):
constraints from geospeedometry and thermal modelling. \textit{J.  
Metam. Geol.},
\textbf{21}, 395-402

\item
Witt-Eickschen, G., Seck, H.~A., Mezger, K., Eggins, S.~M. and  
Altherr, R. (2003)
Lithospheric mantle evolution beneath the Eifel (Germany):  
constraints from
Sr-Nd-Pb isotopes and trace element abundances in spinel peridotite and
pyroxenite xenoliths. \jp, \textbf{44}, 1077-1095

\item
Altherr, R. and Siebel, W. (2002) I-type plutonism in a continental  
back-arc
setting: Miocene granitoids and monzonites from the central Aegean  
Sea, Greece.
\cmp, \textbf{143}, 397-415

\item
Paquin, J. and Altherr, R. (2002) Subduction-related lithium  
metasomatism during
exhumation of the Alpe Arami ultrahigh-pressure garnet peridotite  
(Central Alps,
Switzerland). \cmp, \textbf{143}, 623-640

\item
Woodland, A.~B., Seitz, H.-M., Altherr, R., Marschall, H., Olker, B.,  
Ludwig, T.
(2002) Li abundances in eclogite minerals: a clue to a crustal or
mantle origin? \cmp, \textbf{143}, 587-601

\item
Woodland, A.~B., Seitz, H.-M., Altherr, R., Marschall, H., Olker, B., 
Ludwig, T.
(2002) Li abundances in eclogite minerals: a clue to a crustal or mantle
origin? Erratum. \cmp, \textbf{144}, 128-129

\item
Embey-Isztin, A., Dobosi, G., Altherr, R. and Meyer, H.-P. (2001)  
Thermal
evolution of the lithosphere beneath the western Pannonian Basin:  
evidence from
deep-seated xenoliths. \textit{Tectonophysics\/}, \textbf{331}, 285-306

\item
Paquin, J. and Altherr, R. (2001) New constraints on the P-T  
evolution of the
Alpe Arami garnet peritotide body (Central Alps, Switzerland). \jp,  
\textbf{42},
1119-1140

\item
Paquin, J. and Altherr, R. (2001) 'New Constraints on the P-T  
Evolution of the
Alpe Arami Garnet Peridotite Body (Central Alps, Switzerland)': Reply  
to Comment
by Nimis \& Trommsdorff (2001). \jp, \textbf{42}, 1781-1787

\end{ownpubl}
\cleardoublepage
%
%--------------------------------------------------------------------
%
\renewcommand{\cvhdr}{Blum}
\section*{J.\ Blum}
\subsection*{Curriculum Vitae}
\begin{cvlayout}
\item[Name:\hfill]
    BLUM, J\"urgen
\item[Address:\hfill]
    Institute for Geophysics and Extraterrestrial Physics, TU Braunschweig, Mendelssohnstr. 3\\
    38106 Braunschweig\\
    Phone / Fax: 0531-3915217 / 0531-3918126\\
    E-mail: j.blum@tu-bs.de
\item[Date and Place of Birth:\hfill]
    April 5, 1962, Gedern, Germany
\item[Nationality:\hfill]
    German
\item[Present position:\hfill]
    University Professor
\item[Background:\hfill]
    Studies in Physics at the
    University of Gie{\ss}en (1981-1983)\\
    Studies in Physics and Astronomy at the
    University of Heidelberg (1983-1987)\\
    Diploma in Physics (1987), MPI f\"ur Astronomie and University of Heidelberg\\
    Dissertation in Physics (Ph.D.) (1990), MPI f\"ur Kernphysik and University of Heidelberg\\
    Habilitation (1999), Faculty of Physics and Astronomy, Univ.\
    Jena
\item[Positions:\hfill]
    MPI f\"ur Kernphysik and TU M\"unchen (1987-1992)\\
    Max Planck Research Group ``Dust in Star-Forming Regions'' Jena (1992-1996)\\
    Astrophysical Institute, University of Jena (1997-1999)\\
    Department of Astronomy, University of Florida and Naval
    Research Laboratory, Washington, D.C. (1999-2000)\\
    Astrophysical Institute, University of Jena (2000-2003)\\
    Institute for Geophysics and Extraterrestrial Physics, TU
    Braunschweig (since 2003)
\item[Awards/Fellowships:\hfill]
    Habilitation Award, University of Jena (2000)
\end{cvlayout}

\subsection*{Research interests and achievements}
\subsubsection*{Laboratory Astrophysics}
Simulation of cosmic processes in the laboratory; synthesis of
analog materials (nano- and micro-particles); ice experiments.
\subsubsection*{Cosmic Dust}
Morphological, collisional, optical, electrostatic, and
aerodynamic properties of cosmic dust; characterization of cosmic
dust by in-situ measurements; dust sampling in the higher
atmosphere (stratosphere and mesosphere); capturing and sampling
of cosmic dust in space experiments.
\subsubsection*{Formation of Planetary Systems}
Intitiation of dust growth in young planetary systems; properties
of protoplanetary dust aggregates; properties of macroscopic
bodies in young planetary systems; properties of planetesimals and
cometesimals.
\subsubsection*{Small Solar-System Bodies}
Experimental simulation of regolith formation; optical and
mechanical properties of regolith; densities and porosities of
primitive solar-system bodies.
\subsubsection*{Microgravity Experiments}
Experiments on dust aggregation, regolith formation, Brownian
motion and rotation, granular matter, photophoresis,
thermophoresis, particle traps under microgravity conditions; use
of drop tower, parabolic flights, sounding rockets, space shuttle,
and ISS.

\subsection*{List of refereed publications in past 5 years}
\begin{ownpubl}

\item Blum, J. (2006) Dust Agglomeration. \textit{Advances in
Physics} (submitted)

\item Dominik, C., Blum, J., Cuzzi, J.~N. and Wurm, G. (2006)
Growth of Dust as Initial Step Toward Planet Formation, in
\textit{Protostars \& Planets V},  University of Arizona Press
(Eds. B. Reipurth, D. Jewitt and K. Keil) (in press)

\item Rei{\ss}aus, P., Waldemarsson, T., Blum, J., Cl\'ement,
D., Llamas, I., Mutschke, H. and Giovane, F. (2006) Sticking
Efficiency of Nanoparticles in High-Velocity Collisions with
Various Target Materials. \textit{Journal of Nanoparticle
Research} (in press)

\item Wurm, G. and Blum J. (2006) Experiments on Planetesimal
Formation, in: \textit{Planet Formation}, Cambridge University
Press (Eds. H. Klahr and W. Brandner), 90-111

\item Tamanai, A., Mutschke, H., Blum, J. and Neuh�user, R. (2006)
Experimental Infrared Spectroscopic Measurement of Light
Extinction for Agglomerate Dust Grains. \textit{Journal of
Quantitative Spectroscopy and Radiative Transfer}, \textbf{100},
373-381

\item Heim, L.~O., Butt, H.-J., Schr\"apler, R. and Blum, J.
(2005) Analysing the Compaction of High-Porosity Microscopic
Agglomerates. \textit{Australian Journal of Chemistry},
\textbf{58}, 671-673

\item Blum, J. and Schr\"apler, R. (2004) Structure and Mechanical
Properties of High-Porosity Macroscopic Agglomerates Formed by
Random Ballistic Deposition. \prl, \textbf{93}, 115-503

\item Blum, J. (2004) Grain Growth and Coagulation, in:
\textit{Astrophysics of Dust}, ASP Conference Series, Vol. 309
(Eds. A. Witt, G. Clayton and B. Draine), 369-391

\item Krause, M. and Blum J., (2004) Growth and Form of Planetary
Seedlings: Results from a Sounding Rocket Microgravity Aggregation
Experiment. \prl, \textbf{93}, 021103

\item Steinbach, J., Blum, J. and Krause, M. (2004) Development of
an Optical Trap for Microparticle Clouds in Dilute Gases.
\textit{Eur. Phys. J. E}, \textbf{15}, 287-291

\item Ehrenfreund, P., Fraser, H.~J., Blum, J., Cartwright,
J.~H.~E., Garc\'\i�a-Ruiz, J.~M., Hadamcik, E., Levasseur-Regourd, A.~C.,
Price, S., Prodi, F., Sarkissian, A. (2003) Physics and chemistry
of icy particles in the universe: answers from microgravity. \pss,
\textbf{51}, 473

\item Blum, J., Giovane, F., Tuzzolino, A.~J., McKibben, R.~B. and
Corsaro, R (2003) The Large-Area Dust Detection Array (LADDA).
\textit{Adv. Space Res.} \textbf{31}, 307-312

\item Poppe, T., Blum, J., Henning, Th. (2002) Experiments on dust
aggregation and their relevance to space missions. \textit{Adv.
Space Res.}, \textbf{29}, 763

\item Blum, J., Wurm, G., Poppe, T., Kempf, S. and Kozasa, T.
(2002) First results from the cosmic dust aggregation experiment
CODAG. \textit{Adv. Space Res.}, \textbf{29}, 497

\item Wurm, G., Blum, J. and Colwell, J.~E. (2001) Aerodynamical
sticking of dust aggregates. \phre, \textbf{64}, 46301

\item Wurm, G., Blum, J. and Colwell, J.~E. (2001) NOTE: A New
Mechanism Relevant to the Formation of Planetesimals in the Solar
Nebula. \ica, \textbf{151}, 318

\item Blum, J. and Wurm, G. (2001) Drop Tower Experiments on
Sticking, Restructuring, and Fragmentation of Preplanetary Dust
Aggregates. \textit{Microgravity Science and Technology},
\textbf{XIII/1}, 29-34

\item Mukai, T., Blum, J., Nakamura, A., Johnson, R.~E. and
Havnes, O. (2001) Physical Processes on Interplanetary Dust, in:
\textit{Interplanetary Dust}, Astronomy and Astrophysics Library,
Springer (Eds. E. Gr\"un, B. Gustafson, S. Dermott and H.
Fechtig), 445-508

\item Tehranian, S., Giovane, F., Blum, J., Xu, Y. and Gustafson,
B.~\AA.~S. (2001) Photophoresis of Micrometer-Sized Particles in
the Free-Molecular Regime. \textit{Int. J. Heat and Mass
Transfer}, \textbf{44/9}, 1649-1657

\end{ownpubl}
\cleardoublepage
%
%--------------------------------------------------------------------
%
\renewcommand{\cvhdr}{Dullemond}
\section*{C.P.\ Dullemond}
\subsection*{Curriculum Vitae}
\begin{cvlayout}
\item[Name:\hfill]
    DULLEMOND, Cornelis Petrus
\item[Address:\hfill]
    Max Planck Institute for Astronomy, K\"onigstuhl 17,\\
    69117 Heidelberg\\
    Phone / Fax: +49-89-528-395 / +49-89-528-246 \\
    E-mail: dullemon@mpia.de
\item[Date and Place of Birth:\hfill]
    December 11, 1970, Nijmegen, the Netherlands
\item[Nationality:\hfill]
    Dutch
\item[Present position:\hfill]
    Research Scientist
\item[Background:\hfill]
    Studies at the University of Amsterdam (1989-1994)\\
    Diplome at the University of Amsterdam (1994)\\
    Dissertation in Physics and Astronomy (Ph.D.) (1999), 
    Leiden University\\
\item[Positions:\hfill]
    MPI f\"ur Astrophysik Garching (1999-2004)\\
    MPI f\"ur Astronomie Heidelberg (2004-)
%\item[Awards/Fellowships:\hfill]
\end{cvlayout}

\subsection*{Research interests and achievements}
\subsubsection*{Radiative transfer in astrophysics}
Developing various methods, algorithms and codes for solving this
fundamental, yet complex problem in multiple dimensional astrophysical
objects, in particular in disk-like configurations.
\subsubsection*{Formation, evolution, structure and spectra of
  protoplanetary disks}
Developing models of disk formation and structure, with the aim
of practical use. Application of these models to observations in
order to draw conclusions about the structure and evolution of
protoplanetary disks. Application of these models as basis upon 
which other calculations can be made (such as grain growth, see
below).
\subsubsection*{Modeling of grain aggregation/growth and grain motion}
Modeling the first steps of the planet formation process: the growth
from primordial dust grains to larger aggregates. Fundamental theoretical
models, but always with the aim of applicability to observations or
elsewhere. 
\subsubsection*{Hot plasma flow around black holes}
Theoretical models of the origin of hot ($10^9$K) plasma flows in the
immediate surroundings of galactic black holes.
\subsubsection*{Dusty gas tori around supermassive black holes}
Modeling the structure of the dusty gas flows around the black holes
in Active Galactic Nuclei. 

\subsection*{List of refereed publications in past 5 years}
\begin{ownpubl}

\item
Dullemond, C.~P., Apai, D. and Walch, S. (2006) Crystalline silicates as a
probe of disk formation history. \apjl, \textbf{640}, 67

\item 
Kessler-Silacci, J., Augereau, J.-C., Dullemond, C.~P., Geers, V., Lahuis,
F., Evans, N.~J., van Dishoeck, E.~F., Blake, G.~A., Boogert, A.~C.~A.,
Brown, J., J{\o}rgensen, J.~K., Knez, C. and Pontoppidan, K.~M. (2006) c2d
Spitzer IRS Spectra of Disks around T Tauri Stars. I. Silicate Emission and
Grain Growth. \apj, \textbf{639}, 275

\item 
Lahuis, F., van Dishoeck, E.~F., Boogert, A.~C.~A., Pontoppidan, K.~M.,
Blake, G.~A., Dullemond, C.~P., Evans, N.~J.~II, Hogerheijde, M.~R., 
J{\o}rgensen, J.~K., Kessler-Silacci, J.E. and Knez, C. (2006) 
Hot organic molecules toward a young low-mass star: A look at 
inner disk chemistry. \apjl, \textbf{636}, 145

\item 
Apai, D., Pascucci, I., Bouwman, J., Natta, A., Henning, Th. and 
Dullemond, C.P. (2005) The onset of planet formation in Brown Dwarf disks.
\sci, \textbf{310}, 834

\item 
van Boekel, R., Dullemond, C.P.\ and Dominik, C. (2005) Flared and
self-shadowed disks around Herbig Ae stars: simulations for 10-micron
interferometers. \aap, \textbf{441}, 563

\item 
J{\o}rgensen, J.~K., Lahuis, F., Sch{\"o}ier, F.~L., van Dishoeck, E.~F.,
Blake, G.~A., Boogert, A.~C.~A., Dullemond, C.~P., Evans, N.~J.,
Kessler-Silacci, J.~E. and Pontoppidan, K.~M. (2005) Protostellar Holes:
Spitzer Space Telescope Observations of the Protostellar Binary IRAS
16293-2422. \apjl, \textbf{631}, 77

\item 
Meijerink, R., Tilanus, R.~P.~J., Dullemond, C.~P., Israel, F.~P. and van der
Werf, P.~P. (2005) A submillimeter exponential disk in M51. \aap, \textbf{430},
427

\item 
Pontoppidan, K.~M., Dullemond, C.~P., van Dishoeck, E.~F., Blake, G.~A.,
Boogert, A.~C.~A., Evans, N.~J., II., Kessler-Silacci, J.~E. and
Lahuis, F. (2005) Ices in the Edge-on Disk CRBR 2422.8-3423:
Spitzer Spectroscopy and Monte Carlo Radiative Transfer Modeling. \apj, 
\textbf{622}, 463

\item 
Semenov, D., Pavlyuchenkov, Ya., Schreyer, K., Henning, Th. and Dullemond, C.~P.
(2005) Millimeter observations and modelling of the AB Aurigae system. \apj,
\textbf{621}, 853

\item 
Dullemond C.~P. and van Bemmel, I. (2005) Clumpy tori around AGN. \aap,
\textbf{436}, 47

\item 
Dullemond C.P. and Dominik, C. (2005) Dust coagulation in protoplanetary disk:
a rapid depletion of small grains. \aap, \textbf{434}, 971

\item 
Dullemond C.~P. and Spruit, H.~C. (2005) Evaporation of ion-irradiated disks.
\aap, \textbf{434}, 415

\item 
Pontoppidan, K. and Dullemond, C.~P. (2005) Projection of circumstellar disks on
their environments. \aap, \textbf{435}, 595

\item 
Acke, B., van den Ancker, M. and Dullemond, C.~P. (2005) [OI] Emission in Herbig
Ae/Be systems: signature of Keplerian rotation. \aap, \textbf{436}, 209\

\item 
Pascucci, I., Wolf, S., Steinacker, J., Dullemond, C.~P., Henning, Th.,
Niccolini, G., Woitke, P. and Lopez, B. 
(2004) The 2-D continuum transfer problem: benchmark results for disk
configurations. \aap, \textbf{417}, 793

\item 
Dullemond, C.~P. and Dominik, C. (2004) Flaring vs.~self-shadowed disks: the SEDs
of Herbig Ae/Be stars. \aap, \textbf{417}, 159

\item 
van Boekel, R., Waters, L.~B.~F.~M., Dominik, C., Dullemond, C.~P.,
Tielens, A.~G.~G.~M., de Koter, A. (2004) Spatially and spectrally resolved
10 $\mu$m emission in Herbig Ae/Be stars. \aap, \textbf{418}, 177

\item 
Dullemond, C.~P. and Dominik, C. (2004) Dust settling in disks: from flaring to
self-shadowing. \aap, \textbf{421}, 1075

\item 
Leinert, Ch., van Boekel, R., Waters, L.~B.~F.~M., Chesneau, O., Malbet, F., 
{K{\"o}hler}, R., {Jaffe}, W., 
{Ratzka}, T., {Dutrey}, A., {Preibisch}, T., {Graser}, U., 
{Bakker}, E., {Chagnon}, G., {Cotton}, W.~D., {Dominik}, C., 
{Dullemond}, C.~P., {Glazenborg-Kluttig}, A.~W., {Glindemann}, A., 
{Henning}, T., {Hofmann}, K.-H., {de Jong}, J., {Lenzen}, R., 
{Ligori}, S., {Lopez}, B., {Meisner}, J., {Morel}, S., 
{Paresce}, F., {Pel}, J.-W., {Percheron}, I., {Perrin}, G., 
{Przygodda}, F., {Richichi}, A., {Sch{\"o}ller}, M., 
{Schuller}, P., {Stecklum}, B., {van den Ancker}, M.~E., 
{von der L{\"u}he}, O. and {Weigelt}, G.
(2004) Mid-infrared sizes of circumstellar disks around
Herbig Ae/Be stars measured with MIDI on the VLTI. \aap, \textbf{423}, 537

\item 
Boogert, C.~A., Pontoppidan, K.~M., Lahuis, F., J\o rgensen, J.~K., Augereau,
J.-C., Blake, G.~A., Brooke, T.~Y., Brown, J., Dullemond, C.~P., Evans,
N.~J., II., Geers, V., Hogerheijde, M.~R., Kessler-Silacci, J., 
Knez, C., Morris, P., Noriega-Crespo, A., Sch\"oier, F.~L., 
van Dishoeck, E.~F., Allen, L.~E., Harvey, P.~M., Koerner, D.~W.,
Mundy, L.~G., Myers, P.~C., Padgett, D.~L., Sargent, A.~L., 
Stapelfeldt, K.~R.  (2004) Spitzer space telescope spectroscopy of ices toward
low mass embedded protostars. \apjs, \textbf{154}, 359

\item 
Acke, B., van den Ancker, M.~E., Dullemond, C.~P., van Boekel, R. and
Waters, L.~B.~F.~M. (2004) Evidence for grain growth in self-shadowed disks
around Herbig Ae/Be stars. \aap, \textbf{422}, 621

\item 
Kamp, I. and Dullemond, C.~P. (2004) The gas temperature in the surface layers of
protoplanetary disks. \apj, \textbf{615}, 991

\item 
Young, C.~H., J\o rgensen, J.~K., Shirley, Y.~L., Kauffmann, J., Huard, T.,
{Lai}, S.-P., {Lee}, C.~W., 
{Crapsi}, A., {Bourke}, T.~L., {Dullemond}, C.~P., 
{Brooke}, T.~Y., {Porras}, A., {Spiesman}, W., {Allen}, L.~E., 
{Blake}, G.~A., {Evans}, N.~J., {Harvey}, P.~M., {Koerner}, D.~W., 
{Mundy}, L.~G., {Myers}, P.~C., {Padgett}, D.~L., {Sargent}, A.~I., 
{Stapelfeldt}, K.~R., {van Dishoeck}, E.~F., {Bertoldi}, F., 
{Chapman}, N., {Cieza}, L., {DeVries}, C.~H., {Ridge}, N.~A., 
{Wahhaj}, Z., (2004) A ``starless'' core that isn't: detection of a source
in the L1014 dense core with the Spitzer Space Telescope. \apjs, \textbf{154},
396

\item 
Sterzik, M., Pascucci, I., Apai, D., van der Bliek, N. and Dullemond, C.~P. 
(2004) Evolution of Young Brown Dwarf Disks in the Mid-Infrared. \aap,
\textbf{427}, 245

\item 
Apai, D., Pascucci, I., Sterzik, M.~F., van der Bliek, N., Bouwman, J., 
Dullemond, C.~P. and Henning, Th.
(2004) Grain growth and dust settling in a
brown dwarf disk: Gemini/T-ReCS observations of CFHT-BD-Tau 4. \aap, 
\textbf{426}, 53

\item 
Dominik, C., Dullemond, C.~P., Cami, J. and van Winckel, H. (2003) The dust disk
of HR4049: Another brick in the wall. \aap, \textbf{397}, 595

\item 
Dominik, C., Dullemond, C.~P., Waters, L.~B.~F.~M. and Walch, S. (2003)
Understanding the spectra of isolated Herbig stars in the frame of a passive disk
model. \aap, \textbf{398}, 607

\item 
van Boekel, R., Waters, L.~B.~F.~M., Dominik, C., Bouwman, J., de Koter, A.,
Dullemond, C.~P., Paresce, F. (2003) Grain growth in the inner regions of Herbig Ae/Be star disks. \aap,
\textbf{400}, L21

\item 
Pascucci, I., Apai, D., Henning, Th. and Dullemond, C.~P. (2003) The First
Detailed Look at a Brown Dwarf Disk. \apj, \textbf{590}, L111

\item 
van Bemmel, I. and Dullemond, C.~P. (2003) New radiative transfer models 
for obscuring tori in active galaxies. \aap, \textbf{404}, 1

\item 
Dullemond, C.~P. and Natta, A. (2003) An analysis of two-layer models 
for circumstellar disks. \aap, \textbf{405}, 597

\item 
Dullemond, C.~P., van den Ancker, M.~E., Acke, B. and van Boekel, R. (2003)
Explaining UXOR variability with self-shadowed disks. \apj, \textbf{594},
L47

\item 
Dullemond, C.~P. and Natta, A. (2003) Dust scattering in protoplanetary
disks: the effect on the SED.  \aap, \textbf{408}, 161

\item 
Alencar, S.~H.~P., Melo, C.~H.~F., Dullemond, C.~P., Andersen, J., Batalha,
C., Vaz, L.~P.~R. and Mathieu, R.~D. (2003) The pre-main sequence
spectroscopic binary AK Sco revisited. \aap, \textbf{409}, 1037

\item 
Deufel.~B., Dullemond,~C.~P. and Spruit.~H. (2002) X-ray spectra from accretion
disks illuminated by protons. \aap, \textbf{387}, 907

\item 
Dullemond,~C.~P., van Zadelhoff, G.-J. and Natta,~A. (2002) Vertical structure
models of T Tauri and Herbig Ae/Be disks. \aap, \textbf{389}, 464

\item 
van Zadelhoff, G-J., Dullemond,~C.~P., van der Tak, F., Yates, J.~A., Doty, S.~D.,
Ossenkopf, V., Hogerheijde, M.~R., Juvela, M., Wiesemeyer, H.,
Sch\"oier, F.~L. (2002) Numerical methods for non-LTE line radiative transfer: performance
and convergence characteristics. \aap, \textbf{395}, 373

\item 
Dullemond,~C.~P. (2002) The 2-D structure of dusty disks around Herbig Ae/Be
stars. I. Models with grey opacities. \aap, \textbf{395}, 853

\item 
Dullemond,~C.~P., Dominik.~C. and Natta.~A. (2001) Passive irradiated
circumstellar disks with an inner hole.  \apj, \textbf{560}, 957

\end{ownpubl}
\cleardoublepage
%
%--------------------------------------------------------------------
%
\renewcommand{\cvhdr}{Gail}
\section*{H.-P.\ Gail}
\subsection*{Curriculum Vitae}
\begin{cvlayout}
\item[Name:\hfill]        
    GAIL, Hans-Peter
\item[Address:\hfill]
    Zentrum f\"ur Astronomie, Heidelberg\\
    Institut f\"ur Theoretische Astrophysik\\
    Albert-\"Uberle-Str. 2\\
    69120 Heidelberg\\
    Phone: 06221-548982\\
    E-mail: gail@ita.uni-heidelberg.de

\item[Date and Place of Birth:\hfill]
    23 Juli 1941, Hamburg, Germany
\item[Nationality:\hfill]
    German
\item[Present position:\hfill]
    apl. Prof.
\item[Background:\hfill]
    Studies in Physics and Mathematics at the 
    University of Hamburg (1961-1968)\\
    Diploma in Physics (1968), Sternwarte Bergedorf\\
    Dissertation in Physics (Dr. rer. nat) (1971), Lehrstuhl f\"ur Theoreti\-sche
    Astrophysik, Uni Heidelberg\\
    Habilitation (1977), Faculty of Physics, Univ.\ Heidelberg
\item[Positions:\hfill]
    Univ. of Heidelberg (1971-1984)\\
    Technical Univ. Berlin (1984-1986)\\
    Univ. of Heidelberg (1986--)
\end{cvlayout}

\subsection*{Research interests and achievements}
\subsubsection*{Formation and evolution of protoplanetary disks}
Studying the structure and evolution of protoplanetary accretion disks.
In particular the chemistry of the gas phase and the evolution of the
dust component by chemical and mineralogical processes have been
modelled self consistently with models for the disk. The role of
mixing processes in protoplanetary disks by turbulent mixing and large
scale flows on the composition of the gas phase and the mineral mixture
were studied. The observations of cometary dust from comets like Hale-Bopp
and the recent results from `Deep Impact' and `Stardust' confirm the
predictions of these models.
 
\subsubsection*{Circumstellar dust}
The theory of dust formation in stellar outflows from AGB stars has
been developed. The basic chemical and physical processes of dust growth and
dust nucleation have been studied and methods have been worked out for
modelling selfconsistently the hydrodynamics of the stellar wind and the
condensation of dust in the cooling flow. The methods developed form the
theoretical basis of the existing codes for dust formation by AGB stars.
Recently it was possible to develop a complete model for the dust return
by AGB stars to the ISM.

\subsection*{List of refereed publications in past 5 years}
\begin{ownpubl}

\item
Tscharnuter, W.~M and Gail, H.-P. (2006) 2-D protoplanetary accretion disks. I.
Hydrodynamics, chemistry, and mixing processes. \aap, (submitted)

\item
Ferrarotti, A.~S. and Gail, H.-P. (2006) Composition and quantities of dust
produced by AGB-stars and returned to the interstellar medium. \aap,
\textbf{447}, 553--576

\item
Gail, H.-P., Duschl, W.~J., Ferrarotti, A.S. and Weis, K. (2005) Dust formation
in LBV envelopes. In: \textit{The Fate of the Most Massive Stars}, ASP Conference
Series, Vol. 332, eds. R. Humphreys and K. Stanek. (San Francisco: Astronomical
Society of the Pacific) p.323

\item
Gail, H.-P. and Tscharnuter, W.~M. (2005) Evolution of protoplanetary disks
including detailed chemistry and mineralogy. In: \textit{Reactive Flow,
Diffusion and Transport}, ed. R. Rannacher et. al. (Springer, Berlin-Heidelberg)
(in press)

\item
Ferrarotti, A.~S. and Gail, H.-P. (2005) Mineral formation in stellar winds. V.
Formation of calcium carbonate. \aap, \textbf{430}, 959-965

\item
Keller, Ch. and Gail, H.-P. (2004) Radial mixing in protoplanetary accretion
disks. VI. Mixing by large-scale radial flows. \aap, \textbf{415}, 1177-1185

\item
Gail, H.-P. (2004) Radial mixing in protoplanetary accretion disks. IV.
Metamorphosis of the silicate dust complex. \aap, \textbf{413}, 571-591

\item
Wehrstedt, M. and Gail, H.-P. (2003) Radial mixing in protoplanetary accretion
disks. V. Models with different element mixtures. \aap, \textbf{410}, 917-935

\item
Ferrarotti, A.~S. and Gail, H.-P. (2003) Mineral formation in stellar winds. IV.
Formation of magnesiow\"ustite. \aap, \textbf{398}, 1029-1039

\item
Gail, H.-P. (2003) Formation and Evolution of Minerals in Accretion Disks and
Stellar Outflows. In: \textit{Astromineralogy} (Ed. Th. Henning) Lecture Notes in
Physics, \textbf{609}. Springer, Heidelberg. p. 55-120

\item
Gail, H.-P. (2002)
Radial mixing in protoplanetary accretion disks. III. Carbon dust oxidation and
abundance of hydrocarbons in comets. \aap, \textbf{390}, 253-265

\item
Wehrstedt, M. and Gail, H.-P. (2002) Radial mixing in protoplanetary accretion
disks. II. Time dependent disk models with annealing and carbon combustion. \aap,
\textbf{385}, 181-204 

\item
Ferrarotti, A.~S. and Gail, H.-P. (2002) Mineral formation in stellar winds. III.
Dust formation in S stars. \aap, \textbf{382}, 256-281

\item
Gail, H.-P. (2001) Radial mixing in protoplanetary accretion disks. I. Stationary
disc models with annealing and carbon combustion. \aap, \textbf{378}, 192-213

\end{ownpubl}
\cleardoublepage
%
%--------------------------------------------------------------------
%
\renewcommand{\cvhdr}{Henning}
\section*{Th.~Henning}
\subsection*{Curriculum Vitae}
\begin{cvlayout}
\item[Name:\hfill]        
    HENNING, Thomas
\item[Address:\hfill]
    Max Planck Institute for Astronomy, K\"onigstuhl 17,\\
    69117 Heidelberg\\
    Phone / Fax: +49-89-528-200 / +49-89-528-246 \\
    E-mail: henning@mpia.de
\item[Date and Place of Birth:\hfill]
    April 9, 1956, Jena, Germany
\item[Nationality:\hfill]
    German
\item[Present position:\hfill]
    Director of the Max Planck Institute for Astronomy, Heidelberg,\\
    Department Planet and Star Formation
\item[Background:\hfill]
    Studies in Physics and Astronomy at University Greifswald (1976-1980) and University of Jena (1980-1981)\\
    Diploma (1981)\\
    PhD in Astrophysics, Jena University (1984)\\
\item[Positions:\hfill]
    Post-Doc, Charles University Prague (1984-1985)\\
    Assistant, University Observatory Jena (1986-1988)\\
    Max Planck Institute for Radioastronomy Bonn (1989-1990)\\
    Head of Max Planck Research Unit "Star Formation", Jena (1991-1996)\\
    Professor for Astrophysics, Friedrich-Schiller University, Jena (1992-1998)\\
    Chair for Astrophysics, Friedrich-Schiller University, Jena (1999-2002)\\
    Co-Chair of the DFG Research Group "Laboratory Astrophysics" (2000-)\\
    Director of the Astrophysical Institute Jena (2000-2002)\\
    Professor at the Universities of Jena and Heidelberg (2002-)\\
    Speaker of the German DFG Programme "Physics of Star Formation" (1995-2002)\\
    Member of the Strategic Working Group of ESO\\
    Member of the Scientific Technical Committee of ESO\\
    Member of the Astronomy Working Group of ESA\\
    Member of the SOFIA Science Council\\
    Member of the European ALMA Board\\
    Member of the ELT Design Study Steering Committee\\
    Chairman of the LBT Beteiligungsgesellschaft\\
    Member of the LBT Board of Directors\\
    President of the Scientific Council of the European Interferometry Initiative
\item[Awards/Fellowships:\hfill]
    Prize for Fundamental Reserarch, State of Thuringia\\
    Member of the Leopoldina Academy (1999-)
\end{cvlayout}

\subsection*{Research interests and achievements}
Star and planet formation, exoplanets, circumstellar disks, physics and
chemistry of the interstellar medium, Laboratory Astrophysics, infrared
instrumentation.

Detection and characterization of protoplanetary disks around young
solar-type stars and Brown Dwarfs, multidimensional radiation hydrodynamics
simulations of disk structure and chemistry, detection of the earliest
stages of massive stars, first comprehensive characterization of cosmic dust
analogues.

\subsection*{List of refereed publications in past 5 years}
\begin{ownpubl}
\item Kley, W., D'Angelo, G., Henning, Th. (2001) Three-dimensional
Simulations of a Planet Embedded in a Protoplanetary Disk,
Astrophys. J. \textbf{547}, 457-464.

\item Hofner, P., Wiesemeyer, H., Henning, Th. (2001) A High Velocity
Outflow from the G 9.62+0.19 Star-forming Region, Astrophys. J.
\textbf{549}, 425-432.

\item Henning, Th., Feldt, M., Stecklum, B., Klein, R. (2001)
High-Resolution Imaging of Ultracompact HII Regions - III. G
11.11-0.40 and G 341.21-0.21, Astron. Astrophys. \textbf{370},
100-111.

\item Steinacker, A., Henning, Th. (2001) Global 3D-MHD Simulations of
Accretion Discs and the Surrounding Magnetosphere, Astrophys. J.
\textbf{554}, 514-527.

\item Fabian, D., Henning, Th., J{\"a}ger, C., Mutschke, H., Dorschner, J.,
Wehrhan, O. (2001) Steps Toward Interstellar Silicate Mineralogy
VI. Dependence of Crystalline Olivine IR Spectra on Iron Content and
Particle Shape, Astron. Astrophys. \textbf{378}, 228-238.

\item Palomba, E., Poppe, T., Colangeli, L., Palumbo, P., Perrin, J. M.,
 Bussoletti, E., Henning, Th. (2001) The Sticking Efficiency of Quartz
 Crystals for Cosmic Sub-Micron Grain Collection, Planetary and Space
 Science \textbf{49}, 919-926.

\item Grady, C.A., Polomski, E.F., Henning, Th., {Stecklum}, B., 
{Woodgate}, B.~E., {Telesco}, C.~M., {Pi{\~n}a}, R.~K., 
{Gull}, T.~R., {Boggess}, A., {Bowers}, C.~W., {Bruhweiler}, F.~C., 
{Clampin}, M., {Danks}, A.~C., {Green}, R.~F., {Heap}, S.~R., 
{Hutchings}, J.~B., {Jenkins}, E.~B., {Joseph}, C., 
{Kaiser}, M.~E., {Kimble}, R.~A., {Kraemer}, S., {Lindler}, D., 
{Linsky}, J.~L., {Maran}, S.~P., {Moos}, H.~W., {Plait}, P., 
{Roesler}, F., {Timothy}, J.~G., {Weistrop}, D.
 (2001) The Disk and
Environment of the Herbig Be Star HD 100546, Astron. J. \textbf{122},
3396-3406.

\item Henning, Th., Wolf, S., Launhardt, R., Waters, R. (2001)
Measurements of the Magnetic Field Geometry and Strength in Bok
Globules, Astrophys. J. \textbf{561}, 871-879.

\item Kemper, F., J\"ager, C., Waters, L.B.F.M., Henning, Th.,
Molster, F.J., Barlow, M.J., Lim, T., de Koter, A. (2002) Detection of
Carbonates in Dust Shells around Evolved Stars, Nature \textbf{415},
295-297.

\item Poppe, T., Blum, J., Henning, Th. (2002) Experiments on Dust
Aggregation and their Relevance to Space Missions, Adv. Space
Res. \textbf{29}, 763-771.

\item Wolf, S., Voshchinnikov, N.V., Henning, Th. (2002) Multiple
Scattering of Polarized Radiation by Non-Spherical Grains: First Results,
Astron. Astrophys. \textbf{385}, 365-376.

\item Wolf, S., Gueth, F., Henning, Th., Kley, W. (2002) Detecting Planets
in Protoplanetary Disks: A Prospective Study, Astrophys.  J. \textbf{566},
L97-L99.

\item Markwick, A.J., Ilgner M., Millar, T.J., Henning, Th. (2002)
Molecular Distributions in the Inner Regions of Protostellar
Disks, Astron. Astrophys. \textbf{385}, 632-646.

\item D'Angelo, G., Henning, Th., Kley, W. (2002) Nested grid Calculations
of Disk-Planet Interaction, Astron. Astrophys. \textbf{385}, 647-670.

\item Keller, L.P., Hony, S., Bradley, J.P., Molster, F.J., Waters,
L.B.F.M., Bouwman, J., de Koter, A., Brownlee, D.E., Flynn, G.J., Henning,
Th., Mutschke, H. (2002) Identification of Iron Sulfide Grains in
Protoplanetary Disks, Nature, \textbf{417}, 148-150.

\item G\"urtler, J., Klaas, U., Henning, Th., Abraham, P., Lemke,
D., Schreyer, K., Lehmann, K. (2002) Detection of Solid Ammonia,
Methanol and Methane with ISOPHOT, Astron. Astrophys. \textbf{390},
1075-1087.

\item Apai, D., Pascucci, I., Henning, Th., Sterzik, M.F., Klein, R.,
Semenov, D., G\"unther, E., Stecklum, B. (2002) Probing Dust around Brown
Dwarfs: The Naked LP944-20 and the Disk of Chamaeleon
H$\alpha$2. Astrophys. J. \textbf{573}, L115-L117.

\item Stecklum, B., Brandl, B., Henning, Th., Pascucci, I., Hayward, T.L.,
Wilson, J. (2002) High-Resolution Mid-Infrared Imaging of
W3(OH). Astron. Astrophys. \textbf{392}, 1025-1029.

\item Schreyer, K., Henning, Th., van der Tak, F.F.S., Boonman, A.M.S., van
Dishoek, E.F. (2002) The Young Intermediate-mass Stellar Object AFGL 490-A
Disk Surrounded by a Cold Envelope. Astron.  Astrophys. \textbf{394},
561-583.

\item Quinten, M., Kreibig, U., Henning, Th., Mutschke, H. (2002)
Wavelength-dependent Optical Extinction of Carbonaceous Particles in
Atmospheric Aerosols and Interstellar Dust, Applied Optics \textbf{41},
7102-7113.

\item Steinacker, J., Bacmann, A., Henning, Th. (2002) Application of
Adaptive Multi-Frequency Grids to Three-Dimensional Astrophysical
Radiative Transfer, J. Quant. Spectr. Rad. Transf. \textbf{75},
765-786.

\item Schrempel, F., J\"ager, C., Fabian, D., Dorschner, J.,
Henning, Th., Wesch, W. (2002) Study of the Amorphization Process of
MgSiO$_2$ by Ion Irradiation as a Form of Dust Processing in
Astrophysical Environments. NIM B \textbf{191}, 411-415.

\item J\"ager, C., Fabian, D., Schrempel, F., Dorschner, J., Henning, Th.,
Wesch, W. (2003) Structural Processing of Enstatite by Ion Bombardement,
Astron. Astrophys. \textbf{401}, 57-65.

\item Steinacker, J., Henning, Th. (2003) Detection of Gaps in
Circumstellar Disks, Astrophys. J. \textbf{583}, L35-L38.

\item D'Angelo, G., Kley, W., Henning, Th. (2003) Orbital Migration and Mass
Accretion of Protoplanets in 3D Global Computations with Nested Grids,
Astrophys. J. \textbf{586}, 540-561.

\item Steinacker, J., Henning, Th., Bacmann, A., Semenov, D. (2003) 3D
Continuum Radiative Transfer in Complex Dust Configurations around Young
Stellar Objects and Active Galactic Nuclei. I. Computational Methods and
Capabilities, Astron. Astrophys. \textbf{401}, 405-418.

\item Wiebe, D., Semenov, D., Henning, Th. (2003) Reduction of Chemical
Networks. I. The Case of Molecular Clouds, Astron. Astrophys. \textbf{399},
197-210.

\item K\"uker, M., Henning, Th., R\"udiger, G.  (2003) Magnetic Star-Disk
Coupling in Classical T Tauri Systems, Astrophys. J. \textbf{589},
397-409.

\item Pascucci, I., Apai, D., Henning, Th., Dullemond, C.P. (2003) The First
Detailed Look at a Brown Dwarf Disk, Astrophys. J. \textbf{590}, L111-L114.

\item Wolf, S., Launhardt, R., Henning, Th. (2003) Magnetic Field
Evolution in Bok Globules, Astrophys. J. \textbf{592}, 233-244.

\item J\"ager, C., Il'in, V., Henning, Th., Mutschke, H., Fabian, D.,
Semenov, D.A., Voshchinnikov, N.V.  (2003) A Database of Optical Constants
of Cosmic Dust Analogs, J. Quant. Spectr. Rad. Transf. \textbf{79-80},
765-774.

\item Klein, R., Apai, D., Pascucci, I., Henning, Th., Waters,
L.B.F.M. (2003) First Detection of Millimeter Dust Emission from Brown
Dwarf Disks, Astrophys. J. \textbf{593}, L57-L60.

\item Cl\'ement, D., Mutschke, H., Klein, R., Henning, Th. (2003) New
Laboratory Spectra of Isolated $\beta$-SiC Nanoparticles: Comparison with
ISO Observations, Astrophys. J. \textbf{594}, 642-650.

\item J\"ager, C., Dorschner, J., Mutschke, H., Posch, Th., Henning,
Th. (2003) Steps towards Interstellar Silicate Mineralogy. VII.  Spectral
Properties and Crystallization Behaviour of Magnesium Silicates Produced by
the Sol-Gel-Method, Astron. Astrophys. \textbf{408}, 193-204.

\item Semenov, D., Henning, Th., Helling, M.Ch., Ilgner, M.,
Sedlmayr, E. (2003) Rosseland and Planck Mean Opacities for
Protoplanetary Discs, Astron. Astrophys. \textbf{410}, 611-621.


\item Posch, T., Kerschbaum, F., Fabian, D., Mutschke, H.,
Dorschner, J., Tamanai, A., Henning, Th.: Infrared Properties of
Solid Titanium Oxides (2003) Exploring Potential Primary Dust
Condensates, Astrophys. J. Suppl. Ser. \textbf{149}, 437-445.

\item D'Angelo, G., Henning, Th., Kley, W. (2003) Thermohydrodynamics of
Circumstellar Disks with High-mass Planets, Astrophys. J. \textbf{599},
548-576.

\item Schreyer, K., Stecklum, B., Linz, H., Henning, Th. (2003) NGC 2264
IRS1: The Central Engine and its Cavity, Astrophys. J. \textbf{599},
335-341.

\item Feldt, M., Puga, E., Lenzen, R., Henning, Th., Brandner, W.,
Stecklum, B., Lagrange, A.M., Gendron, E., Rousset, G. (2003) Discovery
of a Candidate for the Central Star of the Ultracompact HII Region
G5.89-0.39, Astrophys. J. \textbf{599}, L91-L94.

\item Forbrich, J., Schreyer, K., Posselt, B., Klein, R., Henning,
Th. (2004) An Extremely Young Massive Stellar Object near IRAS
07029-1215, Astrophys. J. \textbf{602}, 843-849.

\item Apai, D., Pascucci, I., Brandner, W., Henning, Th., Lenzen,
R., Potter, D.E., Lagrange, A.-M., Rousset, G. (2004) NACO Polarimetric
Differential Imaging of TW Hya: A Sharp Look at the Closest T
Tauri Disk, Astron. Astrophys. \textbf{415}, 671-676.

\item Ilgner, M., Henning, Th., Markwick, A.J., Millar, T.J. (2004)
Transport Processes and Chemical Evolution in Steady Accretion
Disk Flows, Astron. Astrophys. \textbf{415}, 643-659.

\item Wang, H., Apai, D., Henning, Th., Pascucci, I. (2004) FU Orionis: A
Binary Star? Astrophys. J. \textbf{601}, L83-L86.

\item Sch{\"u}tz, O., Nielbock, M., Wolf, S., Henning, Th., Els, S. (2004)
SIMBA's view of the $\epsilon$ Eri disk, Astron. Astrophys. \textbf{414},
L9-L12.

\item Sukhorukov, O., Staicu, A., Diegel, E., Rouill\'e, G., Henning, Th.,
Huisken, F. (2004) D$_2$ $\leftarrow$ D$_0$ Transition of the Anthracene
Cation Observed by Cavity Ring-Down Absorption Spectroscopy in a Supersonic
Jet, Chem. Phys. Letters \textbf{386}, 259-264.

\item Rouill\'e, G., Krasnokutski, S., Huisken, F., Henning, Th.,
Sukhorukov, O., Staicu, A. (2004) UV Spectroscopy of Pyrene in a Supersonic
Jet and in Liquid Helium droplets, J. Chem. Phys. \textbf{120}, 6028-6034.

\item Semenov, D., Wiebe, D., Henning, Th. (2004) Reduction of Chemical
Networks. II. Analysis of the Fractional Ionisation of Protoplanetary Discs,
Astron. Astrophys. \textbf{417}, 93-106.

\item Pascucci, I., Wolf, S., Steinacker, J., Dullemond, C.P., Henning, Th.,
Niccolini, G., Woitke, P., Lopez, B. (2004) The 2D Continuum Radiative
Transfer Problem. Benchmark Results for Disk Configurations,
Astron. Astrophys. \textbf{417}, 793-805.

\item Grady, C.A., Woodgate, B., Torres, C.A.O., Henning, Th., Apai, D.,
Rodmann, J., Wang, H., Stecklum, B., Linz, H., Williger, G.M., Brown, A.,
Wilkinson, E., Harper, G.M., Herczeg, G.J., Danks, A., Vieira, G.L.,
Malumuth, E., Collins, N.R., Hill, R.S. (2004) The Environment of the
Optically Brightest Herbig Ae Star HD 104237, Astrophys. J. \textbf{608},
809-830.

\item Leinert, Ch., van Boekel, R., Waters, L.B.F.M., Chesneau, O., Malbet,
F., K{\"o}hler, R., Jaffe, W., Ratzka, Th., Dutrey, A., Preibisch, Th.,
Graser, U., Bakker, E., Chagnon, G., Cotton, W.D., Dominik, C., Dullemond,
C.P., Glazenborg-Kluttig, A.W., Glindemann, A., Henning, Th., Hofmann,
K.-H., de Jong, J., Lenzen, R., Ligori, S., Lopez, B., Meisner, J., Morel,
S., Paresce, F., Pel, J.-W., Percheron, I., Perrin, G., Przygodda, F.,
Richichi, A., Sch\"oller, M., Schuller, P., Stecklum, B., van den Ancker,
M.E., von der L{\"u}he, O., Weigelt, G (2004) Mid-Infrared Sizes of
Circumstellar Disks around Herbig Ae/Be Stars Measured with MIDI on the
VLTI, Astron. Astrophys. \textbf{423}, 537-548.

\item Mutschke, H., Andersen, A.C., J\"ager, C., Henning, Th., Braatz, A.
(2004) Optical Data of Meteoritic Nano-Diamonds from Far-Ultraviolet to
Far-Infrared Wavelengths, Astron. Astrophys. \textbf{423}, 983-993.

\item Sch\"utz, O., B\"ohnhardt, H., Pantin, E., Sterzik, M., Els, S., Hahn,
J., Henning, Th. (2004) A Search for Circumstellar Dust Disks with ADONIS,
Astron.  Astrophys. \textbf{424}, 613-618.

\item Meyer, M.R., Hillenbrand, L.A., Backman, D.E., Beckwith, S.V.W.,
Bouwman, J., Brooke, T.Y., Carpenter, J.M., Cohen, M., Gorti, U., Henning,
Th., Hines, D.C., Hollenbach, D., Kim, J.S., Lunine, J., Malhotra, R.,
Mamajek, E.E., Metchev, S., Moro-Martin, A., Morris, P., Najita, J.,
Padgett, D.L., Rodmann, J., Silverstone, M.D., Soderblom, D.R., Stauffer,
J.R., Stobie, E.B., Strom, S.E., Watson, D.M., Weidenschilling, S.J., Wolf,
S., Young, E., Engelbracht, C.W., Gordon, K.D., Misselt, K., Morrison, J.,
Muzerolle, J., Su, K. (2004) The Formation and Evolution of Planetary
Systems: First Results from a Spitzer Legacy Science Program,
Astrophys. J. Suppl. Ser. \textbf{154}, 422-427.

\item Puga, E., Alvarez, C., Feldt, M., Henning, Th., Wolf, S.: AO-assisted
Observations of G61.48+0.09 (2004) Massive Star Formation at High Resolution,
Astron. Astrophys. \textbf{425}, 543-552.

\item Pascucci, I., Apai, D., Henning, Th., Stecklum, B., Brandl, B. (2004)
The Hot Core-Ultracompact HII Connection in G10.47+0.03,
Astron. Astrophys. \textbf{426}, 523-534.

\item Prieto, A.M., Meisenheimer, K., Marco, O., Reunanen, J., Contini, M.,
Clenet, Y., Davies, R.I., Gratadour, D., Henning, Th., Klaas, U., Kotilanen,
J., Leinert, C., Lutz, D., Rouan, D., Thatte, N. (2004) Unveiling the
Central pc Region of AGN: The Circinus Nucleus in the Near-IR with the VLT,
Astrophys. J. \textbf{614}, 135-141.

\item Alvarez, C., Feldt, M., Henning, Th., Puga, E., Brandner, W.,
Stecklum, B. (2004) Near-IR Sub-Arcsecond Observations of Ultra-Compact H II
Regions, Astrophys. J. Suppl. Ser. \textbf{155}, 123-148.

\item Schr\"apler, R., Henning, Th. (2004) Dust Diffusion, Sedimentation,
and Gravitational Instabilities in Protoplanetary Disks,
Astrophys. J. \textbf{614}, 960-978.

\item Apai, D., Pascucci, I., Sterzik, M.F., van der Bliek, N., Bouwman, J.,
Dullemond, C.P., Henning, Th.: Grain Growth and Dust Settling in a Brown
Dwarf Disk (2004) Gemini/T-ReCs Observations of CFHT-BD-Tau4,
Astron. Astrophys. \textbf{426}, L53-L57.

\item Steinacker, J., Lang, B., Burkert, A., Bacmann, A., Henning,
Th. (2004) Three-Dimensional Continuum Radiative Transfer Images of a
Molecular Cloud Core Evolution, Astrophys. J. \textbf{615}, L157-L160.

\item Staicu, A., Rouill\'e, G., Sukhorukov, O., Henning, Th., Huisken,
F. (2004) Cavity Ring-Down Laser Absorption Spectroscopy of Jet-Cooled
Anthracene, Mol. Phys. \textbf{20}, 1777-1783.
    
\item van Boekel, R., Min, M., Leinert, Ch., Waters, L. B. F. M., Richichi,
A., Chesneau, O., Dominik, C., Jaffe, W., Dutrey, A., Graser, U., Henning,
Th., de Jong, J., K\"ohler, R., de Koter, A., Lopez, B., Malbet, F., Morel,
S., Paresce, F., Perrin, G., Preibisch, Th., Przygodda, F., Sch\"oller, M.,
Wittkowski, M. (2004) The Building Blocks of Planets within the
`Terrestrial' Region of Protoplanetary Disks, Nature \textbf{432}, 479-482.

\item Wang, H., Mundt, R., Henning, Th., Apai, D. (2004) Optical Outflows in
the R CrA Molecular Cloud, Astrophys. J. \textbf{617}, 1191-1203.

\item Voshchinnikov, N.V., Il'in, V.B., Henning, Th. (2005) Modelling the
Optical Properties of Composite and Porous Interstellar Grains,
Astron. Astrophys. \textbf{429}, 371-381.

\item Linz, H., Stecklum, B., Henning, Th., Hofner, P., Brandl, B. (2005)
The G9.62+0.19-F Hot Molecular Core. The Infrared View on Very Young Massive
Stars, Astron. Astrophys. \textbf{429}, 903-921.

\item Carpenter, J.M., Wolf, S., Schreyer, K., Launhardt, R., Henning,
Th. (2005) Evolution of Cold Circumstellar Dust Around Solar-Type Stars,
Astron. J. \textbf{129}, 1049-1062.

\item Semenov, D., Pavlyuchenko, Y., Schreyer, K., Henning, Th., Dullemond,
C., Bacmann, A. (2005) Millimeter Observations and Modeling of the AB
Aurigae System, Astrophys. J. \textbf{621}, 853-874.

\item Cl\'ement, D., Mutschke, H., Klein, R., J\"ager, C., Dorschner, J.,
Sturm, E., Henning, Th. (2005) Detection of Silicon Nitride Particles in
Extreme Carbon Stars, Astrophys. J. \textbf{621}, 985-990.

\item Apai, D., T{\'o}th, L.V., Henning, Th., Vavrek, R., Kov{\'a}cs, Z.,
Lemke, D. (2005) HST/NICMOS Observations of a Proto-Brown Dwarf Candidate,
Astron. Astrophys. \textbf{433}, L33-L36.

\item Steinacker, J., Bacmann, A., Henning, Th., Klessen, R., Stickel,
M. (2005) 3D Continuum Radiative Transfer in Complex Dust
Configurations. II. 3D Structure of the Dense Molecular Cloud Core $\rho$
Oph D, Astron. Astrophys. \textbf{434}, 167-180.

\item Umbreit, S., Burkert, A., Henning, Th., Mikkola, S., Spurzem,
R. (2005) The Decay of Accreting Triple Systems as Brown Dwarf Formation
Scenario, Astrophys. J. \textbf{623}, 940-951.

\item Gouliermis, D., Brandner, W., Henning. Th. (2005) The Initial Mass
Function toward the Low-Mass End in the Large Magellanic Cloud with Hubble
Space Telescope WFPC2 Observations, Astrophys. J. \textbf{623}, 846-859.

\item Apai, D., Linz, H., Henning, Th., Stecklum, B. (2005) Infrared
Portrait of the Star-Forming Region IRAS 09002-4732,
Astron. Astrophys. \textbf{434}, 987-1003.

\item Masciadri, E., Mundt, R., Henning, Th., Alvarez, C., Barrado y
Navascu\'es, D. (2005) A Search for Hot Massive Extrasolar Planets around
Nearby Young Stars with the Adaptive Optics System NACO,
Astrophys. J. \textbf{625}, 1004-1018.

\item Carmona, A., van den Ancker, M.E., Thi, W.-F., Goto, M., Henning, Th.
(2005) Upper Limits on CO 4.7 $\mu$m Emission from Disks around Five Herbig
Ae/Be Stars, Astron. Astrophys. \textbf{436}, 977-982.

\item Wang, H., Stecklum, B., Henning, Th. (2005) New Herbig-Haro Objects in
the L1617 and L1646 Dark Clouds, Astron. Astrophys. \textbf{437}, 169-175.

\item Schartmann, M., Meisenheimer, K., Camenzind, M., Wolf, S., Henning,
Th. (2005) Towards a Physical Model of Dust Tori in Active Galactic
Nuclei. Radiative Transfer Calculations for a Hydrostatic Torus Model,
Astron. Astrophys. \textbf{437}, 861-881.

\item Pfalzner, S., Umbreit, S., Henning, Th. (2005) Disk-Disk Encounters
between Low-Mass Protoplanetary Accretion Discs, Astrophys. J. \textbf{629},
526-534.

\item Kim, J.S., Hines, D.C., Backman, D.E., Hillenbrand, L.A., Meyer, M.R.,
Rodmann, J., Moro-Martin, A., Carpenter, J.M., Silverstone, M.D., Bouwman,
J., Mamajek, E.E., Wolf, S., Malhotra, R., Pascucci, I., Najita, J.,
Padgett, D.L., Henning, Th., Brooke, T.Y., Cohen, M., Strom, S.E., Stobie,
E.B., Engelbracht, C.W., Gordon, K.D., Misselt, K., Morrison, J.E.,
Muzerolle, J., Su, K.Y.L. (2005) Formation and Evolution of Planetary
Systems: Cold Outer Disks Associated with Sun-like Stars,
Astrophys. J. \textbf{632}, 659-669.

\item Boudet, N., Mutschke, H., Nayral, C., J\"ager, C., Bernard, J.-Ph.,
Henning, Th., Meny, C. (2005) Temperature Dependence of the Submillimeter
Absorption Coefficient of Amorphous Silicate Grains,
Astrophys. J. \textbf{633}, 272-281.

\item Apai, D., Pascucci, I., Bouwman, J., Natta, A., Henning, Th.,
Dullemond, C.P. (2005) The Onset of Planet Formation in Brown Dwarf Disks,
Science \textbf{310}, 834-836.

\item Klein, R., Posselt, B., Schreyer, K., Forbrich, J., Henning,
Th. (2005) A Millimeter Continuum Survey for Massive Protoclusters in the
Outer Galaxy, Astrophys. J. Suppl. Ser. \textbf{161}, 361-393.

\item Voshchinnikov, N.V., Il'in, V.B., Henning, Th., Dubkova, D.N.: Dust
Extinction and Absorption (2006) The Challenge of Porous Grains,
Astron. Astrophys. \textbf{445}, 167-177.

\item Chen, X.P., Henning, Th., Boekel, R., Grady, C.A.  (2006) VLT/NACO
  Adaptive Optics Imaging of the Herbig Ae Star HD 100453,
  Astron. Astrophys. \textbf{445}, 331-335.

\item Gouliermis, D., Brandner, W., Henning. Th. (2006) The Low-Mass
Pre-Main-Sequence Population of the Stellar Association LH52 in the Large
Magellanic Cloud Discovered with Hubble Space Telescope WFPC2 Observations,
Astrophys. J. \textbf{636}, L133-L136.


\item Johansen, A., Klahr, H., Henning, Th. (2006) Gravoturbulent Formation
of Planetesimals, Astrophys. J. \textbf{636}, 1121-1134.

\item Rodmann, J., Henning, Th., Chandler, C.J., Mundy, L.G., Wilner,
D.J.(2006) Large Dust Particles in Disks around T Tauri Stars,
Astron. Astrophys. \textbf{446}, 211-221.

\item Schreyer, K., Semenov, D., Henning, Th., Forbrich, J. (2006) A
Rotating Disk around the Very Young Massive Star AFGL 490,
Astrophys. J. \textbf{637}, L129-L132.

\item Hines, D.C., Backman, D.E., Bouwman, J., Hillenbrand, L.A., Carpenter,
J.M., Meyer, M.R., Kim, J.S., Silverstone, M.D., Rodmann, J., Wolf, S.,
Mamajek, E.E., Brooke, T.Y., Padgett, D.L., Henning, Th., Moro-Martin, A.,
Stobie, E., Gordon, K.D., Morrison, J.E., Muzerolle, J., Su, K.Y.L. (2006)
The Formation and Evolution of Planetary Systems (FEPS): Discovery of an
Unusual Debris System Associated with HD 12039, Astrophys. J. \textbf{638},
1070-1079.

\item
Silverstone, M.D., Meyer, M.R., Mamajek, E.E., Hines, D. C., Hillenbrand,
L.A., Najita, J., Pascucci, I., Bouwman, J., Kim, J.S., Carpenter, J.M.,
Stauffer, J.R., Backman, D.E., Moro-Martin, A., Henning, Th., Wolf, S.,
Brooke, T.Y., Padgett, D. L. (2006) Formation and Evolution of Planetary
Systems (FEPS): Primordial Warm Dust Evolution from 3-30 MYR around Sun-Like
Stars Astrophys. J. \textbf{639}, 1138-1146.

\item Koike, C., Mutschke, H., Suto, H., Naoi, T., Chihara, H., Henning,
Th., J{\"a}ger, C., Tsuchiyama, A., Dorschner, J., Okuda, H. (2006)
Temperature Effects on the Mid-and Far-infrared Spectra of Olivine
Particles, Astron. Astrophys. \textbf{449}, 583-596.

\item \'Abrah{\'a}m, P., Mosoni, L., Henning, Th., K\'osp\'al, \'A.,
Leinert, Ch., Quanz, S.P., Ratzka, Th. (2006) First AU-scale Observations of
V1647 Orionis with VLTI/MIDI, Astron. Astrophys. \textbf{449}, L13-L16.

\item Gouliermis, D., Brandner, W., Henning, Th.: The Low-mass Initial Mass
Function of the Field Population in the Large Margellanic Cloud with Hubble
Space Telescope WFPC2 Observations.  Astrophys. J. (2006), in press.

\item Johansen, A., Henning, Th., Klahr, H.: Dust Sedimentation and
Self-Sustained Kelvin-Helmholtz Turbulence in Protoplanetary Disk
Mid-Planes, Astrophys. J. (2006), in press.

\item Mo\'or, A., \'Abrah{\'a}m, P., Derekas, A., Kiss, Cs., Kiss, L.L.,
Apai, D., Grady, C., Henning, Th.: Nearby Debris Disk Systems with High
Fractional Luminosity Reconsidered, Astrophys. J. (2006), in press.

\item Wang, H., Henning, Th.: A Search for Optical Outflows from Brown
Dwarfs in the Chamaeleon I Molecular Cloud, Astrophys. J. (2006), in press.

\item Puga, E., Feldt, M., Alvarez, C., Henning, Th., Apai, D., Le Coarer,
E., Chalabaev, A., Stecklum, B.: Outflows, Disks and Stellar Content in a
Region of High-Mass Star Formation: G5.89-0.39 with Adaptive Optics,
Astrophys. J. (2006), in press.

\item Janson, M., Brandner, W., Henning, Th., Zinnecker, H.: Early ComeOn+
Adaptive Optics Observation of GQ Lup and its Substellar Companion,
Astron. Astrophys. (2006), in press.

\item Steinacker, J., Bacmann, A., Henning, Th.: Ray-Tracing for Complex
Astrophysical High-Opacity Structures, Astrophys. J. (2006), in press.

\item Pavlyuchenkov, Y., Wiebe, D., Launhardt, R., Henning, Th.: CB17:
Inferring the Dynamical History of a Prestellar Core with Chemo-Dynamical
Models, Astrophys. J. (2006), in press.

\end{ownpubl}
\cleardoublepage
%
%--------------------------------------------------------------------
%
\renewcommand{\cvhdr}{Klahr}
\section*{H.\ Klahr}
\subsection*{Curriculum Vitae}
\begin{cvlayout}
\item[Name:\hfill]        
    KLAHR, Hermann Hubertus 
\item[Address:\hfill]
    Max-Planck-Institut f\"ur Astronomie, K\"onigstuhl 17\\
    67245 Heidelberg\\
    Phone / Fax: 06221-528255 / 06221-528246\\
    E-mail: klahr@mpia.de
\item[Date and Place of Birth:\hfill]
    July 21, 1966, Bad Kreuznach, Germany
\item[Nationality:\hfill]
    German
\item[Present position:\hfill]
    Research Scientist (wissenschaftlicher Angestellter)
\item[Background:\hfill]
    Diploma in Physics (1994), University of Karlsruhe\\
    Dissertation in Physics (Ph.D.) (1998), Universtity of Jena\\
%    Habilitation (1920), Faculty of Physics, Univ.\ T\"ubingen
\item[Positions:\hfill]
     PhD student, Max-Planck Research Unit, ``Dust in Star Forming Regions'',  Jena (1994-1997)\\
     PostDoc, Univ. of Jena (1998)\\
     PostDoc, UCO/Lick Observatory, Santa Cruz (1998-2000)\\
     Assistent, Univ. of T\"ubingen (2000-2002)\\
\item[Awards/Fellowships:\hfill]
    NASA Origins research grant fellowship: NAG5-9526 (2000-2003)\\
\end{cvlayout}

\subsection*{Research interests and achievements}
\subsubsection*{Theory of Planet and Star Formation (Dust, Protoplanetary Disks, Giant Planets):}
I studied the effects of turbulent features (vortices, density maxima and convection cells)
on the particle concentration in various papers, both in the context of protoplanetary and
debris disks. I also showed that the time for giant planet formation can be reduced in the
presence of such features.
\subsubsection*{Astrophysical Fluid Dynamics and Super Computing:}
I developed a 3D Radiation Hydro code with van Leer and with 
peace-wise parabolic advection, used for various astrophysical applications, including the
first 3D Radiation Hydro simulations of planet disk interaction.
\subsubsection*{Theory of Accretion Disks: magneto and radiation hydro:}
I found accretion disks to develop and maintain long lived vortices.
I showed by analytical means that a radial entropy gradient in the accretion disk
is usually not sufficient for the formation of vortices. We concluded that
convection (just like other kind of turbulence which is not producing an effective viscosity)
leads to vortices by creating a flow unstable to the  Rossby wave instability.
I also conducted several studies on the diffusion properties of 
magnetohydrodynamically unstable flows.

\subsection*{List of refereed publications in past 5 years}
\begin{ownpubl}

\item
Klahr, H., Rozyczka, M., Dziourkevitch, N., W\"unsch, R. and  
Johansen, A. (2006) Turbulence in Protoplanetary Accretion Disks:  Driving
Mechanisms and Role in Planet Formation. In: \textit{Planet Formation: Theory,
Observation and  Experiments} (Eds. H.\ Klahr and W.\ Brandner), (in press)

\item Johansen, A., Klahr, H. and Mee, A.J. (2006)
Diffusion properties of magnetorotational turbulence in accretion disks:
Effects of an imposed magnetic field. \mn, in press,
ArXiv Astrophysics e-prints arXiv:astro-ph/0603765

\item
Johansen, A., Henning, T., Klahr, H.\ (2006) Dust Sedimentation and Self-Sustained Kelvin-Helmholtz Turbulence in
Protoplanetary Disc Mid-Planes.\ Astrophysical Journal, in press (astro-ph/0512272)

\item
de Val-Borro, M., Edgar, R., Artymowicz, A., Ciecielag, P., Cresswell, P., 
D�Angelo, G., Delgado-Donate, E., Dirksen, G., Fromang, S., Gawryszczak, A., 
Klahr, H., Kley, W., Lyra, W., Masset, F., Mellema, G., Nelson, R., 
Paardekooper, S.-J., Peplinski, A., Pierens, A., Plewa, T., Rice, K., Sch\"afer, C., 
Speith, R.\ (2006) A comparative study of disc-planet interaction.\ \mn, in press

\item
W{\"u}nsch, R., Gawryszczak, A., Klahr, H., R{\'o}{\.z}yczka, M.\ 2006.\
Two-dimensional models of layered protoplanetary discs - II. The effect of a
residual viscosity in the dead zone.\ \mn \textbf{159}, 773--780

\item
Johansen, A., Klahr, H. and  Henning, Th. (2006) Gravoturbulent formation of
planetesimals. \apj,  {\bf 636}, 1121--1134

\item
Klahr, H.  and Kley, W. (2006) 3D-Radiation Hydro Calculations of Disk-Planet
Interaction. \aap, {\bf 445}, 747--758

\item
Klahr, H. and Bodenheimer, P. (2006) Planet Formation via Core Accretion in a
Vortex. \apj, {\bf 639}, 432-440

\item
Johansen, A. and Klahr, H. (2005) Dust diffusion in protoplanetary discs by
magnetorotational turbulence. \apj,  {\bf 634}, 1353--1371

\item
W\"unsch, R., Klahr, H. and Rozyczka, M. (2005) 2-D models of layered
protoplanetary disks: I. The ring  instability. \mn, 362, 361

\item
Klahr H. and Lin, D.~N.~C. (2005) Dust Distribution in Gas Disks II: Self Induced
Ring Formation through a Clumping Instability. \apj, \textbf{632}, (in press)

\item
Klahr, H. (2004) The Global Baroclinic Instability in Accretion Disks. II: Local
Linear Analysis. \apj, \textbf{606}, 1070

\item
Klahr, H. and Bodenheimer, P. (2003) Turbulence in Accretion Disks. Vorticity
Generation and Angular Momentum Transport via the Global Baroclinic Instability.
\apj, \textbf{582}, 869 

\item
Wolf, S. and Klahr, H. (2002) Large-scale Vortices in Protoplanetary Disks:
On the Observability of possible early stages of Planet Formation, \apj,
\textbf{578}, L79

\item
Klahr, H. and Lin, D.N.C. (2001) Dust Distribution in Gas Disks. A Model for the
Ring Around HR 4796A. \apj, \textbf{ 554}, 1095

\end{ownpubl}
\cleardoublepage
%
%--------------------------------------------------------------------
%
\renewcommand{\cvhdr}{Kley}
\section*{W.\ Kley}
\subsection*{Curriculum Vitae}
\begin{cvlayout}
\item[Name:\hfill]
    KLEY, Wilhelm
\item[Address:\hfill]
    Institut f\"ur Astronomie und Astrophysik\\
    Universit\"at T\"ubingen\\
    Auf der Morgenstelle 10\\
    72076 T\"ubingen\\
    Phone / Fax: 07071-2972043 / 07071-295889\\
    E-mail: wilhelm.kley@uni-tuebingen.de
\item[Date and Place of Birth:\hfill]
    February 19, 1958, Soest, Germany
\item[Nationality:\hfill]
    German
\item[Present position:\hfill]
    Full Professor
\item[Background:\hfill]
    Studies in Physics and Astronomy at the Universities
    Bochum, Sussex, and M\"unchen (1978-1985)\\
    Physics Diploma (1985), Univ.~M\"unchen\\
    Dissertation in Physics (Ph.D.) (1988), Univ.~M\"unchen
\item[Positions:\hfill]
    Univ.~M\"unchen (1986-1990)\\
    Univ. of Santa Cruz (1990-1993)\\
    Univ. of London (1992-1993)\\
    Max-Planck Arbeitsgruppe Gravitationstheorie, Jena (1993-1996)\\
    Univ.\ of Jena (1996-1999)\\
    Max-Planck Institute for Astronomie, Heidelberg (1999-2000)\\
    Univ.~T\"ubingen (2000-2006)
\end{cvlayout}

\subsection*{Research interests and achievements}
\subsubsection*{Planet Formation}
Protoplanetary disks, planet-disk interaction, formation of resonant planets,
planets in binary stars
\subsubsection*{Accretion disk theory}
Structure, evolution and dynamics of accretion disks
\subsubsection*{Numerical Methods}
Numerical hydrodynamics, radiation transport, relativistic hydrodynamics, 
magnetohydrodynamics
\subsubsection*{Relativistic Astrophysics}
Oscillations and mergers of neutron stars


\subsection*{List of refereed publications in past 5 years}
\begin{ownpubl}

\item
S\'andor, Z. and Kley W. (2006)
On the evolution of the resonant planetary system HD~128311.
  \aap, Letters, in press

\item
Kley, W. and Dirksen G. (2006)
Disk eccentricity and embedded planets.
  \aap, \textbf{447}, 369

\item
Klahr, H.H., and  Kley, W. (2006)
 3D-radiation hydro simulations of disk-planet interactions
  - I. Numerical algorithm and test cases. \aap, \textbf{445}, 747

\item
Kley, W., Lee, M.-H., Murray, N., and Peale, S. (2005) Modeling the resonant
planetary system GJ 876. \aap, \textbf{437}, 727

\item
Kley, W., Peitz, J. and Bryden, G. (2004) Evolution of Planetary Systems in
Resonance. \aap, \textbf{414}, 735

\item
Sch\"afer, Chr., Speith, R., Hipp, M. and Kley, W. (2004) Simulations of
planet-disc interactions using Smoothed Particle Hydrodynamics. \aap,
\textbf{418}, 325

\item
G\"unther, R., Sch\"afer, Chr. and Kley, W. (2004) Evolution of irradiated
Circumbinary Disks. \aap, \textbf{423}, 559

\item
Speith, R. and Kley, W. (2003) Stability of the viscously spreading ring. \aap,
\textbf{ 399}, 395

\item
D'Angelo, G., Kley, W. and Henning, Th. (2003) Orbital Migration and Mass
Accretion of Protoplanets in Three-dimensional Global Computations with Nested
Grids. \apj, \textbf{586}, 540

\item
Dreizler, S., Hauschildt, P.~H., Kley, W., Rauch, T., Schuh, S.~L.,
Werner, K, Wolff, B. (2003)
OGLE-TR-3: A possible new transiting plane. \aap, \textbf{402}, 791

\item
D'Angelo, G., Henning, Th. and Kley, W. (2003) Thermo-Hydrodynamics of
Circumstellar Disks with High-mass Planets. \apj, \textbf{599}, 548

\item
G\"unther, R. and Kley W. (2002) Circumbinary Disk evolution. \aap, \textbf{387},
550

\item
Dreizler, S., Rauch, T., Hauschildt, P., Schuh, S.~L., Kley, W.,
Werner, K. (2002)
Spectral types of planetary host star candidates: Two new transiting planets?
\aap, \textbf{391}, 17

\item
Kley, W., D'Angelo G. and Henning, Th. (2001) Three-Dimensional Simulations of a
Planet Embedded in a Protoplanetary Disk. \apj, \textbf{547}, 457

\item
D'Angelo G., Henning Th. and Kley, W. (2001) Nested-Grid calculations of
planet-disk interaction. \aap, \textbf{385}, 647

\item
Wolf, S., Gueth, F., Henning, Th. and Kley, W. (2001) Detecting planets in
protoplanetary disks: A prospective study. \apj, \textbf{566}, L97

\end{ownpubl}
\cleardoublepage
%
%--------------------------------------------------------------------
%
\renewcommand{\cvhdr}{Lattard}
\section*{D.\ Lattard}
\subsection*{Curriculum Vitae}
\begin{cvlayout}
\item[Name:\hfill]
    LATTARD, Dominique
\item[Address:\hfill]
    Mineralogisches Institut, INF 236\\
    Universit\"at Heidelberg\\
    Im Neuenheimer Feld 236\\
    69120 Heidelberg\\
    Germany\\
    Phone / Fax: +49-6221-544810 / 544805\\
    E-mail: dlattard@min.uni-heidelberg.de
\item[Date and Place of Birth:\hfill]
    April 4, 1950, Nesle (Somme), France
\item[Nationality:\hfill]
    French
\item[Present position:\hfill]
    University Professor (C3)
\item[Background:\hfill]
    Studies in Geology, Mineralogy at Universit\'e Paris VI  (1967/72)\\
    Diploma  (1972)\\
    Doctorat de 3i\`eme cycle (1974; Universit\'e Paris VI)\\
    Dr. rer.nat. (1980, Univ. Bochum)\\
    Habilitation  (1994)  (Technical Univ. Berlin)
\item[Positions:\hfill]
    Univ. Pierre et Marie Curie, Paris (1973-75)\\
    Research fellow at Ruhr-Universit\"at Bochum (1974)\\
    Employed by Ruhr-Universit\"at Bochum (1975-1982)\\
    Employed by RWTH Aachen (1983-1986)\\
    Guest lecturer at Universit\"at Kiel (1986-1989)\\
    Employed by TU Berlin (1989-1996)\\
    Univ. Heidelberg (since 1996), Professor (Experimental Mineralogy)\\
    Studiendekanin, Faculty of Earth Sciences (1999/2002)\\
    Frauenbeauftragte, Faculty of Earth Sciences (2000/2002)\\
    Gleichstellungsbeauftragte, Univ. Heidelberg (2002- 2005)\\
    Associate Editor of European Journal of Mineralogy (since 1998)\\
    Member of Curatorium of the GeoForschungsZentrum Potsdam (1999-2005)\\
    Member Selection Committee for the Mineralogical Society of America Award
\end{cvlayout}

\subsection*{Research interests and achievements}
\subsubsection*{Stability conditions of minerals from the Earth's crust and Earth's mantle}
High-temperature, high-pressure experimental investigations of the
stabilities of subduction zone minerals.  Effect of redox conditions
on the pressure-temperature stability of iron-bearing minerals.

\subsubsection*{Experimental modelling of the differentiation of basaltic magmas, with emphasis on the redox conditions with relevance for the Earth and other planets:}
Differentiation of basaltic magmas in closed system conditions. Iron
redox state in basaltic melts.

\subsubsection*{Development and improvement of geothermo-barometers for the Earth and the Moon:}
Temperature-dependency of the Zr content in Fe-Ti oxides with relevance to
lunar petrology. New calibration of the Fe-Ti oxide thermo-oxybarometer.

\subsubsection*{Crystal-chemical properties of silicates and oxides:}
Crystal-chemical effects of Mn3+ in silicates.
High-temperature cation vacancies in titanomagnetite.
Magnetic properties of Fe-Ti oxides. 

\subsubsection*{Field and laboratory investigations of metamorphic rocks}
Subduction-related metamorphism in the Western Alps (Doctorat de 3i\`eme
cycle) Pan-African metamorphism in West-Sudan.

\subsubsection*{Methods:}
High-pressure high-temperature experiments under controlled redox conditions.
Optical microscopy, X-ray diffraction methods, IR and M\"ossbauer spectroscopy
Microanalytical methods (Scanning electron Microscope, Electron Microprobe)
Petrological field investigations 


\subsection*{List of refereed publications in past 5 years}
\begin{ownpubl}
\item Lattard, D., Sauerzapf, U., K\"asemann, M. (2005): New calibration
data for the Fe-Ti oxide thermo-oxybarometers from experiments in the
Fe-Ti-O system at 1 bar, 1000-1300 $^o$C and a large range of oxygen
fugacities. Contributions to Mineralogy and Petrology, \textbf{149}, 535-574.
\item Wilke, M., Partzsch, G.M., Bernhardt, R., Lattard, D. (2004):
Determination of the iron oxidation state in basaltic glasses using
XANES at the K-Edge. Chemical Geology, \textbf{213}, 71-87.  Erratum:
Chemical Geology, \textbf{220}, 143-161.
\item Partzsch, G.M., Lattard, D., McCammon, C. (2004): M\"ossbauer
spectroscopic determination of Fe3+/Fe2+ in synthetic basaltic glass: a test
of empirical fO2 equations under superliquidus and subliquidus
conditions. Contributions to Mineralogy and Petrology, \textbf{147}, 565-580.
\item Lattard, D. (2001): Comments on ``Mexican peridotite xenoliths and
tectonic terranes: Correlations among location, texture, temperature,
pressure and oxygen fugacity'' by J.F. Luhr \& J.J. Aranda-Gomez
(1997). Journal of Petrology, \textbf{42}, 847-851.
\item Lattard, D., \& Partzsch, G.M. (2001): Magmatic crystallization
experiments at 1 bar in systems closed to oxygen: A new/old experimental
approach. European Journal of Mineralogy, \textbf{13}, 467-478.
\item Lattard, D. (2001): Ein R\"uckblick auf 150 Jahre Geschichte der
Mineralogie an der Ruprecht-Karls-Universit�t Heidelberg
(1817-1967). Berichte der Deutschen Mineralogischen Gesell\-schaft. Beihefte
zum European Journal of Mineralogy, \textbf{13 (1)}, 1-14.
\end{ownpubl}
\cleardoublepage
%
%--------------------------------------------------------------------
%
\renewcommand{\cvhdr}{Pucci}
\section*{A.\ Pucci}
\subsection*{Curriculum Vitae}

\begin{cvlayout}
\item[Name:\hfill]        
    PUCCI, Annemarie
\item[Address:\hfill]
    Kirchhoff Institute of Physics, Im Neuenheimer Feld 227\\
    69120 Heidelberg\\
    Phone / Fax: 06221-549863 / 06221-549869\\
    E-mail: pucci@kip.uni-heidelberg.de
\item[Date and Place of Birth:\hfill]
    May 31, 1954, Weimar, Germany
\item[Nationality:\hfill]
    German
\item[Present position:\hfill]
    Univ. Prof.
\item[Background:\hfill]
    Studies in Physics at the 
    Friedrich-Schiller University Jena (1972-1977)\\
    Diploma in Physics (1977), Friedrich-Schiller University of Jena\\
    Dissertation in Physics (Dr. rer. nat.) (1983), Rostock University\\
    Habilitation (1992), Friedrich-Schiller University Jena\\
\item[Positions:\hfill]
    Dept. of Physics, Rostock University (1977-1986)\\
    Agricultural College\ Weimar (1986-1987)\\
    Institute of Solid-State Physics, University of Jena (1987-1991)\\
    Dept. of Physics, Free University Berlin (1991-1995)\\
    Institute of Applied Physics (later Kirchhoff Institute of Physics), University of Heidelberg (since 1995)\\

\item[Awards/Fellowships:\hfill]
    Honorary Fellow of the G. Daimler and C. Benz Foundation (2004)\\
    \end{cvlayout}


\subsection*{Research interests and achievements}
\subsubsection*{Growth and dielectric function studies of ultrathin films und nanostructures
with infrared spectroscopy}
Ultra-high vacuum (UHV) studies of the infrared (IR) and dc conductivity
during the formation of metal films, preparation of well defined morphologies,
classical and quantum-size effects of the conductivity, antenna resonances
of metal nanowires, vibrational spectroscopy of semiconducting and
insulating films, studies of adsorbate and temperature induced changes,
supplementary morphologic studies with atomic force microscopy. 

\subsubsection*{SEIRA and SERS}
Surface enhanced infrared absorption (SEIRA) of adsorbate layers on
ultrathin metal films studied under ultra-high vacuum (UHV) conditions,
correlation to surface enhanced Raman scattering (SERS), adsorbates
on rough metal surfaces, in-situ investigation of SEIRA of biological
molecules in liquid environment. 

\subsubsection*{HREELS}
High resolution electron-energy loss spectroscopy (HREELS) of adsorbate
vibrations in relation to surface roughness.

\subsection*{List of refereed publications in past 5 years}
\begin{ownpubl}

\item 
Siemes, C., Bruckbauer, A., Goussev, A., Otto, A., Sinther, M.  and Pucci, A. (2001) 
SERS-active sites on various copper substrates.
\textit{Journal of Raman Spectroscopy\/}, \textbf{32}, 231

\item 
Fahsold, G., Singer,  K. and Pucci, A. (2001) 
In-situ IR-transmission study of vibrational and electronic properties during 
the formation of thin-film �-FeSi2.
\textit{Journal of Applied Physics\/}, \textbf{91}, 145

\item 
Sinther, M., Pucci, A., Otto, A., Priebe, A., Diez, S., and Fahsold, G. (2001) 
Enhanced Infrared Absorption of SERS-active lines of ethylene on Cu.
\textit{physica status solidi (a)\/}, \textbf{188}, 1471

\item 
Priebe, A., Fahsold, G., Geyer, W., and Pucci, A. (2002) 
Enhanced infrared Absorption of CO on smooth iron ultrathin films in correlation 
to their crystalline quality.
\textit{Surface Science\/}, \textbf{502}, 388

\item 
Lust, M., Priebe, A., Fahsold, G., and Pucci, A. (2002) 
SERS-active sites on various copper substrates.Infrared spectroscopic study 
of the CO-mediated decrease of the percolation threshold during the growth 
of ultrathin metal films on MgO(001).
\textit{Surface and Interface Analysis\/}, \textbf{33}, 487

\item 
Fahsold, G., Sinther, M., Priebe, A., Diez, S., and Pucci, A. (2002) 
Adsorbate-induced changes in the broadband infrared transmission of ultrathin 
metal films.
\textit{Phys. Rev. B\/}, \textbf{65}, 235408

\item 
Fahsold, G., Priebe, A., Magg, N., and Pucci, A. (2003) 
Non-contact measurement of conductivity during growth metal ultrathin films.
\textit{Thin Solid Films\/}, \textbf{428}, 107

\item 
Priebe, A., Sinther, M., Fahsold, G., and Pucci, A. (2003) 
The correlation between film thickness and adsorbate line shape in surface 
enhanced infrared absorption.
\textit{J. of Chemical Physics\/}, \textbf{119}, 4887

\item
Fahsold, G.  and Pucci, A. (2003) Non-contact measurement of thin-film 
conductivity by ir spectroscopy In: \textit{Advances in Solid State Physics },
\textbf{39} (Ed. B. Kramer). Springer Press, p. 833

\item 
Yaginuma, S., Nagao, T., Sadowski, J. T. , Pucci, A., Fujikawa, Y., 
and Sakurai, T. (2003) 
Surface pre-melting and surface flattening of Bi nanofilms on Si(1 1 1)-7x7.
\textit{Surface Science\/}, \textbf{547}, L877

\item 
Fahsold, G., Sinther, M., Priebe, A., Diez, S., and Pucci, A. (2004)
Influence of thin film morphology on the adsorbate-induced changes in the 
broadband infrared transmission.
\textit{Phys. Rev. B\/}, \textbf{70}, 115406

\item 
Priebe, A., Fahsold, G., and Pucci, A. (2004) 
Strong pyramidal growth of metal films studied with infrared spectroscopy.
\textit{J. Phys. Chemistry B\/}, \textbf{8}, 18174

\item 
Pucci, A. (2005) 
IR spectroscopy of  adsorbates on ultrathin metal films.
\textit{physica status solidi (b)\/}, \textbf{242}, 2740
\item 

Meng, F., Fahsold, G., and Pucci, A. (2005) 
Growth of silver on MgO(001) and IR optical properties.
\textit{physica status solidi (c)\/}, \textbf{2}, 3963

\item 
Hein, M., Dumas, P., Sinther, M., Priebe, A., Bruckbauer, A., Pucci, A. 
and  Otto, A. (2006) 
Relation between surface resistance, infrared-, surface enhanced infrared- and 
Raman spectroscopies of CO and C2H4 on copper.
\textit{, Surface Science\/}, \textbf{600}, 1017

\item 
Lust, M., Pucci, A., and  Otto, A. (2006) 
SERS and Infrared Reflection-Absorption spectroscopy of NO on cold-deposited Cu.
\textit{Journal of Raman Spectroscopy\/}, \textbf{37}, in press

\item 
Priebe, A., Pucci, A., and Otto, A. (2006) 
Infrared reflection-absorption spectra of C2H4 and C2H6 on Cu: effect of surface roughness.
\textit{J. of Physical Chemistry B\/}, \textbf{110}, 1673

\item 
Cornelius, T. W., Toimil-Molares, M. E., Lovrincic, R., Karim, S., 
Neumann, R., Pucci, A. and Fahsold, G. (2006) 
Quantum size effects manifest in infrared spectra of single bismuth nanowires.
\textit{Applied Physics Letters\/}, \textbf{88}, 103114

\item 
Kolb, T., Kost, F., Neubrech, F., Toimil-Molares, M.E., Cornelius, T.,
Neumann, R., Pucci, A., Fahsold, G. (2006) 
IR spectroscopy and preparation of nanoslits in metal thin films.
\textit{, Infrared Physics \& Technology\/}, in press

\item 
Priebe, A., Meng, F., and Pucci, A. (2006) 
IR spectra of adsorbates on rough metal films.
\textit{Asian J. Phys.\/}, \textbf{15}, 239

\item 
Priebe, A., Pucci, A., Akemann, W., Grabhorn, H., Otto, A. (2006) 
Staggered ethane changes to eclipsed conformation upon adsorption.
\textit{Journal of Raman Spectroscopy\/}, in press

\item 
Enders, D., Nagao, T., Pucci, A., and Nakayama, T. (2006) 
Reversible adsorption of Au nanoparticles on SiO2/Si �An in-situ ATR-IR study.
\textit{Surface Science\/}, \textbf{600}, L71

\item 
Enders, D. and Pucci, A. (2006) 
Surface enhanced infrared absorption of octadecanethiol 
on wet-chemically prepared Au nanoparticle films.
\textit{Applied Physics Letters\/}, in press

\end{ownpubl}
\cleardoublepage
%
%--------------------------------------------------------------------
%
\renewcommand{\cvhdr}{Speith}
\section*{R.\ Speith}
\subsection*{Curriculum Vitae}
\begin{cvlayout}
\item[Name:\hfill]        
    SPEITH, Roland
\item[Address:\hfill]
    Institut f\"ur Astronomie und Astrophysik\\
    Universit\"at T\"ubingen\\
    Auf der Morgenstelle 10\\
    72076 T\"ubingen\\
    Phone / Fax: 07071-2972043 / 07071-295889\\
    E-mail: speith@tat.physik.uni-tuebingen.de
\item[Date and Place of Birth:\hfill]
    June 23, 1966, Hamburg, Germany
\item[Nationality:\hfill]
    German
\item[Present position:\hfill]
    Postdoctoral Research Fellow
\item[Background:\hfill]
    Studies in Physics at the Universities 
    Karlsruhe and T\"ubingen (1987-1994)\\
    Physics Diplom (1994), Univ.~Karlsruhe\\
    Dissertation in Physics (Ph.D.) (1998), Univ.~T\"ubingen
\item[Positions:\hfill]
    Univ.~Karlsruhe (1989-1992)\\
    Univ.~T\"ubingen (1993-2001)\\
    Univ.\ of Leicester, UK (2001-2003)\\
    Univ.~T\"ubingen (2003-2006)
\end{cvlayout}

\subsection*{Research interests and achievements}
\subsubsection*{Accretion disc theory}
Simplified accretion disc modells; General stability
\subsubsection*{Accretion discs in binaries}
Cataclysmic Variables; Superhumps; Stream-disc interaction
\subsubsection*{Protoplanetary accretion discs}
Planet-disc interaction; Pre-planetesimal collisions
\subsubsection*{Computational fluid dynamics}
Algorithms; Smoothed Particle Hydrodynamics

\subsection*{List of refereed publications in past 5 years}
\begin{ownpubl}

\item
Matthews, O.M., Speith, R., Truss, M.R., Wynn, G.A. (2005)
The steady state structure of accretion discs in central magnetic
fields. \mn, \textbf{356}, 66

\item
Rosswog, S., Speith, R., Wynn, G.A. (2004)
Accretion dynamics in neutron star-black hole binaries. 
\mn, \textbf{351}, 1121

\item
Sch\"afer, Chr., Speith, R., Hipp, M. and Kley, W. (2004) Simulations of
planet-disc interactions using Smoothed Particle Hydrodynamics. \aap,
\textbf{418}, 325

\item
Matthews, O.M., Speith, R.,  Wynn, G.A. (2002)
Outbursts of young stellar objects. \mn, \textbf{347}, 873

\item
Speith, R. and Kley, W. (2003) Stability of the viscously spreading ring. \aap,
\textbf{ 399}, 395

\item
Kunze, S., Speith, R., Hessman, F.V. (2001)
Substantial stream-disc overflow found in three-dimensional SPH
simulations of cataclysmic variables.
\mn, \textbf{322}, 499

\end{ownpubl}
\cleardoublepage
%
%--------------------------------------------------------------------
%
\renewcommand{\cvhdr}{Stephan}
\section*{T.\ Stephan}
\subsection*{Curriculum Vitae}
\begin{cvlayout}
\item[Name:\hfill]        
    STEPHAN, Thomas
\item[Address:\hfill]
    Institut f\"ur Planetologie, Wilhelm-Klemm-Str. 10\\ 
    48149 M\"unster\\
    Phone / Fax: 0251 8339050 / 0251 8336301\\
    E-mail: stephan@uni-muenster.de
\item[Date and Place of Birth:\hfill]
    February 27, 1963, Herbolzheim, Germany
\item[Nationality:\hfill]
    German
\item[Present position:\hfill]
    Hochschuldozent
\item[Background:\hfill]
    Abitur, Justus-Liebig-Schule, Darmstadt (1982)\\
    Studium der Physik und der Astronomie, Universit\"at Heidelberg (1982-1987)\\
    Diplom in Physik am Max-Planck-Institut f\"ur Kernphysik/Universit\"at Heidelberg (1987)\\
    Promotion in Physik am Max-Planck-Institut f\"ur 
    Kernphysik/Universit\"at Heidelberg (1989)\\ 
    Habilitation, Westf\"alische Wilhelms-Universit\"at M\"unster 
    venia legendi in Planetologie (2000)
\item[Employed by:\hfill]
    Wissenschaftlicher Mitarbeiter Max-Planck-Institut f\"ur Kernphysik, Heidelberg (1989-1990)\\
    Wissenschaftlicher Mitarbeiter Physikalisches Institut / Institut f\"ur Planetologie, Universit\"at M\"unster (1990-1993)\\
    Wissenschaftlicher Mitarbeiter Max-Planck-Institut f\"ur Kernphysik, Heidelberg (1993-1996)\\
    Wissenschaftlicher Assistent Institut f\"ur Planetologie, Universit\"at M\"unster (1996-2000)\\
    Hochschuldozent Institut f\"ur Planetologie, Universit\"at M\"unster (seit 2000)\\
    Gesch\"aftsf\"uhrender Direktor, Institut f\"ur Planetologie, Universit\"at M\"unster (10/2003�03/2005)
%\item[Awards/Fellowships:\hfill]
\end{cvlayout}
\subsection*{Research interests and achievements}
\subsubsection*{Planetary science, cosmochemistry, astrophysics}
\subsubsection*{Interplanetary, interstellar, and presolar dust, comets, meteorites, Mars}
\subsubsection*{Mass spectrometry, SIMS, TOF-SIMS}

\subsection*{List of refereed publications in past 5 years}
\begin{ownpubl}

\item Wies C., Jessberger E. K., Kl\"ock W., Maetz M., Rost D., Stephan T., Traxel K. and Wallianos A. (2001) Mineral-specific trace element contents of interplanetary dust particles. Nucl. Instr. and Meth. \textbf{B 181}, 539-544.

\item Jessberger E. K., Stephan T., Rost D., Arndt P., Maetz M., Stadermann
F. J., Brownlee D. E., Bradley J. P. and Kurat G. (2001) Properties of
Interplanetary Dust: Information from Collected Samples. In Interplanetary
Dust (eds. E. Gr�n, B. �. S. Gustafson, S. F. Dermott and H. Fechtig),
pp. 253-294. Springer-Verlag, Berlin, Heidelberg, New York.

\item Stephan T. (2001) TOF-SIMS in cosmochemistry. Planet. Space
Sci. \textbf{49}, 859-906

\item Stephan T., Jessberger E. K., Heiss C. H. and Rost D. (2003) TOF-SIMS
analysis of polycyclic aromatic hydrocarbons in Allan Hills
84001. Meteorit. Planet. Sci. \textbf{38}, 109-116.

\item Semenenko V. P., Jessberger E. K., Chaussidon M., Weber I., Stephan
T. and Wies C. (2005) Carbonaceous xenoliths in the Krymka LL3.1 chondrite:
Mysteries and established facts. Geochim. Cosmochim. Acta \textbf{69},
2165-2182.

\item Geisler T., P\"oml P., Stephan T., Janssen A. and Putnis A. (2005)
Experimental observation of an interface-controlled pseudomorphic
replacement reaction in a natural crystalline pyrochlore. Am. Mineral. 
\textbf{90},
1683-1687.

\item Hoppe P., Stadermann F. J., Stephan T., Floss C., Leitner J., Marhas
K. K. and H\"orz F. (2006) SIMS studies of Allende projectiles fired into
Stardust-type aluminum foils at 6 km/s. Meteorit. Planet. Sci., \textbf{41},
197-210.

\item Stephan T., Butterworth A. L., H\"orz F., Snead C. J. and Westphal
A. J (2006) TOF-SIMS analysis of Allende projectiles shot into silica
aerogel. Meteorit. Planet. Sci., \textbf{41}, 211-216

\item Weber I., Semenenko V. P., Stephan T. and Jessberger E. K. (2006) TEM 
studies and the shock history of a "mysterite" inclusion from the Krymka LL 
chondrite. Meteorit. Planet. Sci. \textbf{41}, 571-580.

\item Morlok A., Bischoff A., Stephan T., Floss C., Zinner E. K. and
Jessberger E. K. (2006) Brecciation and chemical heterogeneities of CI
chondrites. Geochim. Cosmochim. Acta, in press.

\end{ownpubl}
\cleardoublepage
%
%--------------------------------------------------------------------
%
\renewcommand{\cvhdr}{Trieloff}
\section*{M.\ Trieloff}
\subsection*{Curriculum Vitae}
\begin{cvlayout}
\item[Name:\hfill]        
    TRIELOFF, Mario
\item[Address:\hfill]
    Mineralogical Institute, Im Neuenheimer Feld 236\\
    69120 Heidelberg\\
    Phone / Fax: 06221-5464805 / 06221-546022\\
    E-mail: trieloff@min.uni-heidelberg.de
\item[Date and Place of Birth:\hfill]
    February 26, 1963, Ludwigshafen/Rh., Germany
\item[Nationality:\hfill]
    German
\item[Present position:\hfill]
    Priv. Doz.
\item[Background:\hfill]
    Studies in Physics and Astronomy at the 
    university of Heidelberg (1982-1990)\\
    Diploma in Physics (1990), MPI f\"ur Kernphysik\\
    Dissertation in Physics (Ph.D.) (1993), MPI f\"ur Kernphysik\\
    Habilitation (1998), Faculty of Geosciences, Univ.\ Heidelberg
\item[Employed by:\hfill]
    Univ. of Jena (1993-1994)\\
    MPI f\"ur Kernphysik (1994-1995)\\
    DFG habilitation stipendiate at MPI f\"ur Kernphysik (1995-1998)\\
    MPI f\"ur Chemie (1998)\\
    DFG Heisenberg stipendiate at Mineralogisches Institut, Univ.\ Heidelberg (1998-2003)\\
    Research fellow at Mineralogisches Institut, Univ.\ Heidelberg (2003-)\\
\item[Awards/Fellowships:\hfill]
    DFG Habilitation Grant (1995)\\
    DFG Heisenberg Grant (1998)\\
    Victor-Moritz-Goldschmidt prize (DMG) 2005
\end{cvlayout}

\subsection*{Research interests and achievements}
\subsubsection*{Evolution of asteroidal-sized planetesimals in the early solar system}
Tracing the thermal evolution of asteroids in the early solar system by using high-precision Ar-40/Ar-39 and Pu-244 fission track thermochronology. Evidence for planetesimal-heating by short-lived isotopes (Al-26), and implications for the accretion of planetesimals and planetary building blocks within the first million years of early solar system history.
\subsubsection*{Cosmochemistry, planetology and terrestrial planet formation}
Identification of early solar wind implanted ions/isotopes in Earth�s mantle, constraints for time scales of the formation of terrestrial planets relative to dissipation of the accretion disk in the early solar system.
\subsubsection*{Noble gas isotopes and the evolution of the Earth}
Using noble gas isotopes as tracers for evolution and structure of the Earth�s mantle, mantle-lithosphere interaction, and formation of the Earth. Constraining the input of extraterrestrial matter into Earth�s mantle.
\subsubsection*{Service to community}
Reviewer for DFG, NASA, private foundations, Nature, Science and 9 international Journals of Earth and Planetary Sciences

\subsection*{List of refereed publications in past 5 years}
\begin{ownpubl}

\item 
Trieloff, M. and Palme, H. (2006) The origin of solids in the early solar system.
In: Planet Formation �" Theory, Observations, and Experiments (Eds. H. Klahr \&
W. Brandner), pp. 64-89, Cambridge University Press.

\item 
Trieloff, M. and Altherr, R. (2006) He-Ne-Ar isotope systematics in Eifel and Pannonian basin mantle xenoliths reveal deep mantle plume-lithosphere interaction beneath the European continent. In: Mantle plumes - A multidisciplinary approach (Eds. J.R.R. Ritter \& U.R. Christensen), Springer, Heidelberg, in press.

\item 
Hopp, J. and Trieloff, M. (2005) Refining the noble gas record of the R\'eunion
mantle plume source: implications on mantle geochemistry.
\epsl, \textbf{240}, 573-588.

\item 
Korochantseva, E.~V., Trieloff, M., Buikin, A.~I., Meyer, H.~P. and Hopp, J. 
(2005) Argon-40/Argon-39 dating, and cosmic ray exposure time of desert
meteorites: Dhofar 300 and Dhofar 007 eucrites and anomalous achondrite NWA 011,
\mps, \textbf{40}, 1433-1454.

\item 
Schwarz, W.~H., Trieloff, M. and Altherr, R. (2005) Subduction of solar type
noble gases from extraterrestrial dust: Constraints from high-pressure 
low-temperature metamorphic deep sea sediments. \cmp, \textbf{149}, 675-684.

\item 
Trieloff, M., Falter, M., Buikin, A.~I., Korochantseva, E.~V., Jessberger,
E.~K., Altherr, R. (2005) Argon isotope fractionation induced by stepwise
heating, \gca, \textbf{69}, 1253-1264.

\item 
Buikin, A.~I., Trieloff, M., Ryabchikov, I.~D. (2005) $^{40}$Ar-$^{39}$Ar dating
of a phlogopite-bearing websterite: Evidence for ancient metasomatism in the
subcontinental lithospheric mantle under the Arabian shield?
\textit{Doklady Earth Sciences\/}, \textbf{400}, 44-48

\item 
Buikin, A.~I., Trieloff, M., Hopp, J., Althaus, T., Korochantseva, E.~V.,
Schwarz, W.~H., Altherr, R.
(2005) Noble gas isotopes suggest deep mantle plume source of late Cenozoic mafic
alkaline volcanism in Europe. \epsl, 
\textbf{230}, 143-162

\item 
Trieloff, M. and Kunz, J. (2005) Isotope systematics of noble gases in the
Earth´s mantle: Possible sources of primordial isotopes and implications for
mantle structure. \textit{Physics of the Earth and Planetary Interiors\/}, 
\textbf{148}, 13-38

\item 
Hopp, J., Trieloff, M. and Altherr, R. (2004)  Neon isotopes in mantle rocks from
the Red Sea region reveal large-scale plume�"lithosphere interaction.
\epsl, \textbf{219}, 61-76

\item 
Trieloff, M., Jessberger, E.~K., Herrwerth, I., Hopp, J., Fi\'eni, C.,
Bourot-Denise, M., Ghelis, M., Pellas, P.
(2003) $^{244}$Pu and $^{40}$Ar-$^{39}$Ar thermochronometries reveal structure
and thermal history of the H-chondrite parent asteroid. \nat, \textbf{422},
502-506. Pressespiegel zur Publikation auf der homepage der Deutschen
Mineralogischen Gesellschaft: 
{\tt http://www.dmg-home.de} (Link ``Presse'')

\item 
Trieloff, M., Falter, M. and Jessberger, E.~K. (2003) The distribution of mantle
and atmospheric argon in oceanic basalt glasses. \gca, \textbf{67}, 1229-1245

\item 
Harrison, D., Burnard, P.~G., Trieloff, M. and Turner, G. (2003) Resolving
Atmospheric Contaminants in Mantle Noble Gas Analyses.
\textit{Geochemistry, Geophysics, Geosystems\/}, \textbf{4}, paper no.
2002GC000325

\item 
Trieloff, M. (2003) Die Entstehung der Erde. In: Weltbilder (Eds. H. Gebhardt and H. Kiesel), Heidelberger Jahrb�cher Bd. 47, 45-70. Springer, Heidelberg.

\item 
Trieloff, M., Kunz, J. and Allègre, C.~J. (2002) Noble gas systematics of the
R\'eunion mantle plume source and the origin of primordial noble gases in Earth´s
mantle. \epsl, \textbf{200}, 297-313

\item 
Trieloff, M., Jessberger, E.~K. and Fi\'eni, C. (2001b) Comment on 
“$^{40}$Ar/$^{39}$Ar age of plagioclase from Acapulco meteorite and the problem
of systematic errors in cosmochronology” by P.~R. Renne. \epsl, \textbf{190},
267-269

\item 
Trieloff, M., Kunz, J., Clague, D.~A., Harrison, D. and All\'egre, C.~J. (2001a)
Noble gases in mantle plumes. \sci, \textbf{291}, 2269

\end{ownpubl}
\cleardoublepage
%
%--------------------------------------------------------------------
%
\renewcommand{\cvhdr}{Tscharnuter}
\section*{W.M. Tscharnuter}
\subsection*{Curriculum Vitae}
\begin{cvlayout}
\item[Name:\hfill]
    TSCHARNUTER, Werner M.
\item[Address:\hfill]
    Zentrum f\"ur Astronomie (ZAH)\\
    Institut f\"ur Theoretische Astrophysik (ITA)\\
    69120 Heidelberg\\
    Phone / Fax: 06221-544815 / 06221-544221\\
    E-mail: wmt@ita.uni-heidelberg.de
\item[Date and Place of Birth:\hfill]
    June 3, 1945, Mitterberg, Austria
\item[Nationality:\hfill]
    Austrian
\item[Present position:\hfill]
    Professor (C4) of Theoretical Astrophysics
\item[Background:\hfill]
    Studies in Mathematics, Physics and Astronomy at the
    University of Vienna, Austria (1963-1968)\\
    Dissertation in Astronomy (Ph.D.), Vienna University (1968)\\
    Habilitation, Universit\"atssternwarte,
       Physics Faculty, Univ.\ G\"ottingen (1975)
\item[Positions:\hfill]
    Research Assistant, Univ. of Vienna (1968--1971)\\
    Research Assistant, Univ. of G\"ottingen (1971--1975)\\
    Lecturer (Privatdozent), Univ.\ of G\"ottingen (1975--1976)\\
    Scientific staff member, Max-Planck-Institute for Physics and Astrophysics,
       Munich and Garching (1975--1981)\\
    Lecturer (Privatdozent), Univ.\ of Munich (1976--1981)\\
    Professor (Ordinarius), Theoretical Astronomy,
       Univ.\ of Vienna, Austria (1981--1987)\\
    Professor (C4), Faculty of Physics and Astronomy,
       Univ.\ of Heidelberg (1987--2004)\\
    Dekan, Faculty of Physics and Astronomy, Univ.\ of Heidelberg (1990--1991)\\
    Prodekan, Faculty of Physics and Astronomy, Univ.\ of Heidelberg (1991--1992)\\
    Deputy director of IWR (1988--1995)\\
    Member of Extended Board of Directors of IWR (since 1995)\\
    Chairman, Collaborative Research Centre (SFB) 328
       ``Evolution of Galaxies'' (1992--1997)\\
    Chairman, Collaborative Research Centre (SFB) 439
       ``Galaxies in the Young Universe'' (1998--2004)\\
    Professor (C4), Zentrum f\"ur Astronomie (ZAH), Univ.\ of Heidelberg (since 2005)
\item[Awards/Fellowships:\hfill]
    "Promotio sub auspiciis praesidentis rei publicae" (1968)\\
\end{cvlayout}

\subsection*{Research interests and achievements}
%
One of the main research areas dealt with at the Institute for
Theoretical Astrophysics, which reflects also my own scientific
interests, is the modeling of gravitational collapse and accretion
phenomena on planetary, stellar, and galactic scales. The
investigation of both collapse and accretion processes is
fundamental for understanding the formation history of galaxies,
stars, and planets. Building up realistic models in this field is
a genuine interdisciplinary task that requires extensive knowledge
in various branches of physics, chemistry, and mineralogy, as well
as in numerical mathematics.

In order to achieve this goal, particularly for the most
challenging problem of planet formation, the basic evolutionary
equations which are the highly non-linear multi-dimensional
partial differential equations of multi-component viscous
radiation hydrodynamics must be combined in a consistent way with
the equations describing the network of the chemical
reactions---comprising gas phase chemistry, combustion,
sublimation and condensation---and mineralogical alterations to,
together with coagulation of, the microscopic dust grains that are
embedded in the hydrodynamical flow. Advanced numerical methods
are necessary to solve this complicated problem. Successful first
steps toward a better understanding of the intriguing interplay
between hydrodynamics, chemistry, and mixing processes in axially
symmetric pre-planetary accretion disks have already been made.

Further research activities, within the framework of the
Collaborative Research Centre (SFB) 439 ``Galaxies in the Young
Universe'', relate to the formation, structure and evolution of
very massive ($600\,\textrm{M}_\odot\gtrsim M\gtrsim
50$\,M$_\odot$) primordial, zero-``metallicity'' (so-called
Population III) stars and their stability on the one hand and, on
the other hand, to the AGB-evolution of Population III stars of
low and intermediate mass ($\lesssim 10$\,M$_\odot$).

\subsection*{List of refereed publications in past 5 years}
\begin{ownpubl}

\item Tscharnuter, W.~M. and Gail, H.-P. (2006) 2-D protoplanetary
accretion disks I. Hydrodynamics, chemistry, and mixing processes.
\aap, (to be submitted May 2006)   
%%% will be submitted before May 15. 

\item Gail, H.-P. and Tscharnuter, W.~M. (2006) Evolution of
protoplanetary disks including detailed chemistry and mineralogy.
In: \textit{Reactive Flow, Diffusion and Transport} (Ed. R.
Rannacher et. al.) (Springer, Berlin-Heidelberg) (accepted)

\item H\"onig, S.~F. and Tscharnuter, W.~M. (2005) Preliminary
Orbital Elements of Four Interferometric Binary Stars. \aj,
\textbf{129}, 1663-1668

\item Wuchterl, G. and Tscharnuter, W.~M. (2003) From clouds to
stars.  Protostellar collapse and the evolution to the pre-main
sequence I. Equations and evolution in the Hertzsprung-Russell
diagram. \aap, \textbf{398}, 1081-1090

\item Straka, C.~W. and Tscharnuter, W.~M. (2001) Massive
zero-metal stars: Energy production and mixing. \aap,
\textbf{372}, 579-582

\end{ownpubl}
\cleardoublepage
%
%--------------------------------------------------------------------
%
\renewcommand{\cvhdr}{Wolf}
\section*{S.\ Wolf}
\subsection*{Curriculum Vitae}
\begin{cvlayout}
\item[Name:\hfill]        
    WOLF, Sebastian
\item[Address:\hfill]
    Max Planck Institute for Astronomy, K\"onigstuhl 17\\
    69117 Heidelberg\\
    Phone / Fax: 06221-528406 / 06221-528246\\
    E-mail: swolf@mpia.de
\item[Date and Place of Birth:\hfill]
    June 1, 1973, Jena, Germany
\item[Nationality:\hfill]
    German
\item[Present position:\hfill]
    Leader of Emmy Noether Research Group
\item[Background:\hfill]
    Studies in Physics at the 
    Friedrich Schiller University of Jena (1992-1997)\\
    Diploma in Physics (1997), University of Jena\\
    Dissertation in Physics (Ph.D.) (2001), University of Jena
\item[Positions:\hfill]
    Thuringian State Observatory Tautenburg (1997-2001)\\
    European Southern Observatory, Santiago de Chile (2001)\\
    Thuringian State Observatory Tautenburg (2001)\\
    Max Planck Institute for Astronomy, Heidelberg (2002)\\
    Jet Propulsion Laborarory / Infrared Processing
    and Analysis Center, Pasadena, USA (2002)\\
    California Institute of Technology, Pasadena, USA (2003)\\
    Max Planck Institute for Astronomy, Heidelberg (since 2004)
\item[Awards/Fellowships:\hfill]
  Examenspreis of the Friedrich Schiller University Jena (1997)\\
  Promotionspreis of the Friedrich Schiller University Jena (2001)\\
  Emmy Noether Gruppe (2004-2007)\\
  Heinz Maier-Leibnitz Prize (2005)
\end{cvlayout}

\subsection*{Research interests and achievements}
\subsubsection*{Planet Formation - Evolution of Circumstellar Disks}
Observations and numerical studies of the dust grain evolution in
circumstellar disks; Predictions concerning the observability of planets in
young and evolved circumstellar disks

\subsubsection*{3D Radiative Transfer}
Development of numerical simulations for self-consistent 3D continuum
radiative transfer, allowing to simulate observable quantities of
dust-shrouded objects; Preparation and performance of the analysis of
Spitzer Space Telescope observations of debris disks

\subsubsection*{Near- and Mid-Infrared Interferometry}
Science case studies for the 2$^{\rm nd}$ generation VLTI instrument
MATISSE; Physical conditions and structure of the planet-forming regions in
T Tauri protoplanetary disks

\subsubsection*{Near-Infrared Polarimetry}
Polarimetric properties of young stellar objects and active galactic nuclei;
Simulation of multiple scattering of optical/infrared polarized radiation in
circumstellar shells containing aligned spheroidal dust grains; Binary star
formation studies

\subsubsection*{Submillimeter Polarimetry -- Magnetic Fields in Star Forming Regions}
Measurement of the structure and strength of the magnetic field in Bok
globules

\subsection*{List of refereed publications in past 5 years}
\begin{ownpubl}

\item Kornet, K., Wolf, S., R${\rm \acute{o}\dot{z}}$yczka, M. (2006)
  Planet formation around stars with various masses, 
  \aap, in press

\item Schegerer, A., Wolf, S., Voshchinnikov, N.V., Przygodda, F., Kessler-Silacci, J.E. (2006)
  Analysis of the dust evolution in the circumstellar disks of T Tauri stars,
  \aap, in press

\item Kornet, K., Wolf, S. (2006) 
  Radial distribution of planets. 
  Predictions based on the Core Accretion / Gas Capture model for planet formation,
  \aap, in press

\item Silverstone, M.D~., Meyer, M.~R., Mamajek, E.~E., Hines, D.~C., Pascucci, I., Hillenbrand, L.~A., 
  Bouwman, J., Kim, J.~M., Carpenter J.~M., Stauffer, J.~R., Najita, J., Moro-Mart\'{\i}n, A., 
  Henning, Th., Wolf, S., Backman, D.~E., Brooke, T.~Y., Padgett, D.~L. (2006)
  Formation and Evolution of Planetary Systems (FEPS): 
  Primordial  warm dust evolution from 3-30 Myr around Sun-like stars,
  \apj, \textbf{639}, 1138

\item Hines, D.~C., Backman, D.~E., Bouwman, J., Hillenbrand, L.~A., Carpenter, J.~M., Meyer, M.~R., 
  Kim, J.~S., Silverstone, M.~D., Rodmann, J., Wolf, S., Mamajek, E.~E., Brooke, T.~Y., Padgett, D.~L., 
  Henning, Th., Moro-Mart\'{\i}n, A., Stobie, E., Gordon, K.~D., Misselt, K., Morrison, J., 
  Muzerolle, J., Su, K. (2006)
  The Formation and Evolution of Planetary Systems (FEPS): 
  Discovery of an Unusual Debris System Associated with HD 12039, 
  \apj, \textbf{638}, 1070

\item Eisner, J.A., Hillenbrand, L.~A., Carpenter, J.~M., Wolf, S. (2005)
  Constraining the Evolutionary Stage of Class I Protostars: 
  Multi-Wavelength Observations and Modeling,
  \apj, \textbf{635}, 396

\item 
Hollenbach, D., Gorti, U., Meyer, M., Kim J.~S., Morris, P., Najita, J., 
Pascucci, I., Carpenter, J., Rodmann, J., Brooke, T., Mamajek, E., 
Padgett, D., Soderblom, D., Wolf, S., Lunine, J. (2005)
Formation and Evolution of Planetary Systems: Upper Limits to the Gas Mass in
HD 105. \apj, \textbf{631}, 1180

\item 
Kim, J.~S., Hines, D.~C., Backman, D.~E., Rodmann, J., Hillenbrand, L.~A.,  
Moro-Mart\'{\i}n, A., Carpenter, J.~M., Silverstone, M.~D., Bouwman, J., 
Mamajek, E.~E., Meyer, M.~R.,  Wolf, S., Malhotra, R., Pascucci, I., 
Najita, J., Henning, Th., Brooke, T.~Y., Strom, S.~E., Padgett, D.~L., 
Stobie, E.~B.,  Engelbracht, Ch., Gordon, K., Misselt, K., Morrison, J., 
Muzerolle, J., Su, K. (2005) 
Formation and Evolution of Planetary Systems: Cold Outer Disks Associated
with Sun-like stars. \apj, \textbf{632}, 659

\item 
Wolf, S. and Hillenbrand, L. (2005) Debris disk radiative transfer simulator.
\cpc, \textbf{171}, 208

\item 
Schartmann, M., Meisenheimer, K., Camenzind, M., Wolf, S. and Henning, Th.  
(2005) Towards a physical model of dust tori in Active Galactic Nuclei. 
Radiative transfer calculations for a hydrostatic torus model. \aap, \textbf{437}, 861

\item 
Metchev, S.~A., Eisner, J.~A., Hillenbrand, L. and  Wolf, S. (2005) Adaptive
Optics Imaging of the AU Microscopii Circumstellar Disk: Evidence for Dynamical
Evolution. \apj, \textbf{622}, 451

\item 
Moro-Mart\'{\i}n, A., Wolf, S. and Malhotra, R. (2005) Signatures of Planets in
Spatially Unresolved Debris Disks. \apj, \textbf{621}, 1079

\item 
Carpenter, J., Wolf, S., Schreyer, K., Launhardt, R. and Henning, Th. (2005)
Evolution of Cold Circumstellar Dust around Solar-type Stars. \apj, \textbf{129},
 1049

\item 
Wolf, S., D'Angelo, G. (2005) On the Observability of Giant Protoplanets in 
Circumstellar Disks. \apj, \textbf{619}, 1114

\item 
Wolf, S., Launhardt, R. and Henning, Th. (2003) Magnetic Field Evolution in Bok
Globules. \apj, \textbf{592}, 233

\item 
Puga, E., Alvarez, C., Feldt, M., Henning, Th. and Wolf, S. (2004) AO-assisted
observations of G61.48+0.09. Massive star formation at high resolution. \aap, 
\textbf{425}, 543

\item 
Wolf, S. and Voshchinnikov, N.~V. (2004) Mie scattering by ensembles of particles
with very large size parameters. \cpc, \textbf{162}, 113

\item 
Meyer, M.~R., Hillenbrand, L.~A., Backman, D.~E., Beckwith, S.~V.~W., Bouwman,
J., Brooke,T.~Y., Carpenter, J.~M., Cohen, M., Gorti, U., Henning, Th., 
Hines, D.~C., Hollenbach, D.,  Kim, J.~S., Lunine, J., Malhotra, R., Mamajek, E.~E., 
Metchev, S., Moro-Mart\'{\i}n, A., Morris, P., Najita, J., Padgett, D.~L., 
Rodmann, J.~R., Silverstone, M.~D., Soderblom, D.~R., Stauffer, J.~R., 
Stobie, E.~B., Strom, S.~E., Watson, D.~M., Weidenschilling, S.~J., Wolf, S., 
Young, E., Engelbracht, C.~W., Gordon, K.~D., Misselt, K., Morrison, J., 
Muzerolle, J., Su, K. (2004) 
The Formation and Evolution of Planetary Systems: First Results
from a Spitzer Legacy Science Program. \apjs, \textbf{154}, 422

\item 
Sch\"utz, O., Nielbock, M., Wolf, S., Henning, Th. and Els, S. (2004) SIMBA's
view of the $\epsilon$ Eri disk. \aap, \textbf{414}, L9 

\item 
Pascucci, I., Wolf, S., Steinacker, J., Dullemond, C.~P., Henning, Th.
Niccolini, G., Woitke, P., Lopez, B.
(2004) The 2D continuum radiative transfer problem. Benchmark results for disk
configurations. \aap, \textbf{417}, 793

\item 
Rengel, M., Fr\"obrich, D., Wolf, S. and Eisl\"offel, J. (2004) Modeling the
Continuum Emission from Class 0 Protostellar Sources. \bala, \textbf{13}, 449

\item 
Agol, E., Barth, A., Wolf, S. and Charbonneau, D. (2004) Spectropolarimetry and 
Modeling of the Eclipsing T Tauri Star KH 15D. \apj, \textbf{600}, 781

\item 
Chesneau, O., Wolf, S. and de Souze, A.~D. (2003) Hot stars mass-loss studied
with Spectro-Polarimetric INterferometry (SPIN). \aap, \textbf{410}, 375

\item 
Pascucci, I., Henning, Th., Steinacker, J. and Wolf, S. (2003) 2D/3D Dust
Continuum Radiative Transfer Codes to Analyze and Predict VLTI Observations.
\ass, \textbf{286}, 113

\item 
Wolf, S. and Hillenbrand, L. (2003) Model Spectral Energy Distributions of
Circumstellar Debris Disks. I. Analytic Disk Density Distributions. \apj,
\textbf{596}, 603

\item 
Wolf, S., Launhardt, R. and Henning, Th. (2004) Evolution of Magnetic fields in
Bok Globules? \ass, \textbf{292}, 239

\item 
Wolf, S., Padgett, D.~L. and Stapelfeldt, K.~R. (2003) The Circumstellar Disk of
the Butterfly Star in Taurus. \apj, \textbf{588}, 373

\item 
Wolf, S. (2003) Efficient Radiative Transfer in Dust Grain Mixtures. \apj,
\textbf{582}, 859

\item 
Wolf, S. (2003) MC3D-3D continuum radiative transfer, Version 2. \cpc,
\textbf{150}, 99

\item 
Wolf, S. and Klahr, H. (2002) Large-Scale Vortices in Protoplanetary Disks: On 
the Observability of Possible Early Stages of Planet Formation. \apj,
\textbf{578}, L79

\item 
Wolf, S., Voshchinnikov, N.~V. and Henning, Th. (2002) Multiple scattering of
polarized radiation by non-spherical grains: First results. \aap, \textbf{385},
365

\item 
Wolf, S., Gueth, F., Henning, Th. and Kley, W. (2002) Detecting Planets in
Protoplanetary Disks: A Prospective Study. \apj, \textbf{566}, L97

\item 
Wolf, S. (2001) Inverse raytracing based on Monte-Carlo radiative transfer
simulations. \aap, \textbf{379}, 690

\item 
Henning, Th., Wolf, S., Launhardt, R. and Waters, L.~B.~F.~M. (2001)
Measurements of the Magnetic Field Geometry and Strength in Bok Globules. \apj,
\textbf{561}, 871

\end{ownpubl}
\cleardoublepage
%
%--------------------------------------------------------------------
%
\renewcommand{\cvhdr}{Wurm}
\section*{Gerhard Wurm}
\subsection*{Curriculum Vitae}
\begin{cvlayout}
\item[Name:\hfill]        
    WURM, Gerhard
\item[Address:\hfill]
    Institut f\"ur Planetologie, Wilhelm-Klemm-Str. 10\\ 
    48149 M\"unster\\
    Phone / Fax: 0251 8339052 / 0251 8336301\\
    E-mail: gwurm@uni-muenster.de
\item[Date and Place of Birth:\hfill]
    April 5, 1967, Siegen, Germany
\item[Nationality:\hfill]
    German
\item[Present position:\hfill]
    Scientist, Leader Emmy Noether Group
\item[Background:\hfill]
    Studies in Physics at the 
    University of Aachen (1987-1993)\\
    Diploma in Physics  (1993)\\
    Dissertation in Physics / Astrophysics (Dr.), University Jena (1997)\\
\item[Positions:\hfill]
    Max-Planck-Research Group\\
	Dust in Starforming Regions, Jena (1994-1996)\\
    AIU, Univ. of Jena (1997-1999)\\
    LASP, Univ. of Colorado, Boulder (2000-2001)\\
    Emmy Noether Group, IfP, Univ. of M\"unster (2002-present)\\
\item[Awards/Fellowships:\hfill]
	Dissertation Price, Univ. of Jena (1998)\\
    Emmy Noether Stipend (2000-2001)\\
    Emmy Noether Group (2002-2006)\\
    
\end{cvlayout}

\subsection*{Research interests and achievements}
\subsubsection*{Planetesimal formation through collisional growth}
Experimental realization of collision processes in protoplanetary disks,
dust cloud experiments, fractal dust aggregates, turbulent growth,
low collision velocities, growth and fragmentation in high velocity
collisions.
\subsubsection*{Gas-particle interaction}
Experiments and models on particle motion in gaseous environments
and gas motion in porous particle aggregates, gas-grain 
coupling times, reaccretion of
collision fragments, erosion of planetesimals by gas drag
\subsubsection*{Optical properties of particles}
Experimental determination of radiation pressure on dust particles,
light scattering by dust aggregates
\subsubsection*{Photophoresis in astrophysical and planetary applications}
Motion of bodies in late protoplanetary disks and planetary atmospheres,
optical and thermal response of porous bodies to light. Surface processes 
and photophoretic erosion.

\subsection*{List of refereed publications in past 5 years}
\begin{ownpubl}

\item
Paraskov, G. B., Wurm, G., and Krauss, O. (2006) Eolian Erosion of Dusty Bodies
in Protoplanetary Disks. \apj, (submitted)

\item 
Dominik, C., Blum, J., Cuzzi, J., and Wurm, G. (2006) Aggregation and Transport of Dust:
in Disks as Initial Steps toward Planet Formation. in: Protostars and Planets V,
(in press)

\item 
Wurm, G. and Blum, J. (2006) Experiments on Planetesimal Formation: in Planet
Formation: Theory, Observation, and Experiment, (in press)

\item
Wurm, G. and Krauss, O. (2006) Dust Eruptions by Photophoresis and Solid 
State Greenhouse Effects. \prl, \textbf{96}, 134301

\item
Wurm, G. and Krauss O. (2006) Concentration and Sorting of Chondrules and
CAIs in the Late Solar Nebula. \ica, \textbf{180}, 487

\item
Krauss, O. and Wurm, G. (2005) Photophoresis and the pile-up of dust in young
circumstellar disks. \apj, \textbf{630}, 1088

\item
Wurm, G., Paraskov, G. and Krauss, O. (2005) Growth of Planetesimals by Impacts
at ~25m/s. \ica, \textbf{178}, 253

\item
Wurm, G., Paraskov, G. and Krauss, O. (2005) Ejection of dust by elastic waves in
collisions between millimeter- and centimeter-sized dust aggregates at 16.5 to
37.5 m/s impact velocities. \phre, \textbf{71}, 21304

\item
Wurm, G., Paraskov, G. and Krauss, O. (2004) On the Importance of Gas Flow
through Porous Bodies for the Formation of Planetesimals. \apj, \textbf{606}, 983

\item
Wurm, G., Relke, H., Dorschner, J. and Krauss, O. (2004) Light scattering
experiments with micron-sized dust aggregates: Results on ensembles of
$\rm SiO_2$ monospheres and of irregularly shaped graphite particles. \jqsrt,
\textbf{89}, 371

\item
Krauss, O. and Wurm, G. (2004) Radiation pressure forces on individual
micron-size dust particles: A new experimental approach. \jqsrt, \textbf{89}, 179

\item
Wurm, G., Relke, H. and Dorschner, J. (2003) Experimental Study of Light
Scattering by Large Dust Aggregates Consisting of Micron-sized $\rm SiO_2$
Monospheres. \apj, \textbf{595}, 891

\item
Wurm, G., Schnaiter, M. (2002) Coagulation as Unifying Element for Interstellar
Polarization. \apj, \textbf{567}, 370

\item
Blum, J., Wurm, G., Poppe, T., Kempf, S. and Kozasa, T. (2002) First results from
the cosmic dust aggregation experiment codag. \textit{Advances in Space
Research\/}, \textbf{29}, 497

\item
Poppe, T., Wurm, G. and Krieg, R. (2002) Optical particle and particle motion
analysis with PATRICIA. \textit{Measurements Science and Technology\/},
\textbf{13}, 796

\item
Schnaiter, M. and Wurm, G. (2002) Experiments on light scattering and extinction
by small micron-sized aggregates of spheres. \textit{Applied Optics\/},
\textbf{41}, 1175

\item
Wurm, G., Blum, J. and Colwell, J.~E. (2001) Aerodynamical sticking of dust
aggregates. \phre, \textbf{64}, 046301

\item
Wurm, G., Blum, J. and Colwell, J.~E. (2001) NOTE: A New Mechanism Relevant to
the Formation of Planetesimals in the Solar Nebula. \ica, \textbf{151}, 318

\item
Blum, J. and Wurm, G. (2001) Drop tower experiments on sticking, restructuring
and fragmentation of preplanetary dust aggregates. \textit{Microgravity Science
and Technology\/}, \textbf{13}, 29

\end{ownpubl}




