
%%%%%%%%%%%%%%%%%%%%%%%%%%%%%%%%%%%%%%%%%%%%%%%%%%%%%%%%
% LaTex Template for proposals within the              %
% DFG Research Unit Program                            %         
%Planet Formation Witnesses and Probes: Transition Discs
% August 2016                                            %                           
%                                                      %
%%%%%%%%%%%%%%%%%%%%%%%%%%%%%%%%%%%%%%%%%%%%%%%%%%%%%%%%
%
% 
%
% This template may be used to prepare proposals in latex.
%
%
% The project description, including publication list, should be no more than 20 pages
% in length. It should be self-explanatory and not require reviewers to read the 
% literature that is quoted or enclosed.

\documentclass[10pt,fleqn,twoside]{article}

%%%% USE ARIAL FONT %%%%%%%%%%%%%%%%%%%%%%%%%%%%%%%%%%%%%%%%%%%%%%%%%%%%%%
\usepackage{helvet}
\renewcommand\familydefault{phv}

%%%% INCLUDE NECESSARY PACKAGES %%%%%%%%%%%%%%%%%%%%%%%%%%%%%%%%%%%%%%%%%%
%\usepackage{babel}
\usepackage[UKenglish]{babel}
\usepackage{amsmath}
\usepackage{amssymb}
\usepackage{fancyhdr}
\usepackage{natbib}
\usepackage{xcolor}
\usepackage{ae,aecompl}
\usepackage{graphicx}
\usepackage{palatino}
\usepackage[T1]{fontenc}
\usepackage{rotating}
\usepackage{epsf}
\usepackage{setspace}
%\usepackage{sfmath}

%%%% PAGE LAYOUT %%%%%%%%%%%%%%%%%%%%%%%%%%%%%%%%%%%%%%%%%%%%%%%%%%%%%%%%%
\setlength{\textheight}{22cm}
\setlength{\topmargin}{-1.2cm}
\setlength{\textwidth}{15.6cm}
\setlength{\oddsidemargin}{0.0cm}
\setlength{\evensidemargin}{0.0cm}
\setlength{\mathindent}{1.5cm}
\setlength{\parindent}{0.0cm}
\setlength{\parskip}{0.08cm}

%%%% PAGE HEADER %%%%%%%%%%%%%%%%%%%%%%%%%%%%%%%%%%%%%%%%%%%%%%%%%%%%%%%%%%
\pagestyle{fancy}
\fancyhead[RE,RO]{}
\fancyfoot[RO]{ \thepage}
\fancyfoot[LE]{ \thepage}
\fancyfoot[CE,CO]{}

%%% FONTS FOR THE TITLE PAGE %%%%%%%%%%%%%%%%%%%%%%%%%%%%%%%%%%%%%%%%%%%%%%
\newfont{\tpfonta}{cmssbx10 scaled 1600}
\newfont{\tpfontb}{cmssbx10 scaled 3200}

%%%% EURO SIGN %%%%%%%%%%%%%%%%%%%%%%%%%%%%%%%%%%%%%%%%%%%%
\newcommand\euro{{\sffamily C%
\makebox[0pt][l]{\kern-.70em\mbox{--}}%
\makebox[0pt][l]{\kern-.68em\raisebox{.25ex}{--}}}}
\newcommand\keuro{k{\sffamily C%
\makebox[0pt][l]{\kern-.70em\mbox{--}}%
\makebox[0pt][l]{\kern-.68em\raisebox{.25ex}{--}}}}

%%%% COLOR DEFINITIONS %%%%%%%%%%%%%%%%%%%%%%%%%%%%%%%%%%%%
\definecolor{blue} {rgb} {0.25,0.25,0.75}

%%%% ADDITONAL EMPHASIS %%%%%%%%%%%%%%%%%%%%%%%%%%%%%%%%%%%
\newcommand{\cem}{\color{blue}}
\newcommand{\eem}{\sl\color{blue}}

%%%% SET THE COLOR OF THE (SUB-) SECTION TITLES %%%%%%%%%%% 
\newcommand{\Tcol}{\color{blue}}

%%%% SET THE COLOR OF THE TITLE BOX BACKGROUND %%%%%%%%%%%%
\definecolor{Background}{rgb} {0.62,0.75,0.5}

%%%% REFERENCE SECTION NAME %%%%%%%%%%%%%%%%%%%%%%%%%%%%%%%
\renewcommand\refname{\Tcol 9. Bibliography}

%%%% COLOR THE SECTION NUMBERS %%%%%%%%%%%%%%%%%%%%%%%%%%%%%%%
\makeatletter
\renewcommand\@seccntformat[1]{\color{blue} {\csname the#1\endcsname}\hspace{0.5em}}
\makeatother
\renewcommand\thesection{\arabic{section}.}
\renewcommand\thesubsection{\arabic{section}.\arabic{subsection}}

%%%% CHANGE THE APPEARANCE OF THE \PARAGRAPH COMMAND  %%%%%%%%%%%%%%%%%%%%%%%%%%%%%%%
\makeatletter
\renewcommand\paragraph{\@startsection{paragraph}{4}{\z@}%
            {-2.5ex\@plus -1ex \@minus -.25ex}%
            {1.25ex \@plus .25ex}%
            {\normalfont\normalsize\bfseries}}
\makeatother
\setcounter{secnumdepth}{4}     % how many sectioning levels to assign numbers to
\setcounter{tocdepth}{4}        % how many sectioning levels to show in ToC


\fancyhead[LE,LO]{\slshape
%%%%  Please edit
%
\color{red}{Please Edit and remove color statement} Your Name: RU Transition Discs Project Description}
%
%
%%%%%


\begin{document}


\newpage

%%%% PROJECT DESCRIPTION STARTS HERE %%%%%%%%%%%%%%%%%%%%%%%%%%%%%%%%%%%

\setcounter{page}{1}

\centerline{\huge\bf\Tcol
%
%
%
%
%%%%  Please edit
%
 Project ABCD12:}

\centerline{\huge\bf\Tcol Title}

%
%%%%
%
%
%
%
\vskip1.0cm

%%%%  Please edit

\noindent{\bf Authors:}\\
\begin{tabular}{ll}
{\textsf{PI:}}                   & Author1 (Institution1) \\
{\textsf{Co-I:}}                & Author2 (Institution2), Author3 (Institution3).....\\
{\textsf{Collaborations:}}      & Collaborator1 (Institution1), Collaborator2 (Institution2).... \\

\end{tabular}

%%%%  Please edit

\vspace{1em}
\noindent{\bf Requested positions: 1PhD student, 1 Postdoc} \\

\vspace{1em}
\noindent{\bf Abstract:}\\

[Text]


\section{\Tcol State of the art and preliminary work}
\renewcommand{\leftmark}{\sc State of the Art and preliminary work}

[Text]

\subsection{\Tcol Project-related publications}

% Please list your own publications related to the proposed project, 
% adhering to the rules of the DFG guidelines 1.91. In brief, please note: 
% - Up to 10 publications
% - The work must be published or accepted.
% - Publications on astro-ph (arXive, SPIRES or articles with a DOI) count as published. 
% - Any work that is only in the status ``accepted'' MUST be attached to the proposal
%    together with the acceptance letter.
% - All publications in this section CAN be attached to the proposal. Please limit these
%    attachments to a minimum and please note that the reviewers may not read the attachments -
%    the proposal has to speak for itself.
% - The number of allowed publications refers to the sum of the publications listed
%    in ``1.1.1 Articles published or officially accepted by publication outlets...'' and 
%    in ``1.1.2 Other publications''. Publications which only exist on public repositories 
%    belong into the category ``Other Publications''.
[Text]

\subsubsection{\Tcol 
Articles published or officially accepted by publication outlets with scientific quality assurance;
book publications}

[Text]

\subsubsection{\Tcol Other publications}

[Text]

\subsubsection{\Tcol Patents}

\paragraph{\Tcol Pending}

[Text]

\paragraph{\Tcol Issued}

[Text]

\section{\Tcol Objectives and work programme}
\renewcommand{\leftmark}{\sc Objectives and work programme}


\subsection{\Tcol Anticipated total duration of the project}

[Text]

\subsection{\Tcol Objectives}

[Text]

\subsection{\Tcol Work programme including proposed research methods}
% for each applicant

[Text]

\subsection{\Tcol Data handling}

[Text]

\subsection{\Tcol Other information}
% Please use this section for any additional information you feel is
% relevant which has not been provided elsewhere.

[Text]

\subsection{\Tcol Information on scientific and financial involvement of international cooperation partners}

[Text]

\section{\Tcol Bibliography}

[Text]

\section{\Tcol Requested modules/funds}
\renewcommand{\leftmark}{\sc  Requested modules/funds}
% Explain each item for each applicant (stating last name, first name).

\subsection{\Tcol Basic Module}

\subsubsection{\Tcol Funding for Staff}
% Please note that funds for your own temporary position (“Eigene Stelleâ€)
% are not to be included here; this belongs to the separate “Module Temporary Positionâ€.

[Text]

\subsubsection{\Tcol Direct Project Costs}

[Text]

\paragraph{\Tcol Equipment up to EUR 10,000, Software and Consumables}

[Text]

\paragraph{\Tcol Travel Expenses}

[Text]

\paragraph{\Tcol Visiting Researchers (excluding Mercator Fellows)}

[Text]

\paragraph{\Tcol Other Costs}

[Text]

\paragraph{\Tcol Project-related publication expenses}

[Text]

\subsubsection{\Tcol Instrumentation}

[Text]

\paragraph{\Tcol Equipment exceeding EUR 10,000} 

[Text]

\paragraph{\Tcol Major Instrumentation exceeding EUR 100,000} 

[Text]

\subsection{\Tcol Module Temporary Position}

[Text]

\subsection{\Tcol Module Replacement Funding}

[Text]

\subsection{\Tcol Module Mercator Fellows}

[Text]

\subsection{\Tcol Module Public Relations Funding}

[Text]

\section{\Tcol Project requirements}
\renewcommand{\leftmark}{\sc Project requirements}

\subsection{\Tcol Employment status information}
% For each applicant, state the last name, first name, and employment
% status (including duration of contract and funding body, if on a
% fixed-term contract).

[Text]

\subsection{\Tcol First-time proposal data}
% Only if applicable: Last name, first name of first-time applicant.

[Text]

\subsection{\Tcol Composition of the project group}
% List only those individuals who will work on the project but will not
% be paid out of the project funds. State each person’s name, academic
% title, employment status, and type of funding.

[Text]

\subsection{\Tcol Cooperation with other researchers}

\subsubsection{\Tcol Planned cooperation on this project}

\paragraph{\Tcol Collaborating researchers for this project within the
  Research Unit}
%Each proposal must be accompanied by a description of how the project
%is integral to the Research Unit, %both in terms of subject matter
%and organisation. This includes a description of the cooperation with
%%others participating within the Research Unit. 

[Text]

\paragraph{\Tcol Collaborating researchers for this project outside of
  the Research Unit}

[Text]

\subsubsection{\Tcol Researchers with whom you have collaborated scientifically within the past three years}
% This information is important for DFG to exclude possible conflicts of interest.
% Please mention not only the names of the cooperation partners but also their institution and city.
% Scientists already mentioned in the previous two subsubsections do not have to be mentioned
% again.

[Text]

\subsection{\Tcol Scientific equipment}
% List larger instruments that will be available to you for the
% project. These may include large computer facilities if computing
% capacity will be needed. 

[Text]

\subsection{\Tcol Project-relevant interests in commercial enterprises}
% Information on connections between the project and the production
% branch of the enterprise.

[Text]


\subsection{\Tcol Additional information}
% If applicable, please list proposals requesting major
% instrumentation and/or those previously submitted to a third party
% here.

[Text]


\end{document}
