%----------------------------------------------------------------------
%                        PROJECT DEFINITION
%----------------------------------------------------------------------
\renewcommand{\projnr}{P}
\renewcommand{\projtitleshort}{Forschergruppen-Professur}
\renewcommand{\projauth}{Kley}
%
\setcounter{section}{0}
\noindent{\normalfont\sffamily\Large\bfseries Project \projnr: \projtitleshort}
%
\section{Full title:}
\hspace{1\baselineskip}\\
\centerline{\large ``Geoscientific and Astrophysical Planetary Research''}
\centerline{\bf (Universit\"at Heidelberg)}
%\centerline{\large ''}
%
\section{Summary (Zusammenfassung)}
One of the major goals of the proposed Forschergruppe is the initiation of
a close collaboration of researchers from different scientific communities
to jointly study the formation of planets around solar type stars.
At the heart of this effort lies the combination of geological/mineralogical
research with astronomy and astrophysics. 
This aim of our research group can be realized best through the creation of
a new Forschergruppen-Professur in the area of 
``Geoscientific and Astrophysical Planetary Research''. 
Heidelberg, where the main part of this Forschergruppe is localized,
constitutes the ideal location to create such a professorship. 
%
%%\subsubsection{Summary:} 
%
\section{Scientific Background (Wissenschaftlicher Hintergrund)}
\subsection{General} 
%
For a long time, the planets in our own Solar System were the only definitely known
planets in our universe. However, the detection of extrasolar planets 
- more than 170 have been found since 1995 - has shown that there are other
worlds that potentially could harbour life. Indeed, the fraction of stars with
planets is thought to lie between 10 and $>$ 95\% - rather ubiquitous than rare - opening
the possibility of billions of possible planetary worlds in our galaxy and equally
billions in the other 10$^{12}$ galaxies. 

In the future, one of the most important tasks of planetary and astrophysical science will
be to place our own Earth, the terrestrial planets and the Solar System as a whole
into context: Are we living in a typical or rather exceptional planetary system? Are conditions
on Earth unique or can other Earth-like planets be expected in a significant number?
These questions can only be answered by 
{\it understanding the origin and formation of planets}.
Their birthplaces - protoplanetary discs - are abundant in star-forming regions, and
advances in observational astronomy during the last decade allow the direct observation
of planet formation processes. These observations must be compared to the much more
detailed information on the origin of the Earth and other Solar System bodies, as
inferred from the analysis of Solar System matter. Indeed, the developments in recent years
in laboratory high-precision techniques and their application to extraterrestrial matter,
meteoritic and cometary,  provide detailed and unique information on how planets were built
from small dust grains in the early Solar System.

Recent advances in laboratory analysis of primitive extraterrestrial matter, astronomical and
in-situ space science observations, and astrophysical modelling are impressive, but
have been developed in two different scientific disciplines, Astrophysics and Earth and
Planetary Sciences. The respective scientific communities have little cross communication,
mostly due to different education, observational/analytical strategies and tools,
i.e. scientific culture. Hence, there is a strong deficit in scientific
cross-field fertilization. An interdisciplinary activity within this interface of
Geoscientific and Astrophysical Planetary research is expected to be highly fruitful.
However, it requires a truly joint research effort in a close and dedicated collaboration,
performed by both astrophysical and geoscientific or cosmochemical working groups
engaged in planetary formation studies.
%
\subsection{The role of Heidelberg} 
%
Heidelberg has a long and outstanding research tradition in the field of planetary
research, cosmochemistry, and isotope geochronology, starting with Wolfgang Gentner
at the Max-Planck-Institut of Kernphysik. In Heidelberg, research groups study planet
formation at the Faculty of Physics and Astronomy 
(Center for Astronomy: W.M. Tscharnuter, J. Wambsganss),
the Max-Planck Institute for Astronomy (Th. Henning)
and the Faculty of Chemistry and Geosciences (Institute of Mineralogy: M. Trieloff).
These groups are now committed to join efforts and team up in this
proposed DFG-Forschergruppe
``The formation of planets: The critical first growth phase''.
The proposed professorship at the University of Heidelberg will constitute a
major link between these different groups and will strengthen future research in this
new and exciting area of science.  
Additional new professorships at the Center for Astronomy at the University of 
Heidelberg (A.~Quirrenbach and R.~Klessen) further strengthen related research fields
(search for planets and star formation).

Constellations of
this kind are rare worldwide, only the Institutes for Geophysics and
Planetology and for Astronomy (University of Hawaii),
the Open University (Milton Keynes), the Center for Star Formation
(Santa Cruz, Nasa Ames and Berkeley) and very few other US localities
have similarly favourable situations. The University of Heidelberg has strongly
supported the astrophysical-geoscientific planetary research in the first
round of the excellence initiative (``Genesis of our universe'',
PI: C. Wetterich), and is doing so in the second round as well
(PI of the astrophysics-physics proposal: J. Wambsganss). 
The newly proposed professorship will allow scientists in this Forschergruppe and
in Heidelberg to be able to play an active role in the field and be
highly competitive internationally.

Geoscientific research related to planet formation and cosmochemistry has no
strong presence at German universities, though the field is of vital international
interest. 
However, the proposed professorship goes far beyond a mere cosmochemistry
professorship: it will have a much broader scope, promoting truly joint research
efforts at the interface of Geoscientific and Astrophysical Planetary Research.
taking into account the newly aroused interest in the fast moving field of
exoplanets, protoplanetary discs and studies of planet formation in our Solar
System, and last not least the question of our place in the Cosmos.

%
\section{Tasks of the Professorship (Aufgaben)}
%
The professorship will focus on the main scientific goal of the Forschergruppe, and
the primary task is the integration of projects that study specific aspects of
planetesimal formation. The candidate shall establish a strong association and
inter-relation of the astrophysical and mineralogic-cosmochemical studies of the
proposed Forschergruppe.

\medskip
\noindent 
The more specific tasks of the planned professorship are summarized in the following
list:
%
\begin{itemize}
\item {\bf
Joint research projects within the proposed Forschergruppe: 
}
Close collaboration shall link analytical studies of extraterrestrial matter,
experimental studies on the mineralogic-chemical evolution and coagulation of
dust with theoretical models. This should result in self-consistent
interdisciplinary scenarios of processes leading to planet formation,
reconciled with observations of the early Solar System and protoplanetary discs in general. 
%
\item {\bf
Laboratory analysis of planetary matter:
}
Excellent research related to planetesimal and planet formation.
Studies to understand the chronology of formation of the first solid aggregates
and planetesimals, and chemical differentiation of Solar System bodies.
This research should focus on the analysis of extraterrestrial matter, mainly
by using the analytical facilities of the Institute for Mineralogy 
(isotope chronology laboratory, electron microprobe analysis, secondary ion
mass spectrometry)
%
\item {\bf
Transfer of geoscientific expertise into the Faculty of Physics and Astronomy.
}
Establishing and maintaining the transfer of know-how to planetary research groups
of the faculty of physics and astronomy engaged in following research areas:
planetary formation studies, astronomical observations of dust and minerals, 
theoretical modelling of dust and
minerals in protoplanetary discs, space missions analyzing dust and minerals in situ.
%
\item {\bf
Strengthening the understanding of Earth as a planet, transfer of
astrophysical expertise in the Faculty of Chemistry and
Geosciences:
}
Earths formation, origin, age, global differentiation and evolution, interaction with
other bodies in the Solar System, with a focus on isotope studies.
This also includes Space-Earth relationships of vital interest,
particularly questions related to the accretion of extraterrestrial matter throughout
time: accretion of the Earths building blocks, or late impacts of large
meteorites causing mass extinctions and endangering civilization. 
%
\item {\bf
Teaching and education 
}
of isotope geology and geochronology (incl. analytical methods like mass spectrometry),
planetary sciences, geochemistry and cosmochemistry, for both students of physics
and astronomy and the Earth sciences.
%
\item {\bf
Promoting educational/public outreach of planetary research,
}
e.g. articles in popular science journals, organization of public symposia and exhibitions.
\end{itemize}
%
%
\section{The Candidate}
%
\subsection{Requirements}
%
\begin{itemize}
%
\item
Major and excellent contributions to the understanding of small body
evolution and planet formation in the early Solar System
%
\item
Research achievements in isotope geology, cosmochemistry, isotope geo- and cosmochronology
%
\item
Experience in planetary science from the astrophysical and from the Earth science view 
%
\item
Teaching expertise in astrophysics as well as in Earth science
%
\item
Record in service to scientific community
%
\item
Approved Funding record
%
\item
Approved activities in public outreach
\end{itemize}
%
\subsection{A specific Candidate}
%
In case of approval, the Forschergruppe-professorship shall be installed within the
first year after starting the Forschergruppe. It will be advertised internationally
to select the most appropriate candidate. 

A possible candidate we may suggest for this
professorship is PD Dr. M. Trieloff, who holds presently a non-tenured position
at the Mineralogical Institute at the University of Heidelberg. 
He is one of the main initiators and Co-Speaker of this Forschergruppe
and plays a vital role in its scientific research efforts.
Dr. Trieloff is an excellent scientist with an impressive research record. 
He has published as first author
in NATURE, SCIENCE, and 9 other Journals of Earth and Planetary Sciences, with
major recognitions of processes of planet formation in the early Solar System.
His research uses noble gas isotopes as tracers and chronological tools.
He studied physics and astronomy, habilitation was performed in the Geoscience
faculty. Teaching includes Solar System formation, planetary research, evolution
of the Earth, focussing on mantle geochemistry, and astrophysical and cosmochemical
concepts of planet formation. He is reviewer of Nature, Science, DFG, NASA,
private foundations and $>$10 journals of Earth and Planetary Sciences,
has an approved DFG funding record, and public outreach by popular articles
in Sterne und Weltraum, Spektrum der Wissenschaft, or organizing public symposia.
Dr. Trieloff received DFG habilitation and Heisenberg fellowships, and
the Victor-Moritz-Goldschmidt-price of the Deutsche Mineralogische Gesellschaft
in 2005 for his ``fundamental research about planetesimals''.

He is presently involved in 3 projects of the Forschergruppe as PI, in many of
them as Co-I.
He has launched initiatives to transfer geoscientific/mineralogic know-how and/or
materials to groups working on minerals in space. Dr. Trieloff is engaged in two 
excellence proposals of the University of Heidelberg (Cluster of excellence
proposal in astrophysics/physics, 
PI: J. Wambsganss, and graduate school Spaceship Earth, PI: W. Shotyk).

%
\section{Funding Request}
%
We request funding for {\bf 1 full professorship} to be established at the
Mineralogical Institute of the University of Heidelberg.

An official statement of the University of Heidelberg in support of
this application, and the agreement to continue this position after
termination of the Forschergruppe will be provided at a later time. 
%

\medskip

\noindent 
Estimated Costs of the Professorship for the first 3 years.\\[0.4cm]
\centerline{\begin{tabular}{||l|r|r|r||}
\hline \hline & Year 1 & Year 2 & Year 3 \\ \hline %
W3-Position            & 80.000  & 80.000  & 80.000  \\
\hline
{\bf Total:} (EUR)     & 80.000  & 80.000 &  80.000 \\
\hline \hline
\end{tabular}
}


