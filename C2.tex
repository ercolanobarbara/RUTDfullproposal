
%%%%%%%%%%%%%%%%%%%%%%%%%%%%%%%%%%%%%%%%%%%%%%%%%%%%%%%%
% LaTex Template for proposals within the              %
% DFG Research Unit Program                            %         
%Planet Formation Witnesses and Probes: Transition Discs
% August 2016                                            %                           
%                                                      %
%%%%%%%%%%%%%%%%%%%%%%%%%%%%%%%%%%%%%%%%%%%%%%%%%%%%%%%%
%
% 
%
% This template may be used to prepare proposals in latex.
%
%
% The project description, including publication list, should be no more than 20 pages
% in length. It should be self-explanatory and not require reviewers to read the 
% literature that is quoted or enclosed.

\documentclass[10pt,fleqn,twoside]{article}

%%%% USE ARIAL FONT %%%%%%%%%%%%%%%%%%%%%%%%%%%%%%%%%%%%%%%%%%%%%%%%%%%%%%
\usepackage{helvet}
\renewcommand\familydefault{phv}

%%%% INCLUDE NECESSARY PACKAGES %%%%%%%%%%%%%%%%%%%%%%%%%%%%%%%%%%%%%%%%%%
%\usepackage{babel}
\usepackage[UKenglish]{babel}
\usepackage{amsmath}
\usepackage{amssymb}
\usepackage{fancyhdr}
\usepackage{natbib}
\usepackage[usenames,dvipsnames]{xcolor}
\usepackage{ae,aecompl}
\usepackage{graphicx}
\usepackage{palatino}
\usepackage[T1]{fontenc}
\usepackage[utf8]{inputenc}
\usepackage{rotating}
\usepackage{epsf}
\usepackage{setspace}
\usepackage{sfmath}

%%%% PAGE LAYOUT %%%%%%%%%%%%%%%%%%%%%%%%%%%%%%%%%%%%%%%%%%%%%%%%%%%%%%%%%
\setlength{\textheight}{22cm}
\setlength{\topmargin}{-1.2cm}
\setlength{\textwidth}{15.6cm}
\setlength{\oddsidemargin}{0.0cm}
\setlength{\evensidemargin}{0.0cm}
\setlength{\mathindent}{1.5cm}
\setlength{\parindent}{0.0cm}
\setlength{\parskip}{0.08cm}

%%%% PAGE HEADER %%%%%%%%%%%%%%%%%%%%%%%%%%%%%%%%%%%%%%%%%%%%%%%%%%%%%%%%%%
\pagestyle{fancy}
\fancyhead[RE,RO]{}
\fancyfoot[RO]{ \thepage}
\fancyfoot[LE]{ \thepage}
\fancyfoot[CE,CO]{}

%%% FONTS FOR THE TITLE PAGE %%%%%%%%%%%%%%%%%%%%%%%%%%%%%%%%%%%%%%%%%%%%%%
\newfont{\tpfonta}{cmssbx10 scaled 1600}
\newfont{\tpfontb}{cmssbx10 scaled 3200}

%%%% EURO SIGN %%%%%%%%%%%%%%%%%%%%%%%%%%%%%%%%%%%%%%%%%%%%
\newcommand\euro{{\sffamily C%
\makebox[0pt][l]{\kern-.70em\mbox{--}}%
\makebox[0pt][l]{\kern-.68em\raisebox{.25ex}{--}}}}
\newcommand\keuro{k{\sffamily C%
\makebox[0pt][l]{\kern-.70em\mbox{--}}%
\makebox[0pt][l]{\kern-.68em\raisebox{.25ex}{--}}}}

%%%% COLOR DEFINITIONS %%%%%%%%%%%%%%%%%%%%%%%%%%%%%%%%%%%%
\definecolor{blue} {rgb} {0.25,0.25,0.75}

%%%% ADDITONAL EMPHASIS %%%%%%%%%%%%%%%%%%%%%%%%%%%%%%%%%%%
\newcommand{\cem}{\color{blue}}
\newcommand{\eem}{\sl\color{blue}}

%%%% SET THE COLOR OF THE (SUB-) SECTION TITLES %%%%%%%%%%% 
\newcommand{\Tcol}{\color{blue}}

%%%% SET THE COLOR OF THE TITLE BOX BACKGROUND %%%%%%%%%%%%
\definecolor{Background}{rgb} {0.62,0.75,0.5}

% ========= hyperref & Colors & Links ===========
\usepackage[breaklinks,colorlinks,linkcolor=Maroon,citecolor=Blue]{hyperref}%removed backref
\usepackage[all]{hypcap} % fixes links to floats
%\usepackage[capitalise,nameinlink]{cleveref} %
\usepackage{aas_macros} %
\setlength{\bibsep}{-0.5pt}

% ========= highlighting important parts of the proposal ===========
\newenvironment{highlight}{\Tcol\itshape}{}

%%%%%%%%%%%%%%%%%%%%%%%%% COMMENT BOXES %%%%%%%%%%%%%%%%%%%%%%%%%%%%%
\usepackage[colorinlistoftodos,textsize=scriptsize,textwidth=2cm]{todonotes}
\newcommand{\til}[1]{\todo[color=LimeGreen,inline]{Til: #1}}
\newcommand{\barbara}[1]{\todo[inline]{Barbara: #1}}

%%%%%%%%%%%%%%%%%%%%%%% REMOVE SPACE IN LISTS %%%%%%%%%%%%%%%%%%%%%%%

\usepackage{enumitem}
\setlist{nosep}

%%%%%%%%%%%%%%%%%%%%%%% NICER REFERENCES %%%%%%%%%%%%%%%%%%%%%%%%%%%%

\usepackage[capitalise,nameinlink]{cleveref}

%%%% REFERENCE SECTION NAME %%%%%%%%%%%%%%%%%%%%%%%%%%%%%%%
\renewcommand\refname{\Tcol 9. Bibliography}

%%%% COLOR THE SECTION NUMBERS %%%%%%%%%%%%%%%%%%%%%%%%%%%%%%%
\makeatletter
\renewcommand\@seccntformat[1]{\color{blue} {\csname the#1\endcsname}\hspace{0.5em}}
\makeatother
\renewcommand\thesection{\arabic{section}.}
\renewcommand\thesubsection{\arabic{section}.\arabic{subsection}}

%%%% CHANGE THE APPEARANCE OF THE \PARAGRAPH COMMAND  %%%%%%%%%%%%%%%%%%%%%%%%%%%%%%%
\makeatletter
\renewcommand\paragraph{\@startsection{paragraph}{4}{\z@}%
            {-2.5ex\@plus -1ex \@minus -.25ex}%
            {1.25ex \@plus .25ex}%
            {\normalfont\normalsize\bfseries}}
\makeatother
\setcounter{secnumdepth}{4}     % how many sectioning levels to assign numbers to
\setcounter{tocdepth}{4}        % how many sectioning levels to show in ToC


\fancyhead[LE,LO]{\slshape
%%%%  Please edit
%
Ercolano \&
Birnstiel: RU Transition Discs project C2}
%
%
%%%%%


\begin{document}


\newpage

%%%% PROJECT DESCRIPTION STARTS HERE %%%%%%%%%%%%%%%%%%%%%%%%%%%%%%%%%%%

\setcounter{page}{1}

\centerline{\huge\bf\Tcol
%
%
%
%
%%%%  Please edit
%
 Project C2:}

\centerline{\huge\bf\Tcol Gone with the wind:}
\centerline{\huge\bf\Tcol Dust entrainment in photoevaporative winds}

%
%%%%
%
%
%
%
\vskip1.0cm

%%%%  Please edit

\noindent{\bf Authors:}\\
\begin{tabular}{ll}
{\textsf{PI:}}                   & B.~Ercolano (LMU) \& T.~Birnstiel (LMU)\\
{\textsf{Co-I:}}                & K.~Dullemond (Heidelberg)\\
{\textsf{Collaborations:}}      &  James Owen (Princeton, USA), P.~Caselli (MPE), G. Picogna (LMU)\\

\end{tabular}

%%%%  Please edit

\vspace{1em}
\noindent{\bf Requested positions: 1PhD student} \\

\vspace{1em}
\noindent{\bf Abstract:}\\

The search for the smoking gun of disc dispersal via photoevaporative
winds, which destroy discs via the formation of Type 1 TDs,  has until
now failed to identify suitable tracers. Quantitative spectroscopy of
YSOs to search for blue-shifted emission lines produced in the wind
relies on an accurate characterisation of the thermochemical
properties of the winds. A central ingredients for the chemical
calculations is the dust content of the wind as micron sized grains
provide the dominant opacity channel in the far-ultraviolet,
furthermore small particles are important players in the temperature
balance of the gas via the photoelectric process.  

We will use realistic radiation-hydrodynamic models of
photoevaporative winds coupled to dust evolution models for the
underlying grain distribution in the disc, to calculate the dust
entrainment in winds to feed to chemical models. The observability of
the continuum emission due to the dust grains in winds from edge-on
discs, a potential new diagnostic, will be estimated both for Herbig
Ae stars and for their fainter T-Tauri counterparts.  

\section{\Tcol State of the art and preliminary work}
\renewcommand{\leftmark}{\sc State of the Art and preliminary work}

The dispersal of protoplanetary discs plays a crucial role in the
planet formation process, and leads to the formation of Type
1 TDs. While photoevaporation from the central star has been proposed
as the dominant disc-dispersal mechanism around low-mass stars
(e.g. Clarke et 2001), to date only tentative evidence exists of a
wind detection, via blue-shifted forbidden line emission of mostly
NeII and OI (e.g. Hartigan, Edwards \& Ghandour 1995; Alexander 2008;
Pascucci \& Sterzik 2009; Schisano, Ercolano \& Guedel 2010; Ercolano
\& Owen 2010, 2016). These lines can only probe the wind on very local
scales and they cannot be inverted to obtain mass loss rates, which
are crucial to pin down the dominating mechanism which drives the disc
photoevaporative wind
(i.e. EUV, FUV or X-ray - or a combination). Different driving
mechanism induce more or less vigorous mass loss at different disc
radii, which  can have dramatic effect on planet formation, both at
the times of planetesimal assembly and for the later dynamical
evolution of planet(esimal)s (e.g. Ercolano \& Rosotti 2015).  

Owen, Ercolano \& Clarke (2011b) demonstrated that in the case of
Herbig Ae/Be stars an EUV-driven wind, the wind selectively entrains
grains of different sizes at different radii resulting in a dust
population that varies spatially and increases with height above the
disc at radii larger than about 10~AU. At near infrared wavelengths
this variable grain population produces a 'wingnut' morphology which
may have already been observed in the case of PDS 144N (Perrin et al.
2006). The work of Owen et al. (2011b) could not however reproduce the
colour gradient of the observations, which show redder emission at
larger heights above the disc. Possibly, the problem was due to the
fact that the synthetic images were dominated by emission from the
smallest grains entrained in the flow. Grain growth in the underlying
disc, neglected in the
Owen et al. (2011b) calculations in the disc is a natural solution to
the colour problem, which needs to be taken into account in future
simulations. 

While it is currently not clear if the PDS 144N observation can be
explained by dust entrainment in a photoevaporative wind, the work of
Owen, Ercolano \& Clarke (2011b) has clearly demonstrated that a
significant amount of small grains (which dominate the opacity in the
FUV) do populate disc winds, hence playing an important part in the
chemistry there and at the base of the flow. 

More recently Hutchison et al. (2016ab) investigated the question of
dust entrainment in protoplanetary discs using smoothed particle
hydrdynamics. While their models are very idealised (non-rotating,
plane-parallel discs), their simulations seem to support most of the
conclusions reported in the analytical work of Owen, Ercolano \&
Clarke (2011b). These models also focus on a EUV-driven wind.  


\subsection*{Dust-entrainment in winds: modelling strategies}

In this section the two main approaches to model dust-entrainment in
winds are described in more detail. The limitations of both approaches
will also be discussed, thus highlighting the knowledge gap that our
project is aiming to fill. 

\subsubsection*{Analytical approach}\label{sec:analytical_approach}

{\color{red}Describe approach used in Owen, Ercolano \& Clarke 2011b

--why can we decouple photoevaporation calculations from dust
entrainment calculation and why is it necessary to do so for our aims

-- discuss shortcomings of current models: The Owen, Ercolano \& Clarke (2011b) calculation are limited to the EUV-case
only and do not include dust-evolution in the underlying disc, for
these reasons their applicability to models aiming at quantitative
spectroscopy of disc winds is rather limited. 
}

Owen, Ercolano \& Clarke (2011b) post-processed
radiation-hydrodynamical simulations of photoevaporating disc winds
around Herbig Ae/Be stars in order to study the evolution and
observational appearance of dust grains entrained in the wind. Their
approach involved three steps: (i) (radiation)-hydrodynamical
calculations of the photoevaporative wind; (ii) calculation of the
dust profile distribution in the wind; (iii) radiative transfer
calculation of the dust distribution to infer the observational
appearance. 

As the work of Owen, Ercolano \& Clarke (2011b) was motivated by the
observations of several edge-on discs around Herbig stars, which
showed extended emission above and below their midplane at NIR
wavelengths (e.g. Padgett et al. 1999, Perrin et al. 2006), these
authors focussed in step (i)  on EUV-driven winds (e.g. Hollenbach et al. 1994;
Font et al. 2004; Alexander, Clarke \& Pringle 2006a,b). Indeed Herbig
stars have generally a much lower X-ray luminosity relative to their
bolometric luminosity, compared to T-Tauri stars, raising questions on
what the driving radiation may be for intermediate mass stars. For
that reason Owen, Ercolano \& Clarke (2011b) chose to adopt the
hydrodynamic solution of Font et al. (2004), which is simple and
scalable, hence allowing them to investigate a wider range of
parameter space. As we will see in Section 2, this option is not
available for us, as we are interested in X-ray driven
photoevaporative winds, which are more likely to occur around T-Tauri
stars. 

In step (ii) Owen, Ercolano \& Clarke (2011b) calculate streamlines
from the base of the flow to the edge of their grid from the solutions
obtained in step (i). Along each of the streamlines the balance
between the drag force, gravity and the centrifugal force is then
calculated for a given grain size: a positive net force is interpreted
as grains of that size being entrained in the flow from that
location. 

From step (ii) the density distribution of dust grains in the wind is
then calculated and post-processed by means of radiative
transfer in step (iii) . Figure ??? shows the synthetic images they
obtained from their models in L,K \& H band and composite which they used to compare
with the observations of Perrin et al. (2006). 


%INCLUDE FIGURE HERE!!!!%%
%\paragraph{Limitations of current models}
The calculations of Owen, Ercolano \& Clarke (2011b) present however
some serious limitations that make them unsuitable for application to our Research
Unit tasks. 

First of all these calculations are limited to the EUV-case only. The
X-ray photoevaporation case, which is likely dominant amongst T-Tauri
stars, is much more complex as the temperatures
of an X-ray ionised gas vary from a few hundred to a few thousand
Kelvin, the hydrogen gas is in a quasi-neutral state of ionisation, with the
exact ionisation level also influencing the efficiency of the X-ray
heating. While the EUV-photoionisation process occurs of via the removal
of a single valence electron, X-rays generally remove an inner shell
electron, with the ejected suprathermal electron producing secondary
ionisation. A further complication is that multiple electrons may be
ejected as a consequence of inner shell ionisations, linking together
non-adjacent ionisation levels.

 The host of microphysics regulating
X-ray ionisation and heating is self-consistently included in the
three-dimensional photoionisation code MOCASSIN (Ercolano et al. 2003,
2005, 2008a) and has been applied to the X-ray photoionisation and
photoevaporation process of protoplanetary discs (Ercolano et
al. 2008b, 2009, 2010, 2016; Owen et al. 2010, 2011, 2012). 
The X-ray photoevaporation models calculated in project B1 are based
on a temperature scheme obtained by the PI with the MOCASSIN code.

Another important shortcoming of Owen, Ercolano \& Clarke (2011b) is
that they do not account for dust evolution in the underlying disc. An
MRN size distribution with standard gas-to-dust ratio of 100 is
assumed everywhere in the disc. The resulting dust density and size
distribution in the wind is thus necessarily incorrect. Given the
central role played by dust grains for the chemistry, our project will
couple dust evolution in the underlying disc to the wind entrainment
problem. 

\subsubsection*{Numerical approaches}

The dynamics of dust grains in protoplanetary discs can be studied either by directly integrating the orbits of a large number of dust 'super-particles' (that sample the local properties of the dust population) or by solving the collisionless Boltzmann equation for the particle distribution function. For a population of very small (tightly coupled with the gas) dust particles, the Boltzmann equation can be reduced to the zero pressure fluid equation (Cuzzi et al. 1993, Garaud et al. 2004). This 'two-fluid' approach has been used to study planet-disc interactions (Paardekooper \& Mellema 2004, 2006; Zhu et al. 2012), however it is limited to a single population of small particles, it cannot account for the full velocity distribution of the grains at a single location, and it is not able to capture strong density gradients. The particle approach has the great advantage to follow the evolution of solid particles with different physical properties recovering perfectly the dust dynamics in the limit where the grains are decoupled from the gas (Youdin \& Johansen 2007; Miniati 2010; Bai \& Stone 2010). This method has also been applied successfully to the study of planet disc interaction adopting both SPH and grid-based codes (Fouchet et al. 2007, 2010; Ayliffe et al. 2012; Lyra et al. 2009; Zhu et al. 2014). The grid codes are preferred because they do not introduce a large artificial viscosity that can affect the evolution of low-mass planets. Moreover, the accuracy needed to properly model the evolution of the gas and dust component in a protoplanetary discs is strongly dependent on the choice of the grid geometry (Lyra et al. 2009; de Val-Borro et al. 2007), requiring more computational effort in a cartesian grid than in a cylindrical or spherical one.

Dr. Picogna (project B1) has implemented a population of dust
particles in the modern grid-based code PLUTO (Mignone et al. 2012)
that can evolve both in a cylindrical and spherical coordinate system
(Picogna, Stoll \& Kley, in prep.). This approach is thus ideal to
study the evolution of different dust particle populations in
protoplanetary discs. As detailed in the next section, this method
coupled with the photoevaporation model implemented in PLUTO will be
adopted to self-consistently model dust particles entrained into the
wind from the disc atmosphere for a number of selected cases.

The recent work Hutchison et al. (2016a,b) used a new algorithm to
treat a wind in a smooth particle hydrodynamics (SPH) code. The wind is
treated using unequal-mass, one-fluid SPH. Using new techniques
developed by these authors they attempt to simulate two-fluid dynamics
in highly stratified atmospheres. The work represents currently still
only a proof of concept, showing that these novel techniques may be in
the future applied to study interesting aspects of gas and dust
dynamics in the wind. At present however the models are very
idealised, approximating discs and winds by a thin, non-rotating,
plane-parallel atmosphere.  This technique is thus not yet mature to
be used for the purposes of our project.

\subsection*{Grain sizes and abundances at the base of the wind}

{\color{green}There are basically two approaches - to zeroth order one
  can use 1d dust evolution models to set the abundance and size
  distribution at the base of the flow. For that it is enough to use
  Birnstiel, Klahr \& Ercolano 2012. 

-- describe what this paper does 

A more sofisticated approach is to use 2d models which resolve the
vertical abundance and size distribution of the grains. 

-- describe such a model -- what is the state of the art on this? -- 

-- how likely it is that one could also find some kind of
parameterisation that could be easily coupled to the grian entrainment calculations?
}


\subsection{\Tcol Project-related publications}

% Please list your own publications related to the proposed project, 
% adhering to the rules of the DFG guidelines 1.91. In brief, please note: 
% - Up to 10 publications
% - The work must be published or accepted.
% - Publications on astro-ph (arXive, SPIRES or articles with a DOI) count as published. 
% - Any work that is only in the status ``accepted'' MUST be attached to the proposal
%    together with the acceptance letter.
% - All publications in this section CAN be attached to the proposal. Please limit these
%    attachments to a minimum and please note that the reviewers may not read the attachments -
%    the proposal has to speak for itself.
% - The number of allowed publications refers to the sum of the publications listed
%    in ``1.1.1 Articles published or officially accepted by publication outlets...'' and 
%    in ``1.1.2 Other publications''. Publications which only exist on public repositories 
%    belong into the category ``Other Publications''.
[Text]

\subsubsection{\Tcol 
Articles published or officially accepted by publication outlets with scientific quality assurance;
book publications}

[Text]

\subsubsection{\Tcol Other publications}

[Text]

%\subsubsection{\Tcol Patents}

%\paragraph{\Tcol Pending}

%[Text]

%\paragraph{\Tcol Issued}

%[Text]

\section{\Tcol Objectives and work programme}
\renewcommand{\leftmark}{\sc Objectives and work programme}


\subsection{\Tcol Anticipated total duration of the project}

36 months

\subsection{\Tcol Objectives}
In this project
we aim to determine the dust content of photoevaporative winds for a range of stellar, disc and wind parameters,
using realistic descriptions for grain evolution in the underlying
disc. 

This will allow us to : 
\begin{enumerate}
\item Build a dust model for photoevaporative winds to be used in
astrochemistry and radiative transfer calculations.
\item Estimate the observability and observation characteristics of
the dust phase in photoevaporative winds.
\end{enumerate}

\subsection{\Tcol Work programme including proposed research methods}
% for each applicant

In this project we will use the semi-analytical approach described in
\cref{sec:analytical_approach} coupled to models for the
evolution of dust grains in the disc in order to efficiently produce
dust entrainment models for the complete set of X-ray photoevaporation
wind solutions calculated in project B1.

There are several reasons for choosing the semi-analytical approach
over fully numerical approaches (SPH or grid-based). Semi-analytical
approaches are much more efficient for our aims. While we will make
use of hydrodynamical calculations, the 
grain entrainment problem from the photoevaporation problem. Indeed
dust grains entrained in the wind do now provide tangible extinction
at X-ray frequencies, hence the wind structure and rate are
not sensitive to the dust distribution in the wind. The dust
entrainment calculation can thus be performed as a sort of
post-processing of the radiation-hydrodynamic calculations, allowing
to use the same wind solution and changing (e.g.) the underlying dust
distribution in the disc, according to different models and
prescriptions. 

It is perhaps worth mentioning that the dust evolution in the
underlying disc is however strongly couple to the evolution of the gas
in the disc. A full radiation-hydrodynamic simulation of the gas and
dust which also simultaneously solves the dust coagulation and drift
is however beyond what is currently feasible. It is however possible
to perform a gas radiation hydrodynamics calculation which includes
dust particles, not accounting for coagulation (e.g. ????REF
Picogna???). If time allows, we will perform a number of such
(expensive) calculations to compare with the semi-analytical
ones. However, a more comprehensive set, will probably be performed
during the second funding period of the Research Unit. 







The work will be carried out by a PhD student co-supervised by
Prof. Ercolano (LMU) and Prof. Birnstiel (LMU), who will provide
guidance, respectively, on the dust entrainment calculations and the
coupling with the dust evolution models for the underlying disc. 

\subsection{Research Tools}

For this project we will need the following tools:
\begin{enumerate}
\item Photoevaporative wind solutions for EUV-driven
models for T-Tauri and Herbig stars. These are easily computable 
using the models that are described in the literature \citep[e.g.,][]{2004ApJ...607..890F}.
\item Photoevaporative wind solutions for X-ray photoevaporated
models. These will be provided from project B1.
\item One-dimensional dust growth models. These codes are published
and available within the group: simulation code of 
 \citet{2010A&A...513A..79B} and the parameterized model of
 \citet{2012A&A...539A.148B}. The one dimensional results will be
 coupled to prescriptions of vertical mixing (e.g. ??????REF)
\item Two-dimensional dust evolution results/parameterisation from the
  ERC-funded projects of Prof. Birnstiel.
\item A 3D radiative transfer code to post-postprocess the wind
models with the calculated grain populations. We will make use  of the
RadMC code developed and maintained by Prof. Dullemond (e.g. REF?????).
\end{enumerate}

\subsection{Research Plan} 

The project will proceed in stages of increasing complexity. We will start
by setting up a framework that can be benchmarked against available
calculations and progressively adding new elements, as they become
available from other sub-projects. The plan has been designed to fit a
PhD student, who will have the opportunity to develop a new theoretical
model as well as acquainting her/himself with standard numerical
techniques (e.g. radiation-hydrodynamics, dust evolution models,
radiative transfer). 

The most important science product from this project is the set of
grain models developed for the  X-ray driven wind. These are needed by
project B2 for the chemical calculations, as the dust
grains dominating the opacities in the FUV are not equally distributed
in the wind (see e.g. Owen, Ercolano \& Clarke, 2011b, Hutchison et
al. 2016ab) and they thus affect the chemistry in the wind
differently in different parts. 

\subsubsection{Months 1-12}
The student will start by producing wind solution for the EUV case
from the work of \citet{2004ApJ...607..890F}, which may be applicable
to Herbig stars. For that she will use the standard version PLUTO code (???REF) in
two-dimensional mode. Note that no additional column density/Stroemgren radius
calculations need to be used in order to implement the Font et
al. (2004) solutions for EUV winds. This solution simply treats the
wind as an isothermal gas with sound speed, $c_s = 10km s^{-1}$. The
number density at the base of the wind is also fixed and is a simple
function of radius, mass of the central star, gas sound speed and
ionising stellar flux (Font et al. 2008, Alexander 2009, Hollenbach et
al. 2004). Ww will be able to validate our solutions with those in the
literature, which is an important step, particularly in a project led
by a PhD student. 

The student will then proceed to calculate the dust
distribution in the wind, under simplifying assumptions for the
underlying dust distribution in the disc as in \citet{2011MNRAS.411.1104O}. In
brief, streamlines from the base of the flow to the edge of the grid
will be computed and along each of them, the force balance between the
drag force, gravity and the centrifugal force will be calculated. A
positive net force on a grain along the streamline will indicate that
the grain is entrained.

Once dust abundance and size distributions have been obtained for the
wind, radiative transfer calculations will be performed to produce
synthetic continuum observations at several disc inclinations. 
These first models will be benchmarked against the solutions
of \citet{2011MNRAS.411.1104O}.

After the benchmarking tests, we will be sure that we have developed a
solid framework which can now be applied, for the
first time, to the calculation of the dust component entrained in an
X-ray driven photoevaporative wind. The wind solutions of Owen et
al. (2010, 2011, 2012) and Ercolano \& Owen (2016) are readily
available and they could provide a starting point, until new 
wind solutions become available form sub-project B1. This may however
not be necessary as the new wind models for the X-ray case
are already being calculated by Dr. Picogna, who is employed to do the
preparatory work from project B1. It is therefore likely that an
initial set of high resolution new X-ray wind solution may already be
available to the student right from the start of project C2.
 
As the first dust models for the X-ray driven wind become available
they will be immediately passed on to project B2 (Astrochemistry), for
inclusion in the chemical models to be then updated when the new
models including dust evolution in the disc become available (see next
section). 

We will perform new radiative transfer calculations of the dusty
X-ray driven wind to produce synthetic continuum observations and
provide a first estimate of the observability of such winds with
current/future instrumentations. Our results may motivate
observational campaigns led by collaborators (e.g. the group of
Prof. Henning), the outcome of which however does not influence the
success of the research aims of our projects. 

 {\color{red} this is very vague what
  instrumentations? in what mode? have such observations already been
  attempted?}


\subsubsection{Months 13-24}
 
At this point we will be in a position to significantly improve on
this work by considering more realistic grain abundances and size
distributions for the underlying disc. 

We will first couple the grain entrainment calculations to simple
prescriptions of dust  evolution \citep[e.g.,][]{2012A&A...539A.148B}
obtained from the one-dimensional models of
\citet{2010A&A...513A..79B}. The one dimensional models describe the
evolution of dust that is mostly in the mid-plane, i.e. well below the
base of the wind, where the grains may be entrained from. The
Birstiel, Klahr \& Ercolano (2012) prescriptions will then need to be
coupled to vertical mixing prescriptions (??????REF) in order to
estimate the grain distributions at the wind-launching location,
obtained from the hydro simulations. 

Complementary to this project the ERC-funded team led by
Prof. Birnstiel aims to develop new two-dimensional (both
radius-height and radius-azimuth) dust evolution
models. When these new models become available, the student will work
closely with Prof. Birnstiel to include elements of these new results
in our calculations. 

\subsubsection{Months 25-36}

In this last year of the first funding period the whole machinery will
be in place. We will now be able apply it to a wide parameter
space of disc winds, producing sets of dust models to be passed on to
project B2 for the chemical calculations. 

We will also perform comprehensive radiative transfer calculations of
the obtained structures to compare with available observations or to
make more detailed observability predictions, which may further guide
future observing proposals. We will join forces with expert
collaborators on scattered light observations (e.g. Prof. Henning) to
plan new proposals, however we stress that failure to obtain new
observations as well as eventual non-detections do not preclude the
main aims of this projects to be achieved. These are the development
of dust models and dust distributions in photoevaporative winds. 

If time allows, the student will collaborate with Dr Picogna (B1) to
produce full hydrodynamical simulations of disc winds, where the dust
component in the disc and wind is treated as particles (e.g. Picogna
\& Kley 2016). These calculations, which are computationally very expensive
will be useful as a piecewise comparison to the simpler methods previously
developed by the student in the project.  

\vspace{0.5em}


\subsection{\Tcol Data handling}

The model data-grids will be made available on the Research Unit
dedicated server for use within the team. Furthermore we will provide
a set of diagnostic models to guide observers in the wider community
on the public partition of the server.
\barbara{Til can we also provide perhaps other useful data from the
dust models?}
\til{Not sure what things observers would like to use. I think we
could provide anything from the full size distribution at every 2D
point to mean grain sizes (averaged in different ways) to calculated
images.}

\subsection{\Tcol Other information}
% Please use this section for any additional information you feel is
% relevant which has not been provided elsewhere.

Not Relevant

\subsection{\Tcol Information on scientific and financial involvement of international cooperation partners}

Not Relevant

\section{\Tcol Bibliography}

\begingroup
\renewcommand{\section}[2]{}%
\bibliographystyle{aa}
\bibliography{bibliography}
\endgroup

\section{\Tcol Requested modules/funds}
\renewcommand{\leftmark}{\sc  Requested modules/funds}
% Explain each item for each applicant (stating last name, first name).

\subsection{\Tcol Basic Module}

\subsubsection{\Tcol Funding for Staff}
% Please note that funds for your own temporary position (“Eigene Stelleâ€)
% are not to be included here; this belongs to the separate “Module Temporary Positionâ€.

We require funding for one PhD student to be supervised at the LMU
jointly by Prof. Birnstiel and Prof. Ercolano.

\subsubsection{\Tcol Direct Project Costs}

\paragraph{\Tcol Equipment up to EUR 10,000, Software and Consumables}

Will be provided by the host institution. 

\paragraph{\Tcol Travel Expenses}

Total: 9900 Euro

Justification: Each year one national trip (e.g., meeting of
Astronomical Society, national meetings) and one international trip
(conference, visit to collaborators). During the course of the PhD 2
one week long visits to our main international collaborator, Dr J.
Owen (currently at Princeton University, will move to Imperial College
London in 2017).

Cost estimate: 
\begin{itemize}
\item National trip: 5 overnight stays, train/airfare,
conference fee; 1000 Euro (3000 over 3 years).
\item International trip: 6 overnight stays, airfare, conference fee;
  1500 Euro (4500 over 3 years).
\item Visit to/from J. Owen: airfare, 6 overnight stay 1200 Euro (2400
  for 2 visits)
\end{itemize}
\til{Should we increase the international trip? Seems a bit cheap to
me if it's in the US at least.}

\paragraph{\Tcol Visiting Researchers (excluding Mercator Fellows)}

Not Relevant

\paragraph{\Tcol Other Costs}

None

\paragraph{\Tcol Project-related publication expenses}

We request 750 Euro py (total 2250 Euro) for publication expenses.

\subsubsection{\Tcol Instrumentation}

None

\paragraph{\Tcol Equipment exceeding EUR 10,000} 

None

\paragraph{\Tcol Major Instrumentation exceeding EUR 100,000} 

None 

\subsection{\Tcol Module Temporary Position}

Not Relevant 

\subsection{\Tcol Module Replacement Funding}

Not Relevant 

\subsection{\Tcol Module Mercator Fellows}

Not Relevant 

\subsection{\Tcol Module Public Relations Funding}

Not Relevant 

\section{\Tcol Project requirements}
\renewcommand{\leftmark}{\sc Project requirements}

\subsection{\Tcol Employment status information}
% For each applicant, state the last name, first name, and employment
% status (including duration of contract and funding body, if on a
% fixed-term contract).

Barbara Ercolano, Professor at the Ludwig-Maximilians-Universit\"at
M\"unchen  (permanent)

Tilman Birnstiel, Professor at the Ludwig-Maximilians-Universit\"at M\"unchen starting 02/2017  (tenure-track)

\subsection{\Tcol First-time proposal data}
% Only if applicable: Last name, first name of first-time applicant.

Not Relevant

\subsection{\Tcol Composition of the project group}
% List only those individuals who will work on the project but will not
% be paid out of the project funds. State each person’s name, academic
% title, employment status, and type of funding.

\textcolor{red}{[Text]}

\subsection{\Tcol Cooperation with other researchers}

\subsubsection{\Tcol Planned cooperation on this project}

\paragraph{\Tcol Collaborating researchers for this project within the
  Research Unit}
%Each proposal must be accompanied by a description of how the project
%is integral to the Research Unit, %both in terms of subject matter
%and organisation. This includes a description of the cooperation with
%%others participating within the Research Unit. 

{\color{red} Expand below - include the researchers that are linked to
  the various projects}

The project will use the wind models calculated in project B1 and then
feed back the dust model to the same project (B1) and to the reduced
chemical network tests of project B2. Dust evolution calculations from C1 will also be used. Observational constraints will
be obtained in collaboration with experts working on project A1 and
stellar properties to guide the models 
from project A2. 


\paragraph{\Tcol Collaborating researchers for this project outside of
  the Research Unit}

Dr. James Owen, currently at Princeton University, from 2017 at
Imperial College London. 

{\color{red} Expand above - mention James' contributions to the state
  of the art and his likely contribution to the project - mention visits}

\subsubsection{\Tcol Researchers with whom you have collaborated scientifically within the past three years}
% This information is important for DFG to exclude possible conflicts of interest.
% Please mention not only the names of the cooperation partners but also their institution and city.
% Scientists already mentioned in the previous two subsubsections do not have to be mentioned
% again.

%BARBARA:
% F.~Niederhofer (STSci, USA),
% M.~Hilker (ESO, Garching),
% N.~Bastian (U. Liverpool, UK),
% M.~Guarcello (U. Palermo, Italy),
% M.~Tazzari (U. Cambridge, UK),
% A.~Natta (Florence, Italy),
% R.~Alexander (U. Leicester),
% D.~Hubber (LMU),
% J.~Dale (U. Hertfordshire, UK),
% C.~Koepferl (LMU),
% I.~Bonnell (U. St. Andrews, UK),
% A.~McLeod (ESO, Garching),
% D.~Boneberg (U. Cambridge, UK),
% R.~Parker (U. Liverpool, UK),
% R.~Wesson (UCL, London, UK),
% M.~Barlow (UCL, London, UK),
% A.~Glassgold (U. Berkeley, USA),
% C.~Manara (ESA, Noordwjik, Netherlands),
% A.~Danekhar (CfA, Harward, USA),
% Q.~Parker (Sidney, Australia),
% S.~Casassus (U. de Chile, Santiago, Chile),
% I.~Pascucci (U. Arizona, USA),
% A.~Bevan (UCL, London, UK).

%TIL:
% S.~Andrews (Harvard, USA),
% X.~Bai (Harvard, USA),
% A.~Banzatti (STScI Baltimore, USA),
% M.~Benisty (IPAG Grenoble, FRA),
% J.~Carpenter (California Institute of Technology),
% C.~Carrasco-González (UNAM, MEX),
% P.~Cazzoletti (MPE, DEU),
% C.~Dominik (Univ. Amsterdam, NLD),
% C.~Dullemond (Univ. Heidelberg, DEU),
% A.~Dutrey (Univ. Bordeaux, FRA),
% M.~Fang (Purple Mountain Obs., CHN),
% M.~Flock (JPL, USA),
% U.~Gorti (SETI Institute, USA),
% S.~Guilloteau (Univ. Bordeaux, FRA),
% T.~Henning (MPIA, DEU),
% M.~Hogerheijde (Leiden Observatory, NLD),
% A.~Isella (Rice Univ., USA),
% A.~Johansen (Lund Univ., SWE),
% M.~Kama (Leiden Observatory, NLD),
% A.~Kataoka (Univ. Heidelberg, DEU),
% H.~Klahr (MPIA, DEU),
% H.~Linz (MPIA, DEU),
% R.~Murray-Clay (UCSB, USA),
% A.~Natta (DIAS, IRL),
% P.~Pinilla (Leiden Observatory, NLD),
% A.~Piso (Harvard, USA),
% A.~Pohl (MPIA, DEU),
% L.~Pérez (MPIfR, DEU),
% L.~Ricci (Harvard, USA),
% V.~Roccatagliata (LMU, DEU),
% K.~Rosenfeld (Harvard, USA),
% D.~Semenov (MPIA, DEU),
% R.~Teague (MPIA, DEU),
% L.~Testi (ESO),
% C.~Walsh (Leiden Observatory, NLD),
% D.~Wilner (Harvard, USA),
% Z.~Zhu (Princeton U., USA),
% M.~de Juan Ovelar (Liverpool Univ., GBR),
% R.~van Boekel (MPIA, DEU),
% E.~van Dishoeck (Leiden Observatory, NLD),
% N.~van der Marel (Leiden Observatory, NLD),
% K.~Öberg (Harvard, USA),

%MERGED:

R.~Alexander (U. Leicester),
S.~Andrews (Harvard, USA),
X.~Bai (Harvard, USA),
A.~Banzatti (STScI Baltimore, USA),
M.~Barlow (UCL, London, UK),
N.~Bastian (U. Liverpool, UK),
M.~Benisty (IPAG Grenoble, FRA),
A.~Bevan (UCL, London, UK).D.~Boneberg (U. Cambridge, UK),
I.~Bonnell (U. St. Andrews, UK),
J.~Carpenter (California Institute of Technology),
C.~Carrasco-González (UNAM, MEX),
S.~Casassus (U. de Chile, Santiago, Chile),
P.~Cazzoletti (MPE, DEU),
J.~Dale (U. Hertfordshire, UK),
A.~Danekhar (CfA, Harward, USA),
C.~Dominik (Univ. Amsterdam, NLD),
C.~Dullemond (Univ. Heidelberg, DEU),
A.~Dutrey (Univ. Bordeaux, FRA),
M.~Fang (Purple Mountain Obs., CHN),
M.~Flock (JPL, USA),
A.~Glassgold (U. Berkeley, USA),
U.~Gorti (SETI Institute, USA),
M.~Guarcello (U. Palermo, Italy),
S.~Guilloteau (Univ. Bordeaux, FRA),
T.~Henning (MPIA, DEU),
M.~Hilker (ESO, Garching),
M.~Hogerheijde (Leiden Observatory, NLD),
D.~Hubber (LMU),
A.~Isella (Rice Univ., USA),
A.~Johansen (Lund Univ., SWE),
M.~Kama (Leiden Observatory, NLD),
A.~Kataoka (Univ. Heidelberg, DEU),
H.~Klahr (MPIA, DEU),
C.~Koepferl (LMU),
H.~Linz (MPIA, DEU),
C.~Manara (ESA, Noordwjik, Netherlands),
A.~McLeod (ESO, Garching),
R.~Murray-Clay (UCSB, USA),
A.~Natta (DIAS, IRL),
F.~Niederhofer (STSci, USA),
R.~Parker (U. Liverpool, UK),
Q.~Parker (Sidney, Australia),
I.~Pascucci (U. Arizona, USA),
P.~Pinilla (Leiden Observatory, NLD),
A.~Piso (Harvard, USA),
A.~Pohl (MPIA, DEU),
L.~Pérez (MPIfR, DEU),
L.~Ricci (Harvard, USA),
V.~Roccatagliata (LMU, DEU),
K.~Rosenfeld (Harvard, USA),
D.~Semenov (MPIA, DEU),
M.~Tazzari (U. Cambridge, UK),
R.~Teague (MPIA, DEU),
L.~Testi (ESO),
C.~Walsh (Leiden Observatory, NLD),
R.~Wesson (UCL, London, UK),
D.~Wilner (Harvard, USA),
Z.~Zhu (Princeton U., USA),
M.~de Juan Ovelar (Liverpool Univ., GBR),
R.~van Boekel (MPIA, DEU),
E.~van Dishoeck (Leiden Observatory, NLD),
N.~van der Marel (Leiden Observatory, NLD),
K.~Öberg (Harvard, USA),


\subsection{\Tcol Scientific equipment}
% List larger instruments that will be available to you for the
% project. These may include large computer facilities if computing
% capacity will be needed. 

The group of Prof. Ercolano has two own computer clusters comprising 

\begin{itemize}
\item 2 CPU Intel Xeon X5650 (Westmere, beginning
2010, 2.66 GHz) 6 cores each 12 cores total (24 virtual) 74 GB ram.

\item 4 CPU Intel Xeon E7-4850 (Ivy Bridge, beginning 2014, 2.30 GHz)
12 cores each 48 cores total (96 virtual) 660 GB ram.

\end{itemize}

Further computational power is provided through the C2PAP facility of the Excellence Cluster to which
the group has guaranteed time. This comprises 126 nodes, each node with 2 CPU Intel Xeon E5-2680 (Sandy
Bridge, beginning 2012, 2.7 GHz) 8 cores each 16 cores total (32
virtual) 64 GB ram. Note that while the future of the Excellence
Cluster Universe is uncertain, the C2PAP facilities will be in any
case supported by the LMU. 

The Leibniz Rechnung Zentrum (LRZ) is also available to us, where still
larger facilities are available with somewhat more constrained and longer queues.

\subsection{\Tcol Project-relevant interests in commercial enterprises}
% Information on connections between the project and the production
% branch of the enterprise.

Not Relevant

\subsection{\Tcol Additional information}
% If applicable, please list proposals requesting major
% instrumentation and/or those previously submitted to a third party
% here.

Not Relevant

\end{document}
