
%%%%%%%%%%%%%%%%%%%%%%%%%%%%%%%%%%%%%%%%%%%%%%%%%%%%%%%%
% LaTex Template for proposals within the              %
% DFG Research Unit Program                            %         
%Planet Formation Witnesses and Probes: Transition Discs
% August 2016                                            %                           
%                                                      %
%%%%%%%%%%%%%%%%%%%%%%%%%%%%%%%%%%%%%%%%%%%%%%%%%%%%%%%%
%
% 
%
% This template may be used to prepare proposals in latex.
%
%
% The project description, including publication list, should be no more than 20 pages
% in length. It should be self-explanatory and not require reviewers to read the 
% literature that is quoted or enclosed.

\documentclass[10pt,fleqn,twoside]{article}

%%%% USE ARIAL FONT %%%%%%%%%%%%%%%%%%%%%%%%%%%%%%%%%%%%%%%%%%%%%%%%%%%%%%
\usepackage{helvet}
\renewcommand\familydefault{phv}

%%%% INCLUDE NECESSARY PACKAGES %%%%%%%%%%%%%%%%%%%%%%%%%%%%%%%%%%%%%%%%%%
%\usepackage{babel}
\usepackage[UKenglish]{babel}
\usepackage{amsmath}
\usepackage{amssymb}
\usepackage{fancyhdr}
\usepackage{natbib}
\usepackage{xcolor}
\usepackage{ae,aecompl}
\usepackage{graphicx}
\usepackage{palatino}
\usepackage[T1]{fontenc}
\usepackage{rotating}
\usepackage{epsf}
\usepackage{setspace}
%\usepackage{sfmath}

%%%% PAGE LAYOUT %%%%%%%%%%%%%%%%%%%%%%%%%%%%%%%%%%%%%%%%%%%%%%%%%%%%%%%%%
\setlength{\textheight}{22cm}
\setlength{\topmargin}{-1.2cm}
\setlength{\textwidth}{15.6cm}
\setlength{\oddsidemargin}{0.0cm}
\setlength{\evensidemargin}{0.0cm}
\setlength{\mathindent}{1.5cm}
\setlength{\parindent}{0.0cm}
\setlength{\parskip}{0.08cm}

%%%% PAGE HEADER %%%%%%%%%%%%%%%%%%%%%%%%%%%%%%%%%%%%%%%%%%%%%%%%%%%%%%%%%%
\pagestyle{fancy}
\fancyhead[RE,RO]{}
\fancyfoot[RO]{ \thepage}
\fancyfoot[LE]{ \thepage}
\fancyfoot[CE,CO]{}

%%% FONTS FOR THE TITLE PAGE %%%%%%%%%%%%%%%%%%%%%%%%%%%%%%%%%%%%%%%%%%%%%%
\newfont{\tpfonta}{cmssbx10 scaled 1600}
\newfont{\tpfontb}{cmssbx10 scaled 3200}

%%%% EURO SIGN %%%%%%%%%%%%%%%%%%%%%%%%%%%%%%%%%%%%%%%%%%%%
\newcommand\euro{{\sffamily C%
\makebox[0pt][l]{\kern-.70em\mbox{--}}%
\makebox[0pt][l]{\kern-.68em\raisebox{.25ex}{--}}}}
\newcommand\keuro{k{\sffamily C%
\makebox[0pt][l]{\kern-.70em\mbox{--}}%
\makebox[0pt][l]{\kern-.68em\raisebox{.25ex}{--}}}}

%%%% COLOR DEFINITIONS %%%%%%%%%%%%%%%%%%%%%%%%%%%%%%%%%%%%
\definecolor{blue} {rgb} {0.25,0.25,0.75}

%%%% ADDITONAL EMPHASIS %%%%%%%%%%%%%%%%%%%%%%%%%%%%%%%%%%%
\newcommand{\cem}{\color{blue}}
\newcommand{\eem}{\sl\color{blue}}

%%%% SET THE COLOR OF THE (SUB-) SECTION TITLES %%%%%%%%%%% 
\newcommand{\Tcol}{\color{blue}}

%%%% SET THE COLOR OF THE TITLE BOX BACKGROUND %%%%%%%%%%%%
\definecolor{Background}{rgb} {0.62,0.75,0.5}

%%%% REFERENCE SECTION NAME %%%%%%%%%%%%%%%%%%%%%%%%%%%%%%%
\renewcommand\refname{\Tcol 9. Bibliography}

%%%% COLOR THE SECTION NUMBERS %%%%%%%%%%%%%%%%%%%%%%%%%%%%%%%
\makeatletter
\renewcommand\@seccntformat[1]{\color{blue} {\csname the#1\endcsname}\hspace{0.5em}}
\makeatother
\renewcommand\thesection{\arabic{section}.}
\renewcommand\thesubsection{\arabic{section}.\arabic{subsection}}

%%%% CHANGE THE APPEARANCE OF THE \PARAGRAPH COMMAND  %%%%%%%%%%%%%%%%%%%%%%%%%%%%%%%
\makeatletter
\renewcommand\paragraph{\@startsection{paragraph}{4}{\z@}%
            {-2.5ex\@plus -1ex \@minus -.25ex}%
            {1.25ex \@plus .25ex}%
            {\normalfont\normalsize\bfseries}}
\makeatother
\setcounter{secnumdepth}{4}     % how many sectioning levels to assign numbers to
\setcounter{tocdepth}{4}        % how many sectioning levels to show in ToC


\fancyhead[LE,LO]{\slshape
%%%%  Please edit
%
Ercolano: RU Transition Discs Project B1}
%
%
%%%%%


\begin{document}


\newpage

%%%% PROJECT DESCRIPTION STARTS HERE %%%%%%%%%%%%%%%%%%%%%%%%%%%%%%%%%%%

\setcounter{page}{1}

\centerline{\huge\bf\Tcol
%
%
%
%
%%%%  Please edit
%
 Project B1:}

\centerline{\huge\bf\Tcol The radiation-hydrodynamics of}

\centerline{\huge\bf\Tcol photoevaporative winds with chemistry}

%
%%%%
%
%
%
%
\vskip1.0cm

%%%%  Please edit
\noindent{\bf Authors:}\\
\begin{tabular}{ll}
{\textsf{PI:}}                  & B.~Ercolano (LMU)\\
{\textsf{Co-I:}}                &P.~Caselli (MPE)\\
{\textsf{Collaborations:}}      & J. Owen (Princeton, USA), E. van Dishoeck (Leiden, MPE)  \
\end{tabular}

%%%%  Please edit

\vspace{1em}
\noindent{\bf Requested positions: 1 Postdoc} \\

\vspace{1em}
\noindent{\bf Abstract:}\\

\noindent{\bf Abstract:}\\
Type 1 TDs are most likely discs in an advanced stage of dispersal
(see introduction to the Research Unit). The
dispersal mechanism of discs has been shown to be of fundamental importance to planet
formation, yet the responsible mechanism is still largely
unconstrained. Photoevaporation from the central star is currently a
promising avenue to investigate, but the models developed to date do
not yet have enough predictive power for a piecewise comparison with
the observations. We aim at building the most
up-to-date radiation-hydrodynamical calculations of irradiated discs, 
coupled to photoionisation, chemistry and radiative transfer
calculations. This will constitute the backbone for the work carried
out in several sub-projects of this proposal (B2, C2), which together
aim at performing quantitative
spectroscopy of disc winds. Comparison with existing and upcoming
observations will allow us to constrain the mass loss rates and the
launching regions of the wind and thus pin down the underlying driving disc
dispersal mechanism. 


\section{\Tcol State of the art and preliminary work}
\renewcommand{\leftmark}{\sc State of the Art and preliminary work}

\subsection{Scientific Background}

Understanding disc dispersal is a key piece in the puzzle of planet
formation. Type 1 TDs, which are considered to be objects caught in
the act of dispersal provide a tight constraints on the underlaying
dispersal mechanism. For example, their (low) frequency, in relation
to the global disc population in a given cluster, implies dispersal
timescales of order 10\% of the global disc timescale, and their
evolution on the colour-colour plane points to an inside-out mode of
dispersal (e.g. Ercolano, Clarke \& Hall, 2011; Koepferl, Ercolano et
al. 2013; Ercolano et al. 2015). 

The most successful of the various disc dispersal mechanisms proposed to
match many of the observational constraints 
is photoevaporation by radiation from the central star
(e.g. Clarke et al. 2001). As described in more details below, once
the mass loss rate due to photoevaporation exceed the mass accretion
rate through disc, the wind can quickly destroy the disc, eroding from the inside-out.

Recent work suggests that magnetohydrodynamic (MHD) turbulence may also
drive disc winds (e.g. Bai \& Stone, 2013), which may lead to disc
dispersal and at the same time remove angular momentum from the
system, thus allowing the inward flow of material, i.e. accretion. These models are still in their infancy and the
mass loss and accretion rates from MHD winds are at present highly
uncertain. What is certain, however, is that if MHD winds are indeed
as vigorous as some authors claim, they would completely change the
way we understand disc evolution and dispersal, invalidating may of
the models that are based on alpha-type discs. Some of the main
sources of uncertainty include the strong dependence of the wind and
accretion rates on the completely unknown magnetic flux distribution
through the disc and, most importantly, magnetic flux evolution as a
function of surface density of the disc (e.g. discussion in Armitage
et al. 2013). Another key ingredient is the level of ionisation in the
atmosphere of discs, which determine the nature of the gas coupling to
this unknown magnetic field. While details of the magnetic field
structure and evolution are difficult to determine at present, the
joint efforts of sub-projects B1, B2 and C2 can deliver the most
detailed assessment to date of the ionisation structure in disc
atmospheres. This is indeed one of the aims, which is described in
subproject B2 (PI: Caselli).  

In the following two sections the state-of-the art for photoevaporation
models and for current disc wind tracers will be discussed.
 
\subsubsection{Photoevaporation models} 

All models of photoevaporation show that radiation from the central
star heats the disc atmosphere, where a thermal  wind is established.
The wind is centrifugally launched from the location where the thermal
energy of the heated gas exceeds the local gravitational binding
energy. 
Disc dispersal then sets in as a consequence of the wind when the mass
loss rate exceeds the accretion rates in the disc. According to
viscous theory, young discs accrete
at a vigorous rate, which naturally decreases as time goes by (REF?????), until,
after a few million years accretion rates fall to values smaller than
the wind rates, allowing photoevaporation to take over the further
evolution of the disc. Once the dispersal sets in the disc is then
quickly eroded from the inside out (see e.g. Alexander et al. 2014 and
Armitage et al. 2011, for recent reviews of this process).  

While the community now agrees on this broad brush picture,
quantitatively speaking, the dispersal mechanism is still largely
unconstrained. There is currently a hot debate in the literature as to
what type of radiation may be the main driver of the wind: Extreme-,
Far-UV or X-ray. This is a fundamental question as the mass-loss-rates
implied by the different models can differ by orders of
magnitudes. Mass loss rates determine the timescales of dispersal for
given initial disc conditions. The wind profile, i.e. the region of
the disc that is most affected by photoevaporation, is also very
different in each scenario (e.g. see Armitage 2012, Alexander et
al. 2014). Figure ??? shows the mass loss profile for the X-ray+EUV
(Owen, Ercolano et al. 2010), EUV -only (Font et al. 2004) and FUV
(Gorti, Dullemond \& Hollenbach 2009) profiles. The X-ray profile is
more extended than the EUV profile, which predicts mass loss only from
a vary narrow range of disc radii. The FUV model is again very
different, showing mass loss from the outer regions of the disc, hence
predicting in some cases an outside-in mode of dispersal. The total
mass loss rates in the disc sets the timescales for the formation of
gas giants, while the wind
profiles sculpt the disc density distributions, profoundly affecting
the evolution of all planetary systems, by putting a stop to migration via
the formation of gaps in the gas. As an example, we have shown that
the photoevaporation profile strongly influences the final semi-major
axis distribution of exo-planets (Ercolano \& Rosotti, 2015) 

All models are incomplete. Some models focus on hydrodynamics and
assume isothermal gas (e.g. the EUV-olny model of Font et al. 2004), others focus on
chemistry but do not perform hydrodynamical calculations (e.g. the FUV
model of Gorti,
Dullemond \& Hollenbach 2009). None of the existing models take into
account dust evolution in the underlying disc and entrainment of
grains in the wind. Together with my previous PhD student, I performed
the only existing radiation hydrodynamic calculations of X-ray  + EUV driven
winds to date (Owen, Ercolano et al. 2010, 2011a, 2012), using realistic gas
temperatures obtained from X-ray photoionisation calculations
(Ercolano et al. 2008, 2009). This led to a fundamental re-think to the
whole problem, as previous isothermal calculations had yielded much
lower mass loss rates (by two orders of magnitude). However, our
models, which still represents the state-of-the-art, also have
crippling limitations, since, most importantly, they do not include
chemistry and ignore the dust phase. Furthermore the resolution at
which the simulations were performed is insufficient to study lines with
emission regions extending close to the disc inner edge. These
limitations make the application of current models to observations
impossible.  

\subsubsection{Disc Wind diagnostics}

The fundamental reason why the theoretical models still wildly
disagree with each other is because until now they could not be
directly tested with observations.

The presence of disc winds has been confirmed via the
observation of a few km/sec blue-shift in the line profiles of a
number of tracers like [NeII]~12.8$\mu$m and [OI] 6300 (e.g. Pascucci
et al 2007, Rigliaco et al. 2013) and a number of collisionally excited lines in the optical
region (Natta et al. 2014). Figure 1, taken from an upcoming review on transition discs by
Ercolano \& Pascucci (2017, to appear in Royal Society Open Science)
shows an example of possible wind diagnostics.  

% INCLUDE FIGURE OF LINES FROM ILARIA'S REVIEW

We have demonstrated, however, that the  [NeII]~12.8$\mu$m and the
optical forbidden lines 
cannot be used to infer the underlying 
mass-loss-rates (e.g. Ercolano \& Owen 2010, Ercolano \& Owen 2016). 
 For example, the intensity and the profile of the [NeII]~12.8$\mu$m line
 can be equally well fitted using an EUV (Alexander 2008) or an X-ray
 photoevaporation model (Ercolano \& Owen 2010), as shown in
 Figure??????. The problem with the 
 [NeII] line is that the Ne+ formation route can occur both via the
 removal of a valence electron in the fully-ionised winds driven by
 EUV radiation, but also by charge exchange of Ne++ with neutral H
 which is abundant in the quasi-neutral winds driven by X-ray. 
The problem with the [OI] 6300 line and all other optical collisionally
excited lines considered to date is the strong temperature dependence
imposed by the Boltzmann term in the emissivity. This means that these
lines are mostly just tracing the hot layer of the wind heated by the
EUV radiation and not actually tracing the bulk of the wind where it
matters (Ercolano \& Owen 2016), hence they
cannot be used to infer mass-loss-rates or to constrain the wind
driving mechanism.   

This picture was further complicated by the recent high spectral
resolution observations of  Simon et al. (2016)  who found that low velocity
emission in [OI] forbidden lines, classically attributed to a
slow-moving disc wind, is present in all T-Tauri stars with
dust disks, even those classified as WTTs, but it is best fit
by a superposition of a broad and a narrow component. 
Most of the broad component emission
arises within 0.5AU and Simon et al. (2016) interpret it
as being produced in a magnetically driven wind, given that the
emitting region is well inside the gravitational 
potential well 
of the central star. The narrow component, which, unlike the broad
component, is always present in also in transition
discs, traces gas further away (0.5-5 AU) and is probably associated
with photoevaporative winds.

The interpretation of the broad component as a tracer of a magnetic
wind is however problematic for a number of reasons. The main problem
is that one would need a very large scale height to overcome the fact
that the emitting volume dominates as one goes to larger
radius. Presumably a very large magnetic pressure would be needed to
achieve this. Furthermore, if an hypothetical magnetic wind would
have enough density to match the observed broad component, it may also absorb
out all UV flux, which would then not be available to irradiate the
wind at larger radii, which is clearly indicated in the
observations. 

An alternative explanation could be that the broad
component is not produced in the wind, but is emitted by bound
material in the disc itself. The broad component, if coming from the
inner disc, cannot have however a thermal origin. We have tested that
with a new higher resolution set of hydrodynamical calculations,
similar to those presented by Ercolano \& Owen (2016), which
extend further into the inner disc, to $r_{\rm in}$=0.04~AU. The line
profiles for the high resolution hydrodynamical calculations are shown
in Figure~???? for R=25000, which represent the resolution of the
Rigliaco et al (2013) data, when the two component had not been
resolved yet, and R=50000, which is more representative
of the work of Simon et al. (2016). The
R=50000 line profiles in our simulations do show broad wings at high disc
inclinations. The wings are due to bound material in the inner
disc. However the wings from our high resolution models are still much
smaller (i.e. do not carry enough flux) than those detected by
Simon et al. (2016). In fact in our calculations the line flux
is completely dominated by the (unbound) emission at larger
radii. This is easy to understand as the flux is proportional to
density squared times the volume and the volume for an isothermal
region in the disc scales like R$^{7/2}$ ($R^2\times H$). Given that
the density does not fall off steeply enough in the heated region then
the emission at larger radii dominates. For 
example a density profile set by the absorption of photons to a fixed
column - indicative of our case - would fall of approximately as
$n\propto1/R$, provided the absorption is dominated at large radius.

Our calculation show nevertheless that bound material in the very
inner disc can indeed produce broad wings. If the column of emitting
bound material were larger then stronger wings would be produced. A
non-thermal process acting at higher columns in the inner disc, as for
example dissociation of the OH molecule (Gorti et al. 2011), could
indeed produce the missing flux in the wings. As our codes currently
lack chemistry we have been unable to test a non thermal
origin of the broad component. This is however an important task, as
if confirmed, there would be no need to invoke magnetic disc winds to
explain the observations.  

Mid-infrared observations of molecular lines (e.g. CO) provide a new
promising alternative to directly measure disc winds. Indeed recent
observations suggest that these lines may be tracing a disc wind which
is slow and partially molecular (e.g. Pontoppidan et al. 2011; Brown et al. 2013). 
The spectro-astrometric survey of molecular gas in the inner regions of
protoplanetary discs using CRIRES, the high-resolution infrared
imaging spectrometer on the Very Large Telescope (Pontoppidan et
al. 2011), showed that for several sources the astrometric signatures
are dominated by gas with strong non-Keplerian (radial) motions. These
authors concluded that the non-Keplerian spectro-astrometric
signatures are likely indicative of the presence of wide-angle disc
winds. 
More observations of this type are planned after the update of
the CRIRES instrument, which is expected to be completed by
2019. Observations with ALMA in molecular lines like e.g. CO J = 2-1
and J = 3-2 emission are also able to trace the presence of a wind (e.g.Klaassen et al. 2013, 2016).  
Molecular lines are sensitive to the mass loss rates since they
sample a significant area of the wind launching regions. However the
exploitation of molecular tracers is currently severely hampered by
the lack of a suitable hydrodynamic wind model coupled to chemistry
and to dust evolution models (which dominate the opacity in the wind)
to interpret the observations.

 While a number of chemical models exist
of the deeper, denser regions of discs, no model is currently
available for the optically thinner disc winds. The work of Gorti \&
Hollenbach (2009), while carrying out detailed chemical calculations
extending to the disc atmosphere, used a hydro{\it static} disc model which
was analysed in a 1+1D fashion. Without hydro{\it dynamics} no predictions
on line profiles can be made.  

Studying the kinematic of the emitting gas is the only way to
constrain the origin and intensity of the disc wind and hence shed
light on the driving mechanism behind the dispersal of discs and the
formation of Type 1 transition discs. 

\subsection{Preliminary work}

The determination gas temperatures in a photoionised gas, in a
photodissociation region (PDR) or in an X-ray dissociated region (XDR)
is computationally expensive. It requires, first of all 
performing a radiative transfer (RT) calculation in order to determine the
radiation field at each point of the region. Then matrices of thermal and
ionisation balance and/or rate equations have to be solved. The RT and
the balance/rate equation are often coupled through the
temperature-dependent gas opacities. There is a host of microphysics
that needs to be taken into account, last but not least the thermal
coupling of the gas and the dust phase. Even a extremely simplified
version of the above results crippling if it needs to be performed at
every time-step of a hydrodynamical calculation. 

In such cases it is convenient to look for parameterisations of the
gas temperature in terms of quantities that are easy to determine in
the hydrodynamics code (e.g. gas properties and/or column density). 
Indeed using such a temperature paremeterisation,
determined via detailed X-ray photoevaporation models using the
MOCASSIN code (Ercolano et al. 2008, 2009) we have performed the
only existing radiation hydrodynamic 
calculations of X-ray + EUV driven photoevaporative winds to date (Owen et
al. 2010, 2011, 2012). The models were run with a version of the {\sc
  zeus2d} code which was modified by us to include a temperature
scheme derived from the detailed X-ray and EUV photoionisation
calculations of protoplanetary discs of Ercolano et al. (2008,
2009). In this work it was shown that, within the
penetration depth of $\sim$1keV X-rays ($\sim 10^{22}cm^{-2}$), the
temperature of a parcel of gas with hydrogen number density, $n_H$, at
distance $r$ from the central star, could be roughly approximated by a
function of the ionisation parameter, defined as $L_X/(r^2 n_H)$,
where$L_X$ is the stellar X-ray luminosity. The error on the
temperature is small for high ionisation parameter values, but it
becomes systematically larger at the low end. 

As the gas temperature enters the hydrodynamics via the square root
dependance on the sound speed, the small error at high values of the
ionisation parameter, typical for the regions where the bulk of the
wind is driven from in primordial discs, is unlikely to produce large
uncertainties in the mass loss rates. For more evolved objects, like
transition discs in the phase of final dispersal, however, the ionisation
parameter decreases dramatically as the cavity becomes larger. The
evolution of transition discs depends thus sensitively on the temperature
of the gas at low ionisation parameters, which is currently poorly
represented by the parameterisation of Ercolano et al. (2008, 2009). 
Note that the recent work of Haworth, Clarke \& Owen (2016) presents a
form of the temperature ionisation parameter relation which is incorrect at
low values of the ionisation parameter. The kink at ionisation
parameters just above $10^{-7}$ (left panel of Figure???) is an artefact of their implementation
of the photoionisation models. We have recently performed new detailed
photoionisation calculations and obtained the curve shown in the right
panel of Figure???? (Ercolano, Picogna \& Owen, 2017). Our
collaborators J. Owen and C. Clarke, have been informed of the problem
and agree with our more recent calculations. Furthermore we have now
found a more accurate scheme that allows us to reduce the error on the
temperature by introducing  column density as an additional
parameter. 

As well as the major shortcoming highlighted above, the Owen et
al. (2010) calculations which were used to make predictions on
the ionised phase of the wind spectra (Ercolano \& Owen 2010;
Ercolano \& Owen 2016), suffered from low spatial resolution,
precluding us from being able to model the inner region of the bound disc, which
may be relevant to interpret the broad wings presented in the recent
work of Simon et al. (2016). Furthermore, a very limited
parameter space was investigated, which included only two values for
stellar mass, 3 values of X-ray luminosity for primordial discs and a
single stellar mass and X-ray luminosity value for transition discs
with 3 values for the cavity radius. This is nowhere near enough to draw any
significant conclusions about trends in possible wind diagnostics. 



\subsection{\Tcol Project-related publications}

% Please list your own publications related to the proposed project, 
% adhering to the rules of the DFG guidelines 1.91. In brief, please note: 
% - Up to 10 publications
% - The work must be published or accepted.
% - Publications on astro-ph (arXive, SPIRES or articles with a DOI) count as published. 
% - Any work that is only in the status ``accepted'' MUST be attached to the proposal
%    together with the acceptance letter.
% - All publications in this section CAN be attached to the proposal. Please limit these
%    attachments to a minimum and please note that the reviewers may not read the attachments -
%    the proposal has to speak for itself.
% - The number of allowed publications refers to the sum of the publications listed
%    in ``1.1.1 Articles published or officially accepted by publication outlets...'' and 
%    in ``1.1.2 Other publications''. Publications which only exist on public repositories 
%    belong into the category ``Other Publications''.

\subsubsection{\Tcol 
Articles published or officially accepted by publication outlets with scientific quality assurance;
book publications}

[Text]

\subsubsection{\Tcol Other publications}

None 
\subsubsection{\Tcol Patents}

\paragraph{\Tcol Pending}

None

\paragraph{\Tcol Issued}

None

\section{\Tcol Objectives and work programme}
\renewcommand{\leftmark}{\sc Objectives and work programme}


\subsection{\Tcol Anticipated total duration of the project}

36 months

\subsection{\Tcol Objectives}


The overarching aim of this project, in common with projects B2 (PI
Caselli) and C2 (PI Ercolano),
is to identify new wind tracers and use them to constrain mass loss rates and hence
disc dispersal models, leading to the formation and evolution of Type 1
transition discs. 

The new comprehensive radiation-hydrodynamics photoevaporation models
developed in this project will enable a quantitative
spectroscopic evaluation of new diagnostics of disc winds, via
detailed astrochemical models developed together with project B2,
using the dust model for the wind and atmosphere from project C2. 

We will perform a comparison between TDs and
primordial disc to provide important constraints on the wind
architecture and the mechanism driving the dispersal. 
Type 1 TDs, are particularly interesting as the streamline architecture of their winds
and the profiles of the lines that are produced in the wind
differ from those of primordial discs. (e.g. Ercolano \& Owen
2010, Ercolano \& Pascucci, 2017, in preparation). Indeed the lines
are expected to be broader and brighter for e.g. inner cavities of a few to 10 AUs.  

The immediate objective of project B1 is to produce a new set of
X-ray+UV photoevaporation models which goes well beyond the current
state-of-the-art, described in the previous section. 

The new models will constitute the backbone for the joint
investigation to be carried out in projects B2 and C2, as well as
allowing us to address the following important unsolved questions in
the formation of Type 1 transition discs and their further evolution: 

\begin{enumerate}
\item How fast do Type 1 TDs evolve/disappear after the inner disc has
  drained?
\item How does the formation \& evolution of Type 1 TDs scale with
  stellar mass and stellar emission properties?
\item What is the role of FUV heating in the late evolution of Type 1
  transition discs? 
\end{enumerate}

%{Test whether Type 2 TDs (large holes, large accretion rates) may
%result from radiative transfer effects on a tilted inner disc. This
%idea stems from  the recent suggestion (e.g. Marino et al. 2015, Montesino et
%al. 2016) that some Type 2 TDs may have a tilted inner disc. 
%A tilted inner disc may strongly influence photoevaporation by
%allowing radiation to reach outer disc regions and may produce the
%large inner holes of (some) Type 2 TDs. This is certainly a worthwhile
%new challenge requiring the development of 3D simulations.  }

%{\color{red} the marino and montesino idea must also be included in
 % the intro and also in the background section of this proposal}

\subsection{\Tcol Work programme including proposed research methods}
% for each applicant
In this project we will significantly expand on the state-of-the-art of these widely used
photoevaporation calculations by constructing a
library of high resolution X-ray+EUV wind solutions for an extended grid of
X-ray luminosities and stellar masses, covering all observed
values. Our new calculations will make use of a new temperature
scheme (Ercolano, Picogna \& Owen, 2017, in preparation), derived from
new more extensive X-ray + EUV photoionisation models.  The new
temperature scheme
significantly reduces the error at low ionisation parameter values,
allowing us to make solid predictions of the late evolution of
transition discs. 

Furthermore, our previous calculations (Owen et al. 2010, 2011, 2012)
could only account for heating 
in the ionised phase of the wind, ignoring that the region beyond the
layer heated by the soft X-ray ($<1keV$) could be heated by FUV
radiation with typical PDR or XDR characteristics. While we show in
Owen et al. (2012), that this should not affect the mass loss rates at
around 1-10~AU, the effect of FUV heating at larger radii may be
important. We have recently coupled our 3D X-ray and EUV Monte Carlo photoionisation
code MOCASSIN to the KROME package to solve the chemistry in the
deeper layers of the disc (Ercolano \& Grassi, 2017, in
preparation). The code has been benchmarked for a simple toy network,
but we will need input from project B2 to include an appropriate
network to model this region and devise a new temperature scheme to
include in our radiation-hydrodynamic simulations.   

Note that we do not plan to run a full
parameter space grid including the effects of the FUV heating. 
Our simulations will be limited to a selected number of cases
aimed at specifically testing how strongly, and for what initial
conditions, FUV heating may affect the evolution of the outer regions
of discs, in particular of those in transition. Indeed one of the
problems with current photoevaporation models is the prediction of a
large number of non-accreting transition disc with large holes
(e.g. Owen ?????REF). In this project we will be able to test the
suggestion that FUV heating may take over the late-stage evolution of
transition discs, speeding up their final complete erosion. Our models
of the PDR in the inner disc regions will also allow us to
investigate a possible non-thermal origin for the broad component of
the low-velocity component of the forbidden [OI] emission, detected by
Simon et al. (2013) and attributed to a magnetically driven winds.\\ 

\subsection{Research Tools}

For this project we will need the following tools: 
\begin{enumerate}
\item A hydrodynamical code which we will modify to include the
  effects of X-ray + EUV irradiation as we did in Owen et al. (2010). For
  that we will use the Pluto code, for which extensive expertise
  exists in our team. 
\item A photoionisation and chemical code which includes X-ray
  heating and ionisation to obtain a new comprehensive temperature
  scheme for the radiation-hydrodynamic simulations. The PI is the
  author of the 3D Monte Carlo photoionisation and dust radiative
  transfer code MOCASSIN (Ercolano et al. 2003, 2005, 2008b),
  which has already been used to calculate the 
  emission line spectra from the ionic phase of X-ray winds (Ercolano
  \& Owen 2010; 2016). The code has
  now been coupled and benchmarked to the KROME code to perform
  arbitrary chemical calculations (Ercolano \& Grassi 2017, in prep)
  and needs now only the appropriate chemical network, which
  we will obtain from project B2.  
\end{enumerate}

\subsection{Research Plan} 

We have divided the work load into two connected blocks which
also have self-contained immediate objectives. Preparatory work for
Block 1 is already being executed by Dr Picogna, employed on a
LMUExcellent initiative grant awarded to the PI in support of
this Research Unit application (End date November 2017). In case of an award Dr Picogna
has already agreed to continue his work for the Research Unit and will
take over the tasks described in Block 1. 

A grid of significantly improved new photoevaporation models including
an accurate temperature scheme which includes X-ray, EUV and FUV
heating will be developed in Block 1, by means of
radiation-hydrodynamic calculations. Block 2 will provide a new
temperature scheme for the hydrodynamical calculations of Block 1, by
performing detailed photoionisation calculations with simple chemistry
using the MOCASSIN+KROME code. The PI will take full responsibility of
the tasks described in Block 2 and will work closely with the postdoc
employed for project B1
for the implementation of the results in Block 1.  

\subsubsection{Block 1: New state of the art radiation-hydrodynamic
  models of photoionised winds.}

\paragraph{Preparatory work: present until beginning of the award}

The hydrodynamical code PLUTO (REF???) is being modified to include the
effects of X-ray and EUV irradiation using the temperature-ionisation
parameter from Ercolano et al. (2008,2009) also employed by Owen et
al. (2010) for the ZEUS2D calculations. 
The obtained solutions will be first of all benchmarked against
those that are already available for a 0.7 and 0.1 M$_\odot$ central
stars (Owen et al. 2010, 2011, 2012), for similar spatial
resolution. This will ensure that we have implemented the algorithm
correctly in PLUTO. 

\paragraph{Months 1-12}
We will then proceed with the implementation of the new, more accurate
temperature scheme described in the previous Section (Ercolano,
Picogna \& Owen, 2017, in preparation). We will compare the resulting
wind rates and profiles for the primordial and transition disc
case. While we do not foresee large changes in the rates for
primordial discs, as described in the previous section the evolution of transition discs will be most
likely affected. 

The new models will have much higher spatial
resolution extending much closer into the star, in order to allow
tracking profile components which may be emitted from the inner bound
atmosphere of the disc. 

This first set of models for $\sim$solar mass stars at a typical X-ray
luminosity ($\sim 10^{30}$erg/sec) will be then passed on to projects
C2 and B2 for the dust and detailed chemistry calculations. 

\paragraph{Months 13-24}

The parameter space of the calculations will be then
significantly extended for the mass of the central star and its X-ray
luminosity. Furthermore models of transitions discs at several stages
of evolution, as tracked by the radius of their inner cavity, will be
performed. 

With the new models we will also be able to investigate how
the process scales with stellar mass and stellar emission
properties. This will allow us to check the
theoretical relations for X-ray photoevaporation 
predicted by means of semi-analytical models and ab-initio arguments
by Owen et al. (2012). While these relations are being widely used in
the literature, they have until now never been tested.\\

\paragraph{Months 25-36} 

At this point a new temperature scheme, including the effects of FUV heating, should become available from
Block 2.  It is likely that a streamlined form of radiative transfer
will be needed in PLUTO at this point, meaning that these calculations will
probably be more expensive to run. Implementation of an efficient
radiative transfer algorithm in PLUTO will be carried out together
with the PI (see Block 2). 

However, as described above, only a
limited number of calculations are planned that include this
effect. These will allow us to assess the relevance of the FUV in the
very final phases of disc dispersal and at large radii, where X-rays
are weak. \\

If time allows, as a further step we plan to perform a very small set of 3D
simulations to explore the effects of asymmetries in the inner
disc. We expect to see dramatic effects in the photoevaporation
profile and in the wind architecture, which may lead to the formation
of large hole TDs. This avenue is never been explored before. It is
however likely that a more comprehensive study of this effect will
have to be delayed until the second funding period. \\ 

\subsubsection{Block 2: Efficient temperature schemes from detailed
  photionisation and chemistry calculations.}

The work described in this section will be performed directly by the
PI, who plans to dedicate 20\% of her research time to perform the
tasks below within the given timescales. 

\paragraph{Preparatory work- present until beginning of the award}

As mentioned in the previous section we have significantly improved on
the original parameterisation of the gas temperature, by performing new detailed calculations
which are now described in terms of gas column as well as ionisation
parameter (Ercolano, Picogna \& Owen, 2017, in preparation). We are currently working at
optimising the interpolation curves for our results, to further
minimise the error on the temperature, and are confident that the new
prescription will be ready by the beginning of the award. 

Furthermore the PI has been working on coupling her 3D Monte Carlo photoionisation
and dust radiative transfer code MOCASSIN (Ercolano et al. 2003, 2005,
2008b) to the KROME package (Grassi et al. 2014), to perform arbitrary
chemical calculations (Ercolano \& Grassi, 2017, in
preparation). Simple photoionisation benchmarks from the set of
Ferland et al. (????) have already been successfully performed with
the new coupled version, and a very basic chemical network has also
been introduced. This is however inadequate for any realistic
modelling of disc chemistry. 

\paragraph{Month 1-18}

We will work in collaboration with the group of Prof. Caselli (B2) on
introducing a reduced chemical network into our
MOCASSIN-KROME code, which captures the temperature distribution obtained
by the more extensive chemical modelling performed in project B2. 
The code will then be used to obtain a new parameterisation of the gas
temperatures that can describe the physical conditions in the PDR and
XDR regions.

While the work will be able to start straight away it is likely that
several iterations will be needed in order to find the optimal
solution. A number of unknown will need to be investigated, including
the issue of whether equilibrium chemistry is justified in the PDR and
XDR region. Another issue is the sensitive dependance of the results on the
specific dust (size) distribution in the PDR and XDR region. We will
produce several models using the prescriptions of Birstiel, Klahr \&
Ercolano (2012) coupled to prescriptions for vertical mixing
(e.g. ?????REF). This will allow us to assess, for what conditions, if
any, FUV heating significantly affects the dispersal of discs and
the formation/evolution of Type 1 transition discs. 

The completion of this task will also depend on the
completion of the tasks in project B2, thus it is likely that the
final parameterisation will only be ready in year 2 of the first
funding period. 

At this point the MOCASSIN+ KROME code, enhanced by the initial
reduced network from B2 (PI Caselli) will allow us to start producing
the first chemical models of the wind and its atmosphere. We will be
able to further investigate the nature of the broad-component of the
neutral hydrogen emission, for which we have suggested a non-thermal
origin coming from OH dissociation in the bound inner disc regions. 
The code chemical models will however most importantly form the basis
to search for new wind diagnostics. In a joint effort with the B2 team
we will produce synthetic observations that will be compared with
available observations under the guidance of the A1 team (PI Testi)
and out external collaborator Prof. van Dishoeck and Prof. Henning. 
This will guide the further development of the chemical model. 

These activities will continue for the duration of the project, as the
models are incrementally improved, by the addition of updated dust
models, networks etc., until we are satisfied that a realistic
synthetic spectrum can be delivered. This will allow us a quantitative
comparison with the observations to finally constrain the models,
allowing the determination of mass loss rates and wind profiles. 

\paragraph{Month 18-36}

The PI will work closely with the postdoc on the development and
validation of a streamlined RT approach to be included in PLUTO for
our means. This is likely necessary to estimate the value of the  FUV field
reaching different regions of the disc atmosphere, where the optical
depth are not high enough to justify the use of a  (grey) flux
limited diffusion method (FLD). 

We note that the PI is experienced in the development of efficient hybrid radiative
transfer schemes for use in hydrodynamical simulation (e.g. Owen,
Ercolano \& Clarke 2012b). 
Finally, as MOCASSIN is able
to solve the problem exactly for a given snapshot of the hydrodynamic
calculation, we are in the rare position to be able to carefully
check the validity of the new methods.  

\subsection{\Tcol Data handling}

A library of the hydrodynamical solutions in steady state (gas density
and velocity)  will be shared initially amongst the Research Unit
members only and will be made available online on the public partition
of the Research Unit server at the end of the first funding period. 

The new temperature schemes will also be published in the relevant
publications together with the mass loss profiles, which are important
for the development of population synthesis models which include disc
dispersal. 

\subsection{\Tcol Other information}
% Please use this section for any additional information you feel is
% relevant which has not been provided elsewhere.

Not Relevant

\subsection{\Tcol Information on scientific and financial involvement of international cooperation partners}

Not Relevant

\section{\Tcol Bibliography}

[Text]

\section{\Tcol Requested modules/funds}
\renewcommand{\leftmark}{\sc  Requested modules/funds}
% Explain each item for each applicant (stating last name, first name).

\subsection{\Tcol Basic Module}

\subsubsection{\Tcol Funding for Staff}
% Please note that funds for your own temporary position (“Eigene Stelleâ€)
% are not to be included here; this belongs to the separate “Module Temporary Positionâ€.

We require funding for one Postdoc to work at the LMU in the group of
Prof. Ercolano. In case of an award Dr Picogna has agreed to
take on the post. Dr Picogna is currently employed in the group of the
PI and is performing preparatory work for the project. Dr Picogna's
expertise in astrophysical fluid dynamics and his familiarity with the
subject is of great advantage for the achievement of the aims of this
project. Dr Picogna's contract ends in November 2017, 

\subsubsection{\Tcol Direct Project Costs}


\paragraph{\Tcol Equipment up to EUR 10,000, Software and Consumables}

Will be provided by the host institution

\paragraph{\Tcol Travel Expenses}

Total: 9900 Euro Justification : Each year one national trip (meeting of Astronomical Society, national
meetings) and one international trip (conference, visit
collaborators). 
During the course of the PhD 2 one week long visits to our main
international collaborator, Dr J. Owen (currently at Princeton
University, will move to Imperial College London in 2017). 

Cost estimate: 
\begin{itemize}
\item National trip: 5 overnight stays, train/airfare,
conference fee; 1000 Euro (3000 over 3 years).
\item International trip: 6 overnight stays, airfare, conference fee;
  1500 Euro (4500 over 3 years).
\item Visit to/from J. Owen: airfare, 6 overnight stay 1200 Euro (2400
  for 2 visits)
\end{itemize}

\paragraph{\Tcol Visiting Researchers (excluding Mercator Fellows)}

Not Relevant 

\paragraph{\Tcol Other Costs}

None

\paragraph{\Tcol Project-related publication expenses}

We request 770 Euro py (total 2250 Euro) for publication expenses.

\subsubsection{\Tcol Instrumentation}

None 

\paragraph{\Tcol Equipment exceeding EUR 10,000} 

None

\paragraph{\Tcol Major Instrumentation exceeding EUR 100,000} 

None 

\subsection{\Tcol Module Temporary Position}

Not Relevant

\subsection{\Tcol Module Replacement Funding}

Not Relevant

\subsection{\Tcol Module Mercator Fellows}

Not Relevant

\subsection{\Tcol Module Public Relations Funding}

Not Relevant

\section{\Tcol Project requirements}
\renewcommand{\leftmark}{\sc Project requirements}

\subsection{\Tcol Employment status information}
% For each applicant, state the last name, first name, and employment
% status (including duration of contract and funding body, if on a
% fixed-term contract).

Barbara Ercolano, Professor at the Ludwig-Maximilians-Universit\"at
M\"unchen  (permanent)

\subsection{\Tcol First-time proposal data}
% Only if applicable: Last name, first name of first-time applicant.

Not Relevant

\subsection{\Tcol Composition of the project group}
% List only those individuals who will work on the project but will not
% be paid out of the project funds. State each person’s name, academic
% title, employment status, and type of funding.

?????


\subsection{\Tcol Cooperation with other researchers}

\subsubsection{\Tcol Planned cooperation on this project}

\paragraph{\Tcol Collaborating researchers for this project within the
  Research Unit}
%Each proposal must be accompanied by a description of how the project
%is integral to the Research Unit, %both in terms of subject matter
%and organisation. This includes a description of the cooperation with
%%others participating within the Research Unit. 


This project will provide the radiation-hydrodynamic models of the
wind which are needed by project B2 (PI: Prof. Caselli) for the
chemical calculations and by project C2 (PI: Ercolano) for the dust
entrainment. Project BI depends on input from project B2 (PI:
Prof. Caselli)  for the reduced network and on
project A2 for observations. Specifically project A2 can
provide insights on the emission properties of the irradiating stars.

\paragraph{\Tcol Collaborating researchers for this project outside of
  the Research Unit}
Dr James Owen (Princeton) will be heavily involved in the
project. Dr Owen developed the original
phoionisation models during his PhD project at the Institute of
Astronomy in Cambridge, which was co-supervised by
Prof. Ercolano. It is envisioned that Dr Owen will pay regular visit
to our group to help with the development of the new models. 

\subsubsection{\Tcol Researchers with whom you have collaborated scientifically within the past three years}
% This information is important for DFG to exclude possible conflicts of interest.
% Please mention not only the names of the cooperation partners but also their institution and city.
% Scientists already mentioned in the previous two subsubsections do not have to be mentioned
% again.


F. Niederhofer (STSci, USA); M. Hilker (ESO, Garching); N. Bastian (U. Liverpool,
UK); M. Guarcello (U. Palermo, Italy); M. Tazzari (U. Cambridge, UK);
A. Natta (Florence, Italy); R. Alexander (U. Leicester); D. Hubber
(LMU); J. Dale (U. Hertfordshire, UK); C. Koepferl (LMU); I. Bonnell
(U. St. Andrews, UK); A. McLeod (ESO, Garching); D. Boneberg
(U. Cambridge, UK); R. Parker (U. Liverpool, UK); R. Wesson (UCL,
London, UK); M. Barlow (UCL, London, UK); A. Glassgold (u. Berkeley,
USA); C. Manara (ESA, Noordwjik, Netherlands); A. Danekhar (CfA,
Harward, USA); Q. Parker (Sidney, Australia); S. Casassus
(U. de Chile, Santiago, Chile); I. Pascucci (U. Arizona, USA);
A. Bevan (UCL, London, UK).

\subsection{\Tcol Scientific equipment}
% List larger instruments that will be available to you for the
% project. These may include large computer facilities if computing
% capacity will be needed. 

The group of Prof. Ercolano has two own computer clusters comprising 

\begin{itemize}
\item 2 CPU Intel Xeon X5650 (Westmere, beginning
2010, 2.66 GHz) 6 cores each 12 cores total (24 virtual) 74 GB ram.

\item 4 CPU Intel Xeon E7-4850 (Ivy Bridge, beginning 2014, 2.30 GHz)
12 cores each 48 cores total (96 virtual) 660 GB ram.

\end{itemize}

Further computational power is provided through the C2PAP facility of the Excellence Cluster to which
the group has guaranteed time. This comprises 126 nodes, each node with 2 CPU Intel Xeon E5-2680 (Sandy
Bridge, beginning 2012, 2.7 GHz) 8 cores each 16 cores total (32
virtual) 64 GB ram. Note that while the future of the Excellence
Cluster Universe is uncertain, the C2PAP facilities will be in any
case supported by the LMU. 

The Leibniz Rechnung Zentrum (LRZ) is also available to us, where still
larger facilities are available with somewhat more constrained and longer queues.

\subsection{\Tcol Project-relevant interests in commercial enterprises}
% Information on connections between the project and the production
% branch of the enterprise.

Not Relevant


\subsection{\Tcol Additional information}
% If applicable, please list proposals requesting major
% instrumentation and/or those previously submitted to a third party
% here.

Not Relevant


\end{document}
