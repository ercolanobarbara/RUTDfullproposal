%----------------------------------------------------------------------
%                        PROJECT DEFINITION
%----------------------------------------------------------------------
\renewcommand{\projnr}{B3}
\renewcommand{\projtitleshort}{High-temperature dust collision experiments}
\renewcommand{\projauth}{Blum, Trieloff}
%
\setcounter{section}{0}
\noindent{\normalfont\sffamily\Large\bfseries Project \projnr: \projtitleshort}
%
\section{Full title:}
\hspace{1\baselineskip}\\
\centerline{\large ``High-temperature dust collision
experiments''}
%\centerline{\large ''}
%
\section{General information}\mbox{}
\subsection{Principle investigators:}
\hspace{-\baselineskip}\\\noindent
%
{\bfseries\itshape Blum}, J\"urgen, Prof.~Dr.\\
C3, tenure\\
Date-of-birth: 05. April 1962, Nationality: German\\
DFG Code number of latest application: Bl 298-6/2\\
Institut f\"ur Geophysik und extraterrestrische Physik\\
Mendelssohnstr.~3\\
38106 Braunschweig\\
Tel: 0531 391 5217\\
Fax: 0531 391 8126\\
Email: j.blum@tu-bs.de\\
Private address: Wendenring 14, 38114 Braunschweig, Tel: 0531 1216432\\
%
\vspace{1em}\\\noindent
{\bfseries\itshape Trieloff}, Mario, Priv. Doz. Dr. \\
non-tenure\\
Date-of-birth: 26.2.63, Nationality: German\\
DFG Code number of latest application: TR333/8-2\\
Mineralogisches Institut der Universit�t Heidelberg\\
Im Neuenheimer Feld 236\\
69120 Heidelberg\\
Tel: 06221-546022\\
Fax: 06221-544805\\
Email: trieloff@min.uni-heidelberg.de\\
Private address:
Zaystr. 48, 76437 Rastatt
Tel: 07222/151875

\subsection{Co-investigators within this Forschergruppe:}
\begin{coilist}
\item G.~Wurm (IFP, Univ. M\"unster)
\end{coilist}


\section{Summary (Zusammenfassung)}
\subsubsection{Summary:}
Coagulation experiments shall be extended to temperatures in the
range 300-1,300~K  and to mineral phases (forsterite, enstatite,
FeNi metal, Ca,Al minerals, Fe sulfides, and Fe-bearing
olivine/pyroxene) that are expected from theoretical and
observational (both astrophysical and cosmochemical) constraints
in protoplanetary disks. In particular, the collision and adhesion
behavior at high temperatures and dust/gas ratios (sintering,
eutectic melting between immiscible phases) could be a key
mechanism to trigger agglomerate growth beyond previously
established critical sizes ($\sim$~dm) and collision velocities (a
few m/s). In addition to that, the importance of sintering for
sticking at high collision velocities shall be investigated.

\subsubsection{Zusammenfassung:}
Koagulationsexperimente sollen auf einen Temperaturbereich von 300
bis 1.300~K und auf Mineralphasen (Forsterit, Enstatit,
FeNi-Metall, Ca,Al-Minerale, Fe-Sulfide und Fe-f\"uhrende
Olivine/Pyroxene) erweitert werden, die auf Grund theoretischer
Modelle und astronomischer bzw. kosmochemischer Beobachtungen in
protoplanetaren Scheiben erwartet werden. Das Sto{\ss}- und
Adh\"asionsverhalten bei hohen Temperaturen und
Staub-Gas-Verh\"altnissen k\"onnte dabei ein
Schl\"usselmechanismus f\"ur das Agglomeratwachstum \"uber die
bislang festgestellte kritische Gr\"o{\ss}e ($\sim$~dm) und
Sto{\ss}geschwindigkeit (einige m/s) sein. Dar\"uber hinaus soll
die Bedeutung des Sinterns f\"ur die Haftung bei hohen
Sto{\ss}geschwindigkeiten untersucht werden.

\section{State of the art (Stand der Forschung)}
Coagulation experiments are crucial for our understanding of
growth from dust to planetesimals. Experiments with
micrometer-sized silica spheres (Poppe et al.~2000a) indicate that
grain-grain sticking dominates only up to relative velocities of a
few m/s. Above this threshold, grain-grain collisions do not
result in growth. Grain-grain experiments with other materials
(Poppe et al.~2000a), such as enstatite, diamond and silicon
carbide indicate similar thresholds. In the solar nebula such high
collision velocities occur mainly among larger ($\gtrsim$ dm)
agglomerates or between larger agglomerates and small dust
particles. For agglomerate-agglomerate collisions, experiments
have shown that collisions are destructive for velocities above a
few m/s (rather independently of grain material), mainly because
the weak van der Waals bonding does not provide the required
agglomerate strength/cohesion to guarantee further agglomerate
growth (Blum \& Wurm 2000; Dominik \& Tielens 1997).
Interestingly, recent experiments indicate that for collision
velocities above $\sim 10$ m/s, impacts of dust agglomerates into
compactified target agglomerates might again lead to a direct growth
(Wurm et al.~2005b). But for intermediate velocities (above a few
m/s and below $\sim 10$ m/s) no direct sticking has ever been
observed for collisions between macroscopic dust agglomerates.

A potential solution to overcome this problem might lie (1) in the
increasing strength of the van der Waals attraction with
increasing temperature ($U_{\rm vdW} \propto T$, Boyer 1975), (2)
in the process of sintering (and, thus, solidification) of dust
aggregates at elevated temperatures (Poppe 2003), or (3) in the
occurrence of eutectic melting between multi-phase agglomerates at
high (T$>300$ K) temperatures: in the region of the disk where the
terrestrial planets and asteroids were formed (within $\sim 4$
AU), disk temperatures can exceed $300$~K, and the calculation of
condensation sequences predicts the following minerals in the
inner parts of the disk: forsterite, enstatite, FeNi metal, Ca and
Al minerals, Fe sulfides, Fe-bearing olivine/pyroxene 
(Grossman \& Larimer 1974, Petaev \& Wood 1998, 2004; Lodders
2003; Gail 2003). There are some indications that agglomeration
and chondrite formation occurred under elevated dust/gas mass ratios 
(dust/gas $\sim 10$, compared to dust/gas $\sim 0.01$ as the "normal" solar value) and,
hence, under more oxidizing conditions than expected for solar
abundances (e.g.\ Sears 2004). At such enhanced dust/gas ratios, which
might occur in the midplane of the solar nebula, and elevated
temperatures prevailing in the inner disk, eutectic melting can
occur when immiscible phases (anorthite, diospide, forsterite, metal, etc.)
are in contact with each other. Melting at grain
boundaries of multi phase agglomerates, could significantly
change collision (and sticking?)  properties and internal
agglomerate strength.
%
% I added the following section(s) (Mario)
%

All above-mentioned three mechanisms cannot yet be constrained due to the
lack of experimental data with appropriate parameters. Two
experimental-parameter extensions are required: First, experiments at
elevated temperatures are needed, as all previous experiments were done at
room temperature. Second, realistic mineral mixtures are needed. Most
previous experiments were performed on glass spherules of quartz composition, 
but quartz is only expected
for fractional condensation with high isolation degree (Petaev \& Wood
\cit{1998}, \cit{2004}). In the early stage of protoplanetary disks,
expected minerals are forsterite, enstatite, FeNi metal, Ca and Al minerals (Melilite, anorthite, diopside)
Fe sulfides, Fe-bearing olivine/pyroxene (Grossman \& Larimer 1974, Petaev
\& Wood 1998, 2004; Lodders 2003; Gail 2003). Material/Mineral properties
may strongly influence sticking mechanisms listed under 1-3 above,
particular mechanism 3 (eutectic melting) only works in a multi-phase
system. Thus, the lack of multi-mineral experiments is a serious limitation
for present models of planetesimal formation.

Scenarios integrating short-lived nuclide chronometries (Kleine et
al.~\cit{2005}; Bizzarro et al.~\cit{2004}) and short-lived nuclide heating
of planetesimals (Trieloff et al.~\cit{2003}) indicate that meteorite parent
bodies formed over the first 4 Myr in the solar nebula (Sears 2004; Trieloff and Palme
2006). This is not as fast as theoretically possible and highlights the
importance that planetesimal growth may be delayed (and then overcome) at
certain specific conditions in the solar nebula (and by analogy in
protoplanetary disks), possibly by conditions and/or mechanisms to be
investigated here.


\section{Preliminary work (Eigene Vorarbeiten)}

%\begin{figure}
%\centerline{\includegraphics[width=8cm]{myfig.eps}}
%\caption{}
%\end{figure}

Scientists in the Institute for Geophysics and extraterrestrial
Physics have extended experience in dust-collision and
dust-aggregation experiments. Earlier work on dust-dust
interactions comprises measurements on cohesion and friction
forces between micrometer-sized dust particles (Heim et al.~1999,
2005), on the sticking probabilities of micrometer-sized dust
grains of various shapes and compositions (Poppe et al.~2000a), on
the formation of fractal dust agglomerates due to Brownian motion
(Blum et al.~2000; Krause \& Blum 2004), sedimentation (Blum et
al.~1999), and gas turbulence (Wurm \& Blum 1998), on impact
compaction and fragmentation of fractal dust aggregates (Blum \&
Wurm 2000), and on impacts of dust agglomerates into dusty targets
(Langkowski \& Blum, in preparation).

For this proposal, the pioneering experiments on the collision and
sticking behavior of single, micrometer-sized dust grains by Poppe
et al.~(2000a) shall be extended to above room temperatures. In the
experiments by Poppe et al.~(2000a), a powder sample of
well-characterized dust particles was deagglomerated by a
fast-rotating cogwheel (Poppe et al.~1997) and accelerated to
velocities $\lesssim 50$~m/s. Individual dust-particle collisions
with flat target surfaces (at arbitrary impact angle) could be
observed by high-speed imaging through a long-distance microscope
and stroboscopic laser illumination. This technique allows the
determination of the collision velocity as well as (in the case of
non-sticking collisions) the measurement of the rebound velocity,
rebound angle, and charge transfer (Poppe et al.~2000b; Poppe \&
Schr\"apler 2005) between projectile and target (see project
\projblum{}). It turned out that spherical projectiles have a
well-defined threshold velocity for sticking which increases with
decreasing particle size. Typical threshold velocities are a few
m/s for spherical, micrometer-sized $\rm SiO_2$ grains. Irregular
particles of micrometer size show a much smoother transition from
high sticking efficiencies at low velocities to low sticking
probabilities at high impact speeds. This effect is caused by the
random orientation of the particle at the moment of contact. Tips
and edges preferentially lead to particle rebound, while flat
surfaces in contact enhance the chance for sticking. It should be
emphasized that all above experiments were carried out at room
temperatures and with refractory dust grains. As we expect a
temperature influence on the strength of the van der Waals
attraction between the grains in contacts as well as on the
``hardness'' (elasticity and plasticity) of the grains, the work
described in this proposal is essential for a thorough
understanding of the growth processes in protoplanetary disks.

In another set of experiments, Poppe (2003) investigated the
influence of elevated temperatures on the structure of
high-porosity dust aggregates consisting of spherical $\rm SiO_2$
particles. The samples were heated in an oven at varying
temperatures and heating durations and were then analyzed by
scanning electron microscopy. Surface diffusion, sintering, and
viscous flow were identified as important transformation
mechanisms in grain-grain contacts. Surface diffusion dominates at
the start of restructuring and viscous flow at higher
temperatures. The formation and growth of inter-particle necks was
observed and could be described by a sintering model. Between the
temperature of neck formation and that of melting, further
restructuring occurs which leads to the dissolution of the
particulate structures and to a compactification of the aggregates.

A method for the formation of macroscopic (cm-sized), monolithic,
and high-porosity dust aggregates that resemble the predicted
morphologies and properties of pre-planetesimal dust aggregates
was established by Blum \& Schr\"apler (2004) who used random
ballistic deposition (RBD)  of individual dust grains. A variety
of studies with these RBD agglomerates for the determination of
their mechanical (Blum \& Schr\"apler 2004; Heim et al.~2005) and
their collision properties (see below) have been carried out.
Studies of the thermal and optical properties of these RBD
agglomerates are under way.

Impacts of mm-sized projectiles into cm-sized, high-porosity dust
aggregates were studied by Langowski \& Blum (in preparation) for
dusty projectiles and by Teiser \& Blum (in preparation) for solid
projectiles. Due to the overwhelming influence of gravity on the
outcome of the collision, these experiments were exclusively
carried out under microgravity conditions. It turned out that
both, solid and aggregated projectiles, stick to the target
agglomerate for not too oblique impacts in the velocity range $0
\ldots \sim 3$~m/s, while rebound and mass transfer was observed
for oblique impacts. An experimental extension for impact
velocities up to $\sim 20$~m/s is foreseen for the near future.
%
% I added the following section(s) (Mario)
%

Co-PI Mario Trieloff is working in the research area of cosmochemistry and
geochemistry, particular the isotope chronology of processes in the
early solar system and their relationship to astrophysical processes in
protoplanetary disks. M. Trieloff conducted a number of studies on early
meteorite chronology and the accretionary history of planetesimals,
i.e.\ meteorite parent bodies (Trieloff et al.~\cit{2003}; Kunz et al.~1995;
Pellas et al.~1997; Korochantseva et al.~2005). He also studied processes
that relate to disk clearing and solar wind implantation into precursor
planetesimals of meteorite parent bodies and the terrestrial planets
(Trieloff et al.~\cit{2000}, \cit{2002}; Trieloff and Kunz 2005; Schwarz et
al.~2005).  Trieloff and Palme 2006 reviewed and outlined the early solar
system history.

M. Trieloff has particular experience with high-temperature
furnaces and ultra-high-vacuum technology for noble gas extraction
from solids, and mass spectrometric analysis for which in-house
constructed high-temperature furnaces are used. Experience with
and control of heated and evacuated devices will be of particular
benefit for the studies proposed here.

%
% Here follows the own refereed publications by the PIs in relation to
% the project proposed here.
%
\ownpubltitle{Own publications related to the Forschergruppe:}
%
% BELOW IS ONLY AN EXAMPLE OF TWO ENTRIES. SEE THE ADDITIONAL FILES
% SENT TO YOU WITH ALL THE REFERENCES FROM THE VORANTRAG
%
\begin{ownpubl}

\item Blum, J., Wurm, G., Poppe, T. and Heim, L.-O. (1999) Aspects
of Laboratory Dust Aggregation with Relevance to the Formation of
Planetesimals. In: \textit{Laboratory Astrophysics and Space
Research}, Astrophysics and Space Science Library, Vol. 236 (Eds.
P. Ehrenfreund, K. Krafft, H. Kochan, V. Pirronello) Kluwer
Academic Publishers, Dordrecht, 399

\item Blum, J. and Wurm, G. (2000) Experiments on Sticking,
Restructuring and Fragmentation of Preplanetary Dust Aggregates.
\ica, \textbf{143}, 138

\item Blum, J., Wurm, G., Kempf, S., Poppe, T., Klahr, H., et al.
(2000) Growth and Form of Planetary Seedlings: Results from a
Microgravity Aggregation Experiment. \prl, \textbf{85}, 2426

\item Blum, J. and Schr\"apler, R. (2004) Structure and Mechanical
Properties of High-Porosity Macroscopic Agglomerates Formed by
Random Ballistic Deposition, \prl, \textbf{93}, 115503

\item Heim, L.-O., Blum, J., Preuss, M. and Butt, H.-J. (1999)
Adhesion and Friction Forces Between Spherical Micrometer-Sized
Particles. \prl, \textbf{83}, 3328

\item Heim, L.-O., Butt, H.-J., Schr\"apler, R. and Blum, J. (2005)
Analyzing the Compaction of High-Porosity Microscopic
Agglomerates, \textit{Australian Journal of Chemistry\/},
\textbf{58(9)} 671

\item Korochantseva, E.~V., Trieloff M., Buikin A.~I., Meyer H.~P.
and Hopp J. (2005) Argon-40/Argon-39 dating, and cosmic ray
exposure time of desert meteorites: Dhofar 300 and Dhofar 007
eucrites and anomalous achondrite NWA 011, \textit{Meteoritics and
Planetary Science \/} \textbf{40}, 1433

\item Krause, M. and Blum J., (2004) Growth and Form of Planetary
Seedlings: Results from a Sounding Rocket Microgravity Aggregation
Experiment. \prl, \textbf{93}, 021103

\item Kunz J., Trieloff M., Bobe K., Metzler K., St\"offler D. and
Jessberger E.~K. (1995) The collisional history of the HED parent
body inferred from 40Ar-39Ar ages of Eucrites. \textit{Planetary
and Space Science\/} \textbf{43}, 527

\item Pellas P., Fieni C., Trieloff M. and Jessberger E.~K. (1997)
The cooling history of the Acapulco meteorite as recorded by the
244Pu and 40Ar-39Ar chronometers. \textit{Geochimica et
Cosmochimica Acta\/} \textbf{61}, 3477

\item Poppe, T., Blum, J. and Henning, Th. (1997) Generating a jet
of deagglomerated small particles in vacuum. \textit{Review of
Scientific Instruments\/}, \textbf{68}, 2529

\item Poppe, T., Blum, J. and Henning, Th. (2000a) Analogous
Experiments on the Stickiness of Micron-Sized Preplanetary Dust.
\apj, \textbf{533}, 454

\item Poppe, T., Blum, J. and Henning, Th. (2000b) Experiments on
Collisional Grain Charging of Micron-sized Preplanetary Dust.
\apj, \textbf{533}, 472

\item Poppe, T. (2003) Sintering of highly porous silica-particle
samples: analogues of early Solar-System aggregates. \ica,
\textbf{164}, 139

\item Poppe, T. and Schr\"apler, R. (2005) Further Experiments on
Collisional Tribocharging of Cosmic Grains. \aap, \textbf{438}, 1

\item Schwarz W.~H., Trieloff M. and Altherr R. (2005) Subduction
of solar type noble gases from extraterrestrial dust: Constraints
from high-pressure low-temperature metamorphic deep sea sediments,
\textit{Contributions to Mineralogy and Petrology\/} \textbf{149},
675

\item Trieloff M. and Kunz J. (2005) Isotope systematics of noble
gases in the Earth�s mantle: Possible sources of primordial
isotopes and implications for mantle structure. \textit{Physics of
the Earth and Planetary Interiors\/} \textbf{148}, 13

\item Trieloff M. and Palme H. (2006) The origin of solids in the
early solar system. In: \textit{Planet Formation -� Theory,
Observations, and Experiments} (Eds. H. Klahr, W. Brandner)
Cambridge University Press, Cambridge, 64

\item Trieloff M., Jessberger E.~K., Herrwerth I., Hopp J.,
Fi\'{e}ni C., Gh\'{e}lis M., Bourot-Denise M. and Pellas P. (2003)
244Pu and 40Ar-39Ar thermochronometries reveal structure and
thermal history of the H-chondrite parent asteroid.
\textit{Nature\/} \textbf{422}, 502

\item Trieloff M., Kunz J. and All\`{e}gre C.~J. (2002) Noble gas
systematics of the R\'{e}union mantle plume source and the origin
of primordial noble gases in Earth�s mantle. \textit{Earth and
Planetary Science Letters\/} \textbf{200}, 297

\item Trieloff M., Kunz J., Clague D.A., Harrison D. and
All\`{e}gre C.J. (2000) The nature of pristine noble gases in
mantle plumes. \textit{Science\/} \textbf{288}, 1036

\item Wurm, G. and Blum, J. (1998) Experiments on Preplanetary
Dust Aggregation. \ica, \textbf{132}, 125
%
% I added the following references (Mario)
%



\end{ownpubl}
%
\section{Goals (Ziele)}

It is the utmost goal of the proposed study to investigate the
influence of elevated temperatures on the collision and sticking
behavior of dust particles and dust agglomerates. We expect
considerable changes of these properties due to an increased van
der Waals force at higher temperatures, due to the occurrence of
sintering within dust aggregates, and due to the possibility of
eutectic melting in collisions between dust grains of different
mineralogical composition. The results of our experimental
investigations shall be included in models of the evolution of the
solid bodies in protoplanetary disks. The following individual
goals have been identified and will be addressed by project
\projblumtrie{}:

\begin{itemize}

\item Study the collision behavior and sticking probabilities in
single-grain impacts in the temperature range from 300~K to
1,300~K and in the velocity range $0 \ldots 50$~m/s. Grain sizes
should range from sub-micrometer to centimeter and grain materials
should comprise of forsterite, enstatite, FeNi metal, Ca,Al
minerals, Fe sulfides, and Fe-bearing olivine/pyroxene. Grain and
target materials should be identical to understand simple mono-phase systems first.

\item Investigate the influence of sintering on the morphology and
mechanical properties (compressive strength, tensile strength) of
high-porosity dust aggregates. Aggregates will be produced by the
RBD process and will consist of micrometer-sized grains.
Individual dust grains will be of mono-phase type, consisting of
forsterite, enstatite, FeNi metal, Ca,Al minerals, Fe sulfides,
and Fe-bearing olivine/pyroxene. 
%These experiments shall study the
%evolution of early dust aggregates in protoplanetary disks before
%extensive mixing or cooling occurs.

\item Study the influence of elevated temperatures on the
structure of aggregates consisting of two-phase minerals and the
occurrence of eutectic melting inside mixed-phase aggregates. This
is relevant for those stages of the protoplanetary disk when
multiple grain materials coexist due to radial mixing or cooling.

\item Understand the collision behavior of realistic multi-phase
mixtures at various temperatures from below to above eutectic
melting points. When two aggregates consisting of different
mineral phases (e.g.\ anorthite, diopside, forsterite) collide at temperatures above the
eutectic melting point (but below the melting point of the
mono-phase minerals), contact melting can occur and can increase
the threshold velocity for sticking. The efficiency of this effect
shall be investigated.

\item Quantify the outcome of collisions for the parameter space
comprising of grain material, grain temperature, impact velocity,
and impact angle in order to use it for astrophysical coagulation
modelling at various evolutionary $(p, T, t)$ stages of
protoplanetary disks (see project \projdul{}).

\end{itemize}


\section{Work schedule (Arbeitsprogramm)}
\subsection{Methods}

We will concentrate on the three most abundant grain materials
that are predicted by condensation-sequence calculations (Petaev
\& Wood \cit{1998}; Sears 2004; 
Grossman \& Larimer {1974}):
%
% I added the references in last line above (Mario)
%
\begin{itemize}
    \item Mg silicates,
    \item Metals (Fe/Ni) and their sulfides,
    \item Ca/Al silicates.
\end{itemize}
Mineral and grain-size separates will be prepared at the
Mineralogical Institute of the University of Heidelberg by
milling, sieving and centrifugation techniques that are already
available. The resulting dust-particle fractions will be analyzed
by SEM with respect to grain-size distribution and grain/surface
morphology.
%
% I added the following section (Mario)
%

As starting materials we will use either synthesized, commercially
available (e.g.\ Crystec GmbH, Berlin/ SPI supplies) or natural materials
(SPI Supplies/C.M. Taylor collection), which are also used for project \projlattard{}. They will be checked for homogeneity
and purity by EMP and possibly by ion microprobe, all
equipment available at the Mineralogical Institute Heidelberg. We expect to succeed in 
obtaining appropriate material, however, in the case these materials have
not the desired level of homogeneity and purity, the synthesis of crystals
can also be performed by high temperature experiments under controlled
oxygen fugacity, either via solid state reactions or in the presence of a
melt phase or a flux. Respective high temperature furnaces are available at
the Mineralogical Institute, or via cooperations with labs specialized in
mineral synthesis.

In a first step, we will produce high-porosity dust aggregates by
the RBD method. Dust aggregates will consist of single minerals
and of mixtures with an emphasized eutectic melting
behavior. Of particular interest is the system Anorthite-Diopside-Forsterite (Hughes 1982; Osborn and Tait 1952; Bowen 1915). Ca,Al minerals are the first condensates to be expected in a high temperature solar nebula. Anorthite and Diopside (to be exact, the Ti-rich variety fassaite) are � besides melilite and spinel - major constituents of Ca,Al rich inclusions (CAIs), the oldest objects from the early solar system. Anorthite and diopside are stable at decreasing temperatures when the Mg-silicates forsterite and enstatite start to condense (Grossman \& Larimer 1974, Petaev\& Wood 1998, 2004; Lodders 2003; Gail 2003), which represent the major mass fraction of silicate material in the solar nebula.

To study the influence of sintering and eutectic melting on the structure of
the dust aggregates for a large variety of dust materials (in the
work by Poppe (2003), only $\rm SiO_2$ was used), the samples will
be heated in a vacuum furnace at a given temperature and for a
pre-set heating duration. Hereafter, they will be recovered for
analytical inspection by optical and secondary electron microscopy
(and also by thin sectioning and electron microprobe analysis
to determine changes in chemical composition at grain boundaries).
Macroscopic samples will also be analyzed with respect to changes
in their mechanical properties (compressive and tensile
strengths); microscopic samples can be studied under the electron
microscope using the nano-manipulator of the Max Planck Institute
for Polymer Research in Mainz.

While these experiments are under way, the existing experimental
setup for the investigation of the collision and sticking behavior
of single, micrometer-sized dust particles will be adapted to
allow high-temperature studies. The sample reservoir, the cogwheel
deagglomerator and accelerator, and the target holder will be
equipped with inductive heaters and temperature sensors. For
thermal insulation of the experiment, the components will be
transferred to a high-vacuum tank (possibly with water cooling).
High-vacuum conditions are required for thermal insulation of the
experimental setup from the hot interior. In addition to that,
oxidization of the minerals will be efficiently prevented. After
completion of the setup, systematic impact experiments of dust
particles consisting of the above-mentioned materials with targets
of identical or different materials will be conducted. Experiments
will start at room temperature and will then be extended to
1,300~K. Changes in the collision and sticking behavior, e.g.\
decreased rebound velocity or increased threshold velocity for
sticking, will be observed through microscopic monitoring of
individual impacts. A difficulty arises from the extremely short
radiation cooling timescales of free micrometer-sized particles of
only a few milliseconds for very hot dust grains. Possible
countermeasures are very short experiment durations (possible for
high-velocity impacts $\gtrsim 10$~m/s or a radiation equilibrium
between particles and environment, which can be reached by a hot
enshrouding of the interaction volume.

For mineral pairs for which eutectic melting in collisions is
expected, we will also setup an accelerator for projectiles in the
size range from 0.1 to 10~mm. Impacts of these macroscopic
particles will be observed by high-speed imaging using telecentric
optics. For these particle sizes, the cooling timescales of free
particles are much longer than an individual experiment duration.
After the collision experiments, the particles will be sampled and
their surfaces will be analyzed by SEM. A model for the
physico-chemical behavior of the contacting dust particles will be
developed and compared to the results of the collision
experiments.

In the following project stages, we will investigate collisions
among macroscopic dust aggregates (including sintered aggregates)
at elevated temperatures. In principle, we have three types of
aggregate-aggregate experiments available for the study of (1) the
collision behavior between mm-sized agglomerates in the velocity
regime between $\sim 0.5$ m/s and $\sim 20$ m/s (Braunschweig),
(2) impacts of mm-sized agglomerates into high-porosity cm-sized
agglomerates for velocities in the range $\sim$ 0.5 m/s \dots\ $\sim$ 20 m/s
(Braunschweig), and (3) impacts of cm-sized agglomerates into
dm-sized target agglomerates of intermediate to low porosities in
the velocity range from $\sim 10$ m/s to $\sim 100$ m/s
(M\"unster). For the impact and collision experiments at higher
temperatures, heating facilities for projectiles and targets must
be developed and implemented. Depending on the experimental
results, it could become necessary to perform the impact and
collision experiments under microgravity conditions. Therefore, we
will foresee two drop-tower campaigns. The existing
(room-temperature) experiments in Braunschweig and M\"unster are
already microgravity-compatible.

In the final stage, we will incorporate the experimental results
on the influence of elevated temperatures, sintering, and eutectic
melting into dust coagulation models for protoplanetary disks (see
project \projdul{}). This will be done in close cooperation with
the other groups involved in this Forschergruppe.

The groups in Braunschweig and M\"unster have worldwide unique experimental
equipment and expertise in coagulation experiments and preparation of high-
and low-porosity dust samples (see e.g.\ Wurm \& Blum 1998, Blum et
al.~1999, Blum et al.~2000, Blum \& Wurm 2000, Wurm \& Blum 2000, Wurm et
al.~2001, Krause \& Blum 2004, Blum \& Schr\"apler 2004, Wurm et al.~2005a,
2005b). This expertise will be interdisciplinarily combined with the
mineralogical and cosmochemical expertise of the group at the Mineralogical
Institute in Heidelberg, which has general broad experience in analytical
and experimental mineralogy and petrology, and also particular expertise in
studying meteorites, and their inferences on the chronology and conditions
of planetesimal accretion in the early solar system (e.g.\ Trieloff et al.~2003).


\subsection{Schedule}
\subsubsection{First year}
%
% I added the following section (Mario)
%
Mineral and grain-size separates will be prepared at the
Mineralogical Institute of the University of Heidelberg by
appropriate techniques (mineral separation, milling, sieving, etc.). As
materials we will use either synthesized, commercially
available (e.g.\ Crystec GmbH, Berlin/ SPI supplies) or natural materials
(SPI Supplies/C.M. Taylor collection), see also project \projlattard{}. If necessary, appropriate minerals can also be synthesized by
high-temperature devices under controlled oxygen fugacity.
Chemical homogeneity and purity will be controlled by secondary
electron microscopy and electron microprobe (if additionally
needed, the ion microprobe) techniques, all equipment available at
the Mineralogical Institute Heidelberg.

With the samples from Heidelberg, room-temperature RBD experiments
for the formation of macroscopic dust aggregates will be
performed. These samples will be heated in vacuum furnaces in
Heidelberg and studied with respect to
morphological/mineralogical/chemical (Heidelberg) and mechanical
(Braunschweig, Mainz) changes. Parallel to these experiments, the
experimental setup for the study of individual collisions of
micrometer-sized grains with flat targets will be adapted to
higher temperatures.

\subsubsection{Second year}
Systematic experiments on the impact and coagulation behavior of
single, micro\-me\-ter-sized dust grains will be conducted for
velocities $\lesssim 50$~m/s and temperatures $300 \ldots 1,300$~K
with mono-phase and binary-phase materials. In addition to that,
we will start to adapt the impact and collision experiments of
dust aggregates to higher temperatures.

\subsubsection{Third year}
Systematic experiments on the collision behavior of mono-phase and
binary-phase dust aggregates will be conducted at elevated
temperatures. The result of the experiments will enable us to
determine the importance and impact of eutectic melting for the
sticking properties of protoplanetary dust aggregates. Finally,
the transfer of the experimental results to protoplanetary
coagulation models will be realized in close cooperation with the
other groups of this Forschergruppe. Publication of the results
closes this work.

\subsection{Literature}
%
% Here follows a general literature list related to the topic of the
% proposal, just like a literature list for a scientific paper.
%
% AGAIN ONLY EXAMPLES ARE LISTED NOW
%
\begin{literature}

\item Bizzarro, M., Baker, J. A. and Haack, H. (2004).  Mg isotope evidence
for contemporaneous formation of chondrules and refractory inclusions,
\textit{Nature} \textbf{431}, 275--278.

\item Blum, J., Wurm, G., Poppe, T. and Heim, L.-O. (1999) Aspects
of Laboratory Dust Aggregation with Relevance to the Formation of
Planetesimals. In: \textit{Laboratory Astrophysics and Space
Research}, Astrophysics and Space Science Library, Vol. 236 (Eds.
P. Ehrenfreund, K. Krafft, H. Kochan, V. Pirronello) Kluwer
Academic Publishers, Dordrecht, 399

\item Blum, J. and Wurm, G. (2000) Experiments on Sticking,
Restructuring and Fragmentation of Preplanetary Dust Aggregates.
\ica, \textbf{143}, 138

\item Blum, J., Wurm, G., Kempf, S., Poppe, T., Klahr, H., et al.
(2000) Growth and Form of Planetary Seedlings: Results from a
Microgravity Aggregation Experiment. \prl, \textbf{85}, 2426

\item Blum, J. and Schr\"apler, R. (2004) Structure and Mechanical
Properties of High-Porosity Macroscopic Agglomerates Formed by
Random Ballistic Deposition, \prl, \textbf{93}, 115503

\item Bowen, N.L. (1915) The crystallization of haplobasaltic, haplodioritic, and related magmas. Am. J. Sci. 4th Ser. \textbf{40}, 161

\item Boyer, T.H. (1975) Temperature Dependence of Van der Waals
Forces in Classical Electrodynamics with Classical Electromagnetic
Zero-Point Radiation, Phys. Rev. A, \textbf{11}, 1650

\item Dominik, C. and Tielens,  A.~G.~G.~M. (1997) The Physics of
Dust Coagulation and the Structure of Dust Aggregates in Space.
\ica, \textbf{480}, 647

\item Gail, H.-P. (2003) Formation and Evolution of Minerals in
Accretion Disks and Stellar Outflows. In:
\textit{Astromineralogy}, Lecture Notes in Physics Vol. 609 (Ed.
Th. Henning), Springer, Heidelberg. 55

\item Grossman, L. \& Larimer, J.~W. 1974 Early chemical history of the
Solar System. Rev. Geophys Space Phys. \textbf{12}, 71-101

\item Heim, L.-O., Blum, J., Preuss, M. and Butt, H.-J. (1999)
Adhesion and Friction Forces Between Spherical Micrometer-Sized
Particles. \prl, \textbf{83}, 3328

\item Heim, L.-O., Butt, H.-J., Schr�pler, R. and Blum, J. (2005)
Analyzing the Compaction of High-Porosity Microscopic
Agglomerates, Australian Journal of Chemistry, \textbf{58(9)} 671

\item Hughes C.J. (1982) Igneous Petrology. Elsevier, Amsterdam, 551pp. 

\item Kleine, T., Mezger, K., Palme, H., Scherer, E. and M\"unker,
C.\ (2005).  Early core formation in asteroids and late accretion of
chondrite parent bodies: Evidence from 182Hf-182W in CAIs, metal-rich
chondrites and iron meteorites.  \textit{Geochim. Cosmochim. Acta}, \textbf{69} 5805. 

\item Korochantseva, E.~V., Trieloff M., Buikin A.~I., Meyer H.~P.
and Hopp J. (2005) Argon-40/Argon-39 dating, and cosmic ray
exposure time of desert meteorites: Dhofar 300 and Dhofar 007
eucrites and anomalous achondrite NWA 011, \textit{Meteoritics and
Planetary Science \/} \textbf{40}, 1433

\item Krause, M. and Blum J., (2004) Growth and Form of Planetary
Seedlings: Results from a Sounding Rocket Microgravity Aggregation
Experiment. \prl, \textbf{93}, 021103

\item Kunz J., Trieloff M., Bobe K., Metzler K., St\"offler D. and
Jessberger E.~K. (1995) The collisional history of the HED parent
body inferred from 40Ar-39Ar ages of Eucrites. \textit{Planetary
and Space Science\/} \textbf{43}, 527

\item Lodders, K. (2003) Solar System Abundances and Condensation
Temperatures of the Elements. \apj, \textbf{591}, 1220

\item Osborn, E.F. and Tait D.B. (1952) The system diopside-forsterite-anorhtite. \textit{Am. J. Sci., Bowen Vol.}
\textbf{250A}, 413.

\item Petaev, M. I. and J. A. Wood (1998) The Condensation with Partial
Isolation (CWPI) model of nebular condensation in the Solar Nebula.
\textit{Meteorit. Planet. Sci.}
\textbf{33}, 1123-1137.

\item Petaev, M. I. and  Wood, J. A. (2004) Meteoritic constraints on
temperatures, pressures, cooling rates, chemical compositions, and modes of
condensation in the solar nebula. In Chondrites and the Protoplanetary Disk
(Eds. A.~N.  Krot, E.~R.~D. Scott, and B. Reipurth), Vol. 341.

\item Pellas P., Fieni C., Trieloff M. and Jessberger E.~K. (1997)
The cooling history of the Acapulco meteorite as recorded by the
244Pu and 40Ar-39Ar chronometers. \textit{Geochimica et
Cosmochimica Acta\/} \textbf{61}, 3477

\item Poppe, T., Blum, J. and Henning, Th. (1997) Generating a jet
of deagglomerated small particles in vacuum. \textit{Review of
Scientific Instruments\/}, \textbf{68}, 2529

\item Poppe, T., Blum, J. and Henning, Th. (2000a) Analogous
Experiments on the Stickiness of Micron-Sized Preplanetary Dust.
\apj, \textbf{533}, 454

\item Poppe, T., Blum, J. and Henning, Th. (2000b) Experiments on
Collisional Grain Charging of Micron-sized Preplanetary Dust.
\apj, \textbf{533}, 472

\item Poppe, T. (2003) Sintering of highly porous silica-particle
samples: analogues of early Solar-System aggregates. \ica,
\textbf{164}, 139

\item Poppe, T. and Schr\"apler, R. (2005) Further Experiments on
Collisional Tribocharging of Cosmic Grains. \aap, \textbf{438}, 1

\item Schwarz W.~H., Trieloff M. and Altherr R. (2005) Subduction
of solar type noble gases from extraterrestrial dust: Constraints
from high-pressure low-temperature metamorphic deep sea sediments,
\textit{Contributions to Mineralogy and Petrology\/} \textbf{149},
675

\item Sears, D. (2004) The origin of chondrules and chondrites.
Cambridge: Cambridge University Press

\item Trieloff, M., Jessberger, E.~K., Herrwerth, I., Hopp, J.,
Fi\'eni, C., Gh\'elis, M., Bourot-Denise, M. and Pellas, P. (2003)
$^{244}$Pu and $^{40}$Ar-$^{39}$Ar thermochronometries reveal
structure and thermal history of the H-chondrite parent asteroid.
\nat, \textbf{422}, 502

\item Trieloff M. and Kunz J. (2005) Isotope systematics of noble
gases in the Earth�s mantle: Possible sources of primordial
isotopes and implications for mantle structure. \textit{Physics of
the Earth and Planetary Interiors\/} \textbf{148}, 13

\item Trieloff M. and Palme H. (2006) The origin of solids in the
early solar system. In: \textit{Planet Formation -� Theory,
Observations, and Experiments} (Eds. H. Klahr, W. Brandner)
Cambridge University Press, Cambridge, 64

\item Wurm, G. and Blum, J. (1998) Experiments on Preplanetary
Dust Aggregation. \ica, \textbf{132}, 125

\item Wurm, G. and Blum, J. (2000) An Experimental Study on the
Structure of Cosmic Dust Aggregates and Their Alignment by Motion
Relative to Gas. \apj, \textbf{529}, L57

\item Wurm, G., Blum, J. and Colwell, J.~E. (2001) Aerodynamical
sticking of dust aggregates. \phre, \textbf{64}, 046301

\item Wurm, G., Paraskov, G. and Krauss, O. (2005a) Ejection of
dust by elastic waves in collisions between millimeter- and
centimeter-sized dust aggregates at 16.5 to 37.5 m/s impact
velocities. \phre, \textbf{71}, 21304

\item Wurm, G., Paraskov, G. and Krauss, O. (2005b) Growth of
Planetesimals by Impacts at ~25m/s. \ica, \textbf{178}, 253




\end{literature}



\section{External/International collaborations}
\begin{collablist}


\item[MPIP Mainz] Dr. L.-O. Heim and Prof. H.-J. Butt (Max Planck
Institute for Polymer Research, Mainz) will aid with their
expertise on the micro-manipulation of small particles and their
knowledge about interaction (adhesion, friction) forces between
such particles.

\item[TU Braunschweig] Dr. M. Morgeneyer (Institute for Particle
Technology) will help to analyze the sintered dust samples with
respect to their mechanical properties.

\end{collablist}



\section{Link to other projects of the Forschergruppe}
\begin{linkproj}


\item[\projtscharn{},\projlattard{}] Projects \projtscharn{} and
\projlattard{} will provide input on abundances of various
minerals to be expected at various parts of the disk. Due to the
detailed chemical and mineralogical modelling in projects
\projtscharn{} and \projlattard{}, much better constraints on the
abundances of the minerals will be available to project
\projblumtrie{}. Thus, the laboratory experiments in project
\projblumtrie{} can always be carried out with dust particles
produced under state-of-the-art knowledge on their relevance.

\item[\projblum{},\projwurm{}] A strong collaborate effort between
the three experimental projects \projblum{},\projwurm{}, and
\projblumtrie{} is mandatory for a successful mapping of the
parameter space for protoplanetary dust agglomeration. Due to
personal and institutional overlap of these three projects, a
strong interaction will be guaranteed.

\item[\projkley{}] Information about the temperature dependence of
the collision behavior of protoplanetary dust aggregates will be
useful and valuable information for the development of the SPH
aggregate collision code to be developed in \projkley{}.

\item[\projklahr{}] Project \projklahr{} will provide input on the
typical impact velocities expected for macroscopic
protoplanetesimal bodies and, thus, for the planning of the
experiments in project \projblumtrie{}.

\item[\projdul{}] The experiments in project \projblumtrie{} will
provide valuable input parameters for the dust-dust interactions
for project \projdul{} and will help to define the reaction and
coagulation kernel in Smoluchowski's equation which is the basis
for project \projdul{}. Vice versa, project \projdul{} will be
able to predict aggregate abundances, collision probabilities and
collision velocities in protoplanetary disks, which all are
important input parameters for the experiments in project
\projblumtrie{}. As part of the project, an extensive iteration
between laboratory work and modelling efforts is envisioned.

\item[D1] The textures of agglomerates coagulated under various conditions 
can be directly compared with fine-grained, primitive chondrite assemblages
and cometary dust particles analysed with transmission electron microscopy
/SEM / TOF-SIMS/Nano-SIMS in D1.

\end{linkproj}



\section{\label{persb3}Team members (Zusammensetzung der Arbeitsgruppe)}
%
% NOTE: Only list non-DFG-funded team members.
% NOTE: Also list technical assistants, students etc involved in the project
%
\begin{teamlist}

\item[Blum, J., Prof.~Dr. (C3)]\mbox{}\\
Team leader. Overlooks the experimental and modelling efforts and
supervises the PhD student funded through this project. Provides
extensive expertise in dust collisions, high-speed imaging
techniques, and microgravity experimentation.

\item[Trieloff, M., Priv. Doz. Dr.]\mbox{}\\
Team leader of isotope geochronology group. Supervises mineral
preparation, agglomerate heating studies and analytical studies
using SEM, EMPA, and SIMS.

\item[Wurm, G., Dr.]\mbox{}\\
Team collaborator. Hosts the PhD student for high-temperature
impact experiments of large dust aggregates. Provides expertise in
high-velocity dust impacts and microgravity experimentation.

\item[Altherr, R., Prof. Dr. (C4)]\mbox{}\\
Internal collaborator. Will assist in petrological/mineralogical
Problems.

\item[Lattard, D., Prof. Dr. (C3)]\mbox{}\\
Internal collaborator. Will assist in mineral synthesis.

\item[Poppe, T., Dr.]\mbox{}\\
Internal collaborator. Has lead the earlier work on impact
sticking and charging and will assist the project.

\item[Schr\"apler, R., Dr.]\mbox{}\\
Internal collaborator. Has led the development of the experimental
setup for the impact experiments of micrometer-sized particles;
has extensive experience in charge measurements and the
acceleration and control of microscopic particles.

\item[Meyer, H.~P., Dr.]\mbox{}\\
Internal collaborator. Assistance in electron microprobe analysis.

\item[Ludwig, Th., Dipl. Phys.]\mbox{}\\
Internal collaborator. Assistance in ion microprobe analysis.

\item[Schwarz, W., Dr.]\mbox{}\\
Internal collaborator. Assistance in high-temperature vacuum
heating experiments and ultra-high-vacuum technology.

\item[Hopp, J., Dr.]\mbox{}\\
Internal collaborator. Assistance in high-temperature vacuum
heating experiments and ultra-high-vacuum technology

\item[Krau{\ss}, O., Dr.]\mbox{}\\
Internal collaborator. Dr. Krau{\ss} has extensive experience in
dust-impact research in the laboratory and in the drop tower.

\item[Stoll, B.]\mbox{}\\
Electronics Engineer. Will assist in all electronics work of the
experimental setups.

\item[Jelting, E.]\mbox{}\\
Electronics Engineer. Will assist in all electronics work of the
experimental setups.

\item[Gebauer, K.]\mbox{}\\
Mechanics Technician. Will assist in the construction of new
experimental hardware.








\end{teamlist}
\vspace{1em}



\section{Funding requested}
The following table gives the full overview of requested
funding:\vspace{1\baselineskip}\\
%
% The table that follows is the overview over the full requested
% funding, including the positions, travel, consumables and ``other
% costs'' (which might include transportation costs of radioactive
% material or the rent of a drop tower or such).
%
\centerline{\begin{tabular}{||l|l|l|l||} \hline \hline & Year 1 & Year 2 &
Year 3 \\ \hline %
Personnel (1 PhD-students: E13/2)   & \hfill 24,000 & 24,000 & 24,000 \\
Equipment                           & \hfill 16,771 & \hfill 2,500 & \hfill - \\
Consumables                         & \hfill 8,408 & \hfill 8,000 & \hfill 7,000 \\
Travel                              & \hfill 5,100 & \hfill 6,700 & \hfill 6,700 \\
Other costs                         & \hfill - & \hfill -     & \hfill -     \\
\hline
{\bf Total:}                        & \hfill 54,279 & \hfill 41,200 & \hfill 37,700 \\
\hline
\hline
\end{tabular}
}
\vspace{1em}\\
Below these costs are explained in more detail:

\subsection{Personnel (Personalbedarf)}
\begin{teamlist}
\item[PhD-Student 1 (E13/2)]\mbox{}\\
The PhD student shall perform independent experimental and
research to reveal the influence of elevated temperatures on the
morphology and mechanical properties and on the collision and
sticking behavior of dust aggregates.
\end{teamlist}

\subsection{Equipment (Ger\"ate)}

For the experiments to be performed within this project: For the
reduction of thermal heat transfer within the vacuum system,
high-vacuum conditions are required. Thus, a mechanical pump is
not sufficient, but needs to be augmented by a turbo-molecular
pump. A complete setup is available from Leybold (PT 50) at a cost
of EUR 7,923. A dedicated vacuum chamber, consisting of a CF-160
double-cross-piece (EUR 3,388 at Leybold) and 6 CF-160 flanges
(EUR 960 at Leybold), will be purchased for the housing of the
experiment. Vacuum gauges are available at the Braunschweig
laboratory at no cost. The samples inside the vacuum chamber will
be heated by inductive heating; we will have to develop a
dedicated inductive-heating system, for which the costs are
estimated to be EUR 2,500 in year 1 and year 2. A work-station PC
with image analysis software for the PhD student is required. The
cost of this PC will be approximately EUR 2,000.

Estimated cost:\vspace{1\baselineskip}\\
\centerline{
\begin{tabular}{||l|l|l|l||} \hline \hline & Year 1 & Year 2 &
Year 3 \\ \hline %
High-vacuum pump &  \hfill 7,923 & \hfill - & \hfill - \\
Vacuum chamber & \hfill 4,348 & \hfill -& \hfill -\\
Inductive heating equipment & \hfill 2,500 & \hfill 2,500 & \hfill
-\\
PC &  \hfill 2,000 & \hfill - & \hfill - \\
\hline
{\bf Total:}    & \hfill 16,771 & \hfill 2,500 & \hfill - \\
\hline \hline
\end{tabular}
}

\subsection{Consumables (Verbrauchsmaterial)}
The experiments require rather large amounts of minerals which
have to be prepared in the Heidelberg laboratory. We estimate the
annual cost for the particles to EUR 1,000. For the heating of the
experimental parts, we require tantalum or molybdenum parts which
need to be manufactured with special tools. An annual expense of
EUR 3,000 is estimated. Thermal insulators (ceramics) for the
experiments will cost EUR 1,000 per year. For the control of
sample and target temperatures, thermo elements (each $\sim$EUR
100) will be used; an annual cost of EUR 1,000 is estimated for
this part. For the connection of the high-vacuum pump to the
experiment chamber, vacuum components (flanges etc.) are required
which will cost EUR 1,000 in year 1 and year 2. The water-cooling
loop for the vacuum components require water-throughput flanges
which cost EUR 204 (Leybold) each. A total of two flanges is
required for the setup. In addition to that, other workshop
materials for the manufacturing of the setups is required. We
estimate an annual cost of EUR 1,000.


Estimated cost:\vspace{1\baselineskip}\\
\centerline{
\begin{tabular}{||l|l|l|l||} \hline \hline & Year 1 & Year 2 &
Year 3 \\ \hline %
Mineral preparation      & \hfill 1,000 & \hfill 1,000 & \hfill 1,000 \\
Vacuum equipment         & \hfill 1,000 & \hfill 1,000 & \hfill -\\
Water-throughput flanges & \hfill   408 & \hfill     - & \hfill
-\\
Workshop material        & \hfill 1,000 & \hfill 1,000 & \hfill 1,000 \\
Ta or Mo parts and tools & \hfill 3,000 & \hfill 3,000 & \hfill 3,000 \\
Ceramics                 & \hfill 1,000 & \hfill 1,000 & \hfill 1,000  \\
Temperature sensors      & \hfill 1,000 & \hfill 1,000 & \hfill 1,000  \\
\hline
{\bf Total:}             & \hfill 8,408 & \hfill 8,000 & \hfill 7,000 \\
\hline \hline
\end{tabular}
}


\subsection{Travel expenses in addition to Project Z (Reisekosten)}
%
% Here only travel expenses not related to usual regular Forschergruppe
% meetings and the overall per capita budget for conferences.
%
Regular project meetings with the participants of this projects
are required. We estimate that two full-day meetings per year (at
various locations) are required. For each person, the cost per
meeting will be EUR 150 for the train, EUR 50 for hotel and EUR 50
for per diem, i.e.\ a total of EUR 250. For each meeting, the
number of travellers is estimated to be two so that a total of EUR
500 is required. In addition to that, two regular one-week
preparation visits per year by the PhD student to Heidelberg and
M\"unster are required. Each trip will cost EUR 150 (train) + EUR
100 (per diem) + EUR 200 (hotel) = EUR 450. Additionally, the PhD
student is required to make extensive (three-week) annual
sample-preparation and measurement campaigns in Heidelberg and
M\"unster. Each campaign will cost EUR 200 (rental car) + EUR 500
(per diem) + EUR 900 (hotel) = EUR 1,600. Two drop-tower campaigns
(three weeks; costs per campaign EUR 200 [rental car] + EUR 500
(per diem) + EUR 900 (hotel) = EUR 1,600) are foreseen for years 2
and 3.

Estimated cost:\vspace{1\baselineskip}\\
\centerline{
\begin{tabular}{||l|l|l|l||} \hline \hline & Year 1 & Year 2 &
Year 3 \\ \hline %
Regular meetings                            & \hfill 1,000 & \hfill 1,000 & \hfill 1,000 \\
Preparation visits                          & \hfill   900 & \hfill   900 & \hfill   900 \\
Sample preparation and measurement campaign & \hfill 3,200 & \hfill 3,200 & \hfill 3,200 \\
Drop-tower campaign                         & \hfill     - & \hfill 1,600 & \hfill 1,600 \\
\hline
{\bf Total:}                                & \hfill 5,100 & \hfill 6,700 & \hfill 6,700 \\
\hline \hline
\end{tabular}
}


\subsection{Other costs (Sonstige Kosten)}
There are no other costs involved.

%%Publication costs will amount to EUR 1,000 for the years 2 and 3.

%Estimated cost per year:\vspace{1\baselineskip}\\
%\centerline{\begin{tabular}{|p{15em}|p{10em}|p{7em}|}
%\hline
%?  & \hfill ? & \hfill ? \\
%\hline
%\end{tabular}}

%Estimated cost:\vspace{1\baselineskip}\\
%\centerline{
%\begin{tabular}{||l|l|l|l||} \hline \hline & Year 1 & Year 2 &
%Year 3 \\ \hline %
%Publication costs & \hfill - & \hfill 1,000 & \hfill 1,000 \\
%\hline
%{\bf Total:}    & \hfill - & \hfill 1,000 & \hfill 1,000 \\
%\hline \hline
%\end{tabular}
%}


\section{Preconditions for carrying out the project at home institution}
%
% This is one of the main subsections of a DFG Normalverfaren proposal.
% Several of the subsubsections in this subsection we have placed in their
% own subsections above (like team members, collaborations). What remains
% are the following three subsections. For those not familiar with these,
% we refer to the DFG Merkblatt on Normalverfahren-proposals.
%
\subsection{\label{equipb3}Scientific equipment available (Apparative Ausstattung)}
%
% Please list those larger instrument available to you for the project (if
% applicable also larger computer equipment in case you need substantial
% amounts of computer time).
%
In the Braunschweig, Heidelberg and M\"unster laboratories, the
following equipment is available for this project:
\begin{itemize}

\item High-speed, high-resolution camera Mikrotron 1310, from 500
frames per second at 1,280 $\times$ 1,024 pixel to 10,000 frames
per second at 65 $\times$ 1,024 pixel.

\item Ruggedized high-speed, high-resolution camera Vossk\"uhler
HCC, from 462 frames per second at 1,024 $\times$ 1,024 pixel to
1,386 frames per second at 256 $\times$ 1,024 pixel for drop-tower
experiments.

\item Long-distance microscope with 80 mm working distance and
$\sim 1~\rm \mu m$ resolution for the impact experiments of
micrometer-sized dust particles.

\item Telecentric objective lenses for the impact experiments with
large projectiles.

\item High-speed flash lamps (up to 1,000 per second) with
ultrashort ($\sim 1~\rm \mu s$) flash duration for the impact and
collision experiments of macroscopic dust agglomerates.

\item Continuous and pulsed laser diode for the impact experiments
of micrometer-sized dust particles.

\item Halogen lamps for the impact and collision experiments of
macroscopic dust agglomerates.

\item Cogwheel-type dust deagglomerator and accelerator (up to
$\sim 50$ m/s) for the impact experiments of micrometer-sized dust
particles.

\item Velocity filter for the precipitation of unwanted particle
sizes and velocities in the impact experiments of micrometer-sized
dust particles.

\item Stepper-motor-driven accelerator (under construction) for
high-porosity dust aggregates and velocities up to $\sim 20$ m/s
(collision experiments of dust aggregates) and solid projectiles.

\item Cross-bow accelerator for low- to medium-porosity dust
aggregates and velocities up to $\sim 100$ m/s (high-speed impact
experiments with medium- to low-porosity dust aggregates).

\item High-precision balance (Sartorius MC 210 P; measurement
sensitivity $10^{-5}$~g at 200~g maximum mass) for target mass
determination.

\item Vacuum gauges for high- and low-vacuum conditions.

\item Various ultra-high-vacuum furnace types: induction heated
glass furnaces, resistance heated double vacuum furnaces with
Ta-crucibles for sample heating.

\item Secondary electron microscope with energy dispersive system.

\item Electron microprobe Cameca SX51.

\item X-ray fluorescence spectrometer SIEMENS 303.

\item Laboratory for mineral separation (milling, magnetic and
heavy liquid separation), optical microscopes, binocular
microscope.

\item Experienced mechanical and electronics workshops.

\item Ion microprobe Cameca 3 mf for trace element analysis,
lateral resolution 10-30 mm.

\end{itemize}

\subsection{Institution's general contribution (Laufende Mittel f\"ur Sachausgaben)}
%
% Please state the annual fund for consumables which comes from the
% institution's budget or any other third party  (please list separately) to
% pay for the research for which your project is part of.  Use estimates where
% applicable.
%

Besides the scientific instrumentation (see Sect. \ref{equipb3})
and personnel costs (see Sect. \ref{persb3}), the proposing
institutions will support this project with EUR 1,000.

%
% Please state the annual fund for consumables which comes from the
% institution's budget or any other third party  (please list separately) to
% pay for the research for which your project is part of.  Use estimates where
% applicable.
%

%We estimate that the running costs per year of our equipment
%are:\vspace{1\baselineskip}\\
%%
%\centerline{\begin{tabular}{|p{18em}|p{7em}|p{7em}|}
%\hline
%?                & \hfil ? & \hfil ? \\
%\hline
%?                & \hfil ? & \hfil ? \\
%\hline
%\end{tabular}}




