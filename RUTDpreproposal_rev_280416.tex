\documentclass[12pt]{article}
\usepackage{graphicx}
\usepackage{amsmath}
\usepackage{amssymb}
\usepackage{float}


%\setlength{\parindent}{0em}
\setlength{\oddsidemargin}{0em}
\setlength{\evensidemargin}{0em}
\setlength{\textwidth}{460pt}
%\setlength{\textheight}{630pt}
\setlength{\textheight}{680pt}
\setlength{\topmargin}{0pt}
\setlength{\voffset}{-25pt}


\def\team#1{{#1\/}}
\def\coll#1{{#1\/}}
%\def\remark#1{{\bf [#1]}}
\def\remark#1{}


\begin{document}

\mbox{}
\vspace{4cm}\\

\centerline{\huge \bf Planet Formation Witnesses and Probes:}
\vspace{0.5cm}
\centerline{\huge \bf Transition Discs}
\vspace{3cm}
\centerline{\Large Speaker: Barbara~Ercolano}
\vspace{0.5cm}
\centerline{\Large Co-speaker: Cornelis Dullemond}
\vspace{1.5cm}

\pagebreak[4]


\section*{Participating institutes and 
principle investigators
%\footnote{Only PIs and Co-PIs listed. Co-investigators and 
%collaborators are listed in individual projects.}
}\label{app-institutes}
\begin{itemize}
\item {\bf Ludwig Maximilians University Munich (LMU):}
 \begin{itemize}
  \item University Observatory\\    
         {\em (Prof.~Dr.~Barbara~Ercolano, Prof.~Dr.~Thomas~Preibisch)}
  \end{itemize} 
\item {\bf European Southern Observatory, Garching:}
 \begin{itemize} 
\item Star and planet formation group \\
         {\em (Dr.~Leonardo~Testi)}
\end{itemize}
\item {\bf Max-Planck-Institute for Extraterrestrial Physics, Garching:}
 \begin{itemize} 
\item Centre for Astrochemical Studies \\
         {\em (Prof.~Dr.~Paola~Caselli)}
\end{itemize}
\item {\bf Ruprecht-Karls-University Heidelberg:}
\begin{itemize}
 \item Institut f\"ur Theoretische Astrophysik (ITA)\\
         {\em (Prof.~Dr.~Cornelis~P.~Dullemond)}
\end{itemize}
\item {\bf University of T\"ubingen:}
\begin{itemize}
  \item Institute for Astronomy \& Astrophysics\\
         {\em (Prof.~Dr.~Wilhelm~Kley)}
\end{itemize}
\end{itemize}
%
\mbox{}\vspace{0.2em}\\

\noindent {\bf Speaker:}\\
\noindent Prof.~Dr.~Barbara~Ercolano \hfill Tel.: 089-2180-6974\\
University Observatory  \\
Ludwig Maximilians University Munich \hfill e-mail ercolano@usm.lmu.de\\
Scheinerstr 1, D-81679 M\"unchen\\

\noindent {\bf Co-speakers:}\\
\noindent Prof.~Dr.~Cornelis~P.~Dullemond \hfill Tel.: 06221-544815\\
Institute for Theoretical Astrophysics\\
 Heidelberg University\hfill e-mail dullemond@uni-heidelberg.de\\
Albert Ueberle Str.\ 2, D-69120 Heidelberg\\


\pagebreak[4]

\fontsize{11}{12}\selectfont

\section*{Summary}
%about half a  page

Recent surveys have shown an overwhelming diversity of extrasolar
planetary systems, prompting the question of how some may end up
looking like our own and being able to sustain life. The environment
in which planets form plays a major role in this issue. Planets are
born out of the dust and gas left over whenever a new star forms: the
protoplanetary disc. The initial conditions for planet formation
are thus determined by the protoplanetary discs, which evolve and disperse as they
give birth to planets. Interestingly, the timescales of disc dispersal
are comparable to those of planet formation, suggesting that the
dispersal mechanism dominates disc evolution right at the time at
which planets form.  Conversely, the planet formation process also
strongly affects the disc, making the combined problem of planet formation
and disc evolution a strongly coupled and complex problem. 

Discs on the verge of dispersal, so-called ``TDs'' (TDs),  are thus particularly important
witnesses of the planet formation process, and they can be used
as probes of the different mechanisms at play at this crucial time of
disc evolution. Latest research has shown, however, that TDs, which are usually
identified as discs showing evidence of an (at least partially)
evacuated inner dust hole, are in reality a diverse class of
objects. Some TDs have relatively small dust holes (a few AUs) and are
weakly accreting, if at all. On the other hand an apparently distinct
population of TDs show evidence for much larger inner dust cavities
(up to several tens of AU) and vigorous accretion, signifying that a
large amount of gas is present inside the dust cavity. Different
physical processes may be at play for the formation of different TD
types (e.g. photoevaporation, dust evolution, planet-disc interactions),
each being a piece of the complex planet formation puzzle. 

Until recently, observations of protoplanetary discs provided very few
constraints on our understanding of disc evolution and planet formation. That was in part due
to the lack of spatial resolution of telescope facilities at infrared and
(sub-)millimetre wavelengths, but also in part because protoplanetary discs
tend to be opaque, and therefore much of the planet formation process is
hidden from view. Both these obstacles impede an unobstructed view of the
physical processes happening inside these planetary nurseries. Both problems may now start to
be overcome with the enormous recent advances in the observational
facilities. At near infrared wavelengths and at millimetre wavelengths we
now start to obtain extraordinarily detailed images of these discs. They
turn out to feature complex (often non-axisymmetric) structures that
challenge our theoretical understanding of these discs. In particular,
many TDs show spectacular structures including lopsided blobs, rings,
spirals etc.\ It is suspected 
that some of these complex structures may be caused by newly formed giant
planets that gravitationally perturb the disc, but exciting new
alternative explanations, which not always involve planets are also emerging. 

In our proposed Research Unit we aim at studying various aspects of
TDs, leading to a better understanding of the different formation
mechanisms of this very diverse class of objects. TDs are
only now really becoming spatially resolvable thanks
to facilities like ALMA and VLT-SPHERE, making their study a timely and
urgent task. Only understanding the disc evolution and the planet-disc
interactions allow the large body of existing and planned observations to be
exploited to answer more complex questions like the formation of planetary
systems capable to host life.

The answers to the many unsolved questions require a focussed effort from
several communities to devise a multi-pronged strategy to approach this
complex problem. Specifically, multiwavelength observations of discs at
different stages of evolution together with exoplanet and disc statistics
should be used to constrain a concerted theoretical modelling effort
including the hydrodynamics of the dust and gas component of discs, with and
without planets, joint to chemical and radiative transfer calculations,
particularly of the surface layers and winds of discs in (or just
before) the transition phase. This is the motivation for the proposed Research Unit.


\section*{Introduction and motivation}

The question of whether the Earth may be a unique and special place for life
in our Universe has been the prime motivation for exoplanet finding missions
and continues to be the driving force behind many observational campaigns and
theoretical investigations in the field. While this question may have
partially been answered by the recent exoplanet surveys, which have shown
 that, statistically speaking, most stars in the Milky Way have planetary
  companions (Cassan et al.\ 2012), other surveys have
also highlighted the diversity of exoplanetary systems (Mullally et
  al.\ 2015). The question however remains as to which
initial conditions may lead to the formation of planets. To answer
this question many more aspects of the planet formation process and of their
subsequent evolution must be understood.

The protoplanetary discs which surround young low mass stars provide the gas
and dust from which planets may form. Therefore, understanding the evolution
and final dispersal of these discs plays a central role in understanding the
initial conditions and timescales of planet formation. For the largest part
of their lives, the evolution of the surface density of discs is well
described by simple viscous theory (e.g.\ Hartmann et al.\ 1998; Lynden-Bell
\& Pringle 1974). This predicts a slow, homogeneous dispersal of the
disc. Observations, however, show that the dispersal is not a
continuous process: after having evolved viscously for a few million
years, discs regularly seem to disappear abruptly  (e.g. Kenyon \& Hartmann, 1995; Luhman
et al.\ 2010). Indeed recent studies have shown that the dispersal
timescales are about 10 times faster than the global disc lifetimes, and
that discs mostly disperse from the inside-out (e.g. Ercolano, Clarke \&
Hall 2011; Koepferl, Ercolano et al.\ 2013). `TDs' (TDs), i.e.\ discs
that have an evacuated inner cavity in dust (or at least an inner region
which is severely depleted in optical depth), may represent discs caught on
the last gasps of their lives and may thus provide key insights on the
mechanism responsible for their evolution. 

Photoevaporation by energetic radiation from the central star is currently accepted as one of
the main players in the late evolution of discs and has seen several
dedicated theoretical efforts (e.g. Clarke et al. 2001, Alexander et
al. 2006, Ercolano et al. 2008, 2009, Owen, Ercolano \& Clarke 2010, 2011a,
2012, Gorti, Dullemond and Hollenbach 2009, 2015). Photoevaporation is
successful in reproducing the observed dispersal timescales, the
inside-out mode of disc dispersal, and can reproduce a subset of the
TD demographics. 

However, disc dispersal by photoevaporation, while certainly an important piece of the
puzzle, does not tell the whole story beyond the formation of TDs. It
has recently become apparent that TDs are a very diverse
class of objects. 
Amongst the TD zoology, at least two separate classes seems
to have emerged, which we will refer to in this proposal as Type 1 and
Type 2 TDs. Type 1 TDs have small inner dust holes and show weak or no accretion of
gas onto the central star. Conversely, Type
2 TDs have much larger inner dust cavities and show evidence of vigorous accretion ($\sim$10$^{-8}$
M$_{\odot}$/yr), with rates not too dissimilar from those measured
on primordial discs (e.g. Manara et al. 2014). Type 2 TDs also tend to have high mm flux levels, meaning that
they still contain large amounts of material in their outer
regions. Owen \& Clarke (2012) showed that it is statistically
unlikely that these two groups of objects may be drawn from the same
underlying population. When talking about TD demographics it is then
important to draw the distinction between the two types, as the
dominant formation mechanism is probably different.  

About half of the global TD population consists of Type 1
discs. The formation of Type 1 TDs is generally well reproduced by
photoevaporation models (e.g. Owen et al. 2010, 2011a). The large accretion rates and large hole
radii of Type 2 TDs, on the other hand, are problematic for all 
photoevaporation (e.g. Rosotti, Ercolano et al. 2013, 2015) or grain
growth (Birnstiel, Andrews \& Ercolano 2012) models to date, and the classical explanation is that
these objects may have instead been carved by dynamical interactions
with a forming giant planet (e.g. Zhu et al. 2011). 
Spatially resolved observations of type
2 TDs have shown many bizarre features that are
not generally seen in primordial disks. For instance, they
  often display a huge dust ring (e.g.~Casassus et al.~2013)
  which is sometimes strongly lopsided (e.g.~van der Marel et al.~2013). The currently favored interpretation is that we see a
  key process of planet formation in action here: the mechanism of dust
  trapping in pressure maxima. Another spectacular kind of features often
  seen in TDs is spiral waves similar to those seen in galaxies
  (e.g.~Muto et al.~2012, Benisty et al. 2015, Wagner et al. 2015). Their origin is currently not
  understood, and trying to understand their physics may teach us about
important processes taking place in these discs. One interesting new
scenario, for example, was proposed to explain the bright ring of scattered light of
HD 142527 which has two dark spots on roughly opposite sides. Marino
et al. (2015) suggest that the shadows cast by an inclined small inner
disk could explain the location of the spots, implying that HD 142527
is an "inclined disc inside a disk". In that case, Montesinos et
al. (2016, astro- ph) claim that two shadows on opposite sides of the
bright rim may cause a brief pressure loss and be at the origin of the
m = 2 spiral waves. If confirmed this scenario opens a pandora of new
physical questions as to how could the inner disc have a different
rotation axis as the outer one?  

Both types of TDs can thus provide complementary information about the planet formation process, as they
can inform us over the disc dispersal mechanism, which influences the
physical conditions in the disc at the time of planet formation and
migration. Type 2 TDs, if indeed formed by dynamical interactions with
giant planets are direct witness of the planet formation
process. Importantly, all disc-destruction mechanisms essentially `open up' the
  disc, so that we can peek inside to see its inner workings, and see
  processes of planet formation in action. TDs therefore
  have the unique potential of unveiling key aspects of the planet formation
process.

As well as disc morphology, the interplay between disc evolution and
dispersal and the planet formation process, of which both types of TDs
are a
by-product, is apparent on many other
 levels. For example the lifetimes of protoplanetary discs (a few
Myrs) are comparable to the timescales for planet formation via the core
accretion model. This highlights the relevance of studying discs at the end
of their lives (i.e.\ TDs), and highlights the importance
of the disc dispersal mechanism (e.g. photoevaporation) which sets the
physical conditions in the disc at the time of planet formation. 
At the same time the similarity in the timescales for disc dispersal
and planet formation may also hint at the possibility that the planet
formation process plays a part in the final dispersal of the discs
(e.g.\ Rosotti, Ercolano et al. 2013, 2015). The inside-out dispersal of
protoplanetary discs, forming TDs, is also an important
factor to consider when studying the final architecture of
exo-planetary systems. While planet migration is necessary to explain
(e.g.) the presence of large planets close to the central star, the
so-called ``hot Jupiters'', this process alone cannot explain the pile-up and
deserts in the semi-major axis distribution of exoplanets. An
additional planet-parking mechanism is required, which may be provided by (e.g.)
photoevaporation which opens a gap in the disc, forming a TD and
stopping further planet migration (e.g.\ Alexander \& Pascucci 2012; Ercolano \& Rosotti
2015).

 %
\vspace{1em}
\noindent {\bf Why is a Research Unit on TDs needed Now?}\\
%
 From the above argumentation it becomes clear that TDs are
  unique laboratory experiments created by Nature that allow us to test and
  update our understanding of disc structure and evolution as well as planet
  formation processes on the small scale (dust growth and dynamics) and
  large scale (planet assembly and planet-disc interaction). Now that the
  era of high-resolution optical/infrared and (sub-)millimetre imaging is
  under way and starts revealing complex structures in these discs, the time
  is ripe for a concerted theoretical/numerical modeling counterpart to
  these observational studies. To be able to understand the observed
  features and statistics of TDs, this effort must combine the
  theory of disc structure, evolution and destruction with dust growth and
  dynamics, as well as with planet-disc interaction and dynamics. It can
  therefore not be done in a single or a set of DFG individual proposal, but
  requires a concerted, closely knit small network of projects involving
  several teams. And to make the link to the observations of these disks,
  which are typically done in scattered light (optical/IR), dust thermal
  emission (sub-millimetre) and molecular rotational lines (sub-millimetre),
  our effort has to also include modeling of disc chemistry and radiative
  transfer, as well as team members with detailed understanding of the
  observations and their limitations. This is the motivation of the research
  group we propose here.


\section*{Contribution of this team to the field }
%
%The process of planet formation cannot be observed directly and
%therefore theories of planet formation theories are built using the
%indirect constraints provided by the observable consequences of
%planet formation on the structure and evolution of protoplanetary
%discs, in particular TDs. Indeed recent ALMA
%observations of discs have provided many surprises posing new
%challenges to existing theories of (e.g.) dust aggregation and
%retention, protoplanetary disc dispersal and planet-disc interactions
%and migration. One of the most famous examples is the image of the
%relatively young disc around HL Tau, which already shows a multiple
%ring structure, speculated to be due to planets forming. Transition
%discs are indeed the best laboratories to study these processes,
%which are all likely to play a role in their formation. 
Our team is composed by theorists and observers, who have all
contributed significantly to the state-of-the art of the field today. \\
{\it Speaker: }{\bf Prof. Ercolano (LMU)} has been studying the link between high
energy radiation from the central star and disc evolution and
dispersal (e.g. Ercolano et al. 2008, 2009, Owen, Ercolano et
al. 2010, 2011, 2012). Before Ercolano et al. (2009), the importance
of X-ray radiation from the central star on the dispersal of discs had
not been recognised. This process is now accepted as one of the major
player for disc dispersal, hence setting the timescale for planet
formation. The models have been tested against observables
(e.g. Ercolano et al. 2010; Owen, Scaife \& Ercolano 2013; Koepferl,
Ercolano et al. 2013; Ercolano, Koepferl et al. 2015), producing
several successes, but also opening new questions (e.g. Ercolano \&
Owen 2016), such as those to be
approached as part of the projects proposed here. Prof. Ercolano and
her team have also investigated the effects of X-ray irradiation on
the final parking radius of exoplanets (Ercolano \& Rosotti, 2015), as
well as on the intrinsic (Ercolano \& Glassgold 2012, Mohanty,
Ercolano \& Turner, 2013) and observed
accretion properties of protoplanetary discs (Ercolano et
al. 2014). With the work of Rosotti, Ercolano et al. (2013, 2015) the
interaction between planet formation and photoevaporation was first
taken into account, in an attempt to match TDs
statistics. Prof. Ercolano is also the main author of the dust RT and
photoionisation code {\sc mocassin}, which is one of the main tools
used for the subpojects in area B. For the development of {\sc
  mocassin} Prof. Ercolano received the Royal Astronomical Society
Fowler Prize for early career achievements in 2010 (https://www.ras.org.uk/news-and-press/157-news2010/1713-ras-honours-outstanding-astronomers-and-geophysicists). \\
{\it Co-Speaker:} {\bf Prof. Dullemond (ZAH)} is an expert in modeling the radiative transfer in protoplanetary
disks, to compute the (vertical) disc structure and the disk's appearance as seen
by observational facilities. He is the author of the popular open source RADMC-3D 
radiative transfer modeling package, which he and his team employ to study
protoplanetary disks, and linking models to observations at infrared and submillimetre
wavelengths. He develops new methods for disc modeling, and has been involved
in the development of new radiation hydrodynamics modules for the PLUTO code
(Kuiper et al. 2010) and ZEUS (Ramsey et al. 2015). His group has also played a
leading role in global disc modeling with dust growth and drift (e.g. Brauer et al. 2008;
Birnstiel et al. 2010), the subsequent planetesimal formation (e.g. Drazkowska 
\& Dullemond 2014), and the link between the disk/dust models and millimetre
and infrared disc observations (e.g. Pinilla et al. 2014; van der Marel et al. 2013;
Kataoka et al. 2015). A key theme in the research of Dullemond's group is the study 
of physical processes and how we can constrain them with observations. The
group often develops its own methods of computation and own codes to implement
the new physics in existing models, and thereby opening new
directions. \\
{\it Co-PI:} {\bf Prof. Kley (University of T\"ubingen}) is an expert in computational astrophysics with emphasis on the
planet formation process, starting from the growth of small dust grains all the way
to full grown planets. The numerical methods developed and used in his group range 
from molecular dynamics, smoothed particle hydrodynamics (SPH) and grid-based 
magneto-hydrodynamics (MHD) including radiative transport. These methods will be used
in the theoretical modeling within the Research Group.   
One focus of his research has been on the important planet-disk
interaction. Through multi-dimensional (2D and 3D) radiation hydrodynamical simulations
his group demonstrated the possibility of strongly reduced or even outward migration
(Kley \& Crida, 2008; Kley, Bitsch \& Klahr, 2009). Recently, models for the origin of the
circumbinary planets have been presented (Kley \& Haghighipour, 2014). Here, longterm models of
disks in the presence of a central binary have been simulated and the motion of an embedded
planet has been followed. Concerning the main focus of this research group, M\"uller \& Kley
(2013) constructed time-dependent hydrodynamics models of transitional disks induced by the
presence of a single planet. Specifically, the models investigated the amount of gas flow past
the planet into the inner hole as a function of the planet mass, disc
parameter and stellar irradiation. With Picogna \& Kley (2016)
significant advances were made in the understanding of the dust phase
response to planet-disc interactions. \\
%On the theory side our team includes
%dynamicists, experts in radiative transfer and astrochemistry, who are
%committed to exploit the synergies of the team to provide a holistic
%approach to understanding planet formation and disc evolution, via the
%study of TDs and their formation. \\
{\it Co-PI:} {\bf Prof. Caselli (MPE)} 
% moved to MPE as Director of the new Centre
%for Astrochemical Studies in April 2014, after spending almost seven
%years as Professor of Astronomy at the School of Physics and Astronomy
%at the University of Leeds, UK. She
is an expert on astrochemical
modelling and observations of the earliest phases of star and planet
formation. She has made important contributions on the chemical structure
of pre-stellar cores (within which future stellar systems form)
%finding that: CO molecules are almost completely ($>$90\%) frozen onto
%dust grain surfaces in the central few thousand astronomical units
%(Caselli et al. 1999), thus thick CO-ice mantles are predicted to be
%present just before a protostar is born; deuterated molecules are the
%best diagnostic tools of the physical properties of the central
%regions, the future stellar cradles (Caselli et al. 2002, 2008;
%Crapsi, Caselli et al. 2005, 2007); the tenuous UV field produced by
%cosmic-ray impacts with H$_2$ molecules plays an important role in
%producing water vapour and regulating the gas-solid phase in these
%dense and cold phases preceding star formation (Caselli et al. 2012);
%a dust opacity increase toward the pre-stellar core centre is needed
%to reproduce the observed temperature drop (Keto \& Caselli 2010),
%suggesting that dust grains have already started to coagulate;
%quasi-static contraction best reproduces the observed molecular line
%profiles toward one of the best studied pre-stellar cores (Keto,
%Caselli \& Rawlings 2015), showing in this case that the dynamical
%evolution toward protostellar birth is a slow process, maybe regulated
%by the action of magnetic fields in partially ionised media. 
She is
now focusing on the link between molecular clouds and protoplanetary
disks using high angular resolution observations, hydro- and
magneto-hydrodynamical simulations, which incorporate various degrees
of chemical complexity, and radiative transfer codes. She is
interested in understanding the effects of different initial
conditions in the physical and chemical evolution of protoplanetary
disks. Already published work in this field include the chemical
structure and ALMA observability of a self-gravitating disc orbiting
around a protostar which will likely evolve into a future F-type main
sequence star (Ilee et al. 2011; Douglas et al. 2013) as well as the
chemical evolution of a self-gravitating disc surrounding a
protosolar-type star (Evans et al. 2015). 
%; all this work on young disks
%has been done with students and former students from the University of
%Leeds, partially funded by the ERC grant PALs 320620).
%Self-gravitating disks are thought to be important during the earliest
%phases of protostellar evolution, although their existence still need
%to be confirmed with ALMA.  
Her expertise on basic astrochemical
processes will be applied to the available and future dynamical models
of transition disks. \\
{\it Co-PI:} {\bf Prof. Testi (ESO)}, has accumulated years of expertise
in the observation and analysis of young stars and their protoplanetary
discs at infrared and millimetre wavelengths. He has investigated the initial stages of planet formation via
extensives studies of properties and evolution of dust in discs (e.g. Testi et
al. 2003, 2014). With his group at ESO has completed the first large observational surveys for dust growth
in disks (Ricci et al. 2010ab) and has developed the first self-consistent analysis tool to constrain dust properties
as a function of radius in discs (Banzatti, Testi et al. 2011, Trotta, Testi et al. 2013, Tazzari, Testi et al. 2016). His group also developed the methodology to perform accurate measurements of the photospheric properties and accretion rates from broad-band XShooter spectra (Manara, Testi et al. 2013) and applied it to study the correlation between disc properties and mass accretion rates in young stars with disks (Manara et al. 2016).
His current role as European ALMA Programme Scientist at ESO
puts him in a very favourable position to lead an effort here in
building a systematic catalogue of the available ALMA observations and help with the interpretation of these data in terms of
the models. The new observational
campaigns that we foresee for phase 2 of the Research Unit will also
strongly 
benefit from his guidance. \\
{\it Co-PI:} {\bf Prof. Thomas Preibisch (LMU)}  has many years of experience in the
fields of stellar X-ray astronomy and infrared observations
of young stellar clusters.
He was deeply involved in the
Chandra Orion Ultradeep Project (COUP;  see Preibisch et al. 2005),
the Chandra Carina Complex Project (CCCP; see Preibisch et al.~2011),
and numerous other projects where
the identification of the X-ray sources with optical
and infrared counterparts was a crucial step for the studies
of the relation between the X-ray properties and the stellar/circumstellar
properties of the young stars. He has also performed several large-scale surveys of star forming
regions in the near-infrared (e.g., Preibisch et al.2011,
Preibisch et al. 2014)
and far-infrared regime (e.g, Preibisch, T. et al. 2012)
with the aim to identify protostars and study disk-bearing young
stellar objects.\\
{\it External Collaborator: } {\bf Prof. T. Henning (MPIA)}  is co-I
on the VLT SPHERE disk programme and therefore has prime access to
high-contrast high-resolution scattered light images of Transition
Disks. He has been involved in many protoplanetary disk projects, both
observationally and theoretically. He is co-author of the Klahr \&
Henning (1995) paper predicting the role of dust trapping in vortices
in protoplanetary disks, something which has recently been
observationally confirmed in transition disks. His involvement in this
Research Unit will be through discussions/collaboration on the
modeling and through comparison with the observations. \\
{\it External Collaborator: } {\bf Prof. E. van Dishoeck (MPE/Leiden)}
is one of the founders of ALMA observatory. In recent years her group
has focussed on studying Transition Disks with ALMA. In 2013 her team
published the first strong evidence, based on ALMA data, for a
dust-trapping vortex in the Transition Disk Oph IRS 48. She has been
involved in many protoplanetary disk studies, both observationally as
well as from an astrochemical perspective. She co-leads a large ALMA
proposal on Transition Disks. Her involvement in this Forschergruppe
will be through discussions/collaboration on the modeling and through
comparison with the observations. \\

\section*{Scientific Objectives}

%The formation of planets in protoplanetary discs is clearly affected
%by the structure and the physical properties of the disc itself. The
%structure and evolution of the disc, however, can be also deeply
%influenced by the planet formation process, as demonstrated by our
%recent calculations (e.g. Rosotti et al. 2013, 2015). This has
%important consequences both in the interpretation of the observed
%exo-planet properties (eg. size and semi-major axis distributions) as
%well as in the interpretation of the disc structures, which are often
%used as tell-tale signs for planet formation, as in the case of the
%TDs. Our understanding of how the two processes
%influence each other is however still at best incomplete. A number of
%key questions need to be answered before we can move towards a more
%complete picture of planet formation and disc evolution. 
In the proposed Research Unit we will address the following two questions:

\begin{enumerate}
\item {\bf How can we use TDs as direct witnesses of the planet formation process?}\\ Which TDs are carved by planets and which are a result of disc evolution? \\ What do the complex shapes of TDs (rings, blobs, spirals) teach us about the physical and
dynamical processes taking place in protoplanetary disks?\\ How does
dust evolve and travel within discs to form planet(esimal)s?

\item {\bf How can we use TDs to learn about the disc dispersal
    mechanism?}\\ What are the mass loss rates of the disc wind and
  what parameter space in the TD demographics can thus be reproduced
  by photoevaporation?\\ What are the dust and gas surface density
  distributions in TDs and how can they be
  explained?\\How are the processes of disc accretion and disk
  dispersal, leading to the formation of TDs,
influenced by the high-energy emission from the central star?
\end{enumerate}

In order to answer these questions we have designed a novel strategy
to tackle the interrelated problem of disc evolution and planet
formation, which conspire to produce the diverse population of TDs
observed. Our project exploits the unique synergies amongst the theorists
and observers in our team. Respectively, the following {\bf immediate objectives} relate to the questions set above: 
%\makeatletter
%\renewcommand{\theenumi}{\Alph{enumi}}
%\renewcommand{\labelenumi}{\theenumi.}
%\renewcommand{\theenumii}{\roman{enumii}}
%\renewcommand{\labelenumii}{\theenumii}
%\makeatother
\paragraph {1.1 Investigate observationally the accretion properties
  and the distributions of dust and gas in the discs and winds of
  primordial compared to transition objects.} To this aim a systematic catalogue of observations will be built, which will provide the constraints to the modelling efforts described below. 
\paragraph{1.2 Dust trapping and growth in TDs.} Dust and
gas hydrodynamical models of discs with and without cavities will be produced to match the
constraints from the observations, these will put constraints onto the
planet-disc interaction and photoevaporation models described below. 
\paragraph {1.3 Planet-disc interaction models with photoevaporation
  and dust trapping.} These models will aim at reproducing
observations of (mainly) Type 2 TDs, including their accretion properties, to pin down the formation mechanism of the observed structures. 

\paragraph{2.1 Determine the mass loss-rates of photoevaporative winds
  to pin down the mechanism producing Type 1 TDs.} To
this aim appropriate wind diagnostics need to be identified via
chemical modelling of  photoevaporative winds, comparing primordial to
TDs. Existing archival observations will be used at first to compare
with the models and a new observing campaign with ALMA will be
devised, perhaps spilling into the second funding period of the Research Unit.
\paragraph{2.2 Determine the dominant dispersal mechanism.} We will
use the archival observations and the models from item 2.1 to analyse
an initial sample of discs, the final statistics will be achieved
however in the second funding period, where a population synthesis of
(Type 1) TDs will be attempted.
\paragraph {2.3 Close the loop using mass loss rates, central star
  properties and accretion measurements to calibrate models.}  At this
point the models will have significantly less free parameters and can
be used to extract the initial conditions for planet formation
(e.g. mass, turbulence in evolved discs). This will make use of the new state-of-the
art simulations and a homogeneous sample of accretion rates and central
star properties including X-ray data. 


\section*{Work plan}
%\vspace{1em}\\

%\noindent{\bf \em Overview over the Forschergruppe}\vspace{0.8em}\\
%
The main goal of this Research Unit is to understand the morphology,
spectroscopy and demographics of TDs, in order to answer
basic questions about the planet formation process. We propose a
four-pronged coordinated effort which includes (A) Observational
studies; (B) Disc dispersal models; (C) Dust physics; (D) Planet-disc
interaction models. The division in subfields is not strict and is
only given here in the aid of clarity. Subfields often overlap and/or
feed back on each other, highlighting the need of strong
collaborations as proposed here. The observations will provide the
constraints to be simultaneously met by models developed in the other
areas. Our team is supported by several external collaborators in particular
Prof. T. Henning (MPIA) and Prof. E. van Dishoeck (Leiden/MPE) have
agreed to take an active part in our project. The theoretical models in the three theory subfields require
expertise in hydrodynamics, astrochemistry, dust evolution and
radiative transfer, which is available in our team. The specific
projects in each area, the support required and the respective members
of the team are summarised in the table below. 
\vspace{1.5em}

\noindent
\begin{tabular}{p{1cm}p{8cm}p{2.0cm}p{3.9cm}}
\hline
{\bf A} & {\bf Observations} & & \\ 
A1 & ``Solids evolution in disks: observational constraints'' & 1 PhD & {\em Testi, Ercolano, Preibisch}\\
A2 & ``New constraints about disc-dissipation processes from the relation between accretion and X-ray activity'' & 1 PhD & {\em Preibisch, Ercolano, Testi}\\
\hline
{\bf B} & {\bf Disc dissipation and chemistry} & & \\ 
B1 & ``Disc mass loss from quantitative spectroscopy of
photoevaporative winds'' & 1 Postdoc & {\em Ercolano, Caselli, Dullemond, Kley}\\
B2 & ``Essential astrochemistry of disc winds' & 1 Postdoc & {\em Caselli, Ercolano}\\
\hline
{\bf C} & {\bf Dust physics} & & \\ 
C1 & ``Trapping the dust: Planet formation 'hotspots' in TDs'' & 1 PhD & {\em Dullemond, Kley, Ercolano}\\
C2 & ``Gone with the wind: Dust entrainment in photoevaporative winds from realistic underlying grain distributions "& 1 PhD & {\em Ercolano, Dullemond}\\
\hline
{\bf D} & {\bf Planet-disc interactions} & & \\ 
D1 & ``TDs and planetary systems''' & 1 Postdoc & {\em Kley, Dullemond}\\
D2 & ``Origin of complex non-axisymmetric structures in TDs "& 1 Postdoc & {\em Dullemond, Kley}\\
\hline
\end{tabular}
\vspace{1.5em}

Figure 1 shows schematically the direct links between the listed sub-projects, which provide the foundation for the Phase 2 study. Details about the major links between sub-projects are summarised in the next Section. 

\begin{figure}[H]
\centerline{\includegraphics[width=13cm]{dependency2.jpg}}
\caption{A diagram showing the links between the various
projects. See text for detail.}

%\vspace{2em}\mbox{}\\
%\centerline{\includegraphics[width=16cm]{sizebar.eps}}
%\vspace{1em}\\
%Fig 2: A diagram showing the coverage
%of the size-range between sub-micron dust particles and
%multi-kilometer-sized planetesimals.
\end{figure}

\vspace{1.0em}


%
\noindent\underline{{\bf Area A: Observations}}\\
\noindent The projects in this area both aim at obtaining
observational constraints on the dust and gas properties of
protoplanetary discs and their central stars. Project A1 focusses on
collecting and analysing existing ALMA data and complementing those
with additional ALMA and EVLA observations. Evidence for grain growth
and planet-disc interactions will be characterised in order to provide
direct observational tests of planet formation and dispersal theories,
necessary to interpret the observational appearance of Type 1 and 2 TDs. The focus of project A2 is on the central star properties and their relation to the accretion properties of the disc, which may be modulated by the disc dispersal mechanisms that lead to the formation of TDs. While both of these projects have self-contained aims, they will also provide the observational goalposts for the theoretical investigations of all projects in areas B, C and D, and indeed a legacy for future theoretical studies of this kind also by other groups.    \\

%

%\vspace{1em}\\

%\pagebreak[4]

\
\mbox{}\vspace{1em}\\
\noindent\underline{\bf Area B: Disc dissipation and chemistry }\\
\noindent The main objective of the projects in this area is to
determine the mass loss rates in disc winds, and constrain once and
for all the disc dissipation mechanism, leading to the formation of
Type 1 TDs. This is crucial to interpret the observed Type 1
demographics. We will perform quantitative spectroscopical modelling
of disc winds, identifying and using new wind diagnostics, in
particular comparing primordial and TDs. In project B1 we will make
use of a newly developed hydrodynamics, photoionisation and radiative
transfer models, self-consistently linked to astrochemical models of
the wind, developed in project B2. The projects aim at matching
existing and new observational constraints coordinated in projects A1
and A2, and also provided by our external collaborators. The dust
content of the wind is also of prime importance for the chemical
modelling and will be constrained in project C2. Planet-disc
interactions may help launch winds at earlier times and may cause
strong asymmetries, which will yield different streamline architectures (see e.g. Rosotti et al 2013, 2015), from which we will be informed by project D1.\\

%
\mbox{}\vspace{1em}\\
\noindent\underline{\bf Area C: Dust Physics}\\
\noindent The dynamics and evolution of dust grains in discs is the
main subjects of this area. In project C1 we will study the growth and
trapping of dust grains at hotspots in TDs that may lead
to breaking through the meter-size barrier of radial drift, thus
allowing the formation of planetesimals. We will use (radiation
hydro)-dynamical modelling and dust coagulation models as well as 3D
radiative transfer tools. Furthermore, we will be able to use the observational sample collected and analysed in project A1 to tackle fundamental open questions on the first stages of planet formation. This project will feedback and take inputs from the photoevaporation modelling performed in project B1 and it will produce the underlying dust distributions for project C2, which aims at determining the dust content of photoevaporative winds. The latter will make use of state-of-the art models of photoevaporating primordial and TDs as well as inputs for the dust distributions from project C1, in order to constrain dust entrainment in the wind, which is an important input to the chemical models in projects B1 and B2. \\


%\pagebreak[4]


\mbox{}\vspace{1em}\\
\noindent\underline{\bf Area D: Planet-disc interactions}\\
\noindent Projects in this area aim at constructing realistic simulations of planet-disc interactions to explain the wealth of new and intriguing observations of TDs. The overarching goal is to use these observation-constrained (from project A1) models to pin down important details of planet-disc interaction processes. This is of fundamental importance to be able to disentangle the message about planet formation which is locked in TDs observations. In particular project D1 aims at significantly pushing forward the state-of-the art of (radiation)-hydrodynamical models of gap-forming giant planets embedded in discs. This project will provide important inputs of density distributions for project B1, particularly informing about the fate of dust grains at planet-gaps, which is also relevant to project C1. Project D2 aims at explaining the surprising non-axisymmetric structures recently observed with ALMA in a number of TDs, via detailed hydrodynamical and radiative transfer models, which will also account for realistic grain size distributions from project C1. Studying the intriguing nature of these objects is likely to provide important insights on planet-disc interaction processes, thus enabling us to use these discs as proxies for planet formation.  \\


% DO WE REALLY NEED THIS IN SUCH A SMALL SPACE?
%\mbox{}\vspace{1em}\\
%\noindent{\bf \em Timeline and Science Products:}\vspace{0.8em}\\
%
%\noindent The first ?? years will be ...

%{\em Area A:} 

%{\em Area B:}


%{\em Area C:} 

%{\em Area D:}

% Examples of such output-input exchange are:
% using collision velocity predictions from hydrodynamic simulations in lab
% experiments, taking into account the results from lab collision experiments
% and theory in coagulation models, using new lab data of thermal processing
% of minerals in theoretical models, using refined models of dust evolution
% and transport in discs for radiative transfer models for infrared spectra.

\mbox{}\vspace{1em}\\
\noindent{\bf \em Future perspectives}\\
%
While all sub-projects presented here are self-contained and will
provide specific intermediate science products, the major strength of this program is the collaborative
work to produce a holistic picture of protoplanetary disc evolution
and dispersal, which through the study of TDs can be used
to inform us on the planet formation process. At the end of Phase 1 of
the Research Unit all theoretical models will have been significantly
improved to allow a much more realistic approach to match the
observational constraints. At the same time the systematic analysis of
the existing observation and the collection of new data will have also
provided a much clearer picture of disc structures, as possible planet
formation signatures. At the end of Phase 1 our team will be then
perfectly posed to perform the most advanced simulations of individual
objects, but also and perhaps more importantly, we will be able to
tackle the issue of demographics of TDs. Via population
synthesis models of TDs including disc dispersal, planet formation, dust
evolution, some simplified chemistry and radiative transfer, we will
be able for the first time to use discs to make predictive models of
planet formation and evolution to match existing exo-planet
statistics. 

We thus foresee a two-pronged approach in Phase 2: 

i) Detailed modelling of individual objects.\\
This will follow mainly from the joint work of areas A (in particular
A1) and D, and will target mainly Type 2 TDs. The insights gained in
Phase 1 of the Research Unit will allow us to construct tailored
models of planet-disc interactions to explain specific observations of
Type 2 TDs (not necessarily obtained by our group). The tailored
models will allow us to decode the message in the many interesting new
features highlighted already by spatially resolved observation, which,
by the beginning of the second funding period, will have surely
delivered more surprises.  

ii) Statistical distributions/demographics of Type 1 TDs. \\
The work carried out in Phase 1 in areas A, B and C will have resulted
in the most advanced disc dispersal models, which would have also been
calibrated for important quantities with direct observations. We will
use these models to construct population synthesis of Type 1
TD demographics (e.g. inner hole radius vs accretion
rate), to compare with available surveys in individual clusters.  
This will be a very strict, direct test of our disc models, and will
allow us to predict realistic initial conditions in a disc population at the
time of planet formation and migration, which are fundamental inputs
for planet formation models (the latter are however beyond the scope of this
proposal).  

It is inappropriate at this stage to design a more detailed program
for a potential Phase 2 of the Research Unit. As well as depending on
the outcome of our Phase 1 projects, the exact focus of a potential
extension would depend on the developments of the field as a
whole. Some general ideas go in the direction of providing the basis for more detailed chemical
and physical evolution of Solar-Nebula-type discs, needed to gain a
better understanding of the Solar System origin and composition,
including the large variety of minerals and organics found in comets
and meteorites. 

\mbox{}\vspace{1em}\\
\noindent{\bf \em Interactions between projects and
  institutions.}\\
\noindent The projects planned are not only collaborative amongst institutions,
but also present strong dependancies from each other, which is the
motivation for setting up a Research Unit. Several members of the
network already work regularly together, so we foresee that this compact
network will benefit from close-knit collaboration amongst its members.
As well as links between the projects we have also links between
various institutions within one project, which are led by people from
different institutions. We request a substantial budget for longer
working visits so that students/postdocs will not work in just one
institute, but effectively work in multiple. Furthermore, we foresee
strong links between our groups and those of our external
collaborators, in particular Prof. T. Henning (MPIA) and Prof. E. van
Dishoeck (Leiden/MPE). Each group will reserve a workplace especially for such exchanges, both within Munich/Garching as well as across the cities. 
Furthermore, we plan to keep communications between teams in
sub-projects as lively as possible. We foresee the following: (1) Kick
off and Wrap up events at the start and end of Phase 1 of the Research
Unit. These events will also be open to non-members, who may provide
external support and collaborate on some of the projects; (2) Twice a
year two days face to face meeting with all members at rotating
locations amongst the institutes in the Research Unit; (3) Once a
month a Video Conference amongst members of a given area, where the young
researchers will present status reports of the various sub-projects. A
concise summary of the meeting will then be compiled on a rotation
basis by the students/postdocs in the area and then distributed to
all members. (4) Collaboration meetings between individual projects.

\mbox{}\vspace{1em}\\
\noindent{\bf \em Promotion of Early career researchers}\\
%
\noindent We have requested a mix of PhD and Postdoc positions, the
latter because of the complex computational aspects of some of the
projects in this building phase of the Research Unit. We are very
committed to the training and the promotion of young researchers, and
indeed most of the Postdocs are planned to be junior positions (within
three years of PhD). The nature of the projects themselves is
favourable to the promotion of early career researchers by providing
them with definite and clear science products as well as training them
in highly sought-after skills, which will propel them in today's
challenging academic world. The Research Unit with the thriving
interactions and the teamwork towards a common goal also provide a
perfect environment to develop a broad view of interrelated, but
fundamentally different research areas, which will especially benefit
PhD students and academically young postdocs. The postdocs will be
encouraged and supported to develop independence, which will set them
up on a career path to individual fellowships. 

\mbox{}\vspace{1em}\\
\noindent{\bf \em Gender Equality Measures}\\
Our PI team is already relatively balanced in terms of gender. We will
continue our efforts to promote this issue within the schemes provided
by our own host institutions and will work towards creating a still
more balanced team of female and male PhDs and Postdocs in this
Research Unit.  

\mbox{}\vspace{1em}\\
\noindent{\bf \em Family Support Measures}\\
Family support is of prime importance in our Research Unit, and we are
aware of the complicated issues of work/family balance. In particular
care will be taken in the scheduling of the biannual Research Unit
meetings, which will necessarily involve traveling, in order to
minimise disruption to family life. When possible we will use
video-conferencing (e.g. for the regular monthly meetings), and will
provide financial support to members who wish to travel with their
families for the face-to-face meetings, which we foresee to be at
least twice a year (on top of the Phase 1 kick-off and wrap-up
meetings), and for the longer collaboration visits. 

\pagebreak[4]



%*******************************************************************************
%*******************************************************************************
%*******************************************************************************

%\pagebreak[4]
%\mbox{}
%\pagebreak[4]

\input projects.tex
\pagebreak[4]

\input references

%*******************************************************************************
%*******************************************************************************
%*******************************************************************************



\end{document}
